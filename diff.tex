\documentclass[12pt]{tufte-book}
%DIF LATEXDIFF DIFFERENCE FILE
%DIF DEL notes-F24-old.tex   Tue Nov  5 23:59:23 2024
%DIF ADD notes-F24.tex       Tue Nov  5 23:51:00 2024
\usepackage{amsthm,amssymb,amsmath,thmtools,datetime,tikz}
\setcounter{secnumdepth}{3}

\declaretheorem[numberwithin=chapter,shaded={bgcolor=Lavender}]{definition}

\declaretheorem[numberwithin=chapter,shaded={bgcolor=Thistle}]{lemma}
\declaretheorem[numberwithin=chapter,shaded={bgcolor=Thistle}]{claim}

\declaretheorem[numberwithin=chapter,shaded={bgcolor=Apricot}]{theorem}

\declaretheorem[numberwithin=chapter,shaded={bgcolor=yellow}]{remark}
\declaretheorem[numberwithin=chapter]{exercise}
\declaretheorem[numberwithin=chapter,shaded={bgcolor=pink}]{construction}
\usepackage[
    type={CC},
    modifier={by-nc-nd},
    version={4.0},
]{doclicense}

\usepackage{graphicx,xcolor,mdframed}
%\usepackage[version=0.96]{pgf}
\usepackage{enumitem}


\def\chpcolor{blue!45}
\def\chpcolortxt{blue!60}

\iffalse
\titleformat{\chapter}%
  {\huge\rmfamily\itshape\color{red}}% format applied to label+text
  {\llap{\colorbox{red}{\parbox{1.5cm}{\hfill\itshape\huge\color{white}\thechapter}}}}% label
  {2pt}% horizontal separation between label and title body
  {}% before the title body
  []% after the title body
\fi

\hypersetup{colorlinks}% uncomment this line if you prefer colored hyperlinks (e.g., for onscreen viewing)

%%
% Book metadata
\title{A Course in Theory of Cryptography}
\author[Sanjam Garg]{Sanjam Garg}
%\publisher{Publisher of This Book}

%%
% If they're installed, use Bergamo and Chantilly from www.fontsite.com.
% They're clones of Bembo and Gill Sans, respectively.
%\IfFileExists{bergamo.sty}{\usepackage[osf]{bergamo}}{}% Bembo
%\IfFileExists{chantill.sty}{\usepackage{chantill}}{}% Gill Sans

%\usepackage{microtype}

%%
% Just some sample text
\usepackage{lipsum}
\newcommand{\ma}{\mathcal{A}}
\newcommand{\cA}{\mathcal{A}}
\newcommand{\mb}{\mathcal{B}}
\newcommand{\cB}{\mathcal{B}}

\newcommand{\getsr}{\xleftarrow{\$}}
\newcommand{\bit}{\{0,1\}}

\newcommand{\Gen}{\mathsf{Gen}}
\newcommand{\gen}{\mathsf{Gen}}
\newcommand{\Enc}{\mathsf{Enc}}
% \newcommand{\enc}{\mathsf{Enc}}
\newcommand{\Dec}{\mathsf{Dec}}
% \newcommand{\dec}{\mathsf{Dec}}
% \newcommand{\pk}{\mathsf{pk}}
% \newcommand{\sk}{\mathsf{sk}}
\newcommand{\ek}{\mathsf{ek}}

% \newcommand{\concat}[0]{\; || \;}

% \newcommand{\bin}{\{0,1\}}
% \newcommand{\adv}{\mathcal{A}}
% \newcommand{\advb}{\mathcal{B}}
% \newcommand{\advc}{\mathcal{C}}
% \newcommand{\fake}{\mathsf{FAKE}}

\newcommand{\Sign}{\mathsf{Sign}}
\newcommand{\Verify}{\mathsf{Verify}}
%\newcommand{\negl}{\mathsf{negl}}
\newcommand{\abort}{\mathsf{abort}}
\newcommand{\Sampler}{\mathsf{Sampler}}
\newcommand{\Eval}{\mathsf{Eval}}
\renewcommand{\tag}{\mathsf{tag}}
\newcommand{\PRF}{\mathsf{PRF}}
\newcommand{\LWE}{\mathsf{LWE}}
%%
% For nicely typeset tabular material
\usepackage{booktabs}

\usepackage[n,advantage,operators,sets,adversary,landau,probability,notions,logic,ff,mm,primitives,events,complexity,oracles,asymptotics,keys]{cryptocode} 
%%
% For graphics / images
\usepackage{graphicx,algpseudocode}
\setkeys{Gin}{width=\linewidth,totalheight=\textheight,keepaspectratio}
\graphicspath{{graphics/}}

% The fancyvrb package lets us customize the formatting of verbatim
% environments.  We use a slightly smaller font.
\usepackage{fancyvrb}
\fvset{fontsize=\normalsize}

%%
% Prints argument within hanging parentheses (i.e., parentheses that take
% up no horizontal space).  Useful in tabular environments.
\newcommand{\hangp}[1]{\makebox[0pt][r]{(}#1\makebox[0pt][l]{)}}

%%
% Prints an asterisk that takes up no horizontal space.
% Useful in tabular environments.
\newcommand{\hangstar}{\makebox[0pt][l]{*}}

%%
% Prints a trailing space in a smart way.
\usepackage{xspace}

%%
% Some shortcuts for Tufte's book titles.  The lowercase commands will
% produce the initials of the book title in italics.  The all-caps commands
% will print out the full title of the book in italics.
\newcommand{\vdqi}{\textit{VDQI}\xspace}
\newcommand{\ei}{\textit{EI}\xspace}
\newcommand{\ve}{\textit{VE}\xspace}
\newcommand{\be}{\textit{BE}\xspace}
\newcommand{\VDQI}{\textit{The Visual Display of Quantitative Information}\xspace}
\newcommand{\EI}{\textit{Envisioning Information}\xspace}
\newcommand{\VE}{\textit{Visual Explanations}\xspace}
\newcommand{\BE}{\textit{Beautiful Evidence}\xspace}

\newcommand{\TL}{Tufte-\LaTeX\xspace}

% Prints the month name (e.g., January) and the year (e.g., 2008)
\newcommand{\monthyear}{%
  \ifcase\month\or January\or February\or March\or April\or May\or June\or
  July\or August\or September\or October\or November\or
  December\fi\space\number\year
}


% Prints an epigraph and speaker in sans serif, all-caps type.
\newcommand{\openepigraph}[2]{%
  %\sffamily\fontsize{14}{16}\selectfont
  \begin{fullwidth}
  \sffamily\large
  \begin{doublespace}
  \noindent\allcaps{#1}\\% epigraph
  \noindent\allcaps{#2}% author
  \end{doublespace}
  \end{fullwidth}
}

% Inserts a blank page
\newcommand{\blankpage}{\newpage\hbox{}\thispagestyle{empty}\newpage}

\usepackage{units}

% Typesets the font size, leading, and measure in the form of 10/12x26 pc.
\newcommand{\measure}[3]{#1/#2$\times$\unit[#3]{pc}}

% Macros for typesetting the documentation
\newcommand{\hlred}[1]{\textcolor{Maroon}{#1}}% prints in red
\newcommand{\hangleft}[1]{\makebox[0pt][r]{#1}}
\newcommand{\hairsp}{\hspace{1pt}}% hair space
\newcommand{\hquad}{\hskip0.5em\relax}% half quad space
\newcommand{\TODO}{\textcolor{red}{\bf TODO!}\xspace}
\newcommand{\ie}{\textit{i.\hairsp{}e.}\xspace}
\newcommand{\eg}{\textit{e.\hairsp{}g.}\xspace}
\newcommand{\na}{\quad--}% used in tables for N/A cells
\providecommand{\XeLaTeX}{X\lower.5ex\hbox{\kern-0.15em\reflectbox{E}}\kern-0.1em\LaTeX}
\newcommand{\tXeLaTeX}{\XeLaTeX\index{XeLaTeX@\protect\XeLaTeX}}
% \index{\texttt{\textbackslash xyz}@\hangleft{\texttt{\textbackslash}}\texttt{xyz}}
\newcommand{\tuftebs}{\symbol{'134}}% a backslash in tt type in OT1/T1
\newcommand{\doccmdnoindex}[2][]{\texttt{\tuftebs#2}}% command name -- adds backslash automatically (and doesn't add cmd to the index)
\newcommand{\doccmddef}[2][]{%
  \hlred{\texttt{\tuftebs#2}}\label{cmd:#2}%
  \ifthenelse{\isempty{#1}}%
    {% add the command to the index
      \index{#2 command@\protect\hangleft{\texttt{\tuftebs}}\texttt{#2}}% command name
    }%
    {% add the command and package to the index
      \index{#2 command@\protect\hangleft{\texttt{\tuftebs}}\texttt{#2} (\texttt{#1} package)}% command name
      \index{#1 package@\texttt{#1} package}\index{packages!#1@\texttt{#1}}% package name
    }%
}% command name -- adds backslash automatically
\newcommand{\doccmd}[2][]{%
  \texttt{\tuftebs#2}%
  \ifthenelse{\isempty{#1}}%
    {% add the command to the index
      \index{#2 command@\protect\hangleft{\texttt{\tuftebs}}\texttt{#2}}% command name
    }%
    {% add the command and package to the index
      \index{#2 command@\protect\hangleft{\texttt{\tuftebs}}\texttt{#2} (\texttt{#1} package)}% command name
      \index{#1 package@\texttt{#1} package}\index{packages!#1@\texttt{#1}}% package name
    }%
}% command name -- adds backslash automatically
\newcommand{\docopt}[1]{\ensuremath{\langle}\textrm{\textit{#1}}\ensuremath{\rangle}}% optional command argument
\newcommand{\docarg}[1]{\textrm{\textit{#1}}}% (required) command argument
\newenvironment{docspec}{\begin{quotation}\ttfamily\parskip0pt\parindent0pt\ignorespaces}{\end{quotation}}% command specification environment
\newcommand{\docenv}[1]{\texttt{#1}\index{#1 environment@\texttt{#1} environment}\index{environments!#1@\texttt{#1}}}% environment name
\newcommand{\docenvdef}[1]{\hlred{\texttt{#1}}\label{env:#1}\index{#1 environment@\texttt{#1} environment}\index{environments!#1@\texttt{#1}}}% environment name
\newcommand{\docpkg}[1]{\texttt{#1}\index{#1 package@\texttt{#1} package}\index{packages!#1@\texttt{#1}}}% package name
\newcommand{\doccls}[1]{\texttt{#1}}% document class name
\newcommand{\docclsopt}[1]{\texttt{#1}\index{#1 class option@\texttt{#1} class option}\index{class options!#1@\texttt{#1}}}% document class option name
\newcommand{\docclsoptdef}[1]{\hlred{\texttt{#1}}\label{clsopt:#1}\index{#1 class option@\texttt{#1} class option}\index{class options!#1@\texttt{#1}}}% document class option name defined
\newcommand{\docmsg}[2]{\bigskip\begin{fullwidth}\noindent\ttfamily#1\end{fullwidth}\medskip\par\noindent#2}
\newcommand{\docfilehook}[2]{\texttt{#1}\index{file hooks!#2}\index{#1@\texttt{#1}}}
\newcommand{\doccounter}[1]{\texttt{#1}\index{#1 counter@\texttt{#1} counter}}

% Generates the index
\usepackage{makeidx}
\makeindex
%DIF PREAMBLE EXTENSION ADDED BY LATEXDIFF
%DIF UNDERLINE PREAMBLE %DIF PREAMBLE
\RequirePackage[normalem]{ulem} %DIF PREAMBLE
\RequirePackage{color}\definecolor{RED}{rgb}{1,0,0}\definecolor{BLUE}{rgb}{0,0,1} %DIF PREAMBLE
\providecommand{\DIFadd}[1]{{\protect\color{blue}\uwave{#1}}} %DIF PREAMBLE
\providecommand{\DIFdel}[1]{{\protect\color{red}\sout{#1}}}                      %DIF PREAMBLE
%DIF SAFE PREAMBLE %DIF PREAMBLE
\providecommand{\DIFaddbegin}{} %DIF PREAMBLE
\providecommand{\DIFaddend}{} %DIF PREAMBLE
\providecommand{\DIFdelbegin}{} %DIF PREAMBLE
\providecommand{\DIFdelend}{} %DIF PREAMBLE
\providecommand{\DIFmodbegin}{} %DIF PREAMBLE
\providecommand{\DIFmodend}{} %DIF PREAMBLE
%DIF FLOATSAFE PREAMBLE %DIF PREAMBLE
\providecommand{\DIFaddFL}[1]{\DIFadd{#1}} %DIF PREAMBLE
\providecommand{\DIFdelFL}[1]{\DIFdel{#1}} %DIF PREAMBLE
\providecommand{\DIFaddbeginFL}{} %DIF PREAMBLE
\providecommand{\DIFaddendFL}{} %DIF PREAMBLE
\providecommand{\DIFdelbeginFL}{} %DIF PREAMBLE
\providecommand{\DIFdelendFL}{} %DIF PREAMBLE
\newcommand{\DIFscaledelfig}{0.5}
%DIF HIGHLIGHTGRAPHICS PREAMBLE %DIF PREAMBLE
\RequirePackage{settobox} %DIF PREAMBLE
\RequirePackage{letltxmacro} %DIF PREAMBLE
\newsavebox{\DIFdelgraphicsbox} %DIF PREAMBLE
\newlength{\DIFdelgraphicswidth} %DIF PREAMBLE
\newlength{\DIFdelgraphicsheight} %DIF PREAMBLE
% store original definition of \includegraphics %DIF PREAMBLE
\LetLtxMacro{\DIFOincludegraphics}{\includegraphics} %DIF PREAMBLE
\newcommand{\DIFaddincludegraphics}[2][]{{\color{blue}\fbox{\DIFOincludegraphics[#1]{#2}}}} %DIF PREAMBLE
\newcommand{\DIFdelincludegraphics}[2][]{% %DIF PREAMBLE
\sbox{\DIFdelgraphicsbox}{\DIFOincludegraphics[#1]{#2}}% %DIF PREAMBLE
\settoboxwidth{\DIFdelgraphicswidth}{\DIFdelgraphicsbox} %DIF PREAMBLE
\settoboxtotalheight{\DIFdelgraphicsheight}{\DIFdelgraphicsbox} %DIF PREAMBLE
\scalebox{\DIFscaledelfig}{% %DIF PREAMBLE
\parbox[b]{\DIFdelgraphicswidth}{\usebox{\DIFdelgraphicsbox}\\[-\baselineskip] \rule{\DIFdelgraphicswidth}{0em}}\llap{\resizebox{\DIFdelgraphicswidth}{\DIFdelgraphicsheight}{% %DIF PREAMBLE
\setlength{\unitlength}{\DIFdelgraphicswidth}% %DIF PREAMBLE
\begin{picture}(1,1)% %DIF PREAMBLE
\thicklines\linethickness{2pt} %DIF PREAMBLE
{\color[rgb]{1,0,0}\put(0,0){\framebox(1,1){}}}% %DIF PREAMBLE
{\color[rgb]{1,0,0}\put(0,0){\line( 1,1){1}}}% %DIF PREAMBLE
{\color[rgb]{1,0,0}\put(0,1){\line(1,-1){1}}}% %DIF PREAMBLE
\end{picture}% %DIF PREAMBLE
}\hspace*{3pt}}} %DIF PREAMBLE
} %DIF PREAMBLE
\LetLtxMacro{\DIFOaddbegin}{\DIFaddbegin} %DIF PREAMBLE
\LetLtxMacro{\DIFOaddend}{\DIFaddend} %DIF PREAMBLE
\LetLtxMacro{\DIFOdelbegin}{\DIFdelbegin} %DIF PREAMBLE
\LetLtxMacro{\DIFOdelend}{\DIFdelend} %DIF PREAMBLE
\DeclareRobustCommand{\DIFaddbegin}{\DIFOaddbegin \let\includegraphics\DIFaddincludegraphics} %DIF PREAMBLE
\DeclareRobustCommand{\DIFaddend}{\DIFOaddend \let\includegraphics\DIFOincludegraphics} %DIF PREAMBLE
\DeclareRobustCommand{\DIFdelbegin}{\DIFOdelbegin \let\includegraphics\DIFdelincludegraphics} %DIF PREAMBLE
\DeclareRobustCommand{\DIFdelend}{\DIFOaddend \let\includegraphics\DIFOincludegraphics} %DIF PREAMBLE
\LetLtxMacro{\DIFOaddbeginFL}{\DIFaddbeginFL} %DIF PREAMBLE
\LetLtxMacro{\DIFOaddendFL}{\DIFaddendFL} %DIF PREAMBLE
\LetLtxMacro{\DIFOdelbeginFL}{\DIFdelbeginFL} %DIF PREAMBLE
\LetLtxMacro{\DIFOdelendFL}{\DIFdelendFL} %DIF PREAMBLE
\DeclareRobustCommand{\DIFaddbeginFL}{\DIFOaddbeginFL \let\includegraphics\DIFaddincludegraphics} %DIF PREAMBLE
\DeclareRobustCommand{\DIFaddendFL}{\DIFOaddendFL \let\includegraphics\DIFOincludegraphics} %DIF PREAMBLE
\DeclareRobustCommand{\DIFdelbeginFL}{\DIFOdelbeginFL \let\includegraphics\DIFdelincludegraphics} %DIF PREAMBLE
\DeclareRobustCommand{\DIFdelendFL}{\DIFOaddendFL \let\includegraphics\DIFOincludegraphics} %DIF PREAMBLE
%DIF LISTINGS PREAMBLE %DIF PREAMBLE
\RequirePackage{listings} %DIF PREAMBLE
\RequirePackage{color} %DIF PREAMBLE
\lstdefinelanguage{DIFcode}{ %DIF PREAMBLE
%DIF DIFCODE_UNDERLINE %DIF PREAMBLE
  moredelim=[il][\color{red}\sout]{\%DIF\ <\ }, %DIF PREAMBLE
  moredelim=[il][\color{blue}\uwave]{\%DIF\ >\ } %DIF PREAMBLE
} %DIF PREAMBLE
\lstdefinestyle{DIFverbatimstyle}{ %DIF PREAMBLE
	language=DIFcode, %DIF PREAMBLE
	basicstyle=\ttfamily, %DIF PREAMBLE
	columns=fullflexible, %DIF PREAMBLE
	keepspaces=true %DIF PREAMBLE
} %DIF PREAMBLE
\lstnewenvironment{DIFverbatim}{\lstset{style=DIFverbatimstyle}}{} %DIF PREAMBLE
\lstnewenvironment{DIFverbatim*}{\lstset{style=DIFverbatimstyle,showspaces=true}}{} %DIF PREAMBLE
%DIF END PREAMBLE EXTENSION ADDED BY LATEXDIFF

\begin{document}
\iffalse
% Front matter
\frontmatter

% r.1 blank page
\blankpage


% v.2 epigraphs
\newpage\thispagestyle{empty}
\openepigraph{%
The public is more familiar with bad design than good design.
It is, in effect, conditioned to prefer bad design, 
because that is what it lives with. 
The new becomes threatening, the old reassuring.
}{Paul Rand%, {\itshape Design, Form, and Chaos}
}
\vfill
\openepigraph{%
A designer knows that he has achieved perfection 
not when there is nothing left to add, 
but when there is nothing left to take away.
}{Antoine de Saint-Exup\'{e}ry}
\vfill
\openepigraph{%
\ldots the designer of a new system must not only be the implementor and the first 
large-scale user; the designer should also write the first user manual\ldots 
If I had not participated fully in all these activities, 
literally hundreds of improvements would never have been made, 
because I would never have thought of them or perceived 
why they were important.
}{Donald E. Knuth}
\fi

% r.3 full title page
\maketitle


% v.4 copyright page
%\newpage
\begin{fullwidth}
~\vfill
\thispagestyle{empty}
\setlength{\parindent}{0pt}
\setlength{\parskip}{\baselineskip}
Copyright \copyright\ \the\year\ \thanklessauthor

%\par\smallcaps{Published by \thanklesspublisher}

\par\smallcaps{This document is continually being updated. Please send us your feedback.}


\par \doclicenseThis
 \index{license}

\par\textit{This draft was compiled on \today.}
\end{fullwidth}

% r.5 contents
\tableofcontents

%\listoffigures

%\listoftables

% r.7 dedication
\iffalse
\cleardoublepage
~\vfill

\begin{doublespace}
\noindent\fontsize{18}{22}\selectfont\itshape
\nohyphenation
Dedicated to those who appreciate \LaTeX{} 
and the work of \mbox{Edward R.~Tufte} 
and \mbox{Donald E.~Knuth}.
\end{doublespace}
\vfill
\vfill

% r.9 introduction
\cleardoublepage
\fi
\chapter*{Preface}
Cryptography enables many paradoxical objects, such as public key encryption, verifiable electronic signatures, zero-knowledge protocols, and fully homomorphic encryption.  The two main steps in developing such seemingly impossible primitives are (i) defining the desired security properties formally and (ii) obtaining a construction satisfying the security property provably. In modern cryptography, the second step typically assumes (unproven) computational assumptions, which are conjectured to be computationally intractable. In this course, we will define several cryptographic primitives and argue their security based on well-studied computational hardness assumptions. However, we will largely ignore the mathematics underlying the assumed computational intractability assumptions.

\section*{Acknowledgements}
These lecture notes are based on scribe notes taken by students in CS 276 over the years. Also, thanks to Peihan Miao, Akshayaram Srinivasan, and Bhaskar Roberts for helping to improve these notes.
%%
% Start the main matter (normal chapters)
\newcommand{\sanjam}[1]{{\color{red} Sanjam: #1}}

\newcommand{\bhaskar}[1]{{\color{ForestGreen} Bhaskar: #1}}

\mainmatter
%\chapter{Mathematical Background}
\label{sec:mb}

In modern cryptography, we typically assume that our attackers cannot run in unreasonably large amounts of time, and we allow security to be broken with a \emph{very small}, but non-zero, probability.

Without these assumptions, we must work in the realm of information-theoretic cryptography, which is often unachievable or impractical for many applications. For example, the one-time pad --- an information-theoretically secure cipher --- is not very useful because it requires very large keys.

In this chapter, we define items (1) and (2) more formally. We require our adversaries to run in polynomial time, which captures the idea that their runtime is not unreasonably large (sections~\ref{ssec:ppt}). We also allow security to be broken with negligible -- very small -- probability (section ~\ref{ssec:nnf}). 


\section{Probabilistic Polynomial Time}
\label{ssec:ppt}
A probabilistic Turing Machine is a generic computer that is allowed to make random choices during its execution. A probabilistic \textit{polynomial time} Turing Machine is one which halts in time polynomial in its input length. More formally:

\begin{definition}[Probabilistic Polynomial Time]
A probabilistic Turing Machine $M$ is said to be PPT (a Probabilistic Polynomial Time Turing Machine) if $\exists c \in \mathbb{Z}^+$ such that $\forall x \in\{0,1\}^*$, $M(x)$ halts in $|x|^c$ steps.
\end{definition}

A {\em non-uniform} PPT Turing Machine is a collection of machines one for each input length, as opposed to a single machine that must work for all input lengths.

\begin{definition}[Non-uniform PPT]
A non-uniform PPT machine is a sequence of Turing Machines $\{ M_1, M_2, \cdots \}$ such that $\exists c \in \mathbb{Z}^+$ such that $\forall x \in\{0,1\}^*$, $M_{|x|}(x)$ halts in $|x|^c$ steps.
\end{definition}



\section{Noticeable and Negligible Functions}
\label{ssec:nnf}
Noticeable and negligible functions are used to characterize the ``largeness'' or ``smallness'' of a function describing the probability of some event.  Intuitively, a noticeable function is required to be larger than some inverse-polynomially function in the input parameter. On the other hand, a negligible function must be smaller than any inverse-polynomial function of the input parameter. More formally:


\begin{definition}[Noticeable Function]
A function $\mu(\cdot): \mathbb{Z}^+ \rightarrow [0,1]$ is noticeable iff $\exists c \in \mathbb{Z}^+, n_0 \in \mathbb{Z}^+$ such that $\forall n \geq n_0 , \; \mu(n) \geq n^{-c}$.
\end{definition}

\paragraph{Example.} Observe that $\mu(n) = n^{-3}$ is a noticeable function.  (Notice that the above definition is satisfied for $c = 3$ and $n_0 = 1$.)

\begin{definition}[Negligible Function]
A function $\mu(\cdot): \mathbb{Z}^+ \rightarrow [0,1]$ is negligible iff $\forall c \in \mathbb{Z}^+ \; \exists n_0 \in \mathbb{Z}^+$ such that $\forall n \geq n_0 , \; \mu(n) < n^{-c}$.
\end{definition}

\paragraph{Example.} $\mu(n) = 2^{-n}$ is an example of a negligible function. This can be observed as follows.
Consider an arbitrary $c \in \mathbb{Z}^+$ and set $n_0 = c^2$. Now, observe that for all $n \geq n_0$, we have that $\frac{n}{\log_2 n} \geq \frac{n_0}{\log_2 n_0} \geq \frac{n_0}{\sqrt{n_0}} = \sqrt{n_0} = c$. This allows us to conclude that $$\mu(n) = 2^{-n} = n^{-\frac{n}{\log_2 n}} \leq n^{-c}.$$

Thus, we have proved that for any $c \in \mathbb{Z}^+$, there exists $n_0 \in \mathbb{Z}^+$ such that for any $n \geq n_0$, $\mu(n) \leq n^{-c}$.

\paragraph{Gap between Noticeable and Negligible Functions.}
At first thought it might seem that a function that is {not} negligible (or, a non-negligible function) must be a noticeable. This is not true!\cite{JC:Bellare02} Negating the definition of a negligible function, we obtain that a non-negligible function $\mu(\cdot)$ is such that $\exists c \in \mathbb{Z}^+$ such that $\forall n_0 \in \mathbb{Z}^+$, $\exists n \geq n_0$ such that $\mu(n) \geq n^{-c}$.
Note that this requirement is satisfied as long as $\mu(n) \geq n^{-c}$ for infinitely many choices of $n \in \mathbb{Z}^+$. However, a noticeable function requires this condition to be true for every $n \geq n_0$.

Below we give example of a function $\mu(\cdot)$ that is neither negligible nor noticeable.
$$\mu(n) = \Big\{
\begin{array}{ll}
  2^{-n} & : x \mod 2 = 0\\
  n^{-3} & : x \mod 2 \neq 0
\end{array}
$$
This function is obtained by interleaving negligible and  noticeable functions. It cannot be negligible (resp., noticeable) because it is greater (resp., less) than an inverse-polynomially function for infinitely many input choices.

\paragraph{Properties of Negligible Functions.} Sum and product of two negligible functions is still a negligible function. We argue this for the sum function below and defer the problem for products to Exercise~\ref{ex:product}.

\begin{exercise}
If $\mu(n)$ and $\nu(n)$ are negligible functions from domain $\mathbb{Z}^+$ to range $[0,1]$ then prove that the following functions are also negligible:
\begin{enumerate}
    \item $\psi_1(n) = \frac{1}{2} \cdot \left(\mu(n) + \nu(n)\right)$
    \item $\psi_2(n) = \mu(n)\cdot \nu(n)$
    \item $\psi_3(n) = \mathsf{poly}(\mu(n))$, where $\mathsf{poly}(\cdot)$ is an unspecified polynomial function.
\end{enumerate}function.
\end{exercise}
\proof 
$ $
\begin{enumerate}
    \item We need to show that for any $c \in \mathbb{Z}^+$, we can find $n_0$ such that $\forall n \geq n_0$, $\psi_1(n) \leq n^{-c}$. Our argument proceeds as follows. Given the fact that $\mu$ and $\nu$ are negligible we can conclude that there exist $n_1$ and $n_2$ such that $\forall n \geq n_1$, $\mu(n) \leq n^{-c}$ and $\forall n \geq n_2$, $g(n) \leq n^{-c}$. Combining the above two facts and setting $n_0 = \max(n_1, n_2)$ we have that for every $n \geq n_0$,
    \begin{align*}
        \psi_1(n) &= \frac{1}{2} \cdot (\mu(n) + \nu(n)) \leq \frac{1}{2} \cdot (n^{-c} + n^{-c}) = n^{-c}
    \end{align*}
    Thus, $\psi_1(n) \leq n^{-c}$ and hence is negligible.
\end{enumerate}
\qed

%\begin{corollary}
%If $f(n)$ is non-negligible and $g(n)$ is negligible, then $h(n) = f(n) - g(n)$ is non-negligible.
%\end{corollary}
%
%\proof If $h(n)$ was negligible, then $f(n) = g(n) + h(n)$ would be the sum of two negligible functions, but would be non-negligible, which is a contradiction.  \qed


%\newcommand{\binset}[1]{\{0,1\}^{#1}}
\newcommand{\binfunc}[2]{\binset{#1}\rightarrow\binset{#2}}
\chapter{One-Way Functions}
\label{sec:owf}

\label{ssec:owf}
Cryptographers often attempt to base cryptographic results on conjectured computational assumptions to leverage reduced adversarial capabilities. Furthermore, the security of these constructions is no better than the assumptions they are based on. 

\begin{quote}
\emph{Cryptographers seldom sleep well.}\footnote{Quote by Silvio Micali in personal communication with Joe Kilian.}
\end{quote}

Thus, basing cryptographic tasks on the \emph{minimal} necessary assumptions is a key tenant in cryptography. Towards this goal, rather can making assumptions about specific computational problem in number theory, cryptographers often consider \emph{abstract primitives}. The existence of these abstract primitives can then be based on one or more computational problems in number theory.

The weakest abstract primitive cryptographers consider is one-way functions. Virtually, every cryptographic goal of interest is known to imply the existence of one-way functions. In other words, most cryptographic tasks would be impossible if the existence of one-way functions was ruled out. On the flip side, the realizing cryptographic taks from just one-way functions would be ideal. 

\section{Definition}
A one-way function $f: \{0,1\}^n \rightarrow \{0,1\}^m$ is a function that is easy to compute but hard to invert. This intuitive notion is trickier to formalize than it might appear on first thought.

\begin{definition}[One-Way Functions]
A function $f : \binset{*} \rightarrow \binset{*}$ is said to be one-way function if:
\begin{itemize}
\item[-] \textbf{Easy to Compute:} $\exists$ a (deterministic) polynomial time machine $M$ such that $\forall x \in \binset{*}$ we have that \[M(x) = f(x)\]
\item[-] \textbf{Hard to Invert:} $\forall$ non-uniform PPT adversary $\mathcal{A}$ we have that
    \begin{equation}\label{eq:owf}
    \mu_{\mathcal{A},f}(n) = \Pr_{x \stackrel{\$}{\leftarrow} \binset{n}}[ \mathcal{A}(1^n, f(x)) \in f^{-1}(f(x))]
     \end{equation}
     is a negligible function,  $x \overset{\$}{\leftarrow} \binset{n}$ denotes that $x$ is drawn uniformly at random from the set $\binset{n}$, $f^{-1}(f(x)) = \{x' \mid f(x) = f(x')\}$, and the probability is over the random choices of $x$ and the random coins of $\mathcal{A}$.
\end{itemize}
\end{definition}

The above definition is rather delicate. We next describe problems in the slight variants of this definition that are insecure.

\begin{enumerate}
\item What if we require that
    $\Pr_{x \stackrel{\$}{\leftarrow} \binset{n}}[ \mathcal{A}(1^n, f(x)) \in f^{-1}(f(x))] = 0$ instead of being negligible?

This condition is false for every function $f$. An adversary $\mathcal{A}$ that outputs an arbitrarily fixed value $x_0$ succeeds with probability at least $1/2^{n}$, as $x_0 = x$ with at least the same probability.

\item  What if we drop the input $1^n$ to $\mathcal{A}$ in Equation~\ref{eq:owf}?

Consider the function $f(x) = |x|$.  In this case, we have that $m = \log_2 n$, or $n = 2^m$.  Intuitively, $f$ should not be considered a one-way function, because it is easy to invert $f$. Namely, given a value $y$ any $x$ such that $|x| = y$ is such that $x \in f^{-1}(y)$.  However, according to this definition the adversary gets an $m$ bit string as input, and hence is restricted to running in time polynomial in $m$. Since each possible $x$ is of size $n = 2^m$, the adversary doesn't even have enough time to write down the answer!  Thus, according to the flawed definition above, $f$ would be a one-way function.

Providing the attacker with $1^n$ ($n$ repetitions of the $1$ bit) as additional input avoids this issue.  In particular, it allows the attacker to run in time polynomial in $m$ and $n$.
\end{enumerate}

\paragraph{A Candidate One-way Function.}
It is not known whether one-way functions exist.  The existence of one-way functions would imply that $P \neq NP$ (see Exercise~\ref{ex:PNP}), and so of course we do not know of any concrete functions that have been proved to be one-way.

However, there are candidates of functions that could be one-way functions.  One example is based on the hardness of factoring.  Multiplication can be done easily in $O(n^2)$ time, but so far no polynomial time algorithm is known for factoring.

One candidate might be to say that given an input $x$, split $x$ into its left and right halves $x_1$ and $x_2$, and then output $x_1 \times x_2$.  However, this is not a one-way function, because with probability $\frac{3}{4}$, $2$ will be a factor of $x_1 \times x_2$, and in general the factors are small often enough that a non-negligible number of the outputs could be factored in polynomial time.

To improve this, we again split $x$ into $x_1$ and $x_2$, and use $x_1$ and $x_2$ as seeds in order to generate large primes $p$ and $q$, and then output $pq$.  Since $p$ and $q$ are primes, it is hard to factor $pq$, and so it is hard to retrieve $x_1$ and $x_2$.  This function is believed to be one-way.


%\section{Modifying One-way Functions}
%
%\subsection{Fixing Certain Values of a One-way Function}
%
%Consider having a one-way function $f$.  Can we use this function $f$ in order to make a one-way function $g$ such that $g(x_0) = y_0$ for some constants $x_0, y_0$, or would this make the function no longer be one-way?\\
%
%Intuitively, the answer is yes - we can specially set $g(x_0) = y_0$, and otherwise have $g(x) = f(x)$.  In this case, the adversary gains the knowledge of how to invert $y_0$, but that will only happen with negligible probability, and so the function is still one-way.
%
%\begin{theorem}
%Given a one-way function $f : \binset{n} \rightarrow \binset{m}$ and constants $x_0 \in \binset{n}$, $y_0 \in \binset{m}$, $\exists g : \binset{n} \rightarrow \binset{m}$ such that $g(x_0) = y_0$ where $g$ is a one-way function.
%\end{theorem}
%
%The proof is given in the appendix.\\
%
%However, this raises an apparent contradiction - according to this theorem, given a one-way function $f$, we could keep fixing each of its values to $0$, and it would continue to be a one-way function.  If we kept doing this, we would eventually end up with a function which outputs 0 for {\em all} of the possible values of $x$.  How could this still be one-way?\\
%
%The resolution of this apparent paradox is by noticing that a one-way function is only required to be one-way in the limit where $n$ grows very large.  So, no matter how many times we fix the values of $f$ to be 0, we are still only setting a finite number of $x$ values to 0.  However, this will still satisfy the definition of a one-way function - it is just that we will have to use larger and larger values of $n_0$ in order to prove that the probability of breaking the one-way function is negligible.

\section{Composability of One-Way Functions}
Given a one-way function $f : \binfunc{n}{n}$, is the function $f^2(x) = f(f(x))$ also a one-way function?  Intuitively, it seems that if it is hard to invert $f(x)$, then it would be just as hard to invert $f(f(x))$. \sanjam{Explain the intuitive reduction that doesn't work.}  However, this intuition is incorrect and highlights the delicacy when working with cryptographic assumptions and primitives. In particular, assuming one-way functions exists we describe a one-way function $f: \{0,1\}^{n/2}\times \{0,1\}^{n/2} \rightarrow \{0,1\}^{n}$ such that $f^2$ can be efficiently inverted.
Let $g: \{0,1\}^n \rightarrow \{0,1\}^n$ be a one-way function then we set $f$ as follows:
$$f(x_1,x_2) = \left\{
\begin{array}{ll}
  0^{n} & : \text{if } x_1 = 0^{n/2} \\
  0^{n/2}\|g(x_2) & : \text{otherwise}
\end{array}
\right.$$
Two observations follow:
\begin{enumerate}
  \item $f^2$ is not one-way. This follows from the fact that for all inputs $x_1, x_2$ we have that $f^2(x_1,x_2) = 0^n$. This function is clearly not one-way!
  \item $f$ is one-way. This can be argued as follows. Assume that there exists an adversary $\mathcal{A}$ such that $\mu_{\mathcal{A},f}(n) = \Pr_{x \stackrel{\$}{\leftarrow} \binset{n}}[ \mathcal{A}(1^n, f(x)) \in f^{-1}(f(x))]$ is non-negligible. Using such an $\mathcal{A}$ we will describe a construction of adversary $\mathcal{B}$ such that $\mu_{\mathcal{B},g}(n) = \Pr_{x \stackrel{\$}{\leftarrow} \binset{n}}[ \mathcal{B}(1^n, g(x)) \in g^{-1}(g(x))]$ is also non-negligible. This would be a contradiction thus proving our claim.

      \textbf{Description of $\mathcal{B}$}: $\mathcal{B}$ on input $y \in\{0,1\}^n$ outputs the $n$ lower-order bits of  $\mathcal{A}(1^{2n}, 0^{n}\|y)$.

      Observe that if $\mathcal{A}$ successfully inverts $f$ then we have that $\mathcal{B}$ successfully inverts $g$. More formally, we have that:
      $$\mu_{\mathcal{B},g}(n) = \Pr_{x \stackrel{\$}{\leftarrow} \binset{n}}\left[ \mathcal{A}(1^{2n}, 0^n || g(x)) \in \{0,1\}^n || g^{-1}(g(x))\right].$$
      Note that
      \begin{align*}
      \mu_{\mathcal{A},f}(2n) =& \Pr_{x_1, x_2 \stackrel{\$}{\leftarrow} \binset{2n}}[ \mathcal{A}(1^{2n}, f(x_1, x_2)) \in f^{-1}(f(\tilde x))]\\
      \leq & \Pr_{x_1 \stackrel{\$}{\leftarrow} \binset{n}}[x_1 = 0^n] +  \Pr_{x_1 \stackrel{\$}{\leftarrow} \binset{n}}[x_1 \neq 0^n] \Pr_{x_2 \stackrel{\$}{\leftarrow} \binset{n}} [ \mathcal{A}(1^{2n}, 0^n || g(x_2)) \in \{0,1\}^n || g^{-1}(g(x_2))]\\
      = & \frac{1}{2^n} + \left( 1-\frac{1}{2^n}\right)\cdot\Pr_{x_2 \stackrel{\$}{\leftarrow} \binset{n}} [ \mathcal{A}(1^{2n}, 0^n || g(x_2)) \in \{0,1\}^n || g^{-1}(g(x_2))]\\
      = & \frac{1}{2^n} + \left( 1-\frac{1}{2^n}\right)\cdot\mu_{\mathcal{B},g}(n).
      \end{align*}
      Rewriting the above expression, we have that $\mu_{\mathcal{B},g}(n) = \frac{\mu_{\mathcal{A},f}(2n) - \frac{1}{2^n}}{1- \frac{1}{2^n}}$ which is non-negligible as long as $\mu_{\mathcal{A},f}(2n)$  is non-negligible.
\end{enumerate}


%\begin{lemma}
%If $f : \binfunc{n}{n}$ is a one-way function, then $g : \binfunc{2n}{2n}$ defined as $g(x) = 0^n \concat f(x_{[1:n]})$ is also one-way.\\
%\end{lemma}
%\proof
%Assume towards contradiction that $g$ is not one-way, and so there is an adversary $A_g$ that inverts $g$ with probability $\mu(2n)$ that is non-negligible.\\
%
%Note that $\mu(2n)$ is also non-negligible with respect to inputs of size $n$.\\

%Then we can define an adversary $A_f$ such that $A_f(y) = (A_g(0^n \concat y))_{[1:n]}$.  Note that $A_g$ breaks $g$ on input $0^n \concat y$ $\implies$ $A_f$ breaks $f$ on input $y$, and so $A_f$ breaks $f$ with at least non-negligible probability $\mu(2n)$.  Contradiction.\\
%
%Thus, $g$ is also one-way.  \qed\\
%
%Now, given a function $f : \binfunc{n}{n}$, we can construct a new one-way function $g : \binfunc{2n}{2n}$.  From $g$, we can construct another one-way function $h : \binfunc{2n}{2n}$ defined by:
%
%$h(x) = \left\{
%\begin{array}{lr}
%  0^{2n} & : x_{[1:n]} = 0^n \\
%  g(x) & : otherwise
%\end{array}
%\right.$

%A generalization of the previous theorem (fixing values in a one-way function) shows that $h$ is also a one-way function.  (In short, we are only fixing the values of $\frac{2^n}{2^{2n}} = \frac{1}{2^n}$ of all of the possible values of $x$.  Since we are only fixing a negligible fraction of the possible values of $x$, the same proof with slight modifications still applies.)\\
%
%So, $h$ is a one-way function.  However, $h^2(x) = h(h(x)) = 0^{2n}$, and so $h^2$ is clearly not a one-way function.  Thus, composing one-way functions is not guaranteed to give another one-way function. \qed



%\newcommand{\bin}{\{0,1\}}
\newcommand{\adv}{\mathcal{A}}
\newcommand{\advb}{\mathcal{B}}
\newcommand{\advc}{\mathcal{C}}
%\usepackage[utf8]{inputenc}
%\usepackage{amsmath,amssymb,fullpage}


% The goal of this section is to illustrate the general strategy for the problems of the form,
% \begin{center}
% \textit{``If $f$ is one-way function, then show that $f'$ (derived from $f$) is not a one-way function"}
% \end{center}
% Some of the examples include:
% \begin{itemize}
% \item If $f$ is a one-way function, prove that $f'$ defined as $f(f(\cdot))$ is not one-way.
% \item If $f$ is a one-way function, prove that $f'$ defined by dropping the first bit the output of $f$ is not one-way.
% \end{itemize}

% In order to give such a proof, we need to give an example of an one-way function $f$ and show that $f'$ (derived from $f$) is not one-way. The general strategy for these types of problems is the following:
% \begin{enumerate}
% \item Come up with a contrived function $g$ and show that $g$ is one-way. 
% \item Construct the new function $g'$ that is derived from $g$.
% \item Show that $g'$ can be inverted with non-negligible probability and thus show that $g'$ is not one-way.
% \end{enumerate}
% The reason why we need to come-up with a contrived function is that for specific one-way function $f$, $f'$ (derived from $f$) could be one-way. To see why this is the case, consider a one-way function $f: \bin^n \rightarrow \bin^n$ that is additionally injective. Then, one can show that $f^2(\cdot)$ is in fact a one-way function.\footnote{Try to prove this!} On the other hand, in the previous section, we showed that there exists a (contrived) function $g$ such that $g$ is one-way but $g^2$ is not one-way.
% Hence, we might not always be able to start from any one-way function $f$ and show that $f'$ (derived from $f$) is not one-way. The first step where we come up a suitable $g$ requires some ingenuity. Once that is done, the second and the third steps would generally be not so hard.

% To illustrate these three steps, let us consider a concrete example. 
Below is another example of a transformation that does not work.
We construct a function $f$ and show that if $f$ is one-way, then $f'$ that is defined by dropping the first bit of the output of $f$ is not one-way.


% \paragraph{Step-1: Designing the function $g$.} 
To do this, we first want to come up with a (contrived) function $g$ and prove that it is one-way.
Let us assume that there exists a one-way function $h : \bin^n \rightarrow \bin^n$. We define the function $g : \bin^{2n} \rightarrow \bin^{2n}$ as follows:
$$
g(x\|y) = \begin{cases}
 0^{n}\|y &\text{    if } x = 0^n\\
1\|0^{n-1}\|g(y) &\text{    otherwise }
\end{cases}
$$
\begin{claim}
If $h$ is a one-way function, then so is $g$.
\end{claim}
\begin{proof}
Assume for the sake of contradiction that $g$ is not one-way. Then there exists a polynomial time adversary $\adv$ and a non-negligible function $\mu(\cdot)$ such that:
$$
\Pr_{x,y}[\adv(1^n,g(x\|y)) \in g^{-1}(g(x\|y))] = \mu(n)
$$
We will use such an adversary $\adv$ to invert $h$ with some non-negligible probability. This contradicts the one-wayness of $h$ and thus our assumption that $g$ is not one-way function is false.

Let us now construct an $\advb$ that uses $\adv$ and inverts $h$. $\advb$ is given $1^n,h(y)$ for a randomly chosen $y$ and its goal is to output $y' \in h^{-1}(h(y))$ with some non-negligible probability. $\advb$ works as follows:
\begin{enumerate}
\item It samples $x \gets \bin^n$ randomly.
\item If $x = 0^n$, it samples a random $y' \gets \bin^n$ and outputs it.
\item Otherwise, it runs $\adv(10^{n-1}\|h(y))$ and obtains $x' \| y'$. It outputs $y'$.
\end{enumerate}

Let us first analyze the running time of $\advb$. The first two steps are clearly polynomial (in $n$) time. In the third step, $\advb$ runs $\adv$ and uses its output. Note that the running time of since $\adv$ runs in polynomial (in $n$) time, this step also takes polynomial (in $n$) time. Thus, the overall running time of $\advb$ is polynomial (in $n$).

Let us now calculate the probability that $\advb$ outputs the correct inverse. If $x = 0^n$, the probability that $y'$ is the correct inverse is at least $\frac{1}{2^n}$ (because it guesses $y'$ randomly and probability that a random $y'$ is the correct inverse is $\geq 1/2^n$). On the other hand, if $x \neq 0^n$, then the probability that $\advb$ outputs the correct inverse is $\mu(n)$. Thus,
\begin{eqnarray*}
\Pr[\advb(1^n,h(y)) \in h^{-1}(h(y))] & \geq & \Pr[x = 0^n](\frac{1}{2^n}) + \Pr[x \neq 0^n]\mu(n)\\
& = & \frac{1}{2^{2n}} + (1 - \frac{1}{2^n}) \mu(n) \\
& \geq & \mu(n) - (\frac{1}{2^{n}} - \frac{1}{2^{2n}})
\end{eqnarray*}

Since $\mu(n)$ is a non-negligible function and $(\frac{1}{2^{n}} - \frac{1}{2^{2n}})$ is a negligible function, their difference is non-negligible.\footnote{Exercise: Prove that if $\alpha(\cdot)$ is a non-negligible function and $\beta(\cdot)$ is a negligible function, then $(\alpha - \beta)(\cdot)$ is a non-negligible function.} This contradicts the one-wayness of $h$.

\end{proof}

% \paragraph{Step-2: Constructing $g'$.} 
We construct the new function $g': \bin^{2n} \rightarrow \bin^{2n-1}$ by dropping the first bit of $g$. That is,
$$
g'(x\|y) = \begin{cases}
 0^{n-1}\|y &\text{    if } x = 0^n\\
0^{n-1}\|g(y) &\text{    otherwise }
\end{cases}
$$

% \paragraph{Step-3: Inverting $g'$.} 
We now want to prove that $g'$ is not one-way. That is, we want to design an adversary $\advc$ such that given $1^{2n}$ and $g'(x \| y)$ for a randomly chosen $x,y$, it outputs an element in the set $g^{-1}(g(x \| y)$. The description of $\advc$ is as follows:

\begin{itemize}
\item On input $1^{2n}$ and $g'(x \| y)$, the adversary $\advc$ parses $g'(x \| y)$ as $0^{n-1} \| \overline{y}$.
\item It outputs $0^{n} \| \overline{y}$ as the inverse.
\end{itemize}
Notice that $g'(0^{n} \| \overline{y}) = 0^{n-1} \| \overline{y}$. Thus, $\advc$ succeeds with probability $1$ and this breaks the one-wayness of $g'$.



%%%%%%%%%%%%%%%%%%%%%%%%%%%%%%%%%%%%%%%%%%%%%%%%%%%%%
\section{Hardness Amplification}
\label{sec:owf:amplify}
In this section, we show that even a very \emph{weak} form of one-way functions suffices from constructing one-way functions as defined previously. For this section, we refer to this previously defined notion as strong one-way functions.
\begin{definition}[Weak One-Way Functions]
A function $f : \binset{n} \rightarrow \binset{m}$ is said to be a weak one-way function if:
\begin{itemize}
\item[-] $f$ is computable by a polynomial time machine, and
\item[-] There exists a noticeable function $\alpha_f(\cdot)$ such that $\forall$ non-uniform PPT adversaries $\mathcal{A}$ we have that
    $$
    \mu_{\mathcal{A},f}(n) =
    \Pr_{x \stackrel{\$}{\leftarrow} \binset{n}}[ \mathcal{A}(1^n, f(x)) \in f^{-1}(f(x))] \leq 1 - \alpha_{f}(n).
    $$
\end{itemize}
\end{definition}

\begin{theorem}\label{theorem:weakstrongOWF}
If there exists a weak one-way function, then there exists a (strong) one-way function.
\end{theorem}

\proof We prove the above theorem constructively. Suppose $f : \binset{n} \rightarrow \binset{m}$ is a weak one-way function, then we prove that the function $g: \binset{nq} \rightarrow \binset{mq}$ for $q = \lceil \frac{2n}{\alpha_{f}(n)} \rceil$ where 
$$g(x_1, x_2, \cdots, x_q) = f(x_1) || f(x_2) || \cdots || f(x_q),$$
 is a strong one-way function.

Assume for the sake of contradiction that there exists an adversary $\mathcal{B}$ such that $\mu_{\mathcal{B},g}(nq) = \Pr_{x \stackrel{\$}{\leftarrow} \binset{nq}}[ \mathcal{B}(1^{nq}, g(x)) \in g^{-1}(g(x))]$ is non-negligible.
%Suppose $\mu_{\mathcal{A},g}(nq) \geq \tilde \mu_{\mathcal{A},g}(nq)$ for arbitrarily large $n$, where $\tilde  \mu_{\mathcal{A},g}$ is a noticeable function.\peihan{to ensure that $T$ is poly}
Then we use $\mathcal{B}$ to construct $\mathcal{A}$ (see Figure~\ref{fig:adv:weak}) that breaks $f$, namely $\mu_{\mathcal{A},f}(n) = \Pr_{x \stackrel{\$}{\leftarrow} \binset{n}}[ \mathcal{A}(1^n, f(x)) \in f^{-1}(f(x))] > 1 - \alpha_f(n)$ for sufficiently large $n$.
\begin{marginfigure}[-5cm]
%\Loop { $T=\frac{4n^2}{\alpha_f(n) \mu_{\mathcal{B}, g}(nq)}$ times}
\begin{enumerate}
    \item $i \stackrel{\$}{\leftarrow} [q]$.
    \item $x_1, \cdots, x_{i-1}, x_i, \cdots, x_q \stackrel{\$}{\leftarrow} \binset{n}$.
    \item Set $y_j = f(x_j)$ for each $j \in [q]\backslash \{i\}$ and $y_i = y$.
    \item $(x'_1, x'_2, \cdots, x'_q) := \mathcal{B} (f(x_1), f(x_2), \cdots, f(x_q))$.
    \item {$f(x'_i) = y$} then output $x'_i$ else $\bot$.
\end{enumerate}
\caption{Construction of $\mathcal{A}(1^n, y)$}
\label{fig:adv:weak}
\end{marginfigure}

Note that: (1) $\mathcal{A}(1^n, y)$ iterates at most $T = \frac{4n^2}{\alpha_f(n)\mu_{\mathcal{B},g}(nq)}$ times each call is polynomial time. (2) $\mu_{\mathcal{B},g}(nq)$ is a non-negligible function. This implies that for infinite choices of $n$ this value is greater than some noticeable function. Together these two facts imply that for infinite choices of $n$ the running time of $\mathcal{A}$ is bounded by a polynomial function in $n$.

It remains to show that $\Pr_{x \stackrel{\$}{\leftarrow} \binset{n}}[ \mathcal{A}(1^n, f(x)) = \bot] < \alpha_f(n)$ for arbitrarily large $n$. A natural way to argue this is by showing that at least one execution of $\mathcal{B}$ should suffice for inverting $f(x)$. However, the technical challenge in proving this formally is that these calls to $\mathcal{B}$ aren't independent. Below we formalize this argument even when these calls aren't independent.\marginnote[-5cm]{\begin{lemma}
Let $A$ be any an efficient algorithm such that $\Pr_{x,r}[A(x,r) =1] \geq \epsilon$. Additionally, let $G = \{x\mid \geq \Pr_{r}[A(x,r) =1] \geq \frac\epsilon2\}$. Then, we have $\Pr_x[x \in G] \geq \frac\epsilon2$.
\end{lemma}
\begin{proof}
The proof of this lemma follows by a very simple counting argument. Let's start by assuming that $\Pr_x[x \in G] < \frac\epsilon2$. Next, observe that
\begin{align*}
\Pr_{x,r}&[A(x,r) =1]& \\&= \Pr_x[x \in G]\cdot\Pr_{x,r}[A(x,r) =1\mid x \in G] \\&+ \Pr_x[x \not\in G]\cdot\Pr_{x,r}[A(x,r) =1\mid x \not\in G]
\\&< \frac\epsilon2\cdot 1 + 1\cdot\frac\epsilon2
\\&< \epsilon,
\end{align*}
which is a contradiction.
\end{proof}
}

Define the set $S$ of ``bad'' $x$'s, which are hard to invert:
$$S := \left\{x \left| \Pr_\mathcal{B}\left[\mathcal{A} \text{ inverts $f(x)$ in a single iteration} \right] \leq \frac{\alpha_f(n) \mu_{\mathcal{B},g}(nq)}{4n} \right. \right\}.$$
We start by proving that the size of $S$ is small. More formally,
$$\Pr_{x \stackrel{\$}{\leftarrow} \binset{n}} [x \in S] \leq \frac{\alpha_f(n)}{2}.$$
Assume, for the sake of contradiction,\marginnote{\begin{lemma}
Let $A$ be any an efficient algorithm such that $\Pr_{x,r}[A(x_1,\ldots x_n,r) =1] \geq \epsilon$. Additionally, let $G = \{x\mid \geq \Pr_{x_1,\ldots x_n,r}[A(x,r) =1\mid \exists i, x = x_i] \geq \frac\epsilon2\}$. Then, we have $\Pr_x[x \in G] \geq \frac\epsilon2$.
\end{lemma}
\begin{proof}
The proof of this lemma follows by a very simple counting argument. Let's start by assuming that $\Pr_x[x \in G] < \frac\epsilon2$. Next, observe that
\begin{align*}
\Pr_{x,r}&[A(x,r) =1]& \\&= \Pr_x[x \in G]\cdot\Pr_{x,r}[A(x,r) =1\mid x \in G] \\&+ \Pr_x[x \not\in G]\cdot\Pr_{x,r}[A(x,r) =1\mid x \not\in G]
\\&< \frac\epsilon2\cdot 1 + 1\cdot\frac\epsilon2
\\&< \epsilon,
\end{align*}
which is a contradiction.
\end{proof}
}
that $\Pr_{x \stackrel{\$}{\leftarrow} \binset{n}} [x \in S]  > \frac{\alpha_f(n)}{2}$. Then we have that:
\begin{align*}
\mu_{\mathcal{B},g}(nq) =& \Pr_{(x_1, \cdots, x_q) \stackrel{\$}{\leftarrow} \binset{nq}}[ \mathcal{B}(1^{nq}, g(x_1, \cdots, x_q)) \in g^{-1}(g(x_1, \cdots, x_q))]\\
=&  \Pr_{x_1, \cdots, x_q}[ \mathcal{B}(1^{nq}, g(x_1, \cdots, x_q)) \in g^{-1}(g(x_1, \cdots, x_q)) \wedge \forall i: x_i \notin S]\\
& + \Pr_{x_1, \cdots, x_q}[ \mathcal{B}(1^{nq}, g(x_1, \cdots, x_q)) \in g^{-1}(g(x_1, \cdots, x_q)) \wedge \exists i: x_i \in S]\\
\leq& \Pr_{x_1, \cdots, x_q}[ \forall i: x_i \notin S]
+ \sum_{i=1}^q \Pr_{x_1, \cdots, x_q}[ \mathcal{B}(1^{nq}, g(x_1, \cdots, x_q)) \in g^{-1}(g(x_1, \cdots, x_q)) \wedge  x_i \in S]\\
\leq& \left( 1-\frac{\alpha_f(n)}{2}\right)^q
+ q \cdot \Pr_{x_1, \cdots, x_q,i}[ \mathcal{B}(1^{nq}, g(x_1, \cdots, x_q)) \in g^{-1}(g(x_1, \cdots, x_q)) \wedge x_i \in S] \\
=& \left( 1-\frac{\alpha_f(n)}{2}\right)^{\frac{2n}{\alpha_f(n)}}
+  q\cdot \Pr_{x \stackrel{\$}{\leftarrow} \binset{n}, \mathcal{B}}[\mathcal{A} \text{ inverts $f(x)$ in a single iteration}  \wedge x \in S]\\
\leq& e^{-n} + q\cdot  \Pr_{x}[x \in S] \cdot \Pr[\mathcal{A} \text{ inverts $f(x)$ in a single iteration} ~|~ x \in S]\\
\leq& e^{-n} + \frac{2n}{\alpha_f(n)} \cdot  1 \cdot \frac{\mu_{\mathcal{B},g}(nq) \cdot \alpha_f(n)}{4n}\\
\leq& e^{-n} + \frac{\mu_{\mathcal{B},g}(nq)}{2}.
\end{align*}
Hence $\mu_{\mathcal{B},g}(nq) \leq 2 e^{-n}$, contradicting with the fact that $\mu_{\mathcal{B},g}$ is non-negligible.
Then we have
\begin{align*}
\Pr_{x \stackrel{\$}{\leftarrow} \binset{n}}&[ \mathcal{A}(1^n, f(x)) = \bot]\\
=& \Pr_x[x \in S] + \Pr_x [x \notin S]\cdot\Pr[\mathcal{B} \text{ fails to invert $f(x)$ in every iteration} | x \notin S]\\
\leq& \frac{\alpha_f(n)}{2}+ \left(\Pr[ \mathcal{B} \text{ fails to invert $f(x)$ a single iteration} | x \notin S] \right)^T\\
\leq & \frac{\alpha_f(n)}{2}+ \left( 1-\frac{\mu_{\mathcal{A},g}(nq) \cdot \alpha_f(n)}{4n}\right)^T\\
\leq& \frac{\alpha_f(n)}{2} + e^{-n} \leq \alpha_f(n)
\end{align*}
for sufficiently large $n$. This concludes the proof.
\qed


\section{Levin's One-Way Function}
\begin{theorem}\label{thm:levin}
If there exists a one-way function, then there exists an explicit function $f$ that is one-way  (constructively).
\end{theorem}

\begin{lemma}\label{lem:n2owf}
If there exists a one-way function computable in time $n^c$ for a constant $c$, then there exists a one-way function computable in time $n^2$.
\end{lemma}
\proof
Let $f: \binset{n} \rightarrow \binset{n}$ be a one-way function computable in time $n^c$.
Construct $g: \binset{n+n^c} \rightarrow \binset{n+n^c}$ as follows:
$$g(x,y) = f(x) || y$$
where $x \in \binset{n}, y \in \binset{n^c}$.
$g(x,y)$ takes time $2n^c$, which is linear in the input length.

We next show that $g(\cdot)$ is one-way.
Assume for the purpose of contradiction that there exists an adversary $\mathcal{A}$ such that $\mu_{\mathcal{A},g}(n+n^c) = \Pr_{(x,y) \stackrel{\$}{\leftarrow} \binset{n+n^c}}[ \mathcal{A}(1^{n+n^c}, g(x,y)) \in g^{-1}(g(x,y))]$ is non-negligible. Then we use $\mathcal{A}$ to construct $\mathcal{B}$ such that $\mu_{\mathcal{B},f}(n) = \Pr_{x \stackrel{\$}{\leftarrow} \binset{n}}[ \mathcal{B}(1^n, f(x)) \in f^{-1}(f(x))]$ is also non-negligible.

$\mathcal{B}$ on input $z \in\{0,1\}^n$, samples $y \stackrel{\$}{\leftarrow} \binset{n^c}$, and outputs the $n$ higher-order bits of  $\mathcal{A}(1^{n+n^c}, z||y)$. Then we have
\begin{align*}
\mu_{\mathcal{B},g}(n) =& \Pr_{x \stackrel{\$}{\leftarrow} \binset{n}, y \stackrel{\$}{\leftarrow} \binset{n^c}}\left[\mathcal{A}(1^{n+n^c}, f(x) || y) \in f^{-1}(f(x)) || \binset{n^c}\right]\\
\geq&\Pr_{x,y}\left[\mathcal{A}(1^{n+n^c}, g(x,y)) \in f^{-1}(f(x)) || y\right]\\
=& \Pr_{x,y}\left[\mathcal{A}(1^{n+n^c}, g(x,y)) \in g^{-1}(g(x,y))\right]
\end{align*}
is non-negligible.
\qed

\bigskip
\proof[of Theorem~\ref{thm:levin}]
We first construct a weak one-way function $h: \binset{n} \rightarrow \binset{n}$ as follows:
$$
h(M,x) = \left\{
\begin{array}{ll}
M || M(x) & \text{if $M(x)$ takes no more than $|x|^2$ steps} \\
M || 0 & \text{otherwise}
\end{array}
\right.
$$
where $|M| = \log n, |x| = n - \log n$ (interpreting $M$ as the code of a machine  and $x$ as its input).
If $h$ is weak one-way, then we can construct a one-way function from $h$ as we discussed in Section~\ref{sec:owf:amplify}.

It remains to show that if one-way functions exist, then $h$ is a weak one-way function, with $\alpha_h(n) = \frac{1}{n^2}$.
Assume for the purpose of contradiction that there exists an adversary $\mathcal{A}$ such that $\mu_{\mathcal{A},h}(n) = \Pr_{(M,x) \stackrel{\$}{\leftarrow} \binset{n}}[ \mathcal{A}(1^{n}, h(M,x)) \in h^{-1}(h(M,x))]\geq 1-\frac{1}{n^2}$ for all sufficiently large $n$.
By the existence of one-way functions and Lemma~\ref{lem:n2owf}, there exists a one-way function $\tilde M$ that can be computed in time $n^2$. Let $\tilde M$ be the uniform machine that computes this one-way function.
We will consider values $n$ such that $n > 2^{|\tilde M|}$. In other words for these choices of $n$, $\tilde M$ can be described using $\log n$ bits.
We construct $\mathcal{B}$ to invert $\tilde M$: on input $y$ outputs the $(n-\log n)$ lower-order bits of $\mathcal{A}(1^n, \tilde M||y)$. Then
\begin{align*}
\mu_{\mathcal{B},\tilde M}(n-\log n) =& \Pr_{x \stackrel{\$}{\leftarrow} \binset{n-\log n}}\left[\mathcal{A}(1^{n}, \tilde M || \tilde M(x)) \in \binset{\log n} || \tilde M^{-1}(\tilde M((x))\right]\\
\geq& \Pr_{x \stackrel{\$}{\leftarrow} \binset{n-\log n}}\left[\mathcal{A}(1^{n}, \tilde M || \tilde M(x)) \in \tilde{M} || \tilde M^{-1}(\tilde M((x))\right].
\end{align*}
Observe that for sufficiently large $n$ it holds that
\begin{align*}
1-\frac{1}{n^2} \leq& \mu_{\mathcal{A},h}(n)\\
=& \Pr_{(M,x) \stackrel{\$}{\leftarrow} \binset{n}}\left[ \mathcal{A}(1^{n}, h(M,x)) \in h^{-1}(h(M,x))\right]\\
\leq& \Pr_{M }[M = \tilde M] \cdot \Pr_{x }\left[ \mathcal{A}(1^{n}, \tilde M||\tilde M(x)) \in  \tilde{M} || \tilde M^{-1}(\tilde M((x))\right] + \Pr_{M }[M \neq \tilde M]  \\
\leq&  \frac{1}{n} \cdot \mu_{\mathcal{B},\tilde M}(n-\log n) +\frac{n-1}{n}.
\end{align*}
Hence $\mu_{\mathcal{B},\tilde M}(n-\log n) \geq \frac{n-1}{n}$  for sufficiently large $n$ which is a contradiction.
\qed

\newpage
\section*{Exercises}
\begin{exercise}
\label{ex:product} If $\mu(\cdot)$ and $\nu(\cdot)$ are negligible functions then show that $\mu(\cdot) \cdot \nu(\cdot)$ is a negligible function.
\end{exercise}

\begin{exercise}
\label{ex:product} If $\mu(\cdot)$ is a negligible function and $f(\cdot)$ is a function polynomial in its input then show that $\mu(f(\cdot))$\footnote{Assume that $\mu$ and $f$ are such that $\mu(f(\cdot))$ takes inputs from $\mathbb{Z}^+$ and outputs values in $[0,1]$.} are negligible functions.
\end{exercise}

\begin{exercise}\label{ex:PNP} Prove that the existence of one-way functions implies $P \neq NP$.
\end{exercise}

\begin{exercise}
Prove that there is no one-way function $f:\{0,1\}^n\to \{0,1\}^{\lfloor \log_2 n\rfloor}$.
\end{exercise}


\begin{exercise} Let $f:\{0,1\}^n\to \{0,1\}^{n}$ be any one-way function then is $f'(x) \stackrel{def}{=} f(x)\oplus x$ necessarily one-way?
\end{exercise}

\begin{exercise}
Prove or disprove: If $f: \{0,1\}^n\rightarrow \{0,1\}^n$ is a one-way function, then $g: \{0,1\}^n\rightarrow \{0,1\}^{n-\log n}$ is a one-way function, where $g(x)$ outputs the $n-\log n$ higher order bits of $f(x)$.
\end{exercise}

\begin{exercise}
Explain why the proof of Theorem~\ref{theorem:weakstrongOWF} fails if the attacker $\mathcal{A}$ in Figure~\ref{fig:adv:weak} sets $i = 1$ and not $i \stackrel{\$}{\leftarrow} \{1, 2, \cdots, q\}$.
\end{exercise}

\begin{exercise}
Given a (strong) one-way function construct a weak one-way function that is not a (strong) one-way function.
\end{exercise}

\begin{exercise}
 Let $f:\{0,1\}^n\to \{0,1\}^{n}$ be a weak one-way permutation (a weak one way function that is a bijection). More formally, $f$ is a PPT computable one-to-one function such that $\exists$ a constant $c >0$ such that $\forall$ non-uniform PPT machine $A$ and $\forall$ sufficiently large $n$ we have that:
    \[\Pr_{x,A}[A(f(x)) \not\in f^{-1}(f(x))] > \frac{1}{n^c}\]

     Show that $g(x) = f^T(x)$ is not a strong one way permutation. Here $f^T$ denotes the $T$ times self composition of $f$ and $T$ is a polynomial in $n$.

     Interesting follow up reading if interested: With some tweaks the function above can be made a strong one-way permutation using explicit constructions of expander graphs. See Section 2.6 in \url{http://www.wisdom.weizmann.ac.il/~oded/PSBookFrag/part2N.ps}
\end{exercise}


%

\section{Levin's One-Way Function}

In this section, we discuss Levin's one-way function, which is an explicit construction of a one-way function that is secure as long as a one-way function exists.
This is interesting because unlike a typical cryptographic primitive that relies on a specific hardness assumption (which may or may not hold in the future), Levin's one-way function is future-proof in the sense that it will be secure as long as atleast one hardness assumption holds (which we may or may not discover).

The high-level intuition behind Levin's construction is as follows: since we assume one-way functions exist, there exists a uniform machine $\tilde M$ such that $|\tilde M|$ is a constant and $\tilde M(x)$ is hard to invert for a random input $x$.
Now, consider a function $h$ that parses the first $\log(n)$ bits of its $n$-bit input as the code of a machine $M$ and the remaining bits as the input to $M$.
For a large enough $n$ that is exponential in $|\tilde{M}|$, note that we will hit the code of $\tilde{M}$ with noticeable probability in $n$, and for those instances, $h$ will be hard to invert.
% that if you parse the input to the function as the code of a uniform machine $M$ followed by the input to the machine, for large enough input length $n$ that is exponential in $|M|$, you'll get the code of a one-way function with some noticeable probability.
It is easy to see that this gives us a weak one-way function which has a noticeable probability of being hard to invert, and we can amplify the hardness of this weak one-way function to get an explicit construction of a one-way function.

\begin{theorem}\label{thm:levin}
If there exists a one-way function, then there exists an explicit function $f$ that is one-way  (constructively).
\end{theorem}

Before we look at the construction and the proof in detail, we first prove a lemma that will be useful in the proof.
In particular, we need a bound on the running time of the one-way function $\tilde M$ so that we can upper bound the execution time of $h$, since there could be inputs to $g$ that do not terminate in polynomial time.
To this end, we prove the following lemma which shows that if a one-way function exists, then there is also a one-way function that runs in time $n^2$, and thus, we can bound $h$ to $n^2$ steps.

\begin{lemma}\label{lem:n2owf}
If there exists a one-way function computable in time $n^c$ for a constant $c$, then there exists a one-way function computable in time $n^2$.
\end{lemma}
\proof
Let $f: \binset{n} \rightarrow \binset{n}$ be a one-way function computable in time $n^c$.
Construct $g: \binset{n+n^c} \rightarrow \binset{n+n^c}$ as follows:
$$g(x,y) = f(x) || y$$
where $x \in \binset{n}, y \in \binset{n^c}$.
$g(x,y)$ takes time $2n^c$, which is linear in the input length.

We next show that $g(\cdot)$ is one-way.
Assume for the purpose of contradiction that there exists an adversary $\mathcal{A}$ such that $\mu_{\mathcal{A},g}(n+n^c) = \Pr_{(x,y) \stackrel{\$}{\leftarrow} \binset{n+n^c}}[ \mathcal{A}(1^{n+n^c}, g(x,y)) \in g^{-1}(g(x,y))]$ is non-negligible. Then we use $\mathcal{A}$ to construct $\mathcal{B}$ such that $\mu_{\mathcal{B},f}(n) = \Pr_{x \stackrel{\$}{\leftarrow} \binset{n}}[ \mathcal{B}(1^n, f(x)) \in f^{-1}(f(x))]$ is also non-negligible.

$\mathcal{B}$ on input $z \in\{0,1\}^n$, samples $y \stackrel{\$}{\leftarrow} \binset{n^c}$, and outputs the $n$ higher-order bits of  $\mathcal{A}(1^{n+n^c}, z||y)$. Then we have
\begin{align*}
\mu_{\mathcal{B},g}(n) =& \Pr_{x \stackrel{\$}{\leftarrow} \binset{n}, y \stackrel{\$}{\leftarrow} \binset{n^c}}\left[\mathcal{A}(1^{n+n^c}, f(x) || y) \in f^{-1}(f(x)) || \binset{n^c}\right]\\
\geq&\Pr_{x,y}\left[\mathcal{A}(1^{n+n^c}, g(x,y)) \in f^{-1}(f(x)) || y\right]\\
=& \Pr_{x,y}\left[\mathcal{A}(1^{n+n^c}, g(x,y)) \in g^{-1}(g(x,y))\right]
\end{align*}
is non-negligible.
\qed

\bigskip
Now, we provide the explicit construction of $h$ and prove that it is a weak one-way function.
Since $h$ is an (explicit) weak one-way function, we can construct an (explicit) one-way function from $h$ as we discussed in Section~\ref{sec:owf:amplify}, and this would prove Theorem~\ref{thm:levin}.
\proof[Proof of Theorem~\ref{thm:levin}]
$h: \binset{n} \rightarrow \binset{n}$ is defined as follows:
$$
h(M,x) = \left\{
\begin{array}{ll}
M || M(x) & \text{if $M(x)$ takes no more than $|x|^2$ steps} \\
M || 0 & \text{otherwise}
\end{array}
\right.
$$
where $|M| = \log n, |x| = n - \log n$ (interpreting $M$ as the code of a machine  and $x$ as its input).

It remains to show that if one-way functions exist, then $h$ is a weak one-way function, with $\alpha_h(n) = \frac{1}{n^2}$.
Assume for the purpose of contradiction that there exists an adversary $\mathcal{A}$ such that $\mu_{\mathcal{A},h}(n) = \Pr_{(M,x) \stackrel{\$}{\leftarrow} \binset{n}}[ \mathcal{A}(1^{n}, h(M,x)) \in h^{-1}(h(M,x))]\geq 1-\frac{1}{n^2}$ for all sufficiently large $n$.
By the existence of one-way functions and Lemma~\ref{lem:n2owf}, there exists a one-way function $\tilde M$ that can be computed in time $n^2$. Let $\tilde M$ be the uniform machine that computes this one-way function.
We will consider values $n$ such that $n > 2^{|\tilde M|}$. In other words for these choices of $n$, $\tilde M$ can be described using $\log n$ bits.
We construct $\mathcal{B}$ to invert $\tilde M$: on input $y$ outputs the $(n-\log n)$ lower-order bits of $\mathcal{A}(1^n, \tilde M||y)$. Then
\begin{align*}
\mu_{\mathcal{B},\tilde M}(n-\log n) =& \Pr_{x \stackrel{\$}{\leftarrow} \binset{n-\log n}}\left[\mathcal{A}(1^{n}, \tilde M || \tilde M(x)) \in \binset{\log n} || \tilde M^{-1}(\tilde M((x))\right]\\
\geq& \Pr_{x \stackrel{\$}{\leftarrow} \binset{n-\log n}}\left[\mathcal{A}(1^{n}, \tilde M || \tilde M(x)) \in \tilde{M} || \tilde M^{-1}(\tilde M((x))\right].
\end{align*}
Observe that for sufficiently large $n$ it holds that
\begin{align*}
1-\frac{1}{n^2} \leq& \mu_{\mathcal{A},h}(n)\\
=& \Pr_{(M,x) \stackrel{\$}{\leftarrow} \binset{n}}\left[ \mathcal{A}(1^{n}, h(M,x)) \in h^{-1}(h(M,x))\right]\\
\leq& \Pr_{M }[M = \tilde M] \cdot \Pr_{x }\left[ \mathcal{A}(1^{n}, \tilde M||\tilde M(x)) \in  \tilde{M} || \tilde M^{-1}(\tilde M((x))\right] + \Pr_{M }[M \neq \tilde M]  \\
\leq&  \frac{1}{n} \cdot \mu_{\mathcal{B},\tilde M}(n-\log n) +\frac{n-1}{n}.
\end{align*}
Hence $\mu_{\mathcal{B},\tilde M}(n-\log n) \geq \frac{n-1}{n}$  for sufficiently large $n$ which is a contradiction.
\qed

\section{Hardness Concentrate Bit}
We start by asking the following question: Is it possible to concentrate the strength of a one-way function into one bit? In particular, given a one-way function $f$, does there exist one bit that can be computed efficiently from the input $x$, but is hard to compute given $f(x)$?
\begin{definition}[Hard Concentrate Bit]
Let $f:\binset{n} \rightarrow \binset{n}$ be a one-way function.
$B:\{0,1\}^n \rightarrow \{0,1\}$ is a hard concentrate bit of $f$ if:
\begin{itemize}
\item[-] $B$ is computable by a polynomial time machine, and
\item[-] $\forall$ non-uniform PPT adversaries $\mathcal{A}$ we have that
	$$\Pr_{x\stackrel{\$}{\leftarrow} \binset{n}}[\mathcal{A}(1^n, f(x)) = B(x)] \leq \frac{1}{2} + \mathsf{negl}(n).$$
\end{itemize}
\end{definition}

\noindent\textbf{A simple example.}
Let $f$ be a one-way function. Consider the one-way function $g(b, x) = 0 || f(x)$ and a hard concentrate bit $B(b, x) = b$.
Intuitively, the value $g(b, x)$ does not reveal any information about the first bit $b$, thus no information about the value $B(b, x)$ can be ascertained. Hence $\mathcal{A}$ cannot predict the first bit with a non-negligible advantage than a random guess. However, we are more interested in the case where the hard concentrate bit is hidden because of computational hardness and not information theoretic hardness.

\bigskip
\begin{remark}
Given a one-way function $f$, we can construct another one-way function $g$ with a hard concentrate bit. However, we may not be able to find a hard concentrate bit for $f$. In fact, it is an open question whether a hard concentrate bit exists for every one-way function.
\end{remark}

\bigskip
Intuitively, if a function $f$ is one-way, it seems that there should be a particular bit in the input $x$ that is hard to compute given $f(x)$. However, we show that is not true:
\begin{claim}
If $f:\binset{n}\rightarrow \binset{n}$ is a one-way function, then there exists a one-way function $g:\binset{n+\log n}\rightarrow\binset{n+\log n}$ such that $\forall i \in [1, n+\log n]$, $B_i(x) = x_i$ is not a hard concentrate bit, where $x_i$ is the $i^\text{th}$ bit of $x$.
\end{claim}
\proof
Define $g:\{0,1\}^{n+\log(n)} \rightarrow \{0,1\}^{n+\log(n)}$ as follows.
$$g(x,y) = f(x_{\bar y}) || x_y || y,$$
where $|x| = n, |y| = \log n$, $x_{\bar y}$ is all bits of $x$ except the $y^\text{th}$ bit, and $x_y$ is the $y^\text{th}$ bit of $x$.

First, one can show that $g$ is still a one-way function. (We leave this as an exercise!)
Next, we show that $B_i$ is not a hard concentrate bit for $\forall i \in [1, n]$ (clearly $B_i$ is not a hard concentrate bit for $i \in [n+1, n+\log n]$).
Construct an adversary $\mathcal{A}_i(1^{n+\log n}, f(x_{\bar y}) || x_y || y)$ that ``breaks'' $B_i$:
\begin{itemize}
\item[-] If $y \not= i$ then output a random bit;
\item[-] Otherwise output $x_y$.
\end{itemize}
\begin{align*}
& \Pr_{x, y}[\mathcal{A}(1^{n+\log n}, g(x,y)) = B_i(x)]\\
=& \Pr_{x, y}[\mathcal{A}(1^{n+\log n}, f(x_{\bar y}) || x_y || y) = x_i]\\
=& \frac{n-1}{n} \cdot \frac{1}{2} + \frac{1}{n} \cdot 1 = \frac{1}{2} + \frac{1}{2n}.
\end{align*}
Hence $\mathcal{A}_i$ can guess the output of $B_i$ with greater than $\frac{1}{2} + \mathsf{negl}(n)$ probability.
\qed


\iffalse
\paragraph{Application: Coin tossing over the phone.} We next describe an application of hard concentrate bits to coin tossing.
Consider two parties trying to perform a coin tossing over the phone. In this setting the first party needs to declare its choice as the second one flips the coin. However, how can the first party trust the win/loss response from the second party?  In particular, if the first party calls out ``head'' and then the second party can just lie that it was ``tails.'' We can use hard concentrate bit of a (one-to-one) one-way function to  enable this applications.

Let $f$ be a (one-to-one) one-way function and $B$ be a hard concentrate bit for $f$. Consider the following protocol:
\begin{itemize}
\item[-] Party $P_1$ samples $x$ from $\{0,1\}^n$ uniformly at random and sends $y$, where $y = f(x)$, to party $P_2$.
\item[-] $P_2$ sends back a random bit $b$ sampled from $\{0,1\}$.
\item[-] $P_1$ sends back $(x, B(x))$ to $P_2$. $P_2$ aborts if $f(x) \neq y$.
\item[-]  Both parties output $B(x)\oplus b$.
\end{itemize}
Note that $P_2$ cannot guess $B(x)$ with a non-negligible advantage than $1/2$ as he sends back his $b$.
On the other hand, $P_1$ cannot flip the value $B(x)$ once it has sent $f(x)$ to $P_2$ because $f$ is one-to-one.
\fi

\subsection{Hard Concentrate Bit of any One-Way Permutation}
We now show that a slight modification of every one-way function has a hard concentrate bit. More formally,
\begin{theorem}\label{thm:hard-concentrate-bit}
Let  $f:\binset{n} \rightarrow \binset{n}$ be a one-way function.
Define a function $g:\binset{2n} \rightarrow \binset{2n}$ as follows:
$$g(x,r) = f(x) || r,$$
where $|x| = |r| =n$. Then we have that $g$ is one-way and that it has a hard concentrate bit, namely $B(x, r) = \sum_{i=1}^n x_i r_i\mod 2$.
\end{theorem}
\begin{remark}
If $f$ is a (one-to-one) one-way function, then $g$ is also a (one-to-one) one-way function with hard concentrate bit $B(\cdot)$.
\end{remark}
\proof
We leave it as an exercise to show that $g$ is a one-way function and below we will prove that the function $B(\cdot)$ describe a hard concentrate bit of $g$.
More specifically, we need to show that if there exists a non-uniform PPT  $\ma$ s.t. $\Pr_{x,r}[\ma(1^{2n},g(x,r)) = B(x,r)] \ge \frac{1}{2} + \epsilon(n)$, where $\epsilon$ is non-negligible, then there exists a non-uniform PPT $\mathcal{B}$ such that $\Pr_{x,r}[\mathcal{B}(1^{2n}, g(x,r)) \in g^{-1}(g(x,r))]$ is non-negligible.
Below we use $E$ to denote the event that $\ma(1^{2n},g(x,r)) = B(x,r)$.
We will present our proof in three steps, where each step progressively increases in complexity:
(1) the super simple case where we restrict to $\ma$ such that $\Pr_{x,r}[E] = 1$,
(2) the simple case where we restrict to $\ma$ such that $\Pr_{x,r}[E] \geq \frac34 + \epsilon(n)$,
and finally (3) the general case where $\Pr_{x,r}[E] \geq \frac12 + \epsilon(n)$.

\medskip
\noindent\textbf{\underline{Super simple case.}}
Suppose $\ma$ guesses $B(\cdot)$ with perfect accuracy:
$$\Pr_{x,r}[E] =1.$$
We now construct $\mathcal{B}$ that inverts $g$ with perfect accuracy.
Let $e^i$ denote the one-hot $n$-bit string $0\cdots 0 1 0 \cdots0$, where only the $i$-th bit is $1$, the rest are all $0$.
$\mathcal{B}$ gets $f(x)||r$ as input, and its algorithm is described in Figure~\ref{alg:super-simple-case}.

\begin{marginfigure}
\begin{algorithmic}
\For {$i=1$ \textbf{to} $n$}
    \State $x'_i \gets \ma(1^{2n}, f(x)||e^i)$
\EndFor
\State \Return $x'_1\cdots x'_n || r$
\end{algorithmic}
\caption{Super-Simple Case $\mathcal{B}$} \label{alg:super-simple-case}
\end{marginfigure}
Observe that $B(x,e^i) = \sum_{j=1}^n x_je^i_j = x_i$. Therefore, the probability that $\mathcal{B}$ inverts a single bit successfully is,
$$\Pr_{x}\left[\ma(1^{2n}, f(x)||e^i)=x_i\right] =  \Pr_{x}\left[\ma(1^{2n}, f(x)||e^i)=B(x,e^i)\right] = 1.$$
Hence $\Pr_{x,r}[\mathcal{B}(1^{2n}, g(x,r)) = (x,r)] = 1$.


\bigskip
\noindent\textbf{\underline{Simple case.}}
Next moving on to the following more demanding case.
$$\Pr_{x,r}[E] \geq \frac{3}{4} + \epsilon(n),$$ where $\epsilon(\cdot)$ is non-negligible.
We describe $\mathcal{B}$'s algorithm for inverting $g$ in Figure~\ref{alg:simple-case}.
Here we can no longer use the super simple case algorithm because we no longer know if $\ma$ outputs the correct bit on input $f(x) \| e^i$.
Instead, we introduce randomness to $\ma$'s input expecting that it should be able to guess the right bit on majority of those inputs since it has a high probability of guessing $B(\cdot)$ in general.
We now also need to make two calls to $\ma$ to isolate the $i$-th bit of $x$.
% On input $f(x)||r$, $\mathcal{B}$ proceeds as follows:
Note that an iteration of $\mathcal{B}$ outputs the right bit if calls to $\ma$ output the correct bit because $B(x,s) \oplus B(x, s\oplus e^i) = x_i$:
\begin{marginfigure}
\begin{algorithmic}
\For {$i = 1$ \textbf{to} $n$}
	\For {$t = 1$ \textbf{to} $T = \frac{n}{2\epsilon(n)^2}$}
		\State $s \stackrel{\$}{\leftarrow} \binset{n}$
    	% \State $x_i^t \leftarrow \ma(f(x)|| s) \oplus \ma(f(x) || s+e^i)$
		\State $\begin{array}{l@{}l}
		x_i^t \leftarrow& \ma(f(x)|| s) \\
		& \oplus~ \ma(f(x) || (s\oplus e^i))
		\end{array}$
	\EndFor
	\State $x'_i \gets $ the majority of $\{x_i^1, \cdots, x_i^T\}$
\EndFor
\State \Return $x'_1\cdots x'_n||R$
\end{algorithmic}
\caption{Simple Case $\mathcal{B}$} \label{alg:simple-case}
\end{marginfigure}
\begin{align*}
B(x,s) \oplus B(x, s\oplus e^i) =& \sum_j x_j s_j \oplus \sum_j x_j (s_j \oplus e^i_j)\\
=& \sum_{j \not= i} (x_j s_j \oplus x_j s_j) \oplus x_i s_i \oplus x_i (s_i \oplus 1)\\
=& ~x_i
\end{align*}
The key technical challenge in proving that $\mathcal{B}$ inverts $g$ with non-negligible probability arises from the fact that the calls to $\ma$ made during one iteration of $\mathcal{B}$ are not independent.
In particular, all calls to $\ma$ share the same $x$ and the calls $\ma(f(x)|| s)$ and $\ma(f(x) || (s \oplus e^i))$ use correlated randomness as well.

We solve the first issue by showing that there exists a large set of $x$ values for which $\ma$ still works with large probability.
The latter issue of lack of independence between $\ma(f(x)|| s)$ and $\ma(f(x) || (s \oplus e^i))$ can be solved using a union bound since the success probability of the adversary $\ma$ is high enough.

Formally, define the set $G$ of ``good'' $x$'s, for which it is easy for $\ma$ to predict the right bit:
$$G := \left\{x \left| \Pr_r \left[ E \right]\geq \frac{3}{4} + \frac{\epsilon(n)}{2} \right. \right\}.$$
Now we prove that $G$ is not a small set. More formally, we claim that:
$$\Pr_{x \stackrel{\$}{\leftarrow} \binset{n}}[x \in G] \geq \frac{\epsilon(n)}{2}.$$
Assume that $\Pr_{x \stackrel{\$}{\leftarrow} \binset{n}}[x \in G] < \frac{\epsilon(n)}{2}$. Then we have the following contradiction:
\begin{align*}
\frac{3}{4} + \epsilon(n) \leq&~ \Pr_{x,r}[E]\\
=&~ \Pr_x [x \in G] \Pr_{r}[E| x\in G] + \Pr_x [x \notin G] \Pr_{r}[E | x\notin G]\\
< &~ \frac{\epsilon(n)}{2} \cdot 1 + 1\cdot \left(\frac{3}{4}+\frac{\epsilon(n)}{2}\right)  = \frac{3}{4} + \epsilon(n).
\end{align*}
Now consider a single iteration for a fixed $x \in G$:
\begin{align*}
&\Pr_{s} \left[ \ma(f(x), s) \oplus \ma(f(x), s\oplus e^i) = x_i \right]\\
&=~ \Pr_{s} \left[ \text{Both $\ma$'s are correct} \right] + \Pr_{s} \left[ \text{Both $\ma$'s are wrong} \right]\\
&\geq~ \Pr_{s} \left[ \text{Both $\ma$'s are correct} \right] = 1- \Pr_{s} \left[ \text{Either $\ma$ is wrong} \right]\\
&\geq~ 1- 2\cdot\Pr_{s} \left[ \text{$\ma$ is wrong} \right]\\
&\geq~ 1-2\left(\frac{1}{4} - \frac{\epsilon(n)}{2} \right) 
= \frac{1}{2} + \epsilon(n).
\end{align*}
Let $Y_i^t$ be the indicator random variable that $x_i^t = x_i$ (namely, $Y_i^t=1$ with probability $\Pr[x_i^t = x_i]$ and $Y_i^t=0$ otherwise).
Note that $Y_i^1, \cdots, Y_i^T$ are independent and identical random variables, and for all $t \in \{1,\ldots, T\}$, we have $\Pr[Y_i^t=1] = \Pr[x_i^t = x_i] \geq \frac{1}{2} + \epsilon(n)$.
Next we argue that majority of $x_i^t$ coincide with $x_i$ with high probability.
\begin{align*}
\Pr[x'_i \neq x_i]
=&~ \Pr\left[\sum_{t=1}^T Y_i^t \leq \frac{T}{2} \right]\\
=&~ \Pr\left[\sum_{t=1}^T  Y_i^t- \left(\frac{1}{2} + \epsilon(n) \right)T \leq \frac{T}{2} - \left(\frac{1}{2} + \epsilon(n) \right)T \right]\\
\leq&~ \Pr\left[ \left| \sum_{t=1}^T  Y_i^t- \left(\frac{1}{2} + \epsilon(n) \right)T \right| \geq \epsilon(n)T \right]\\
& \text{Let $X_1,\cdots,X_m$ be i.i.d. random variables taking values 0 or 1. Let $\Pr[X_i=1] = p$.}\\
& \text{By Chebyshev's Inequality, $\Pr\left[ \left| \sum X_i - pm \right| \geq \delta m \right] \leq \frac{1}{4\delta^2 m}$.}\\
\leq&~ \frac{1}{4\epsilon(n)^2T} = \frac{1}{2n}.
\end{align*}
Then, completing the argument, we have
\begin{align*}
&\Pr_{x,r}[\mathcal{B}(1^{2n}, g(x,r)) = (x,r)]\\
&\geq~ \Pr_x [x \in G] \Pr[x'_1 = x_1, \cdots x'_n = x_n | x \in G]\\
&\geq~ \frac{\epsilon(n)}{2} \cdot \left(1- \sum_{i=1}^n\Pr[x'_i \neq x_i | x \in G]\right)\\
&\geq~ \frac{\epsilon(n)}{2} \cdot \left(1- n \cdot\frac{1}{2n} \right) =  \frac{\epsilon(n)}{4}.
\end{align*}



\bigskip
\noindent\textbf{\underline{Real Case.}} Now, we describe the final case where $\Pr_{x,r}[E] \geq \frac{1}{2} + \epsilon(n)$ and $\epsilon(\cdot)$ is a non-negligible function.
The key technical challenge in this case is that we cannot make two related calls to $\ma$ as was done in the simple case above since we can't argue that both calls to $\ma$ will be correct with high enough probability.
However, just using one call to $\ma$ seems insufficient.
The key idea is to just guess one of those values.
Very surprisingly, this idea along with careful analysis magically works out.
Just like the previous two cases, we start by describing the algorithm $\mathcal{B}$ in Figure~\ref{alg:real-case}.


\begin{marginfigure}
\begin{algorithmic}
\State $T = \frac{2n}{\epsilon(n)^2}$
	\For {$\ell = 1$ \textbf{to} $\log T$}
		\State $s_\ell \stackrel{\$}{\leftarrow} \binset{n}$
		\State $b_\ell \stackrel{\$}{\leftarrow} \{0,1\}$
	\EndFor
\For {$i = 1$ \textbf{to} $n$}
	\ForAll {$L \subseteq \{1,2,\cdots, \log T\}$}
		\State $S_L :=\bigoplus_{j \in L} s_j$
		\State $B_L := \bigoplus_{j \in L} b_j$
		\State $x_i^L \leftarrow B_L \oplus \ma(f(x) || S_L\oplus e^i)$
	\EndFor
	\State $x'_i \gets $ majority of $\{x_i^\emptyset, \cdots, x_i^{[\log T]}\}$
\EndFor
\State \Return $x'_1\cdots x'_n||R$
\end{algorithmic}
\caption{Real Case $\mathcal{B}$} \label{alg:real-case}
\end{marginfigure}

\medskip
In the beginning of the algorithm, $\mathcal{B}$ samples $\log T$ random strings $\{ s_\ell \}_{\ell}$ and bits $\{ b_\ell \}_{\ell}$.
Since there are only $\log T$ values, with probability $\frac{1}{T}$ (which is polynomial in $n$) all the $b_{\ell}$'s are correct, i.e., $b_\ell = B(x, s_\ell)$. In the rest of this proof, we denote this event as $F$.
Now note that if $F$ happens, then $B_L$ as defined in the algorithm is also equal to $B(x, S_L)$ (we denote the $k^{\text{th}}$-bit of $s$ with $(s)_k$): 

\begin{align*}
B(x,S_L) &= \sum_{k=1}^n x_k (\bigoplus_{j \in L} s_j)_k \\ 
         &=  \sum_{k=1}^n x_k \sum_{j \in L} \left(s_j\right)_k \\
         &= \sum_{j \in L} \sum_{k=1}^n x_k (s_j)_k \\
         &= \sum_{j \in L} B(x,s_j) \\
         &= \sum_{j\in L} b_j  \\
         &= B_L
\end{align*}

Thus, with probability $\frac{1}{T}$, we have all the right guesses for one of the invocations, and we just need to bound the probability that $\ma(f(x) || S_L \oplus e^i) = B(x, S_L \oplus e^i)$.
However there is a subtle issue. Now the events $Y_i^\emptyset, \cdots, Y_i^{[\log T]}$ are no longer independent.
Nevertheless, we can still show that they are pairwise independent, and the Chebyshev's Inequality still holds. Now we give the formal proof.

Just as in the simple case, we define the set $G$ as
$$G := \left\{x \left| \Pr_r \left[ E \right]\geq \frac{1}{2} + \frac{\epsilon(n)}{2} \right. \right\},$$
and with an identical argument we obtain that:

$$\Pr_{x \stackrel{\$}{\leftarrow} \binset{n}}[x \in G] \geq \frac{\epsilon(n)}{2}$$

% Correctness of $\mathcal{B}$ follows from the fact in case $b_\ell = B(x,s_\ell)$ for every $\ell \in [\log T]$ then $\forall L \subseteq [\log T]$, it holds that
Next, given $\{ b_\ell = B(x,s_\ell) \}_{\ell \in [\log T]}$ and $x\in G$, we have:
\begin{align*}
&\Pr_{r} \left[  B_L \oplus \ma(f(x) || S_L \oplus e^i) = x_i \right] \\
&=~ \Pr_{r} \left[ B(x,S_L) \oplus \ma(f(x) || S_L \oplus e^i) = x_i \right]\\
&=~ \Pr_{r} \left[ \ma(f(x) || S_L \oplus e^i) =  B(x,S_L \oplus e^i) \right]\\
&\geq~ \frac{1}{2} + \frac{\epsilon(n)}{2}
\end{align*}
For the same $\{ b_\ell \}_\ell$ and $x\in G$, let $Y_i^L$ be the indicator random variable that $x_i^L = x_i$.
Notice that $Y_i^\emptyset, \cdots, Y_i^{[\log T]}$ are pairwise independent and $\Pr[Y_i^L=1] = \Pr[x_i^L = x_i] \geq \frac{1}{2} + \frac{\epsilon(n)}{2}$.
\begin{align*}
\Pr[x'_i \neq x_i] =& \Pr\left[\sum_{L \subseteq [\log T]} Y_i^L \leq \frac{T}{2} \right]\\
=& \Pr\left[\sum_{L \subseteq [\log T]} Y_i^L - \left(\frac{1}{2} +  \frac{\epsilon(n)}{2} \right)T \leq \frac{T}{2} - \left(\frac{1}{2} +  \frac{\epsilon(n)}{2} \right)T \right]\\
\leq& \Pr\left[ \left| \sum_{L \subseteq [\log T]} Y_i^L - \left(\frac{1}{2} +  \frac{\epsilon(n)}{2} \right)T \right| \geq \frac{\epsilon(n)}{2} T \right]\\
& \text{(By Theorem~\ref{thm:Chebyshev})}\\
\leq& \frac{1}{4\left( \frac{\epsilon(n)}{2}\right)^2T} = \frac{1}{2n}.
\end{align*}
Completing the proof, we have that:
\begin{align*}
& \Pr_{x,r}[\mathcal{B}(1^{2n}, g(x,r)) = (x,r)]\\
&\geq~  \Pr_{\{ b_\ell, s_\ell \}_\ell}\left[ F \right] \cdot  \Pr_x [x \in G] \cdot \Pr[x'_1 = x_1, \cdots x'_n = x_n ~|~ x \in G ~\wedge~ F]\\
&\geq~ \frac{1}{T} \cdot \frac{\epsilon(n)}{2} \cdot \left(1- \sum_{i=1}^n\Pr[x'_i \neq x_i ~|~  x \in G ~\wedge~ F]\right)\\
&\geq~ \frac{\epsilon(n)^2}{2n} \cdot \frac{\epsilon(n)}{2} \cdot \left(1- n \cdot\frac{1}{2n} \right) =  \frac{\epsilon(n)^3}{8n}
\end{align*}
\qed

\marginnote[-17cm]{%
\noindent\textbf{Pairwise Independence and Chebyshev's Inequality.} For the sake of completeness, we prove the Chebyshev's Inequality here.
\begin{definition}[Pairwise Independence]
A collection of random variables $\{X_1,\cdots,X_m\}$ is said to be \emph{pairwise independent} if for every pair of random variables $(X_i, X_j), i \neq j$  and every pair of values $(v_i,v_j)$, it holds that
\[\Pr[X_i = v_i, X_j = v_j] = \Pr[X_i = v_i]\Pr[X_j = v_j]\]
\end{definition}

\begin{theorem}[Chebyshev's Inequality]\label{thm:Chebyshev}
Let $X_1,\hdots,X_m$ be pairwise independent and identically distributed binary random variables. In particular, for every $i \in [m]$, $\Pr[X_i = 1] = p$ for some $p\in [0,1]$ and $\Pr[X_i=0]=1-p$. Then it holds that
$$\Pr\left[\left|\sum_{i=1}^m X_i - pm\right| \geq \delta m\right] \leq \frac{1}{4\delta^2m}.$$
\end{theorem}

\proof
Let $Y = \sum_i X_i$. Then
\begin{align*}
&\Pr\left[\left|\sum_{i=1}^m X_i - pm\right| > \delta m\right] \\
&=~ \Pr\left[\left(\sum_{i=1}^m X_i - pm\right)^2> \delta^2 m^2\right]\\
&\leq~ \frac{\mathbb{E}\left[\left|Y - pm\right|^2\right]}{\delta^2m^2} = \frac{\text{Var}(Y)}{\delta^2m^2} \\
\end{align*}
Observe that
\begin{align*}
\text{Var}(Y) &= \mathbb{E}\left[Y^2\right] - \left(\mathbb{E}[Y]\right)^2\\
&= \sum_{i=1}^m \sum_{j=1}^m \left( \mathbb{E}\left[X_iX_j\right] - \mathbb{E}\left[X_i\right] \mathbb{E}\left[X_j\right]\right)\\
& \text{By pairwise independence, for $i \neq j$,} \\  
& \text{$\mathbb{E}\left[X_i X_j\right] = \mathbb{E}\left[X_i\right] \mathbb{E}\left[X_j\right]$.}\\
&= \sum_{i=1}^m \mathbb{E}\left[X_i^2\right] - \mathbb{E}\left[X_i\right]^2\\
&= mp(1-p).
\end{align*}
Hence
$$\Pr\left[\left|\sum_{i=1}^m X_i - pm\right| \geq\delta m\right] \leq \frac{mp(1-p)}{\delta^2m^2} \leq \frac{1}{\delta^2m}.$$
\qed
}


\newpage
\section*{Exercises}
\begin{exercise}
\label{ex:product} If $\mu(\cdot)$ and $\nu(\cdot)$ are negligible functions then show that $\mu(\cdot) \cdot \nu(\cdot)$ is a negligible function.
\end{exercise}

\begin{exercise}
\label{ex:product} If $\mu(\cdot)$ is a negligible function and $f(\cdot)$ is a function polynomial in its input then show that $\mu(f(\cdot))$\footnote{Assume that $\mu$ and $f$ are such that $\mu(f(\cdot))$ takes inputs from $\mathbb{Z}^+$ and outputs values in $[0,1]$.} are negligible functions.
\end{exercise}

\begin{exercise}\label{ex:PNP} Prove that the existence of one-way functions implies $P \neq NP$.
\end{exercise}

\begin{exercise}
Prove that there is no one-way function $f:\{0,1\}^n\to \{0,1\}^{\lfloor \log_2 n\rfloor}$.
\end{exercise}


\begin{exercise} Let $f:\{0,1\}^n\to \{0,1\}^{n}$ be any one-way function then is $f'(x) \stackrel{def}{=} f(x)\oplus x$ necessarily one-way?
\end{exercise}

\begin{exercise}
Prove or disprove: If $f: \{0,1\}^n\rightarrow \{0,1\}^n$ is a one-way function, then $g: \{0,1\}^n\rightarrow \{0,1\}^{n-\log n}$ is a one-way function, where $g(x)$ outputs the $n-\log n$ higher order bits of $f(x)$.
\end{exercise}

\begin{exercise}
Explain why the proof of Theorem~\ref{theorem:weakstrongOWF} fails if the attacker $\mathcal{A}$ in Figure~\ref{fig:adv:weak} sets $i = 1$ and not $i \stackrel{\$}{\leftarrow} \{1, 2, \cdots, q\}$.
\end{exercise}

\begin{exercise}
Given a (strong) one-way function construct a weak one-way function that is not a (strong) one-way function.
\end{exercise}

\begin{exercise}
 Let $f:\{0,1\}^n\to \{0,1\}^{n}$ be a weak one-way permutation (a weak one way function that is a bijection). More formally, $f$ is a PPT computable one-to-one function such that $\exists$ a constant $c >0$ such that $\forall$ non-uniform PPT machine $A$ and $\forall$ sufficiently large $n$ we have that:
    \[\Pr_{x,A}[A(f(x)) \not\in f^{-1}(f(x))] > \frac{1}{n^c}\]

     Show that $g(x) = f^T(x)$ is not a strong one way permutation. Here $f^T$ denotes the $T$ times self composition of $f$ and $T$ is a polynomial in $n$.

     Interesting follow up reading if interested: With some tweaks the function above can be made a strong one-way permutation using explicit constructions of expander graphs. See Section 2.6 in \url{http://www.wisdom.weizmann.ac.il/~oded/PSBookFrag/part2N.ps}
\end{exercise}




%\subsection{Proof: Fixing a Value in a One-way Function}
%
%\begin{theorem}
%Given a one-way function $f : \binset{n} \rightarrow \binset{m}$ and constants $x_0 \in \binset{n}$, $y_0 \in \binset{m}$, $\exists g : \binset{n} \rightarrow \binset{m}$ such that $g(x_0) = y_0$ where $g$ is a one-way function.\\
%\end{theorem}
%
%Main Idea:  Set $g$ to be $f$, except at $x_0$, where $g(x_0) = y_0$.  If there exists an adversary that can break $g$, then that adversary will also break $f$, because the adversary can only know negligibly more information about $g$ than $f$.\\
%
%\proof  Define the function $g$ as follows:
%
%$g(x) = \left\{
%\begin{array}{lr}
%  y_0 & : x = x_0 \\
%  f(x) & : x \neq x_0
%\end{array}
%\right.$
%
%Suppose there is an adversary $A$ that can break $g$ with non-negligible probability $\mu(n)$.\\
%
%So, we have $\mu(n) = \underset{x \overset{\$}{\leftarrow} \binset{n}}{Pr} [ A(g(x)) \in g^{-1}(g(x)) ] = \sum\limits_{x \in \binset{n}} Pr(X = x) Pr [ A(g(x)) \in g^{-1}(g(x)) ]$\\
%
%Since $x$ is uniformly distributed, $Pr[X = x] = \frac{1}{2^n}$.  We can split it into the cases $x : g(x) = y_0$ and $x : g(x) \neq y_0$:\\
%
%$\mu(n) = \big[ \frac{1}{2^n} \sum\limits_{x \in \binset{n}, g(x) = y_0} Pr [ A(y_0) \in g^{-1}(y_0)) ] \big] + \big[ \frac{1}{2^n} \sum\limits_{x \in \binset{n}, g(x) \neq y_0} Pr [ A(g(x)) \in g^{-1}(g(x)) ] \big]$.\\
%
%Let $p = | \{ x : g(x) = y_0 \} |$.  Consider the adversary $M$ where $M(y) = x_1$ for any $y$, where $x_1$ is a value of $x$ where $f(x_1) = y_0$.  Thus, $M$ breaks $f$ for any input where $f(x) = y_0$, of which there are $p - 1$ or $p$ (depending on whether $f(x_0) = y_0$).  So, the probability with which $M$ breaks $f$ is $\frac{p-1}{2^n}$ or $\frac{p}{2^n}$.  Either way, since $f$ is a one-way function, this implies that $\frac{p}{2^n}$ is a negligible function.\\
%
%Now, since $Pr [ A(y_0) \in g^{-1}(g(x_0)) ] \leq 1$, we have:\\
%
%$\mu(n) \leq \frac{p}{2^n} + \sum\limits_{x \in \binset{n}, g(x) \neq y_0} Pr [ A(g(x)) \in g^{-1}(g(x)) ]$\\
%
%Notice that for any $x$ such that $g(x) \neq y_0$, we have $f(x) = g(x)$ and $f^{-1}(f(x)) = g^{-1}(g(x))$.\\
%
%So $\mu(n) \leq \frac{p}{2^n} + \frac{1}{2^n}\sum\limits_{x \in \binset{n}, g(x) \neq y_0} Pr [ A(f(x)) \in f^{-1}(f(x)) ]$\\
%
%Thus, if we consider $A$ as an adversary for $f$, then we get:\\
%
%$\underset{x \overset{\$}{\leftarrow} \binset{n}}{Pr} [ A(f(x)) \in f^{-1}(f(x)) ] \; \geq \; \frac{1}{2^n}\sum\limits_{x \in \binset{n}, g(x) \neq y_0} Pr [ A(f(x)) \in f^{-1}(f(x)) ] \; \geq \; \mu(n) - \frac{p}{2^n}$\\
%
%$\mu(n)$ is non-negligible and $\frac{p}{2^n}$ is negligible, and so, $\mu(n) - \frac{p}{2^n}$ is non-negligible.  Thus $A$ is an adversary that breaks $f$ with non-negligible probability.  \qed
%

%\section{Pseudorandom Generators}
Now, we can define pseudorandom generators, which intuitively generates a polynomial number of bits that are indistinguishable from being uniformly random:
\begin{definition}
A function $G:\{0,1\}^n\rightarrow \{0,1\}^{n+m}$ with $m = poly(n)$ is called a \emph{pseudorandom generator} if
\begin{itemize}
\item $G$ is computable in polynomial time.
\item $U_{n+m}\approx G(U_n)$, where $U_k$ denotes the uniform distribution on $\{0,1\}^k$.
\end{itemize}
\end{definition}


\subsection{PRG Extension}
In this section we show that any pseudorandom generator that produces one bit of randomness can be extended to create a polynomial number of bits of randomness.

\begin{construction}
Given a PRG $G: \{0, 1\}^n \rightarrow \{0, 1\} ^ {n+1}$,
we construct a new PRG $F: \{0, 1\}^n \rightarrow \{0, 1\} ^{n+l}$ as follows ($l$ is polynomial in $n$).
\begin{enumerate}[label=(\alph*)]
    \item Input: $S_0 \xleftarrow{\$} \{0, 1\}^n$.
    \item $\forall i \in [l] = \{1, 2, \cdots, l\}$, $(\sigma_i, S_i) := G(S_{i-1})$, where $\sigma_i \in \{0, 1\}, S_i \in \{0, 1\}^n$ .
    \item Output: $\sigma_1 \sigma_2 \cdots \sigma_l S_l$.
\end{enumerate}
\end{construction}

\begin{theorem}
The function $F$ constructed above is a PRG.
\end{theorem}

\proof
We prove this by hybrid argument. Define the hybrid $H_i$ as follows.
\begin{enumerate}[label=(\alph*)]
	\item Input: $S_0 \xleftarrow{\$} \{0, 1\}^n$.
    \item $\sigma_1, \sigma_2, \cdots, \sigma_i \xleftarrow{\$} \{0, 1\}$, $S_i \gets S_0$.\\
     $\forall j \in \{i+1, i+2, \cdots, l\}$, $(\sigma_j, S_j) := G(S_{j-1})$, where $\sigma_j \in \{0, 1\}, S_j \in \{0, 1\}^n$ .
    \item Output: $\sigma_1 \sigma_2 \cdots \sigma_l S_l$.
\end{enumerate}
Note that $H_0 \equiv F$, and $H_l \equiv U_{n+l}$.

Assume for the sake of contradiction that there exits a non-uniform PPT adversary $\ma$ that can distinguish $H_0$ form $H_l$.
Define $\epsilon_i := \Pr[\ma(1^n, H_i)=1]$ for $i = 0, 1, \cdots, l$.
Then there exists a non-negligible function $v(n)$ such that $|\epsilon_0 - \epsilon_l| \geq v(n)$.
Since
\[
|\epsilon_0 - \epsilon_1| +
|\epsilon_1 - \epsilon_2| +
\cdots +
|\epsilon_{l-1} - \epsilon_l| \geq
|\epsilon_0 - \epsilon_l|
\geq v(n),
\]
there exists $k \in \{0, 1, \cdots, l-1\}$ such that
\[
|\epsilon_{k} - \epsilon_{k+1}| \geq \frac{v(n)}{l}.
\]
$l$ is polynomial in $n$, hence $\frac{v(n)}{l}$ is also a non-negligible function.
That is to say, $\ma$ can distinguish $H_{k}$ from $H_{k+1}$.
Then we use $\ma$ to construct an adversary $\mathcal{B}$ that can distinguish $U_{n+1}$ from $G(U_n)$ (which leads to a contradiction):
On input $T \in \{0, 1\}^{n+1}$ ($T$ could be either from $U_{n+1}$ or $G(U_n)$), $\mathcal{B}$ proceeds as follows:
\begin{itemize}
\item $\sigma_1, \sigma_2, \cdots, \sigma_k \xleftarrow{\$} \{0, 1\}$, $(\sigma_{k+1}, S_{k+1}) \gets T$.
\item $\forall j \in \{k+2, k+3, \cdots, l\}$, $(\sigma_j, S_j) := G(S_{j-1})$, where $\sigma_j \in \{0, 1\}, S_j \in \{0, 1\}^n$ .
\item Output: $\ma(1^n, \sigma_1 \sigma_2 \cdots \sigma_l S_l)$.
\end{itemize}

First, since $\ma$ and $G$ are both PPT computable, $\mathcal{B}$ is also PPT computable.

Second, if $T\gets G(U_n)$, then $\sigma_1 \sigma_2 \cdots \sigma_l S_l$ is the output of  $H_{k}$; if $T \stackrel{\$}\leftarrow U_{n+1}$, then $\sigma_1 \sigma_2 \cdots \sigma_l S_l$ is the output of $H_{k+1}$.
Hence
\begin{align*}
&\big|\Pr[\mathcal{B}(1^n, G(U_n)) = 1] - \Pr[\mathcal{B}(1^n, U_{n+1}) = 1]\big|\\
=& \big|\Pr[\ma(1^n,H_k) = 1] - \Pr[\ma(1^n,H_{k+1}) = 1]\big|\\
=&
|\epsilon_{k} - \epsilon_{k+1}| \geq \frac{v(n)}{l}.
\end{align*}
\qed

\subsection{PRG from OWP (One-Way Permutations)}
In this section we show how to construct pseudorandom generators under the assumption that one-way permutations exist.

\begin{construction}
Let $f: \{0, 1\}^n \rightarrow \{0, 1\}^n$ be a OWP. We construct $G: \{0, 1\}^{2n} \rightarrow \{0, 1\}^{2n+1}$ as
\[
G(x, r) = f(x) || r || B(x, r),
\]
where $x, r \in \{0, 1\}^n$, and $B(x, r)$ is a hard concentrate bit for the function $g(x,r) = f(x) || r$.
\end{construction}

\begin{remark}
The hard concentrate bit $B(x,r)$ always exists. Recall Theorem~\ref{thm:hard-concentrate-bit},
\[B(x,r) = \left(\sum_{i=1}^n x_i r_i\right)\mod 2\]
is a hard concentrate bit.
\end{remark}

\begin{theorem}
The $G$ constructed above is a PRG.
\end{theorem}

\proof
Assume for the sake of contradiction that $G$ is not PRG.
We construct three ensembles of probability distributions:
\[H_0 := G(U_{2n}) = f(x) || r || B(x, r), \text{ where } x, r \xleftarrow{\$} \{0, 1\}^n;\]
\[H_1 := f(x) || r || \sigma, \text{ where } x, r \xleftarrow{\$} \{0, 1\}^n, \sigma \xleftarrow{\$} \{0, 1\};\]
\[H_2 := U_{2n+1}.\]

Since $G$ is not PRG, there exists a non-uniform PPT adversary $\ma$ that can distinguish $H_0$ from $H_2$.
Since $f$ is a permutation, $H_1$ is uniformly distributed in $\{0, 1\}^{2n+1}$, i.e., $H_1 \equiv H_2$.
Therefore, $\ma$ can distinguish $H_0$ from $H_1$,
that is, there exists a non-negligible function $v(n)$ satisfying
\[
\big| \Pr[\ma(H_0)=1] - \Pr[\ma(H_1)=1] \big| \geq v(n).
\]

Next we will construct an adversary $\mathcal{B}$ that ``breaks'' the hard concentrate bit (which leads to a contradiction).
Define a new ensemble of probability distribution
\[
H_1' = f(x) || r || (1-B(x, r)) , \text{ where } x, r \xleftarrow{\$} \{0, 1\}^n.
\]
Then we have
\begin{align*}
\Pr[\ma(H_1) = 1]
=& \Pr[\sigma = B(x, r)] \Pr[A(H_0) = 1] + \Pr[\sigma = 1 - B(x, r)] \Pr[A(H_1') = 1]\\
=& \frac{1}{2} \Pr[A(H_0) = 1] + \frac{1}{2}\Pr[A(H_1') = 1].
\end{align*}
Hence
\begin{align*}
&\Pr[A(H_1) = 1] - \Pr[A(H_0) = 1]
=  \frac{1}{2}\Pr[A(H_1') = 1] - \frac{1}{2} \Pr[A(H_0) = 1],
\\
&\frac{1}{2} \left|\Pr[A(H_0) = 1] - \Pr[A(H_1') = 1] \right|
= \left| \Pr[A(H_1) = 1] - \Pr[A(H_0) = 1] \right|
\geq v(n),
\\
&\left|\Pr[A(H_0) = 1] - \Pr[A(H_1') = 1] \right|
\geq 2v(n).
\end{align*}

Without loss of generality, we assume that
\[
\Pr[A(H_0) = 1] - \Pr[A(H_1') = 1]
\geq 2v(n).
\]
Then we construct $\mathcal{B}$ as follows:
\[
\mathcal{B}(f(x)|| r) :=
\begin{cases}
\sigma, & \text{if } \ma(f(x)|| r||\sigma) = 1\\
1 - \sigma, & \text{if } \ma(f(x)||r|| \sigma) = 0
\end{cases},
\]
where $\sigma \xleftarrow{\$} \{0, 1\}$.
Then we have
\begin{align*}
& \Pr[\mathcal{B}(f(x)|| r) = B(x, r)]\\
=& \Pr[\sigma = B(x, r)] \Pr[ \ma(f(x)|| r||\sigma)=1 | \sigma = B(x, r)] + \\
& \Pr[\sigma = 1 - B(x, r)] \Pr[ \ma(f(x)|| r||\sigma) = 0 | \sigma = 1- B(x, r)] + \\
=& \frac{1}{2} \big( \Pr[\ma(f(x)||r||B(x, r)) = 1] + 1 - \Pr[\ma(f(x)|| r|| 1- B(x, r)) = 1] \big)\\
=& \frac{1}{2} + \frac{1}{2} \big( \Pr[A(H_0) = 1] - \Pr[A(H_1') = 1] \big)\\
\geq & \frac{1}{2} + v(n).
\end{align*}
Contradiction with the fact that $B$ is a hard concentrate bit.
\qed


%\section{Pseudorandom Functions}
In this section, we first define pseudorandom functions, and then show how to  construct a pseudorandom function from a pseudorandom generator.

Considering the set of all functions $f: \{0, 1\}^n \rightarrow \{0, 1\}^n$, there are $(2^n)^{2^n}$ of them.
To describe a random function in this set we need $n \cdot 2^n$ bits.
Intuitively, a pseudorandom function is one that cannot be distinguished from a random one,
but needs much fewer bits (e.g., polynomial in $n$) to be described.
Note that we restrict the distinguisher to only being allowed to ask the function poly($n$) times and decide whether it is random or pseudorandom.

\subsection{Definitions}

\begin{definition}[Function Ensemble]
A \emph{function ensemble} is a sequence of random variables $F_1, F_2, \cdots, F_n, \cdots$ denoted as $\{F_n\}_{n \in \mathbb{N}}$ such that
$F_n$ assumes values in the set of functions mapping $n$-bit input to $n$-bit output.
\end{definition}

\begin{definition}[Random Function Ensemble]
We denote a random function ensemble by $\{R_n\}_{n \in \mathbb{N}}$.
\end{definition}

\begin{definition}[Efficiently Computable Function Ensemble]
A function ensemble is called \emph{efficiently computable} if
\begin{enumerate}[label=(\alph*)]
    \item \textbf{Succinct}:
        $\exists$ a PPT algorithm $I$ and a mapping $\phi$ from strings to functions such that
        $\phi(I(1^n))$ and $F_n$ are identically distributed.
        Note that we can view $I$ as the description of the function.
    \item \textbf{Efficient}:
        $\exists$ a poly-time machine $V$ such that
        $V(i, x) = f_i(x)$ for every $x \in \{0, 1\}^n$, where $i$ is in the range of $I(1^n)$, and $f_i = \phi(i)$.
\end{enumerate}
\end{definition}

\begin{definition}[Pseudorandom Function Ensemble]
A function ensemble $F = \{F_n\}_{n \in \mathbb{N}}$ is \emph{pseudorandom} if
for every non-uniform PPT oracle adversary $\ma$, there exists a negligible function $\epsilon(n)$ such that
\[
\big| \Pr[\ma^{F_n} (1^n) = 1] - \Pr[\ma^{R_n} (1^n) = 1]  \big| \leq \epsilon(n).
\]
Here by saying ``oracle'' it means that $\ma$ has ``oracle access'' to a function (in our definition, the function is $F_n$ or $R_n$), and each call to that function costs 1 unit of time.
\end{definition}

Note that we will only consider efficiently computable pseudorandom ensembles in the following.

\subsection{Construction of PRF from PRG}

\begin{construction}
Given a PRG $G: \{0, 1\}^n \rightarrow \{0, 1\}^{2n}$,
let $G_0(x)$ be the first $n$ bits of $G(x)$, $G_1(x)$ be the last $n$ bits of $G(x)$.
We construct $F^{(K)}: \{0, 1\}^n \rightarrow \{0, 1\}^n$ as follows.
\[
F^{(K)}_n(x_1 x_2 \cdots x_n) := G_{x_n}(G_{x_{n-1}} (\cdots(G_{x_1}(K)) \cdots  )),
\]
where $K \in \{0,1\}^n$ is the key to the pseudorandom function. Here $i$ is an $n$-bit string, which is the seed of the pseudorandom function.
\end{construction}
The construction can be viewed as a binary tree of depth $n$, as shown in Figure \ref{fig:binary-tree}.

\begin{marginfigure}
    \centering
    \includegraphics[width=\textwidth]{Old Scribe Notes/binary-tree.pdf}
    \caption{View the construction as a binary tree}
    \label{fig:binary-tree}
\end{marginfigure}

\begin{theorem}\label{theorem:ggm}
The function ensemble $\{F_n\}_{n \in \mathbb{N}}$ constructed above is pseudorandom.
\end{theorem}

\proof
Assume for the sake of contradiction that $\{F_n\}_{n \in \mathbb{N}}$ is not PRG.
Then there exists a non-uniform PPT oracle adversary $\ma$ that can distinguish $\{F_n\}_{n \in \mathbb{N}}$ from $\{R_n\}_{n \in \mathbb{N}}$. Below, via a hybrid argument, we prove that this contradicts the fact that $G$ is a PRG.

Consider the sequence of hybrids $H_i$ for $i \in \{ 0, 1, \cdots, n\}$ where the hybrid $i$ is defined as follows:
\[H_{n,i}^{(K)} (x_1x_2\ldots x_n ):= G_{x_n}(G_{x_{n-1}} (\cdots(G_{x_{i+1}}(K(x_ix_{i-1}\ldots x_1))) \cdots  )), \]
where $K$ is a random function from $\{0,1\}^{i}$ to $\{0,1\}^n$. Intuitively, hybrid $H_i$ corresponds to a binary tree of depth $n$ where the nodes of levels $0$ to $i$ correspond to random values and the nodes at levels $i+1$ to $n$ correspond to pseudorandom values. By inspection, observe that hybrids $H_0$ and $H_n$ are identical to a pseudorandom function and a random function, respectively. There it suffices to prove that hybrids $H_i$ and $H_{i+1}$ are computationally indistinguishable for each $i \in \{ 0, 1, \cdots, n\}$.

We show that $H_{i}$ and $H_{i+1}$ are indistinguishable by considering a sequence of sub-hybrids $H_{i,j}$ for $j \in \{0,\ldots q_{i+1}\}$, where $q_{i+1}$ is the number of the distinct $i-bit$ prefixes of the queries of $\mathcal{A}$.\footnote{Observe that $q_{i+1}$ for each appropriate choice of $i$ is bounded by the running time of $\mathcal{A}$. Hence, this value is bounded by a polynomial in the security parameter.}

We define hybrid $H_{i,j}$ for $j =0$ to be same as hybrid $H_{i}$. Additionally, for $j >0$ hybrid $H_{i,j}$ is defined to be exactly the same as hybrid $H_{i,j-1}$ except the response provided to the attacker for the $j^{th}$ distinct $i-bit$ prefix query of $\mathcal{A}$. Let this prefix be $x^*_n x^*_{n-1} \ldots x^*_{i}$. Note that in hybrid $H_{i,j-1}$ the children of the node $x^*_n x^*_{n-1} \ldots x^*_{i}$ correspond to two pseudorandom values. In hybrid $H_{i,j}$ we replace these two children with random values. By careful inspection, it follows that hybrid $H_{i,q_{i+1}}$ is actually $H_{i+1}$. All we are left to prove is that hybrid $H_{i,j}$ and $H_{i,j+1}$ are indistinguishable for the appropriate choices of $j$ and we prove this below.


Now we are ready to construct an adversary $\mathcal{B}$ that  distinguishes $U_{2n}$ from $G(U_n)$: On input $T \in\{0, 1\}^{2n}$ ($T$ could be either from $U_{2n}$ or $G(U_n)$),
construct a full binary tree of depth $n$ that is exactly the same as $H_{i,j}$ except replacing the children of  $x^*_n x^*_{n-1} \ldots x^*_{i}$ by the value $T$.
Observe that the only difference between $H_{i,j}$ and $H_{i,j+1}$ is that values corresponding to nodes $x_n^*\ldots x_i^* 0$ and $x_n^*\ldots x_i^* 1$ are pseudorandom or random respectively. $\mathcal{B}$ uses the value $T$ to generate these two nodes. Hence success in  distinguishing hybrids $H_{i,j}$ and $H_{i,j+1}$ provides a successful attack for $\mathcal{B}$ in violating security of the pseudorandom generator.
\qed


%




\section{PRFs from DDH: Naor-Reingold PRF}
We will now describe a PRF function family $F_n: \mathcal{K} \times \{0,1\}^n \rightarrow \mathbb{G}_n$ where DDH is assumed to be hard for  $\{\mathbb{G}_n\}$ and $\mathcal{K}$ is the key space.
The seed for the PRF $F_n$ will be $K =  (h, u_1, \ldots u_n)$, where $u,u_0\ldots u_n$ are sampled uniformly from $|\mathbb{G}_n|$, $g$ is the generator of $\mathbb{G}_n$ and $h = g^u$. 

\[F_n(K,x) = h^{\prod_{i} u_i^{x_i}}\]

Next, we will prove that the function $F_n$ is a pseudo-random function or that $\{F_n\}$ is a pseudo-random function ensemble.\footnote{Here, we require that adversary distinguish the function $F_n$ from a random function from $\{0,1\}^n$ to $\mathbb{G}_n$. Note that the output range of the function is $\mathbb{G}_n$. Note that the distribution of random group elements in $\mathbb{G}_n$ might actually be far from uniformly random strings.}
\begin{lemma}
Assuming the DDH Assumption (see Definition~\ref{def:ddh}) for $\{\mathbb{G}_n\}$ is hard, we have that $\{F_n\}$ is a pseudorandom function ensemble.
\end{lemma}
\begin{proof}
The proof of this lemma is similar to the proof of Theorem~\ref{theorem:ggm}.

Let $R_n^j$ be random function from $\{0,1\}^j \rightarrow \mathbb{G}_n$. Then we want to prove that for all non-uniform PPT adversaries $\mathcal{A}$ we have that:
\[\mu(n) = \left|\Pr[\mathcal{A}^{F_n}(1^n) =1] -  \Pr[\mathcal{A}^{R_n^n}(1^n) =1]\right|\]
is a negligible function. 

For the sake of contradiction, we assume that the function $F_n$ is not pseudorandom. Next, towards a contradiction, we consider a sequence of hybrid functions $F_n^0 \ldots F_n^n$. 
For any $j \in \{0,\ldots n\}$, let $F^j_n((h,u_{j}\ldots u_n),x) = (R_n^j(x_1\ldots x_j))^{\prod_{i=j+1}^n u_i^{x_i}}$, where $R_n^0(\epsilon)$ is the constant function with output $h$. Observe that $F_n^0$ is the same as the function $F_n$ and $F_n^n$ is the same as the function $R_n^n$. Thus, by a hybrid argument, we conclude that there exists $k \in \{0,\ldots n-1\}$, such that 
\[\left|\Pr[\mathcal{A}^{F_n^k}(1^n) =1] -  \Pr[\mathcal{A}^{F_n^{k+1}}(1^n) =1]\right|\]
is a non-negligible function. Now all we are left to show is that this implies an attacker that refutes the DDH assumption. The proof of this claim follows by a sequence of $T$ sub-hybrids, where $T$ is the running time of $\mathcal{A}$. Without loss of generality we assume that $\mathcal{A}$ never makes the same query twice. 

More specifically, we consider a sequence of functions $F_n^{k,t}$ where $t \in \{0,T\}$, $F_n^{k,0}$ is same as $F_n^{k}$ and $F_n^{k,T}$ is same as $F_n^{k+1}$. In particular, we explain how $F_n^{k,t}$ answers queries by $\mathcal{A}$.\footnote{As assumed earlier, keep in mind that $\mathcal{A}$ never makes the same query twice.} Let $x^1, \ldots x^t$ be the first $t$ queries made by $\mathcal{A}$. For any query, $x$ made by $\mathcal{A}$ such that the first $k$ bits of $x$ match the first $k$ bits of one of $x_1, \ldots x_y$ answer as $F_n^{k+1}$ else answer as $F_n^{k}$. Now we can conclude that there exists a $t$ such that $F_n^{k,t}$ and $F_n^{k,t+1}$ are distinguishable with non-negligible probability. 

Finally, we will show that using an adversary that can distinguish between $F_n^{k,t}$ and $F_n^{k,t+1}$ we need to construct an adversary $\mathcal{B}$ that refutes the DDH assumption. We leave construction of this adversary as an exercise.
\end{proof}


\newpage
\section*{Exercises}
\begin{exercise}
\newcommand{\bit}{\{0,1\}}

Prove or disprove: If $f$ is a one-way function, then the following function $B:\bit^*\to\bit$ is a hardconcentrate predicate for $f$. The function $B(x)$ outputs the inner product modulo 2 of the first $\lfloor |x|/2\rfloor$ bits of $x$ and the last $\lfloor |x|/2\rfloor$ bits of $x$.
\end{exercise}

\begin{exercise}
Let $\phi(n)$ denote the first $n$ digits of $\pi = 3.141592653589\ldots$ after the decimal in binary ($\pi$ in its binary notation looks like $11.00100100001111110110101010001000100001\ldots$).

   Prove the following: if one-way functions exist, then there exists a one-way function $f$ such that the function $B:\{0,1\}^* \rightarrow \{0,1\}$ is not a hard concentrate bit of $f$. The function $B(x)$ outputs $\langle x, \phi(|x|)\rangle$, where
    \[\langle a, b\rangle := \sum_{i=1}^n a_i b_i \mod 2\]
    for the bit-representation of $a = {a_1a_2\cdots a_n}$ and $b= {b_1b_2\cdots b_n}$.
\end{exercise}

\begin{exercise}
 If $f: \{0,1\}^{n}\times \{0,1\}^n\rightarrow \{0,1\}^n$  is PRF, then in which of the following cases is $g: \{0,1\}^{n}\times \{0,1\}^n\rightarrow \{0,1\}^n$ also a PRF? \begin{enumerate} \item $g(K,x) = f(K,f(K,x))$ \item $g(K,x) = f(x,f(K,x))$ \item $g(K,x) = f(K,f(x,K))$
    \end{enumerate}
\end{exercise}

\begin{exercise}[Puncturable PRFs.] Puncturable PRFs are PRFs for which a key can be given out such that, it allows evaluation of the PRF on all inputs, except for one designated input.

\newcommand{\negl}{\mathsf{negl}}
\newcommand{\A}{\mathcal{A}}
\newcommand{\F}{F}
\newcommand{\KeyF}{\mathsf{Key}_{\F}}
\newcommand{\PunctureF}{\mathsf{Puncture}_{\F}}
\newcommand{\EvalF}{\mathsf{Eval}_{\F}}


A puncturable pseudo-random function $\F$ is given by a triple of efficient algorithms ($\KeyF$,$\PunctureF$, and $\EvalF$), satisfying the following conditions:
\begin{itemize}
\item[-] \textbf{Functionality preserved under puncturing}: For every $x^*, x \in \{0,1\}^{n}$ such that $x^* \neq x$, we have that:
    $$\Pr[\EvalF(K,x) = \EvalF(K_{x^*},x) : K \gets \KeyF(1^n), K_{x^*} = \PunctureF(K,x^*)] = 1$$
\item[-] \textbf{Pseudorandom at the punctured point}: For every $x^*\in \{0,1\}^n$ we have that for every polysize adversary $\A$ we have that:
    $$|\Pr[\A(K_{x^*}, \EvalF(K,x^*)) = 1] - \Pr[\A(K_{x^*}, \EvalF(K,U_n)) = 1]|= \negl(n)$$
    where $K \gets \KeyF(1^n)$ and $K_S = \PunctureF(K,x^*)$. $U_n$ denotes the uniform distribution over $n$ bits.
\end{itemize}

Prove that: If one-way functions exist, then there exists a puncturable PRF family that maps $n$ bits to $n$ bits. \\ 
\textbf{Hint:} The GGM tree-based construction of PRFs from a length doubling pseudorandom generator (discussed in class) can be adapted to construct a puncturable PRF. Also note that $K$ and $K_{x^*}$ need not be the same length.
\end{exercise}
%
%\subsection{Application}
%Consider an interesting game: Alice and Bob are talking on the phone.
%Alice flips a coin, and Bob guesses whether it's head or tail.
%But the problem is how can Alice convince Bob that the coin is indeed head or tail?
%If we have pseudorandom functions, the problem could be easily solved.
%
%Assume we have a PRF $F_n: \{0, 1\}^n \rightarrow \{0, 1\}^n$.
%Alice and Bob have a shared key $i \in \{0, 1\}^n$, then $f_i(\cdot)$ is shared information.
%Now Alice has a message $m \in \{0, 1\}^n$ and wants to let Bob guess it,
%the procedure consists of three steps.
%\begin{enumerate}[(a)]
%    \item Alice chooses a string $r \in \{0, 1\}^n$, and sends to Bob  $m' = f_i(r) \oplus m$ ;
%    \item Bob guesses $m$;
%    \item Alice sends $r$ to Bob.
%\end{enumerate}
%In step (a), since $F_n$ is PRF, all the information that Bob gets is a random $n$-bit string, so it will not influence his behavior in step (b).
%Then in step (c), Bob receives $r$ and will be convinced that the true value of $m$ is $f_i(r) \oplus m'$.
%
%\newcommand{\nonnegl}{\mathsf{nonnegl}}


\section{Fixed-length MACs}

Previously, we defined what a MAC is, and specified correctness and security definitions for MACs. In this section, we'll define a fixed-length MAC for length $\ell(n)$.

\begin{theorem}
    If $F : \{0, 1\}^n \to \{0, 1\}^n$ is a PRF, then $\Pi = (\mathsf{Gen}, \mathsf{Mac}, \mathsf{Verify})$ (constructed below) is a MAC, as defined with EF-CMA security.

    \begin{algorithmic}[1]
        \Function{Gen}{$1^n$}
            \State \Return $k \in \{0, 1\}^n$
        \EndFunction
        \Statex
        \Function{Mac}{$k$, $m$}
            \State \Return $F_k(m)$
        \EndFunction
        \Statex
        \Function{Verify}{$k$, $m$, $t$}
            \State \Return $t \overset{?}{=} F_k(m)$
        \EndFunction
    \end{algorithmic}

    That is, we just compute the PRF on our message as the MAC.
\end{theorem}

\begin{proof}
    To prove security, suppose for contradiction that there exists an adversary $A$ that breaks the security for $\Pi$. We'd like to construct an adversary $B$ that breaks the security of the PRF.

    Here, the adversary $A$ expects queries for tags, given messages as input. $B$ can simply forward these requests on to $F$, and return the response back to $A$. Further, $A$ outputs a pair $(m^*, t^*)$, which $B$ can send $m^*$ to $F$, and output whether $t = t^*$.

    \begin{center}
        \begin{tikzpicture}
            \draw (3, 0) rectangle (8, 5.5);
            \draw (5, 0.5) rectangle (7.5, 5);
            \node at (3.25, 5.25) {$B$};
            \node at (5.25, 4.75) {$A$};

            \draw (5.5, 4) edge[->] node[right, pos=0] {$m$} (2.5, 4)
                (2.5, 3.5) edge[->] node[right, pos=1] {$t$} (5.5, 3.5);
            \node[left] at (2.5, 3.75) {$F_k$};
            \node at (5.5, 3.1) {$\vdots$};
            \node at (4.5, 3.1) {$\vdots$};
            \draw (5.5, 2.5) edge[->] node[right, pos=0] {$(m^*, t^*)$} (4.5, 2.5);

            \draw (3.5, 2.5) edge[->] node[right, pos=0] {$m^*$} (2.5, 2.5);
            \node[left] at (2.5, 2.25) {$F_k$};
            \draw (2.5, 2) edge[->] node[right, pos=1] {$t$} (3.5, 2);
            \draw (3.5, 1) edge[->] node[right, pos=0] {$t \overset{?}{=} t^*$} (2.5, 1);
        \end{tikzpicture}
    \end{center}

    Analyzing the probability for $B$, we have
    \[
        \abs{\Pr(B^{F_k(\cdot)}(1^n) = 1) - \Pr(B^{R_n(\cdot)}(1^n) = 1)}
        = \abs{\varepsilon_A(n) - \frac{1}{2^n}}
        = \nonnegl(n)
    .\]
    Here, the first term is because the correctness follows immediately from the correctness of $A$, and the second term is due to the fact that the output of $R_n$ is random.
\end{proof}

\section{Variable-length MACs}

Now, let us look at messages with lengths that are a multiple of $n$. In particular, we have a few blocks $m_1, \ldots, m_{\ell}$, each of size $n$. There are a few ways to do this, but we'll look at a method similar to the counter mode we looked at last time.

\begin{center}
    \begin{tikzpicture}
        \node (m1) at (0, 2) {$m_1$};
        \node (m2) at (1, 2) {$m_2$};
        \node (m3) at (2, 2) {$m_3$};
        \node (mdots) at (3, 2) {$\cdots$};
        \node (ml) at (4, 2) {$m_{\ell}$};

        \node[draw] (fk1) at (0, 0) {$F_k$};
        \node[draw] (fk2) at (1, 0) {$F_k$};
        \node[draw] (fk3) at (2, 0) {$F_k$};
        \node at (3, 0) {$\ldots$};
        \node[draw] (fkl) at (4, 0) {$F_k$};

        \node[outer sep=0pt, inner sep=0pt, draw, circle] (fk1+m2) at (1, 1) {$+$};
        \node[outer sep=0pt, inner sep=0pt, draw, circle] (fk2+m3) at (2, 1) {$+$};
        \node[outer sep=0pt, inner sep=1pt] (xor-dots) at (3, 1) {$\ldots$};
        \node[outer sep=0pt, inner sep=0pt, draw, circle] (dots+ml) at (4, 1) {$+$};

        \draw (m1) edge[->] (fk1)
            (m2) edge[->] (fk1+m2)
            (m3) edge[->] (fk2+m3)
            (ml) edge[->] (dots+ml);

        \draw[->] (fk1) -- ++(0.5, 0) |- (fk1+m2);
        \draw[->] (fk2) -- ++(0.5, 0) |- (fk2+m3);
        \draw[->] (fk3) -- ++(0.5, 0) |- (xor-dots);

        \draw[->] (xor-dots) -- (dots+ml);
        \draw (fk1+m2) edge[->] (fk2)
            (fk2+m3) edge[->] (fk3)
            (dots+ml) edge[->] (fkl);
        \draw (fkl) edge[->] ++(1, 0);
    \end{tikzpicture}
\end{center}

This construction avoids having to store a tag equal in length to the message, but this is not secure, due to length extension attacks. In particular, suppose we query for the tag $t$ associated with $0^n$. We can then query another tag $t'$ for $0^n \oplus t$. Observe here that $t'$ is also the tag for $0^{2n}$.

A solution is to use different keys for each PRF, but this isn't too efficient, since we're still calling the PRF once per block of length $n$. We'll instead improve this to use only one block cipher call---we do some preprocessing and only call $F_k$ once on the output of the preprocessing.

In particular, we'll claim that applying a universal hash function to the input and then applying the block cipher is a secure MAC.

\begin{definition}[Universal Hash Function]
    A function $h : \mathcal{F} \times \mathcal{F}^* \to \mathcal{F}$ (where $\mathcal{F}$ is a field of size $2^m$) is a universal hash function if for all $m, m' \in \mathcal{F}^{\le \ell}$ (i.e. $m$ and $m'$ have length at most $\ell$),
    \[
        \Pr_s(h(s, m) = h(s, m')) \le \frac{\ell}{\abs{F}}
    .\]
    That is, the probability of collision is small.
\end{definition}

Crucially here, we fix $m$ and $m'$, and we sample $s$. (If we fix an $s$, we can almost surely find an $m$ and $m'$ that collide.)

Today, we'll look at the following function:
\[
    h(s, m_0, \ldots, m_{\ell - 1}) = m_0 + m_1 s + m_2 s^2 + \cdots + m_{\ell - 1} s^{\ell - 1} + s^{\ell}
.\]

\begin{claim}
    The function defined by
    \[
        h(s, m_0, \ldots, m_{\ell - 1}) = m_0 + m_1 s + m_2 s^2 + \cdots + m_{\ell - 1} s^{\ell - 1} + s^{\ell}
    \]
    is a universal hash function.
\end{claim}

\begin{proof}
    We'd like to argue that for a fixed $m$ and $m'$, and a random $s$, the probability that there is a collision is at most $\frac{\ell}{\abs{\mathcal{F}}}$.

    We'll look at
    \[
        h(x, m_0, \ldots, m_t) - h(x, m_0', \ldots, m_t') = (m_0 - m_0') + \cdots + (m_{t - 1} - m_{t-1}') x^{\ell - 1}
    .\]
    If there is a collision, this difference is 0. The probability that this polynomial of degree at most $\ell$ has a zero at $x$ is at most $\frac{\ell}{\abs{\mathcal{F}}}$, since it has at most $\ell$ zeroes. This means that $h$ is indeed a universal hash function.
\end{proof}

\begin{claim}
    The MAC given by $F_k(h(s, m_1, \ldots, m_{\ell}))$, for the universal hash function $h$ given prior, is secure. (This is a slight variation on the Carter--Wegman MAC.)
\end{claim}

\begin{proof}
    Suppose for contradiction that there exists a nu-PPT $A$ that breaks the security of this scheme.

    Here, for appropriately generated $k$ and $s$, $A$ makes queries $m \mapsto F_k(h_s(m))$, and outputs $(m^*, t^*)$.

    We'd like to create an adversary $B$ that either breaks the security of the PRF, or breaks the security of the universal hash function.

    $B$ will start by sampling $s \in \mathcal{F}$. When given the query for $m_1$, it computes $h_s(m_1)$ and queries for $F_k(h_s(m_1))$, which it sends back to $A$. If $F_k$ was actually pseudorandom, then $A$ is given a pseudorandom input, and if $F_k$ was random $R_n$, then $A$ is given a random input.

    $A$ must still be able to generate pairs $(m^*, t^*)$ even when given a random input, due to the security of the PRF.

    \begin{center}
        \begin{tikzpicture}
            \draw (2, 0) rectangle (8, 5.5);
            \draw (5, 0.5) rectangle (7.5, 5);
            \node at (2.25, 5.25) {$B$};
            \node at (5.25, 4.75) {$A$};

            \node at (3.5, 4.5) {sample $s \in \mathcal{F}$};

            \node (hsm) at (3.5, 4) {$h_s(m)$};
            \draw (5.5, 4) edge[->] node[right, pos=0] {$m$} (hsm)
                (hsm) edge[->] (1.5, 4)
                (1.5, 3.5) edge[->] node[right, pos=1] {$t$} (5.5, 3.5);
            \node[left] at (1.5, 3.75) {$F_k$};
            \node at (5.5, 3.1) {$\vdots$};
            \node at (4.5, 3.1) {$\vdots$};

            \node (hash-m-star) at (3.5, 2.5) {$h_s(m^*)$};
            \draw (5.5, 2.5) edge[->] node[right, pos=0] {$(m^*, t^*)$} (hash-m-star)
                (hash-m-star) edge[->] (1.5, 2.5);
            \node[left] at (1.5, 2.25) {$F_k$};
            \draw (1.5, 2) edge[->] node[right, pos=1] {$t$} (3.5, 2);
            \draw (3.5, 1) edge[->] node[right, pos=0] {$t \overset{?}{=} t^*$} (1.5, 1);
        \end{tikzpicture}
    \end{center}

    Let $E$ be the event that there exists an $m, m' \in L \cup \{m^*\}$, such that $h_s(m) = h(m')$. If $E$ does not happen, then the hash function never collides. This means that the attacker only sees random values depending on distinct inputs, so this reduces to the case from earlier (when the MAC is just $F_k$).

    As such, we'd like to show that collisions in $h_s(\cdot)$ occur with negligible probability.

    To show this, suppose for contradiction that collisions actually do occur with non-negligible probability. We then want to construct an adversary $B$ utilizing $A$ that just outputs $m$ and $m'$ such that when $s$ is sampled, $h_s(m) = h_s(m')$ with high probability.

    $B$ will pick a random $i, j \in \{1, \ldots, q+1\}$ (here suppose $i < j$), where $q$ is the number of MAC queries. We then run $A$ until the $j$th query. Taking the $i$th and $j$th query, we then output $m_i$ and $m_j$ as our pair of messages. We still need to entertain the queries made by $A$, so we can just return random values for tags (giving the same value if it requests it for the same message).

    \begin{center}
        \begin{tikzpicture}
            \draw (2, 0) rectangle (8, 6);
            \draw (5, 0.5) rectangle (7.5, 5.5);
            \node at (2.25, 5.75) {$B$};
            \node at (5.25, 5.25) {$A$};

            \node[align=center] at (3.5, 5) {\small $i, j \xleftarrow{\$} \{1, \ldots, q+1\}$};

            \draw (5.5, 4) edge[->] node[right, pos=0] {$m_1$} (4.5, 4)
                (4.5, 3.5) edge[->] node[left, pos=0] {sample $t_1$} node[right, pos=1] {$t_1$} (5.5, 3.5);
            \node at (5.5, 3.1) {$\vdots$};
            \node at (4.5, 3.1) {$\vdots$};

            \draw (5.5, 2.5) edge[->] node[right, pos=0] {$m_i$} (4.5, 2.5)
                (4.5, 2) edge[->] node[left, pos=0] {sample $t_i$} node[right, pos=1] {$t_i$} (5.5, 2);
            \node at (5.5, 1.6) {$\vdots$};
            \node at (4.5, 1.6) {$\vdots$};

            \draw (5.5, 1) edge[->] node[right, pos=0] {$m_j$} (4.5, 1);

            \draw (3, 0.5) edge[->] node[right, pos=0] {$(m_i, m_j)$} (1.5, 0.5);
        \end{tikzpicture}
    \end{center}

    By assumption, we know that $E$ occurs with non-negligible probability. That is, among the queries made by $A$, there is a non-negligible probability that $h_s(m_i) = h_s(m_j)$. Since here the implementation of $B$ just picks out a pair of random queries from those made by $A$, the pair $(m_i, m_j)$ output by $B$ also has a collision with non-negligible probability. (In particular, with probability $\Pr(E) / q^2$.

    This breaks the definition of a universal hash function, which is a contradiction.
\end{proof}

So far, we know how to generate tags of fixed length, and of lengths that are a multiple of $n$. If we have a message that is not a multiple of $n$, we could potentially just pad the input with 0's, but this causes an issue, as $m$ and $m \concat 0$ have the same tag.

Instead, one solution is to put the size of the message in the first block, and we can still put the padding at the end. This way, if the messages differ by length, the first block will be different, and if the messages do not differ by length, then we're essentially just ignoring the padding. This gives us a MAC for arbitrary-length messages.

\section{Authenticated Encryption Schemes}

We've talked about confidentiality and integrity separately, but generally we want both properties---when Alice sends a message to Bob, we'd like for any eavesdropper to be unable to recover the message, \emph{and} we'd like Bob to be able to verify that the message actually came from Alice.

A scheme that achieves both of these conditions is called an \emph{authenticated encryption scheme}.

\begin{definition}[Authenticated Encryption Scheme]
    A scheme $\Pi$ is an \emph{authenticated encryption scheme} if it is CPA-secure, and it has ciphertext integrity (CI).
\end{definition}

\begin{definition}[Ciphertext Integrity (CI)]
    Consider the following game for the scheme $\Pi = (\mathsf{Gen}, \mathsf{Enc}, \mathsf{Dec})$.

    \begin{algorithmic}[1]
        \Function{CI${}_{\Pi}^A$}{$n$}
            \State $k \gets \mathsf{Gen}(1^n)$
            \State $c^* \gets A^{\mathsf{Enc}(k, \cdot)}(1^n)$
            \State $L \gets$ the list of queries made by $A$
            \State \Return $(\mathsf{Dec}(k, c^*) \ne \bot) \land (c^* \notin L)$
        \EndFunction
    \end{algorithmic}

    A scheme has ciphertext integrity if for all nu-PPT $A$, $\Pr(\mathrm{CI}_{\Pi}^A)$ is negligible.
\end{definition}

Observe that an authenticated encryption scheme is also CCA-secure, since the CI property says that the adversary can never generate a valid ciphertext. This means that whenever an adversary requests the decryption of a ciphertext, we can always return $\bot$ (unless they previously requested a ciphertext for a message, and wants to decode that ciphertext). This means that the decryption oracle is essentially useless, and this reduces to the CPA case.

Next, we'll construct an authenticated encryption scheme, called ``Encrypt-then-MAC'', utilizing a CPA-secure encryption scheme and an EF-CMA MAC scheme.

\begin{claim}
    Let $\Pi_e = (\mathsf{Gen}_e, \mathsf{Enc}_e, \mathsf{Dec}_e)$ be a CPA-secure encryption scheme, and let $\Pi_m = (\mathsf{Gen}_m, \mathsf{Mac}_m, \mathsf{Verify}_m)$ be an EF-CMA-secure MAC scheme.

    The following scheme $\Pi = (\mathsf{Gen}, \mathsf{Enc}, \mathsf{Dec})$ is an authenticated encryption scheme.

    \begin{algorithmic}[1]
        \Function{Gen}{$1^n$}
            \State $k_e \gets \mathsf{Gen}_e(1^n)$
            \State $k_m \gets \mathsf{Gen}_m(1^n)$
            \State \Return $(k_e, k_m)$
        \EndFunction
        \Statex
        \Function{Enc}{$(k_e, k_m), m$}
            \State $c \gets \mathsf{Enc}_e(k_e, m)$
            \State $t \gets \mathsf{Mac}_m(k_m, c)$
            \State \Return $(c, t)$
        \EndFunction
        \Statex
        \Function{Dec}{$(k_e, k_m), (c, t), m$}
            \If {$\mathsf{Verify}_m(k_m, c, t)$}
                \State \Return $\mathsf{Dec}_e(k_e, c)$
            \Else
                \State \Return $\bot$
            \EndIf 
        \EndFunction
    \end{algorithmic}
\end{claim}

\begin{proof}
    Suppose for contradiction that we have an adversary $A$ that breaks the CPA security of $\Pi$. The CPA game allows for queries of the ciphertext for messages $m$, produces a pair $m_0, m_1$, and then gets $c^* = \mathsf{Enc}(k, m_B)$, and $A$ eventually outputs $b'$ to identify which message was encrypted.

    We'd like to construct another adversary $B$, which breaks the CPA-security of $\Pi_e$. The only difference here is the MACs, so $B$ can sample a $k_m \gets \mathsf{Gen}_m(1^n)$, and perform all of the MACs itself.

    In particular, when $A$ asks for the ciphertext of $M$, we pass it to the oracle for $\Pi_e$, and attach $t \gets \mathsf{Mac}_m(k_m, c)$. If $A$ is able to distinguish between ciphertexts of $M_0$ and $M_1$, then we can use the same bit to distinguish between ciphertexts for $\Pi_e$.
    \begin{center}
        \begin{tikzpicture}
            \draw (1.5, -1) rectangle (8, 6);
            \draw (5, -0.5) rectangle (7.5, 5.5);
            \node at (1.75, 5.75) {$B$};
            \node at (5.25, 5.25) {$A$};

            \node at (3.25, 5) {$k_m \gets \textsc{Gen}_m(1^n)$};

            \node (enc) at (3.25, 3.5) {$C = \textsc{Enc}_e(k_e, m)$};
            \node (mac) at (3.25, 3) {$(C, \textsc{Mac}_m(k_m, C))$};

            \draw (5.5, 4) edge[->] node[right, pos=0] {$m$} (1, 4)
                (1, 3.5) edge[->] (enc)
                (mac) edge[->] node[right, pos=1] {$c$} (5.5, 3);
            \node[left] at (1, 3.75) {$\textsc{Enc}_e(k_e, \cdot)$};
            \node at (5.5, 2.6) {$\vdots$};
            \node at (4.5, 2.6) {$\vdots$};

            \draw (5.5, 2) edge[->] node[right, pos=0] {$m_1, m_2$} node[left, pos=1] {$m_1, m_2$} (1, 2);
            \node (enc-mb) at (3.25, 1.5) {$C^*$};
            \node (mac-mb) at (3.25, 1) {$(C^*, \textsc{Mac}_m(k_m, C^*))$};
            \draw (1, 1.5) edge[->] node[left, pos=0] {$\textsc{Enc}_e(k_e, m_b)$} (enc-mb);
            \draw (mac-mb) edge[->] node[right, pos=1] {$c^*$} (5.5, 1);

            \node at (5.5, 0.5) {$\vdots$};
            \node at (4.5, 0.5) {$\vdots$};

            \draw (5.5, 0) edge[->] node[right, pos=0] {$b'$} (1, 0);
        \end{tikzpicture}
    \end{center}

    To prove ciphertext integrity, suppose we have an adversary $A$ that breaks the ciphertext integrity of $\Pi$. Here, $A$ asks for ciphertext queries, and eventually returns a new ciphertext that is valid.

    We'd like to construct an adversary $B$ that is able to generate a new message and a tag, given oracle access to the MAC scheme. The construction will follow similarly to the prior proof on CPA security.

    Here, our adversary $B$ can sample $k_e \gets \mathsf{Gen}_e(1^n)$. When $A$ asks for the encryption of $M$, $B$ can send $m = \mathsf{Enc}_e(k_e, M)$ to the MAC oracle, and it returns $c = (m, t)$ to $A$.

    When $A$ returns $C^* = (c^*, t^*)$, $B$ can also just return the same, since the tag $t^*$ is being computed on $c^*$.

    \begin{center}
        \begin{tikzpicture}
            \draw (1.5, 1) rectangle (8, 6);
            \draw (5, 1.5) rectangle (7.5, 5.5);
            \node at (1.75, 5.75) {$B$};
            \node at (5.25, 5.25) {$A$};

            \node at (3.25, 5) {$k_e \gets \textsc{Gen}_e(1^n)$};

            \node (enc) at (3.25, 4) {$C = \textsc{Enc}_e(k_e, m)$};
            \node (mac) at (3.25, 3.5) {$T = \textsc{Mac}_m(k_m, C)$};
            \node (enc-mac) at (3.25, 3) {$(C, T)$};

            \draw (5.5, 4) edge[->] node[right, pos=0] {$m$} (enc)
                (enc) edge[->] (1, 4)
                (1, 3.5) edge[->] (mac)
                (enc-mac) edge[->] node[right, pos=1] {$c$} (5.5, 3);
            \node[left] at (1, 3.75) {$\textsc{Mac}_m(k_m, \cdot)$};
            \node at (5.5, 2.6) {$\vdots$};
            \node at (4.5, 2.6) {$\vdots$};

            \node (gen-ciphertext) at (3.25, 2) {$(C^*, T^*)$};
            \draw (5.5, 2) edge[->] node[right, pos=0] {$c^*$} (gen-ciphertext)
                (gen-ciphertext) edge[->] (1, 2);
        \end{tikzpicture}
    \end{center}
\end{proof}

As an example, AES-GCM is the most popular authenticated encryption scheme that is used, and also has the ability to authenticate additional data. (AES-GCM basically just appends the associated data to the ciphertext, so that the encryption is only on the message, but the MAC is on both the ciphertext and the associated data.) This scheme uses a counter-mode encryption scheme, and the MAC that we saw, but makes this more efficient.


%% !TEX root = collection.tex

\chapter{Digital Signatures}

In this chapter, we will introduce the notion of a digital signature. At an intuitive level, a digital signature scheme helps providing authenticity of messages and ensuring non-repudiation. We will first define this primitive and then construct what is called as one-time secure digital signature scheme. An one-time digital signature satisfies a weaker security property when compared to digital signatures. We then introduce the concept of collision-resistant hash functions and then use this along with a one-time secure digital signature to give a construction of digital signature scheme.

\section{Definition}

A digital signature scheme is a tuple of three algorithms $(\Gen,\Sign,\Verify)$ with the following syntax:
\begin{enumerate}
\item $\Gen(1^n)\to (vk,sk)$: On input the message length (in unary) $1^n$, $\Gen$ outputs a secret signing key $sk$ and a public verification key $vk$.
\item $\Sign(sk, m) \to \sigma$: On input a secret key $sk$ and a message $m$ of length $n$, the $\Sign$ algorithm outputs a signature $\sigma$.
\item $\Verify(vk, m, \sigma) \to \{0,1\}$: On input the verification key $vk$, a message $m$ and a signature $\sigma$, the $\Verify$ algorithm outputs either $0$ or $1$.
\end{enumerate}

We require that the digital signature to satisfy the following correctness and security properties.\\
\medskip
\noindent\textbf{Correctness.} For the correctness of the scheme, we have that
$\forall m \in \bin^n$,
\[\Pr \left[ (vk,sk) \gets \Gen(1^n), \sigma \leftarrow \Sign(sk,m) : \Verify(vk, m, \sigma) = 1 \right] = 1.\]

\medskip
\noindent\textbf{Security.} Consider the following game between an adversary and a challenger
.

\begin{enumerate}
    \item The challenger first samples $(vk,sk) \gets \Gen(1^n)$. The challenger gives $vk$ to the adversary.
    \item \textbf{Signing Oracle.} The adversary is now given access to a signing oracle. When the adversary gives a query $m$ to the oracle, it gets back $\sigma \gets \Sign(sk,m)$.
    \item \textbf{Forgery.} The adversary outputs a message, signature pair $(m^*,\sigma^*)$ where $m^*$ is different from the queries that adversary has made to the signing oracle.
    \item The adversary wins the game if $\Verify(vk,m^*,\sigma^*) = 1$.
\end{enumerate}
We say that the digital signature scheme is secure if the probability that the adversary wins the game is $\negl(n)$.

\section{One-time Digital Signature}
\label{lampart}
An one-time digital signature has the same syntax and correctness requirement as that of a digital signature scheme except that in the security game the adversary is allowed to call the signing oracle only once (hence the name one-time). We will now give a construction of one-time signature scheme from the assumption that one-way functions exists.

Let $f: \bin^n \rightarrow \bin^n$ be a one-way function.
\begin{itemize}
\item $\Gen(1^n)$: On input the message length (in unary) $1^n$, $\Gen$ does the following:
\begin{enumerate}
    \item Chooses $x_{i,b} \gets \bin^n$ for each $i \in [n]$ and $b \in \bin$.
    \item Output $vk = \left[ \begin{array}{ccc}
f(x_{1,0}) & \ldots & f(x_{n,0}) \\
f(x_{1,1}) & \ldots & f(x_{n,1}) \\
\end{array} \right]$ and $sk = \left[ \begin{array}{ccc}
x_{1,0} & \ldots & x_{n,0} \\
x_{1,1} & \ldots & x_{n,1} \\
\end{array} \right]$
\end{enumerate}
\item $\Sign(sk, m)$: On input a secret key $sk$ and a message $m \in \bin^n$, the $\Sign$ algorithm outputs a signature $\sigma = x_{1,m_1}\|x_{2,m_2}\| \ldots \| x_{n,m_n}$.
\item $\Verify(vk, m, \sigma)$: On input the verification key $vk$, a message $m$ and a signature $\sigma$, the $\Verify$ algorithm does the following:
\begin{enumerate}
    \item Parse $\sigma = x_{1,m_1}\|x_{2,m_2}\| \ldots \| x_{n,m_n}$.
    \item Compute $vk'_{i,m_i} = f(x_{i,m_i})$ for each $i \in [n]$.
    \item Check if for each $i \in [n]$, $vk'_{i,m_i} = vk_{i,m_i}$. If all the checks pass, output 1. Else, output 0.
\end{enumerate}
\end{itemize}

Before we prove any security property, we first observe that this scheme is completely broken if we allow the adversary to ask for two signatures. This is because the adversary can query for the signatures on $0^n$ and $1^n$ respectively and the adversary gets the entire secret key. The adversary can then use this secret key to sign on any message and break the security. 

We will now argue the one-time security of this construction. Let $\adv$ be an adversary who breaks the security of our one-time digital signature scheme with non-negligible probability $\mu(n)$. We will now construct an adversary $\advb$ that breaks the one-wayness of $f$. $\advb$ receives a one-way function challenge $y$ and does the following:
\begin{enumerate}
    \item $\advb$ chooses $i^*$ uniformly at random from $[n]$ and $b^*$ uniformly at random from $\bin$.
    \item It sets $vk_{i^*,b^*} = y$
    \item For all $i \in [n]$ and $b \in \bin$ such that $(i,b) \neq (i^*,b^*)$, $\advb$ samples $x_{i,b} \gets \bin^n$. It computes $vk_{i,b} = f(x_{i,b})$.
    \item It sets $vk = \left[ \begin{array}{ccc}
vk_{1,0} & \ldots& vk_{n,0} \\
vk_{1,1} & \ldots& vk_{n,1} \\
\end{array} \right]$ and sends $vk$ to $\adv$.
\item $\adv$ now asks for a signing query on a message $m$. If $m_{i^*} = b^*$ then $\advb$ aborts and outputs a special symbol $\abort_1$. Otherwise, it uses it knowledge of $x_{i,b}$ for $(i,b) \neq (i^*,b^*)$ to output a signature on $m$.
\item $\adv$ outputs a valid forgery $(m^*,\sigma^*)$. If $m^*_{i^*} = m_{i^*}$ then $\advb$ aborts and outputs a special symbol $\abort_2$. If it does not abort, then it parses $\sigma^*$ as ${1,m_1}\|x_{2,m_2}\| \ldots \| x_{n,m_n}$ and outputs $x_{i^*,b^*}$ as the inverse of $y$.
\end{enumerate}
We first note that conditioned on $\advb$ not outputting $\abort_1$ or $\abort_2$, the probability that $\advb$ outputs a valid preimage of $y$ is $\mu(n)$. Now, probability $\advb$ does not output $\abort_1$ or $\abort_2$ is $1/2n$ (this is because $\abort_1$ is not output with probability $1/2$ and conditioned on not outputting $\abort_1$, $\abort_2$ is not output with probability $1/n$). Thus, $\advb$ outputs a valid preimage with probability $\mu(n)/2n$. This completes the proof of security.

We now try to extend this one-time signature scheme to digital signatures.

%
\section{Collision Resistant Hash Functions}

As the name suggests, collision resistant hash function family is a set of hash functions $H$ such that for a function $h$ chosen randomly from the family, it is computationally hard to find two different inputs $x,x'$ such that $h(x) = h(x')$. We now give a formal definition.

\subsection{Definition of a family of CRHF}

A set of function ensembles
\[ \{H_n = \{h_i : D_n \to R_n \}_{i \in I_n} \}_n\]
where $|D_n| < |R_n|$ is a family of collision resistant hash function ensemble if there exists efficient algorithms $(\Sampler,\Eval)$ with the following syntax:
\begin{enumerate}
\item $\Sampler(1^n) \to i:$ On input $1^n$, $\Sampler$ outputs an index $i \in I_n$.
\item $\Eval(i,x) = h_i(x):$ On input $i$ and $x \in D_n$, $\Eval$ algorithm outputs $h_i(x)$. 
\item $\forall$ PPT $\adv$ we have
\[\Pr[i \gets \Sampler(1^n), (x,x') \gets \adv(1^n,i) : h_i(x) = h_i(x') \wedge x \neq x'] \leq \negl(n)\]
\end{enumerate}


\subsection{Collision Resistant Hash functions from Discrete Log}
We will now give a construction of collision resistant hash functions from the discrete log assumption. We first recall the discrete log assumption:
\begin{definition}[Discrete-Log Assumption]
We say that the discrete-log assumption holds for the group ensemble $\mathcal{G} =\{ \mathbb{G}_n\}_{n \in \mathbb{N}}$, if for every non-uniform PPT algorithm $\mathcal{A}$ we have that
\[\mu_\mathcal{A}(n) := \Pr_{x \leftarrow |G_n|}[\mathcal{A}(g,g^x) = x]\]
is a negligible function.
\end{definition}

We now give a construction of collision resistant hash functions.  

\begin{itemize}
\item $\Sampler(1^n):$ On input $1^n$, the sampler does the following:
\begin{enumerate}
    \item It chooses $x \gets |\mathbb{G}_n|$.
    \item It computes $h = g^x$.
    \item It outputs $(g,h)$.

\end{enumerate}
\item $\Eval((g,h),(r,s)):$ On input $(g,h)$ and two elements $(r,s) \in |\mathbb{G}_n|$, $\Eval$ outputs $g^rh^s$.
\end{itemize}

We now argue that this construction is collision resistant. Assume for the sake of contradiction that an adversary gives a collision $(r_1,s_1) \neq (r_2,s_2)$. We will now use this to compute the discrete logarithm of $h$. We first observe that:
\begin{eqnarray*}
r_1+xs_1 &=& r_2 + xs_2\\
(r_1 - r_2) &=& x(s_2 - s_1)
\end{eqnarray*}
We infer that $s_2 \neq s_1$. Otherwise, we get that $r_1 = r_2$ and hence, $(r_1,s_1) = (r_2,s_2)$. Thus, we can compute $x = \frac{r_1-r_2}{s_1 - s_2}$ and hence the discrete logarithm of $h$ is computable.


\section{Multiple-Message Digital Signature}

We now explain how to combine collision-resistant hash functions and one-time signatures to get a signature scheme for multiple messages. We first construct an intermediate primitive wherein we will still have the same security property as that of one-time signature but we would be able to sign messages longer than the length of the public-key.\footnote{Note that in the one-time signature scheme that we constructed earlier, the length of message that can be signed is same as the length of the public-key.}


\subsection{One-time Signature Scheme for Long Messages}
We first observe that the CRHF family $H$ that we constructed earlier compresses $2n$ bits to $n$ bits (also called as 2-1 CRHF). We will now give an extension that compresses an arbitrary long string to $n$ bits using a 2-1 CRHF.
\paragraph{Merkle-Damgard CRHF.} The sampler for this CRHF is same as that of 2-1 CRHF. Let $h$ be the sampled hash function. To hash a string $x$, we do the following. Let $x$ be a string of length $m$ where $m$ is an arbitrary polynomial in $n$. We will assume that $m = kn$ (for some $k$) or otherwise, we can pad $x$ to this length. We will partition  the string $x$ into $k$ blocks of length $n$ each. For simplicity, we will assume that $k$ is a perfect power of $2$ or we will again pad $x$ appropriately. We will view these $k$-blocks as the leaves of a complete binary tree of depth $\ell = \log_2 k$. Each intermediate node is associated with a bit string $y$ of length at most $\ell$ and the root is associated with the empty string. We will assign a $\tag \in \bin^n$ to each node in the tree. The $i$-th leaf is assigned $\tag_i$ equal to the $i$-block of the string $x$. Each intermediate node $y$ is assigned a $\tag_y = h(\tag_{y\|0}\| \tag_{y \| 1})$. The output of the hash function is set to be the $\tag$ value of the root. Notice that if there is a collision for this CRHF then there are exists one intermediate node $y$ such that for two different values $\tag_{y\|0},\tag_{y\|1}$ and $\tag'_{y\|0},\tag'_{y\|1}$ we have, $h(\tag_{y\|0},\tag_{y\|1}) = \tag'_{y\|0},\tag'_{y\|1}$. This implies that there is a collision for $h$. 

\paragraph{Construction.} We will now use the Merkle-Damgard CRHF and the one-time signature scheme that we constructed earlier to get a one-time signature scheme for signing longer messages. The main idea is simple: we will sample a $(sk,vk)$ for signing $n$-bit messages and to sign a longer message, we will first hash it using the Merkle-Damgard hash function to $n$-bits and then sign on the hash value. The security of the construction follows directly from the security of the one-time signature scheme since the CRHF is collision-resistant. 

\subsection{Signature Scheme for Multiple Messages}
We will now describe the construction of signature scheme for multiple messages. Let $(\Gen',\Sign',\Verify')$ be a one-time signature scheme for signing longer messages. 
\begin{enumerate}
    \item $\Gen(1^n):$ Run $\Gen'(1^n)$ using to obtain $sk,vk$. Sample a PRF key $K$. The signing key is $(sk,K)$ and the verification key is $vk$.
    \item $\Sign((sk,K),m):$ To sign a message $m$, do the following:
    \begin{enumerate}
        \item Parse $m$ as $m_1m_2\ldots m_{\ell}$ where each $m_i \in \bin$.
        \item Set $sk_0 = sk$ and $m_0 = \epsilon$ (where $\epsilon$ is the empty string).
        \item For each $i \in [\ell]$ do:
        \begin{enumerate}
            \item Evaluate $\PRF(m_1\|\ldots\|m_{i-1}\|0)$ and $\PRF(m_1\|\ldots\|m_{i-1}\|1)$ to obtain $r_0$ and $r_1$ respectively. Run $\Gen'(1^n)$ using $r_0$ and $r_1$ as the randomness to obtain $(sk_{i,0},vk_{i,1})$ and $(sk_{i,1},vk_{i,1})$.
            \item Set $\sigma_i = \Sign(sk_{i-1,m_{i-1}},vk_{i,0}\|vk_{i,1})$
            \item If $i = \ell$, then set $\sigma_{\ell+1} = \Sign(sk_{i,m_i},m)$.
            
        \end{enumerate}
        \item Output $\sigma = (\sigma_1,\ldots,\sigma_{\ell+1})$ along with all the verification keys as the signature.
    \end{enumerate}
    \item $\Verify(vk,\sigma,m)$: Check if all the signatures in $\sigma$ are valid.
\end{enumerate}

To prove security, we will first use the security of the PRF to replace the outputs with random strings. We will then use the security of the one-time signature scheme to argue that the adversary cannot mount an existential forgery.

\section*{Exercises}
\begin{exercise}
\textbf{Digital signature schemes can be made deterministic.} Given a digital signature scheme $(\mathsf{Gen}, \mathsf{Sign}, \mathsf{Verify})$ for which $\mathsf{Sign}$ is probabilistic, provide a construction of a digital signature scheme $(\mathsf{Gen}', \mathsf{Sign}', \mathsf{Verify}')$ where $\mathsf{Sign}'$ is deterministic.
\end{exercise}

%\section{Random Oracle Model}

We looked at RSA-FDH in the last section, and in this section we'll continue on and provide some semblance of a security analysis of the scheme.

As a note, collision resistance of the hash function isn't quite enough for the security of the RSA-FDH scheme. In particular, if we can find three messages $m_1, m_2, m_3$ such that $H(m_1) \cdot H(m_2) = H(m_3) \pmod{N}$ (this isn't protected against with collision resistance), then we can break the scheme, assuming that we use the RSA trapdoor function. Here, we'd have
\begin{align*}
    <<<<<<< HEAD
    \sigma_1 \sigma_2 & = f^{-1}(H(m_1)) \cdot f^{-1}(H(m_2)) \\
                      & = H(m_1)^d H(m_2)^d \pmod{N}          \\
                      & = (H(m_1) H(m_2))^d \pmod{N}          \\
                      & = H(m_3)^d \pmod{N}
    =======
    \sigma_1 \sigma_2 & = f^{-1}(H(m_1)) \cdot f^{-1}(H(m_2)) \\
                      & = H(m_1)^d H(m_2)^d \pmod{N}          \\
                      & = (H(m_1) H(m_2))^d \pmod{N}          \\
                      & = H(m_3)^d \pmod{N}
    >>>>>>> 6918b01 (added lec 11)
\end{align*}
Ideally, we'd like to have a proof of the security of this scheme, but nobody has been able to come up with one yet. Instead, we can only hope to find some kind of \emph{evidence} for the security of the scheme.

This evidence comes from the \emph{random oracle model} (ROM), otherwise known as the \emph{random oracle methodology}.

Suppose we're given a scheme $\Pi^H = (A^H, B^H, C^H, \ldots)$, where calls to the hash function $H$ is explicit. (Some functions may not call the hash function, but that's okay.)

We'd like to perform some analysis on these schemes, even though we may not fully understand the properties of the hash function---we'd like to abstract it out. To do this, we instead prove the security of $\Pi^O = (A^O, B^O, C^O, \ldots)$, where the hash function is replaced with an oracle $O$ for a truly random function.

This oracle assumption is a very strong one, and is perhaps not the most indicative of the security of the original scheme---there are cases where the scheme $\Pi^O$ under an oracle $O$ is secure, but replacing the oracle with \emph{any} instantiation breaks the security of the scheme.

When we're trying to prove security of $\Pi^O$, we'll look at an adversary $\mathcal{A}^O$, which has access to $O$. Here, observe that we can provide the answers to the oracle queries---we just need to find a contradiction to the existence of the function $\mathcal{A}$, regardless of what the oracle $O$ does.

Note here that the adversary $\mathcal{A}$ in this case is forced to explicitly call the oracle for its hash function queries---the fact that we can see these calls is called \emph{observability}. In the standard model, we can't actually see the queries that the adversary makes, since it just runs the predefined hash function itself.

Another property is called \emph{programmability}: since we're working with a random oracle, the only thing that matters is that the output of the oracle looks uniformly random. This means that we can replace a uniform output $x$ of the oracle with $f(x)$, for some one-way permutation $f$. This allows us to control some secret parameter that affects the output distribution of the oracle $O$. In the standard model, we don't have programmability---we again just have a fixed hash function that we can't change after the fact.

\begin{theorem}
    RSA-FDH is EUF-CMA secure in the ROM, assuming $\{f_s\}_s$ is a secure family of trapdoor permutations.
\end{theorem}

\begin{proof}
    Suppose we have an adversary $\mathcal{A}$ in this model.

    The first thing it is given is a public key $\mathrm{pk} = s$. The adversary then gets to make signature queries: $m \mapsto f_t^{-1}(O(m))$. At the very end, it must output a forged signature $(m^*, \sigma^*)$. This adversary is also allowed to make separate hash queries to the random oracle: $m \mapsto O(m)$. %TODO: draw the oracle calls at the top or the right side of the box

    \begin{center}
        \begin{tikzpicture}
            \draw (0, 0) rectangle (3, 4);
            \node at (0.25, 3.75) {$\mathcal{A}$};

            \draw (-0.5, 3) edge[->] node[pos=0, left] {$\mathrm{pk} = s$} (0.5, 3);

            \draw (-0.5, 2) edge[<-] node[pos=1, right] {$m$} (0.5, 2);
            \draw (-0.5, 1.5) edge[->] node[pos=1, right] {$f_t^{-1}(O(m))$} (0.5, 1.5);
            \node[anchor=east] at (-0.6, 1.75) {Signature queries};

            \draw (-0.5, 0.5) edge[<-] node[pos=1, right] {$(m^*, \sigma^*)$} (0.5, 0.5);

            \draw (2.5, 3) edge[->] node[pos=0, left] {$m$} (3.5, 3);
            \draw (2.5, 2.5) edge[<-] node[pos=0, left] {$O(m)$} (3.5, 2.5);
            \node[anchor=west] at (3.6, 2.75) {Hash queries to $O$};
        \end{tikzpicture}
    \end{center}

    WLOG, suppose that for every message $m$ that $\mathcal{A}^O$ queries for a signature, it has already made a query for the same message to the hashing oracle. (Otherwise, we can simply make a wrapper around $\mathcal{A}^O$ that does this.) We can also assume WLOG that when the adversary outputs $(m^*, \sigma^*)$, it has also made the hashing query $O(m^*)$. Let's call this hybrid $H_0$.

    For the hybrid $H_1$, we'll abort the machine if for any $m, m'$ in the hash queries, we have $O(m) = O(m')$, essentially removing all collisions from the oracle. This happens with negligible probability ($q^2 / 2^n$), so this hybrid is still indistinguishable from $H_0$.

    Next, we'll construct an adversary $\mathcal{B}$ using $\mathcal{A}$, and inverts the trapdoor permutation. In particular, given $(s, y^*)$, where $y^* = f^{-1}(x^*)$, the goal is to output $x^*$.

    Suppose $\mathcal{A}$ makes $q_s$ signing queries and $q_h$ hashing queries.

    We pass in $s$ as $\mathrm{pk}$. $\mathcal{B}$ first samples an $i^* \gets \{1, \ldots, q_h\}$. We then set the output of the $i$th hash query to $y^*$. In particular, we have $O(m_{q_{i^*}}) = y^*$. If the adversary happens to call a signing query on $i^*$, we'll abort.

    We still need to specify what happens on all other queries, and we want to make sure that we can respond with a signature query on all of these other queries. For $i \ne i^*$, we sample $x \in D_s$, and compute $y = f_s(x)$. On the $i$th hashing query, we then set $O(m_i) = y$. If the adversary later requests a signature on the same $m_i$, then we output $x$ for the signing query. (This is because $f^{-1}(O(m_i)) = f^{-1}(y) = x$.)

    In particular, the adversary must have called the hashing query for its output $m^*$, and with some probability, this is the $i^*$th query, in which case the message $m^*$ is our inverse $x^*$.

    Analyzing the probabilities, we have that
    \begin{align*}
        \Pr(\text{$\mathcal{B}$ outputs $f^{-1}(x^*)$})
         & = \Pr(\text{$\mathcal{A}$ successful} \land \text{no sign query on $m_{i^*}$} \land m^* = m_{q_i^*}) \\
         & = \varepsilon \times \left(1 - \frac{1}{q_h}\right)^{q_s} \times \frac{1}{q_h}                       \\
         & \approx \frac{\varepsilon}{q_h}
    \end{align*}
    which is non-negligible, assuming $\mathcal{A}$ is successful with non-negligible probability.
\end{proof}

We'll now talk about a different scheme and analyze its security under the random oracle model.

This scheme is called the \emph{Schnorr signature scheme}. Given a group $G$ of prime order $q$ and a hash function $H : \{0, 1\}^* \to \mathbb{Z}_q$, we define
\begin{itemize}
    \item $\textsc{Gen}(1^n) = (\mathrm{pk} = g^x, \mathrm{sk} = x \gets \mathbb{Z}_q)$
    \item $\textsc{Sign}(\mathrm{sk}, m)$:
          \begin{algorithmic}
              \State $k \gets \mathbb{Z}_q$
              \State $r = g^k$
              \State $h = H(m \concat r)$
              \State $s = k + hx$
              \State $\sigma = (h, s)$
          \end{algorithmic}

    \item $\textsc{Verify}(\mathrm{pk}, m, \sigma)$:
          \begin{algorithmic}
              \State output $h \overset{?}{=} H(m \concat \frac{g^s}{\mathrm{pk}^h})$
          \end{algorithmic}
\end{itemize}

\begin{theorem}
    The Schnorr signature scheme is EUF-CMA secure in the ROM, assuming the discrete log problem is hard.
\end{theorem}

\begin{proof}
    The adversary $\mathcal{A}^O$ gets a public key $\mathrm{pk} = g^x$, can make signing queries $m \mapsto \textsc{Sign}(x, m)$ and hashing queries $(m, z) \mapsto O(m \concat z)$, for $z \in G$. $\mathcal{A}$ then returns a forgery $(m^*, \sigma^*)$.
    \begin{center}
        \begin{tikzpicture}
            \draw (0, 0) rectangle (3, 4);
            \node at (0.25, 3.75) {$\mathcal{A}$};

            \draw (-0.5, 3) edge[->] node[pos=0, left] {$\mathrm{pk} = g^x$} (0.5, 3);

            \draw (-0.5, 2) edge[<-] node[pos=1, right] {$m$} (0.5, 2);
            \draw (-0.5, 1.5) edge[->] node[pos=1, right] {$\sigma = \textsc{Sign}(x, m)$} (0.5, 1.5);
            \node[anchor=east] at (-0.6, 1.75) {Signature queries};

            \draw (-0.5, 0.5) edge[<-] node[pos=1, right] {$(m^*, \sigma^*)$} (0.5, 0.5);

            \draw (2.5, 3) edge[->] node[pos=0, left] {$(m, z)$} (3.5, 3);
            \draw (2.5, 2.5) edge[<-] node[pos=0, left] {$O(m \concat z)$} (3.5, 2.5);
            \node[anchor=west] at (3.6, 2.75) {Hash queries to $O$};
        \end{tikzpicture}
    \end{center}

    WLOG, we can assume that $m^* \concat r^*$ is in the list of hash queries (where $r^*$ was the value computed in the output signature $\sigma^*$).

    We'll define a modified signing algorithm as follows:
    \begin{algorithmic}
        \Function{Sign'}{$\mathrm{sk}, m$}
        \State $h, s \in \mathbb{Z}_q$ uniformly
        \State $g^k \gets \frac{g^s}{g^{hx}} = \frac{g^s}{\mathrm{pk}^h}$
        \State $h = H(m \concat g^k)$
        \State output $(h, s)$
        \EndFunction
    \end{algorithmic}
    The main idea here is to provide a random signature, consistent with the definition of $\textsc{Sign}$, so that \textsc{Verify} will still succeed.

    We'll then define a wrapper $\mathcal{A}'^{O}$, which performs these modified signing queries by itself, since it no longer requires the secret key $x$. As such, $\mathcal{A}'^O$ only makes hashing queries, and produces $(m^*, \sigma^*)$. In particular, $\mathcal{A}'$ depends on $\mathrm{pk}, q_1, h_1, \ldots, q_H, h_H$, but all of the queries $q_1, \ldots, q_H$ are deterministic depending on the previous hash output (or dependent on pk in the case of $q_1$). This means that $\mathcal{A}'$ can actually be thought of as a function of
    \[
        \mathcal{A}'(\mathrm{pk}, h_1, h_2, \ldots, h_H)
        .\]
    The main insight that we'll use is that we can run $\mathcal{A}'$ until the $(i^* - 1)$th query, and on the $i^*$th query, we run the adversary twice, on two different possible responses: $h_{i^*}$ and $h'_{i^*}$. These two executions share the first $i^* - 1$ hashing queries, and both are perfectly valid executions of the adversary. We'll use these two executions to break the discrete log problem.

    Let us define $\mathcal{B}$ that breaks the discrete log problem, given as input $(g, g^x)$. Here, we'll let $g^x$ be the public key.

    In response to hashing queries, if $\mathcal{A}'$ asks for the hash of $m \concat z$, we respond with a random value (or the same value as before if queried multiple times) as $O(m \concat z)$.

    Now, we'd like to be able to find $x$, utilizing the behavior of $\mathcal{A}'$.
    At the $i^*$th query, we run the adversary twice, with $h_{i^*}$ as the hash in the first execution, and $h'_{i^*}$ as the hash in the second execution. In the first execution, we would have gotten queries $q_{i^*}, q_{i^*+1}, \ldots, q_h$, and outputted $(m^*, \sigma^*)$. In the second execution we would have gotten queries $q_{i^*}', q_{i^*+1}', \ldots, q_h'$, and outputted $(m'^*, \sigma'^*)$.

    Now, in the $i^*$th query, note that $s = k + hx$ in the first execution, and $s' = k + h' x$ in the second execution. Crucially, the value of $k$ is the same here, since the query utilizes the same value of $r$, and we can solve for $x = \frac{s - s'}{h - h'}$.

    The probability that the adversary $\mathcal{A}'$ succeeds in producing a forgery while utilizing $i^*$ is $\mu(n) = \varepsilon(n) / q_h$. For ease, let us also define the two halves of the input to $\mathcal{A}'$ as $\alpha = (\mathrm{pk}, h_1, \ldots, h_{i^* - 1})$ and $\beta = (h_{i^*}, \ldots, h_{q_h})$. We then define the ``good set'' as
    \[
        S = \left\{\alpha \mid \Pr_{\beta}(\mathcal{A}'(\alpha, \beta)\ \text{outputs a forgery}) \ge \frac{\mu(n)}{2}\right\}
        .\]
    We can also see that $\Pr(\alpha \in S) \ge \frac{\mu(n)}{2}$; to see why, suppose by contradiction $\Pr(\alpha \in S) < \frac{\mu(n)}{2}$. Here, we have
    \begin{align*}
        \Pr(\text{$\mathcal{A}'$ succeeds}) & = \Pr(\text{$\mathcal{A}'$ succeeds} \mid \alpha \in S) \Pr(\alpha \in S) + \Pr(\text{$\mathcal{A}'$ succeeds} \mid x \notin S) \Pr(\alpha \notin S) \\
                                            & < 1 \cdot \frac{\mu(n)}{2} + \frac{\mu(n)}{2} \cdot 1 < \mu(n)
    \end{align*}
    which is a contradiction.

    The probability that $\mathcal{B}$ succeeds is thus
    \[
        \Pr(\alpha \in S) \Pr(\text{$\mathcal{A}'(\alpha, \beta)$ succeeds} \mid \alpha \in S) \Pr(\text{$\mathcal{A}'(\alpha, \beta')$ succeeds} \mid \alpha \in S)
        \ge \left(\frac{\mu(n)}{2}\right)^3
        ,\]
    due to the definition of $S$ from earlier.

    This means that in total, our probability of success is
    \[
        \Pr(\text{$\mathcal{B}$ succeeds}) \ge \frac{\varepsilon^3(n)}{8q_h^3}
        ,\]
    which is non-negligible if $\mathcal{A}'$ succeeds with non-negligible probability, giving us our contradiction.
\end{proof}

%% !TEX root = collection.tex



\chapter{Public Key Encryption}
Public key encryption deals with a setting where there are two parties who wish
to communicate a secret message from one of them to the other. Unlike the symmetric
setting, in which the two parties share a secret key, the public-key setting has
asymmetry in who can decrypt a given message. This allows one party to
announce the public key to everyone so messages can be encrypted, but keep the
secret key private so no one else can preform decryption.

\begin{definition}[Public Key Encryption]

A public key crypto-system consists of three algorithms: $\gen$, $\enc$,
and $\dec$ with properties as follows:

\begin{enumerate}
\item $\gen(1^k)$ outputs a pair of keys $(\pk, \sk)$; the public and private
keys respectively.

\item $\enc(\pk, m)$ encrypts a message $m$ under
public key $\pk$.

\item $\dec(\sk, c)$ decrypts a ciphertext $c$
under secret key $\sk$.
\end{enumerate}
\end{definition}

There are other properties about these algorithms which we will discuss next
in order for these algorithms to be useful. The first of these, correctness,
ensures that the decryption of an encrypted message returns the original plaintext.
The second is the security property, which says that an attacker with access
to the encrypted message learns nothing about the plaintext.

Note that the $\gen$ algorithm must be a randomized algorithm. If not, it would
always output the same public-private key pair, and would not be very useful. We will
later show that $\enc$ must also be a randomized algorithm, or else the security
properties will not hold. Finally, $\dec$ may be a randomized algorithm but is not
required to be one.


\section{Correctness}

In order for the encryption and decryption to satisfy our intuition of what these
algorithms should do, we require that decrypting an encrypted value with the
correct keys yields the original message.

\begin{definition}[Correctness]
An public key algorithm $(\gen, \enc, \dec)$ is \emph{correct} if
\begin{equation*}
\forall m . \Pr[\dec(\sk, \enc(\pk, m))=m | (\pk,\sk) \gets \gen(1^k)] = 1
\end{equation*}
\end{definition}

Some definitions may relax this constraint from being equal to one to being greater
than $1-neg(\cdot)$. However, since this probability is equal to one, we can restate
the definition

\begin{lemma}[Correctness]
An public key algorithm $(\gen, \enc, \dec)$ is \emph{correct} if
\begin{equation*}
\forall m, \pk,\sk . (\pk,\sk) \gets \gen(1^k) \implies \dec(\sk, \enc(\pk, m))=m.
\end{equation*}
\end{lemma}

There is no statement involved about what happens when encrypting or decrypting
under the wrong key, nor is there anything about decrypting a malformed ciphertext.
It only requires that decryption of an encrypted message under corresponding
keys produces the original message.

\section{Indistinguishability and Semantic Security}

Not only must public key encryption be correct, we would also like it to hide
the values which are encrypted. There are two different definitions which we
will show are identical.

\subsection{Indistinguishability Security}
Our first definition, indistinguishability, states that the ciphertexts
obtained from encrypting any message must look identical to those from encrypting
any other message. In particular this implies that encryption must be a randomized
algorithm.

\begin{definition}[Indistinguishability]
A public key encryption scheme satisfies the indistinguishability property if
the distributions
$A_{m_1}$ and $A_{m_2}$ are computationally indistinguishable for all $m_1, m_2 \in M$
such that $|m_1| = |m_2|$ where
\begin{align*}
A_{m_1} = \{\pk, \enc(\pk, m_1) : (\pk, \sk) \gets \gen(1^k)\}_k \\
A_{m_2} = \{\pk, \enc(\pk, m_2) : (\pk, \sk) \gets \gen(1^k)\}_k
\end{align*}
\end{definition}

It is required that we talk about computational indistinguishability. The encryption
of a message must reveal at least something in an information-theoretic setting.
It is also required that the sizes of the two messages be equal. It would be very
easy to tell the difference between the encryption of a one-bit message and an
arbitrarily large message by just comparing their sizes. It is possible to work around
this requirement by having an encryption scheme pad messages so that they are all
of equal size if an upper bound is known.

\subsection{Semantic Security}
An alternate definition, which we will later prove is identical, is semantic security.
This definition intuitively states that given a ciphertext, you should learn nothing about
the original message other than it's length.

\begin{definition}[Semantic Security]
An encryption scheme is semantically secure if there exists
a simulator $S$ such that the two following processes generate computationally
indistinguishable outputs.

\begin{tabular}{cccc}
\parbox[t]{2.5cm}{\noindent \\}
&
\parbox[t]{5.8cm}{
1. $(m,z) \gets M(1^k)$ \\
2. $(\pk, \sk) \gets \gen(1^k)$ \\
3. $\text{Output\,} A(\pk, \enc(\pk, m), z)$
}
&
\parbox[t]{1cm}{
\noindent\\
$\approx$
}
&
\parbox[t]{4.5cm}{
1. $(m,z) \gets M(1^k)$ \\
2. $\text{Output\,} S^A(1^k, z)$
}
\end{tabular}
\\

Where $M$ is a machine that randomly samples message from the message space and
arbitrary additional information and $A$ is the adversary.

\end{definition}


\subsection{Equivalence of Definitions}

\begin{theorem}[Equivalence of Definitions]
A public key encryption scheme $(gen, enc, dec)$ is semantically secure if and only if
it satisfies the indistinguishability property.
\end{theorem}

\proof
We begin by proving that semantic security implies indistinguishability.
That is, we need to show that for every PPT adversary $A$, $A_{m_1}$ and $A_{m_2}$ are computationally indistinguishable for every pair of $m_1, m_2 \in M$.
We prove by a hybrid argument.
In hybrid $H_0$ let $M$ always output the value $m_1$, feed into $A$ and output $A(\pk, \enc(\pk, m_1))$.
In hybrid $H_1$ replace the output by $S^A(1^k)$.
In hybrid $H_2$ let $M$ always output the value $m_2$, feed into $A$ and output $A(\pk, \enc(\pk, m_2))$.
By semantic security the output of these hybrids are computationally indistinguishable, hence $A$ cannot distinguish $A_{m_1}$ and $A_{m_2}$.



Now we prove the other direction.
The simulator generates a public/secret key pair $(\pk, \sk) \gets \gen(1^k)$, encrypts the message $0^{|m|}$ and feeds into $A$.
It outputs $A(\pk, \enc(\pk, 0^{|m|}), z)$.
By the indistinguishability property, $A$ cannot distinguish $(\pk, \enc(\pk, 0^{|m|}))$ from $(\pk, \enc(\pk, m))$, hence the outputs of $A$ and $S$ are computationally indistinguishable.
 \qed



\section{Public Key Encryption from Trap-Door OWP}

We now show how it is possible to create a public key encryption scheme from a trapdoor
one way permutation.

\begin{theorem}
Let $(G,F,A)$ be a trap-door one-way permutation The following is then a public-key
encryption scheme:
\begin{enumerate}
\item $\gen(1^k) = (i,t_i) \gets G(1^k)$.
\item $\enc(\pk, m) = (F_i(x) || r, B(x, r) \oplus m)$ where $(x,r)$ are sampled uniformly, and $B(\cdot)$ is a hard-core bit.
\item $\dec(\sk, c) = B(A(i,t_i, y), r) \oplus c_0$ where $c = (y||r,c_0)$.
\end{enumerate}

\end{theorem}

\proof
Clearly the encryption scheme is correct, since $\alpha \oplus \alpha \oplus m = m$ for any $\alpha$.

Now we  prove  that the scheme is indistinguishable by showing that $\enc(\pk, 0)$ and $\enc(\pk, 1)$
are computationally indistinguishable. Note that distinguishing the two distributions is by definition equivalent to guessing the hard-core bit.
Hence they are indistinguishable.
\qed


\section{Indistinguishability in a Chosen Plaintext Attack}

We now define a new security requirement, IND-CPA, which is stronger than indistinguishability
or semantic security. Consider the following scheme:

The challenger samples $(\pk, \sk) \gets \gen(1^k)$  and gives $\pk$ to the attacker.
The attacker then replis with two messages $(m_0, m_1)$. The challenger then picks
one of these uniformly at random, encrypts it generating $c = \enc(\pk, m_b)$ and sends
it to the attacker. The attacker must then guess which message was encrypted.

Here, the attacker gets to choose two messages which he likes, ask the challenger to
encrypt the two messages, and only must be able to distinguish between the two.

Then a scheme is IND-CPA if $\forall A . \Pr[\text{Experiment}^{A(1^k)}] < 1/2+neg(k)$ where
A is a PPT machine. That is,
an attacker which can ask the user to encrypt specific messages after learning the
public keys can still not learn anything about the values of the encrypted messages. In
particular, there can not be any easy-to-identify weak messages for a given public key.

\begin{definition}
A public key encryption scheme (PKE) given by the three efficient procedures $(G,E,D)$ is
IND-CPA-secure if no adversary $A$ has a significant advantage in the game
represented in Table~\ref{tab:cpa}.
\begin{table}[ht]
\centering
\begin{tabular}{r c l}
Challenger & & Adversary \\
%\mright{(\pk,\sk)\gets G(1^k)}{\pk}{}
%\mleft{}{m'_0,m'_1}{}
%$b \xleftarrow{\$} \{0,1\}$ & & \\
%\mright{c^*=E(\pk,m'_b)}{c^*}{}
%\mleft{}{b'}{}
Output 1 if $(b'=b)$ & & \\
\end{tabular}
\caption{CPA security.}\label{tab:cpa}
\end{table}

This means that every probabilistic polynomial-time (PPT) adversary $A$ has only a
negligible advantage (over a random guess)
when guessing which message ($m'_0$ or $m'_1$) the challenger used to
obtain $c^*$.

Note that, since the adversary has the public key $\pk$, he is able to encrypt any
polynomial number of plaintexts during the game.
\end{definition}


\begin{theorem}
IND-CPA is a stronger definition than semantic security.
\end{theorem}

\proof
First, observe that IND-CPA is at least as strong as semantic security. An
attack which showed an encryption scheme is not semantically secure can be used
identically to distinguish the two messages the attacker sends in the IND-CPA protocol.

Now we only need to show that it is stronger.
In the following we construct an encryption scheme which is semantically secure, but is not secure
under IND-CPA.
Assume we have
a scheme $(G, E, D)$ which is semantically secure. Now we will construct a new one
$(G', E', D')$ which  is still semantically secure, but is definitely not IND-CPA.
\begin{align*}
& G'(1^k) = ((\pk, x_0, x_1), \sk), \,\text{where}\, (\pk,\sk) \gets G(1^k), x_0, x_1 \gets M(1^k) \\
& E'((\pk, x_0, x_1), m) = \text{if } m \not\in\{x_0,x_1\}, \,\text{then}\, (\mathsf{y},E(\pk,m)) \,\text{else}\, (\mathsf{n},m) \\
& D'(\sk, c) = \text{if } c \text{ starts with } \mathsf{y}, \text{then}\, D(\sk,E(\pk,m)) \,\text{else}\, m 
\end{align*}

This scheme is still semantically secure. $E'$ is different from $E$ on only two
inputs, which is certainly negligible in $k$. However, this
scheme is not IND-CPA. After seeing the public key pair $(\pk, x_0, x_1)$, the attacker
knows exactly to pick values $x_0$ and $x_1$ will be able to distinguish between
those two with probability $1$. \qed



%%%%%%%%%%%%%%%%%%%%%%%%%%%%%%%%%%%%%%%%%%%%%%%%%%%%%%%%%%%%%%%%%%%%%%%%%%%%%
\section{Chosen Ciphertext Attack for Public Key Encryption}


\begin{definition}
A public key encryption scheme (PKE) given by the three efficient procedures $(G,E,D)$ is
IND-CCA1-secure if no adversary $A$ has a significant advantage in the game
represented in Table~\ref{tab:cca1}.
\begin{table}[ht]
\centering
\begin{tabular}{r c l}
Challenger & & Adversary \\
%\mright{(\pk,\sk)\gets G(1^k)}{\pk}{}
%\mleft{}{c_1}{}
%\mright{m_1=D(\sk,c_1)}{m_1}{}
% & \vdots & \\
%\mleft{}{c_q}{}
%\mright{m_q=D(\sk,c_q)}{m_q}{}
%\mleft{}{m'_0,m'_1}{}
%$b \xleftarrow{\$} \{0,1\}$ & & \\
%\mright{c^*=E(\pk,m'_b)}{c^*}{}
%\mleft{}{b'}{}
Output 1 if $(b'=b)$ & & \\
\end{tabular}
\caption{Non-adaptive CCA security.}\label{tab:cca1}
\end{table}

In this definition the adversary may send some polynomial number of queries
to be decrypted before he receives the challenge ciphertext $c^*$.
\end{definition}



\begin{definition}
A public key encryption scheme (PKE) given by the three efficient procedures $(G,E,D)$ is
IND-CCA2-secure if no adversary $A$ has a significant advantage in the game
represented in Table~\ref{tab:cca2}.
\begin{table}[t!]
\centering
\begin{tabular}{r c l}
Challenger & & Adversary \\
%\mright{(\pk,\sk)=G(1^k)}{\pk}{}
%\mleft{}{c_1}{}
%\mright{m_1=D(\sk,c_1)}{m_1}{}
% & \vdots & \\
%\mleft{}{c_q}{}
%\mright{m_q=D(\sk,c_q)}{m_q}{}
%\mleft{}{m'_0,m'_1}{}
%$b \xleftarrow{\$} \{0,1\}$ & & \\
%\mright{c^*=E(\pk,m'_b)}{c^*}{}
%\mleft{}{c_{q+1}}{c_{q+1}\neq c^*}
%\mright{m_{q+1}=D(\sk,c_{q+1})}{m_{q+1}}{}
% & \vdots & \\
%\mleft{}{c_{q+l}}{c_{q+l}\neq c^*}
%\mright{m_{q+l}=D(\sk,c_{q+l})}{m_{q+l}}{}
%\mleft{}{b'}{}
Output 1 if $(b'=b)$ & & \\
\end{tabular}
\caption{Adaptive CCA security.}\label{tab:cca2}
\end{table}

Note that, in this adaptive version, the adversary is able to send more queries to the
challenger even after having seen the challenge ciphertext $c^*$. The only thing we require
is that he does not pass the challenge ciphertext $c^*$ itself for those queries.
\end{definition}

\begin{theorem}
Given an IND-CPA-secure public key encryption scheme $(G,E,D)$, it is possible
to construct an IND-CCA1-secure public key encryption scheme $(G',E',D')$.
\end{theorem}
\proof
Let $(\pk_1,\sk_1)\leftarrow G(1^k)$, $(\pk_2,\sk_2)\leftarrow G(1^k)$ be two pairs
of keys generated by the IND-CPA scheme $(G,E,D)$.
We claim that there is a NIZK proof system that is
able to prove that $c_1$ and $c_2$ are ciphertexts obtained by the encryption
of the same message $m$ under the keys $\pk_1$ and $\pk_2$, respectively.

More precisely, we claim that there is a NIZK proof system for the language
$$L = \{(c_1,c_2) \mid \text{$\exists r_1,r_2,m$ such that $c_1=E(\pk_1,m;r_1)$ and
$c_2=E(\pk_2,m;r_2)$}\},$$
where $E(\pk_i,m;r_i)$ represents the output of $E(\pk_i,m)$ when the random coin flips
of the procedure $E$ are given by $r_i$.

The language $L$ is clearly in $NP$, since for every $x=(c_1,c_2)\in L$ there is a witness
$w=(r_1,r_2,m)$ that proves that $x\in L$. Given $x$ and $w$, there is an efficient procedure
to verify if $w$ is a witness to the fact that $x\in L$.

By Theorem~\ref{the:NIZK_NP}, there is a NIZK proof system for any language in $NP$, so our
claim holds.
Let this NIZK proof system be given by the procedures $(K,P,V)$.
We can assume that this is an adaptive NIZK, because it is always possible to construct an
adaptive NIZK from a non-adaptive NIZK proof system.
Then we define our public key encryption scheme $(G',E',D')$ as follows:

\begin{align*}
G'(1^k):\quad & (\pk_1,\sk_1) \leftarrow G(1^k) \\
              & (\pk_2,\sk_2) \leftarrow G(1^k) \\
              & \sigma \leftarrow K(1^k) \\
              & \text{let $\pk'=(\pk_1,\pk_2,\sigma)$ and $\sk'=\sk_1$} \\
              & \text{return $(\pk',\sk')$} \\
& \\
E'(\pk',m):\quad & \text{let $(\pk_1,\pk_2,\sigma)=\pk'$} \\
                 & c_1 \leftarrow E(\pk_1,m;r_1) \\
                 & c_2 \leftarrow E(\pk_2,m;r_2) \\
                 & \text{let $x=(c_1,c_2)$ // statement to prove} \\
                 & \text{let $w=(r_1,r_2,m)$ // witness for the statement} \\
                 & \pi \leftarrow P(\sigma,x,w) \\
                 & \text{let $c=(c_1,c_2,\pi)$} \\
                 & \text{return $c$} \\
& \\
D'(\pk',c):\quad & \text{let $\sk_1=\sk'$} \\
                 & \text{let $(c_1,c_2,\pi)=c$} \\
                 & \text{let $x=(c_1,c_2)$} \\
                 & \text{if $V(\sigma,x,\pi)$} \\
                 & \quad\text{then return $D(\sk_1,c_1)$} \\
                 & \quad\text{else return $\bot$} \\
\end{align*}

The correctness of $(G',E',D')$ is easy. If the keys were generated correctly and the
messages were encrypted correctly, then $\pi$ is a valid proof for the fact that
$c_1$ and $c_2$ encrypt the same message. So the verifier $V(\sigma,x,\pi)$ will
output true and the original message $m$ will be obtained by the decryption $D(\sk_1,c_1)$.

\begin{table}[t!]
\centering
\begin{tabular}{l|l}
Game 0 &
Game 1 \\
 & \\
$(\pk_1,\sk_1),(\pk_2,\sk_2) \leftarrow G(1^k)$ &
$(\pk_1,\sk_1),(\pk_2,\sk_2) \leftarrow G(1^k)$ \\
$\sigma \leftarrow {K(1^k)}$ &
$\textcolor{red}{\sigma \leftarrow {Sim(1^k)}}$ \\
let $\pk'=(\pk_1,\pk_2,\sigma)$ and $\sk'={\sk_1}$ &
let $\pk'=(\pk_1,\pk_2,\sigma)$ and $\sk'={\sk_1}$ \\
$(m_0,m_1) \leftarrow A^{D'(\sk',\cdot)}(\pk')$ &
$(m_0,m_1) \leftarrow A^{D'(\sk',\cdot)}(\pk')$ \\
$c_1 \leftarrow E(\pk_1,{m_0};r_1), c_2 \leftarrow E(\pk_2,{m_0};r_2)$ &
$c_1 \leftarrow E(\pk_1,{m_0};r_1), c_2 \leftarrow E(\pk_2,{m_0};r_2)$ \\
$\pi \leftarrow {P(\sigma,(c_1,c_2),(r_1,r_2,{m_0}))}$ &
$\textcolor{red}{\pi \leftarrow {Sim(c_1,c_2)}}$ \\
$b \leftarrow A(\pk',c_1,c_2,\pi)$ &
$b \leftarrow A(\pk',c_1,c_2,\pi)$ \\
\hline
Game 2 &
Game 2' \\
 & \\
$(\pk_1,\sk_1),(\pk_2,\sk_2) \leftarrow G(1^k)$ &
$(\pk_1,\sk_1),(\pk_2,\sk_2) \leftarrow G(1^k)$ \\
$\sigma \leftarrow {Sim(1^k)}$ &
$\sigma \leftarrow {Sim(1^k)}$ \\
let $\pk'=(\pk_1,\pk_2,\sigma)$ and $\sk'={\sk_1}$  &
let $\pk'=(\pk_1,\pk_2,\sigma)$ and $\sk'=\textcolor{red}{\sk_2}$  \\
$(m_0,m_1) \leftarrow A^{D'(\sk',\cdot)}(\pk')$ &
$(m_0,m_1) \leftarrow A^{\textcolor{red}{D'(\sk',\cdot)}}(\pk')$ \\
$c_1 \leftarrow E(\pk_1,{m_0};r_1), c_2 \leftarrow E(\pk_2,\textcolor{red}{m_1};r_2)$ &
$c_1 \leftarrow E(\pk_1,{m_0};r_1), c_2 \leftarrow E(\pk_2,{m_1};r_2)$ \\
$\pi \leftarrow {Sim(c_1,c_2)}$ &
$\pi \leftarrow {Sim(c_1,c_2)}$ \\
$b \leftarrow A(\pk',c_1,c_2,\pi)$ &
$b \leftarrow A(\pk',c_1,c_2,\pi)$ \\
\hline
Game 3 &
Game 3' \\
 & \\
$(\pk_1,\sk_1),(\pk_2,\sk_2) \leftarrow G(1^k)$ &
$(\pk_1,\sk_1),(\pk_2,\sk_2) \leftarrow G(1^k)$ \\
$\sigma \leftarrow {Sim(1^k)}$ &
$\sigma \leftarrow {Sim(1^k)}$ \\
let $\pk'=(\pk_1,\pk_2,\sigma)$ and $\sk'={\sk_2}$  &
let $\pk'=(\pk_1,\pk_2,\sigma)$ and $\sk'=\textcolor{red}{\sk_1}$  \\
$(m_0,m_1) \leftarrow A^{D'(\sk',\cdot)}(\pk')$ &
$(m_0,m_1) \leftarrow A^{\textcolor{red}{D'(\sk',\cdot)}}(\pk')$ \\
$c_1 \leftarrow E(\pk_1,\textcolor{red}{m_1};r_1), c_2 \leftarrow E(\pk_2,{m_1};r_2)$ &
$c_1 \leftarrow E(\pk_1,{m_1};r_1), c_2 \leftarrow E(\pk_2,{m_1};r_2)$ \\
$\pi \leftarrow {Sim(c_1,c_2)}$ &
$\pi \leftarrow {Sim(c_1,c_2)}$ \\
$b \leftarrow A(\pk',c_1,c_2,\pi)$ &
$b \leftarrow A(\pk',c_1,c_2,\pi)$ \\
\hline
Game 4 & \\
 & \\
$(\pk_1,\sk_1),(\pk_2,\sk_2) \leftarrow G(1^k)$ & \\
$\sigma \leftarrow {K(1^k)}$ & \\
let $\pk'=(\pk_1,\pk_2,\sigma)$ and $\sk'={\sk_1}$  & \\
$(m_0,m_1) \leftarrow A^{D'(\sk',\cdot)}(\pk')$ & \\
$c_1 \leftarrow E(\pk_1,{m_1};r_1), c_2 \leftarrow E(\pk_2,{m_1};r_2)$ & \\
$\textcolor{red}{\pi \leftarrow {P(\sigma,(c_1,c_2),(r_1,r_2,{m_1}))}}$ & \\
$b \leftarrow A(\pk',c_1,c_2,\pi)$ & \\
\end{tabular}
\caption{Games used for the hybrid argument.}\label{tab:games}
\end{table}

Now we want to prove that $(G',E',D')$ is IND-CCA1-secure.
Let $Sim$ be a simulator of the NIZK proof system.
Consider the games shown in Table~\ref{tab:games}.
In these games, the adversary is given a decrypting oracle during the first phase
and at the end the adversary should output a bit to distinguish which game he is playing.
We now show that the adversary $A$ cannot distinguish Game 0 and Game 4.

Let $\fake$  be the event that $A$ submits a query $(c_1,c_2,\pi)$ for its decryption oracle such that $D(\sk_1,c_1) \neq D(\sk_2, c_2)$ but $V(\sigma,(c_1,c_2),\pi)=1$.
This represents the event where the adversary is able to trick the verifier into returning true on a bad pair of ciphertexts (i.e., a pair not in the language).
The probability of event $\fake$ in Game 0 is negligible because of the
soundness of the proof system.

Game 1 differs from Game 0 by the use of a simulator for the random string and proof generation.
They are indistinguishable by reduction to the zero-knowledge property of the proof system.
In particular, if the adversary could detect the difference, then the adversary could be used to distinguish real proofs from simulated proofs.
And the probability of event $\fake$ in Game 1 is also negligible by the
zero-knowledge property.

Game 2 differs from Game 1 in that it obtains ciphertext $c_2$ from the message $m_1$.
They are indistinguishable by reduction to the IND-CPA property of the underlying PKE,
which guarantees that encryptions of $m_0$ cannot be distinguished from encryptions of $m_1$.
$\Pr[\fake]$ is the same as in Game 1, namely negligible.

Game 2' differs from Game 2 by using the key $\sk_2$ for decryption instead of $\sk_1$.
The adversary's view of the two games only differs if the event $\fake$ occurs, and it happens with
negligible probability.

Game 3 differs from Game 2' in that it obtains ciphertext $c_1$ from the message $m_1$.
They are indistinguishable by reduction to the IND-CPA property of the underlying PKE,
which guarantees that encryptions of $m_0$ cannot be distinguished from encryptions of $m_1$.
$\Pr[\fake]$ stays the same, namely negligible.

Game 3' differs from Game 3 by using the key $\sk_1$ for decryption instead of $\sk_2$.
The adversary's view of the two games only differs if the event $\fake$ occurs, and it happens with
negligible probability.

Game 4 differs from Game 3' by the use of the real NIZK proof system instead of a simulator.
They are indistinguishable by reduction to the zero-knowledge property of the proof system.
\qed



%\input{lec12-F24}
%
\section{Post-Quantum Public Key Encryption}
The public-key encryption schemes discussed so far, rely on the difficulty of problems like factorization or discrete logarithms. In 1994, Peter Shor showed that these problems can be solved efficiently on a quantum computer~\cite{FOCS:Shor94}. Even though we do not have large-scale quantum computers capable of breaking current encryption schemes, there are two reasons to begin the transition of public-key encryption to quantum-resistant schemes:
\begin{itemize}
    \item Encrypted messages captured today can be stored and decrypted in the future when a large scale quantum computer is available. This is commonly referred to as the "harvest now, decrypt later" risk.
    \item Transition to new encryption schemes is a slow process and it is important to start the transition well before large scale quantum computers are available.
\end{itemize}

The national institute of standards and technology (NIST) opened a call for post-quantum public-key encryption and signature schemes in 2016. In Nov 2019, it received 59 submissions for public-key encryption and 23 submissions for digital signatures. In July 2022, NIST announced the first batch of winners for public-key encryption and digital signatures. In Aug 2024, the final standard for these schemes was published and they are now making their way into existing infrastructure. There was one winner (CRYSTALS-Kyber) for public-key encryption (based on lattices) and three winners for digital signatures (two based on lattices and one based on hash functions).

\begin{figure}[!htbp]
    \begin{tikzpicture}[node distance=1.5cm and 2cm,
            every node/.style={align=center}, % Adjust width as needed
            >=stealth] % Nicer arrow tips
        \node[align=center](lwe) {LWE \\ \cite{STOC:Regev05}};
        \node[right=of lwe] (lweske) {LWE-SKE};
        \node[right=of lweske, align=center] (lwepke) {LWE-PKE \\ \cite{STOC:Regev05} \\ LWE + LHL };
        \node[right=of lwepke, align=center] (comlwepke) {Compact LWE-PKE \\ \cite{RSA:LinPei11} \\ LWE + LWE};
        \node[below=of lweske, align=center] (rlwe) {RLWE \\ \cite{EC:LyuPeiReg10}};
        \node[right=of rlwe, align=center] (rlwepke) {RLWE-PKE \\ \cite{EC:LyuPeiReg13} \\ RLWE + RLWE};
        \node[below=of rlwe, align=center] (mlwe) {MLWE \\ \cite{ITCS:BraGenVai12,DCC:LanSte15}};
        \node[below=of comlwepke, align=center] (kyber) {CRYSTALS-Kyber \\ MLWE + MLWE};

        \draw[->] (lwe) -- (lweske);
        \draw[->] (lweske) -- (lwepke);
        \draw[->] (lwepke) -- (comlwepke);
        \draw[->] (lwe) -- (rlwe);
        \draw[->] (lwe) -- (mlwe);
        \draw[->] (rlwe) -- (rlwepke);
        \draw[->] (comlwepke) -- (rlwepke);
        \draw[->] (mlwe) -- (kyber);
        \draw[->] (comlwepke) -- (kyber);
        \draw[->] (rlwepke) -- (kyber);
        % Bounding box
        % \draw[dashed, thick] ($(lwe.north west)+(-0.5,0.5)$) rectangle ($(kyber.south east)+(0.5,-3)$);
    \end{tikzpicture}
    \caption{A roadmap of lattice-based public-key encryption schemes ending in the standardized CRYSTALS-Kyber scheme.}
\end{figure}


\begin{definition}[Learning With Errors Assumption (Search)]
    Let $m,n,q \in \mathbb{N}$ and $\chi$ be a distribution over $\mathbb{Z}_q$. The Learning With Errors (LWE) $\LWE_{n,m,q,\chi}$ problem is defined as follows:
    \[\Pr[\adv(A,b) \to s \mid s \sample \mathbb{Z}_q^n, A \sample \mathbb{Z}_q^{m \times n}, b = A \cdot s + e ] \leq \negl\]
\end{definition}

\begin{definition}[Learning With Errors (Decision)]
    Let $m,n,q \in \mathbb{N}$ and $\chi$ be a distribution over $\mathbb{Z}_q$. The Learning With Errors (LWE) $\LWE_{n,m,q,\chi}$ problem is defined as follows:
    \begin{multline*}
        |\Pr[\adv(A,b) \to 1 \mid s \sample \mathbb{Z}_q^n, A \sample \mathbb{Z}_q^{m \times n}, b = A \cdot s + e ] \\
        - \Pr[\adv(A,b) \to 1 \mid s \sample \mathbb{Z}_q^n, A \sample \mathbb{Z}_q^{m \times n}, b \sample \mathbb{Z}_q^m ]|\leq \negl
    \end{multline*}
\end{definition}
The Learning With Errors assumption is commonly referred to as a Lattice based assumption because there is a reduction from Search/Decision LWE to a ``worst-case'' lattice problem.

The above assumptions have been stated with respect to some abstract distribution $\chi$ over $\mathbb{Z}_q$. But what do we actually choose this distribution to be? In the extreme case of $\chi = 0$, the LWE problem is trivial as one can simply use Gaussian Elimination. In the other extreme if $\chi$ is uniform over $\mathbb{Z}_q$, then the LWE problem is information theoretically hard (but not very useful for cryptography). We will be interested in the intermediate case where $\chi$ is a \emph{small} distribution over $\mathbb{Z}_q$, centered around 0. For eg: $\chi$ is a uniform distribution over $[-B, B]$ for some $B \ll q/2$. This will allow us to do build interesting cryptographic primitives like public key encryption and signature schemes. For provable reductions to lattice problems, we set $\mathsf{stddev}(\chi) = \Omega(\sqrt{n})$. However, there is a gap between the best known attacks on LWE and the best known reductions to lattice problems. As a result, much more aggressive parameters are used in practice, chosen based on the best known attacks. Typical parameters for LWE are $n = 512$, $q = 3329$, $\mathsf{supp}(\chi) = [-3,3]$, and $m = 768$. \cite{EPRINT:AlbPlaSco15} provides a lattice estimator \url{https://github.com/malb/lattice-estimator} that can be used to estimate the number of bits of security provided by a given set of LWE parameters.

\subsection{LWE $\to$ LWE-SKE}
As a first step, we will see how to build a symmetric key encryption scheme from LWE.

\begin{itemize}
    \item $\Gen$: Sample $s \sample \mathbb{Z}_q^n$ and output $\sk = s$.
    \item $\Enc(\sk,\mu)$: Sample $a \sample \mathbb{Z}_q^n$, $e \sample \chi$, and compute $b \gets \langle a, s \rangle + e$. Output $c = (c_0 = a, c_1 = b + \lfloor\frac{q}{2}\rfloor\mu)$.
    \item $\Dec(\sk,c)$: Parse $c = (c_0, c_1)$ and compute $m \gets \mathsf{Decode}(c_1 - \langle c_0, s \rangle)$, where $\mathsf{Decode}(\hat{\mu}) \to \{0,1\}$ takes a value from $\mathbb{Z}_q$ and outputs $0$ if $\hat{\mu}$ is closer to $0$ than to $\lfloor\frac{q}{2}\rfloor$ and $1$ otherwise.
\end{itemize}

\begin{theorem}
    If the decisional variant of LWE is hard, then the above scheme is a secure symmetric key encryption scheme.
\end{theorem}
\begin{proof}
    Let the ciphertext $c = (c_0, c_1)$. Then $\hat{\mu} = c_1 - \langle c_0,s\rangle = b + \lfloor\frac{q}{2} - \langle a,s\rangle = \lfloor\frac{q}{2}\rfloor\mu + e$.

\end{proof}
%\section{Cramer-Shoup Construction}

The Cramer-Shoup cryptosystem is a public-key encryption scheme that achieves security against adaptive chosen ciphertext attacks (CCA2). To understand its significance, let's first examine the ElGamal encryption scheme and its limitations.

\subsection{Background: ElGamal Encryption}

ElGamal encryption is a simple and elegant public-key system based on the Diffie-Hellman key exchange. The scheme operates as follows:

\begin{itemize}
    \item $\textsc{Gen}(1^n)$:
        \begin{algorithmic}
            \State Select prime $q$ where $|q|=n$
            \State Choose generator $g$ of group $G$ of order $q$
            \State Sample $x \gets \mathbb{Z}_q$ randomly
            \State Compute $h = g^x$
            \State Output $(\mathrm{pk}=(g,h), \mathrm{sk}=x)$
        \end{algorithmic}
    
    \item $\textsc{Enc}(\mathrm{pk}, m \in G)$:
        \begin{algorithmic}
            \State Sample $r \gets \mathbb{Z}_q$ randomly
            \State Compute $c_1 = g^r$
            \State Compute $c_2 = m \cdot h^r$
            \State Output $(c_1, c_2)$
        \end{algorithmic}

    \item $\textsc{Dec}(\mathrm{sk}, (c_1, c_2))$:
        \begin{algorithmic}
            \State Compute $m = c_2/c_1^x$
            \State Output $m$
        \end{algorithmic}
\end{itemize}

\textbf{Proof of Correctness:}
\begin{align*}
    c_2/c_1^x &= (m \cdot h^r)/(g^r)^x \\
    &= (m \cdot (g^x)^r)/(g^r)^x \\
    &= m \cdot g^{xr}/g^{rx} \\
    &= m
\end{align*}

However, ElGamal is malleable, meaning an adversary can modify a ciphertext to create a related ciphertext. Given a ciphertext $(c_1, c_2)$ encrypting message $m$, an adversary can create $(c_1, k \cdot c_2)$ which will decrypt to $k \cdot m$ for any $k$. This malleability makes ElGamal insecure against chosen-ciphertext attacks.

\subsection{Hash Proof Systems}

To address these limitations, we introduce Hash Proof Systems (HPS), a powerful tool for constructing CCA-secure encryption schemes.

\subsection{Formal Definition}

Let $\mathbb{X}$ be a set and $\mathbb{L} \subset \mathbb{X}$ be a language defined by:
\[ \mathbb{L} = \{x \in \mathbb{X} \mid \exists w \text{ s.t. } (x,w) \in \mathbb{R}\} \]
where $\mathbb{R}$ is a binary relation and $w$ is called a witness.

A Hash Proof System consists of three algorithms:
\begin{itemize}
    \item $\mathcal{K}\mathcal{G}\text{en}(1^n)$: Generates key pair $(pk, sk)$
    \item $\mathcal{H}_{sk}: \mathbb{X} \rightarrow \Pi$ (private evaluation)
    \item $\mathcal{H}_{pk}: \mathbb{L} \times \mathbb{W} \rightarrow \Pi$ (public evaluation)
\end{itemize}

\subsection{Key Properties}

\begin{enumerate}
    \item \textbf{Correctness:} For all $(x,w) \in \mathbb{R}$:
    \[ \mathcal{H}_{sk}(x) = \mathcal{H}_{pk}(x,w) \]
    
    \textbf{Proof:}
    This follows from the construction where both evaluations compute the same mathematical operation, but $\mathcal{H}_{pk}$ requires a witness while $\mathcal{H}_{sk}$ uses the secret key.

    \item \textbf{Smoothness:} For all $x \not\in \mathbb{L}$:
    \[ \{pk, \mathcal{H}_{sk}(x)\} \stackrel{s}{\approx} \{pk, U_\Pi\} \]
    where $U_\Pi$ is uniform over $\Pi$.
    
    \textbf{Proof Sketch:}
    This property follows from the DDH assumption in our concrete construction. When $x \not\in \mathbb{L}$, the hash value appears random to any computationally bounded adversary.

    \item \textbf{Universal Property:} For all $x \in \mathbb{L}$ and $y_1, \ldots y_t \in \mathbb{L}$:
    \[ \{pk, \mathcal{H}_{sk}(x), \{\mathcal{H}_{pk}(y_i)\}_i\} \stackrel{s}{\approx} \{pk, U_\Pi, \{\mathcal{H}_{sk}(y_i)\}_i\} \]
\end{enumerate}

\subsection{Concrete DDH-based Construction}

Let's construct a specific HPS based on the Decision Diffie-Hellman (DDH) assumption:

\begin{itemize}
    \item Let $G$ be a cyclic group of prime order $q$
    \item Fix generators $g_1, g_2 \in G$
    \item Define $\mathbb{X} = G^2$
    \item Define $\mathbb{L} = \{(h_1,h_2) \in \mathbb{X} \mid \exists r \in \mathbb{Z}_q: h_1=g_1^r \text{ and } h_2=g_2^r\}$
\end{itemize}

The construction:
\begin{align*}
    \mathcal{K}\mathcal{G}\text{en}(1^n): &\text{ Sample } d_1,d_2 \gets \mathbb{Z}_q \\
    &\text{ Output } (pk=(g_1^{d_1}g_2^{d_2}), sk=(d_1,d_2)) \\
    \mathcal{H}_{sk}(x=(h_1,h_2)) &= h_1^{d_1}h_2^{d_2} \\
    \mathcal{H}_{pk}(x=(h_1,h_2),w=r) &= pk^r
\end{align*}

\textbf{Proof of Correctness:}
For $(x,w) \in \mathbb{R}$ where $x=(g_1^r, g_2^r)$:
\begin{align*}
    \mathcal{H}_{sk}(x) &= (g_1^r)^{d_1}(g_2^r)^{d_2} \\
    &= g_1^{rd_1}g_2^{rd_2} \\
    &= (g_1^{d_1}g_2^{d_2})^r \\
    &= pk^r \\
    &= \mathcal{H}_{pk}(x,w)
\end{align*}

\subsection{Basic Scheme and Its Limitations}

Let's analyze our first attempt at constructing a secure encryption scheme using HPS:

$\mathcal{G}\text{en}(1^n)$:
\begin{itemize}
    \item $(pk,sk) \leftarrow \mathcal{K}\mathcal{G}\text{en}(1^n)$
\end{itemize}

$\text{Enc}(pk,m \in G)$:
\begin{enumerate}
    \item Sample $x \leftarrow \mathbb{L}$ along with witness $w$
    \item Compute $e = \mathcal{H}_{pk}(x,w) \cdot m$
    \item Output $(x,e)$
\end{enumerate}

$\text{Dec}(sk,(x,e))$:
\begin{itemize}
    \item Output $e/\mathcal{H}_{sk}(x)$
\end{itemize}

\subsection{Security Analysis of Basic Scheme}

While this construction achieves CPA security through the following hybrid argument:

\begin{enumerate}
    \item $\mathcal{H}_0$: Real encryption game
    \item $\mathcal{H}_1$: Replace $e = \mathcal{H}_{sk}(x) \cdot m$ (using $sk$ instead of $pk$)
        \begin{itemize}
            \item This is identical to $\mathcal{H}_0$ because $x \in \mathbb{L}$ and by HPS correctness
            \item Formally: For any $(x,w) \in \mathbb{R}$, $\mathcal{H}_{pk}(x,w) = \mathcal{H}_{sk}(x)$
        \end{itemize}
    \item $\mathcal{H}_2$: Replace $x \leftarrow \mathbb{X} \setminus \mathbb{L}$ 
        \begin{itemize}
            \item Indistinguishable by property 2 of HPS
            \item Adversary cannot tell if $x$ is sampled from $\mathbb{L}$ or $\mathbb{X} \setminus \mathbb{L}$
        \end{itemize}
    \item $\mathcal{H}_3$: Replace $e$ with uniform random element
        \begin{itemize}
            \item Indistinguishable by smoothness property of HPS
            \item When $x \not\in \mathbb{L}$, $\mathcal{H}_{sk}(x)$ is statistically close to uniform
        \end{itemize}
\end{enumerate}

However, this scheme fails to achieve CCA-2 security. Here's a concrete attack:

\begin{theorem}[CCA-2 Attack on Basic Scheme]
The basic HPS construction is not CCA-2 secure.
\end{theorem}

\begin{proof}
Consider the following attack in the CCA-2 game:
\begin{enumerate}
    \item Adversary receives challenge ciphertext $(x^*, e^*)$ for message $m_b$
    \item Adversary creates modified ciphertext $(x^*, k \cdot e^*)$ for some $k \in G$
    \item Adversary submits modified ciphertext to decryption oracle
    \item Let $m'$ be the decrypted result. Then:
    \begin{align*}
        m' &= (k \cdot e^*)/\mathcal{H}_{sk}(x^*) \\
        &= k \cdot (e^*/\mathcal{H}_{sk}(x^*)) \\
        &= k \cdot m_b
    \end{align*}
    \item Since $k$ is known, adversary can recover $m_b$ and win the CCA-2 game
\end{enumerate}
\end{proof}

\subsection{The Need for Ciphertext Integrity}

The key insight is that we need to prevent ciphertext manipulation. The full Cramer-Shoup construction addresses this by:

\begin{enumerate}
    \item Adding a second HPS instance ($\mathcal{H}'$) that acts as a "proof of well-formedness"
    \item Using a collision-resistant hash function to bind all components together
    \item Verifying the proof $\pi$ before decryption
\end{enumerate}

The second HPS instance must satisfy a stronger security property called "2-smoothness":

\begin{definition}[2-Smoothness]
For all $x_1,x_2 \in \mathbb{L}$ and $t_1,t_2 \in T$ with $t_1 \neq t_2$:
\[ \{pk, \mathcal{H}_{sk}(x_1,t_1), \mathcal{H}_{sk}(x_2,t_2)\} \stackrel{s}{\approx} \{pk, U_\Pi, U_\Pi\} \]
\end{definition}

\subsection{CCA-2 Secure Construction}

The final Cramer-Shoup construction achieves CCA-2 security by combining two HPS instances with a collision-resistant hash function:

\subsection{The Scheme}

$\mathcal{G}\text{en}(1^n)$:
\begin{algorithmic}
    \State $(pk,sk) \leftarrow \mathcal{K}\mathcal{G}\text{en}(1^n)$
    \State $(pk',sk') \leftarrow \mathcal{K}\mathcal{G}\text{en}(1^n)$
    \State Output $((pk,pk'),(sk,sk'))$
\end{algorithmic}

$\text{Enc}(pk,m)$:
\begin{algorithmic}
    \State Sample $x \leftarrow \mathbb{L}$ with witness $w$
    \State $e = \mathcal{H}_{pk}(x,w) \cdot m$
    \State $t = \text{CRHF}(x,e)$
    \State $\pi = \mathcal{H}'_{pk}(x,w,t)$
    \State Output $(x,e,\pi)$
\end{algorithmic}

$\text{Dec}(sk,(x,e,\pi))$:
\begin{algorithmic}
    \If{$\pi \neq \mathcal{H}'_{sk}(x)$}
        \State Output $\perp$
    \EndIf
    \State Output $e/\mathcal{H}_{sk}(x)$
\end{algorithmic}
\subsection{Final Security Theorem}

\begin{theorem}
If $\mathcal{H}$ is a 1-smooth HPS, $\mathcal{H}'$ is a 2-smooth HPS, and CRHF is a collision-resistant hash function, then the scheme is CCA-2 secure.
\end{theorem}

\begin{proof}
We prove security through a sequence of hybrid games, showing each transition is indistinguishable to a PPT adversary. Let $A$ be any PPT adversary in the CCA2 game.

First, we establish a crucial lemma:

\begin{lemma}[Key Soundness Lemma]
For any ciphertext decryption query $(x,e,\pi)$ that $A$ makes, if $\text{Dec}((sk,sk'),(x,e,\pi)) \neq \perp$ then $x \in \mathbb{L}$ except with negligible probability.
\end{lemma}

\begin{proof}[Proof of Lemma]
Suppose for contradiction that $x \notin \mathbb{L}$ but the decryption doesn't return $\perp$. This means:
\[ \pi = \mathcal{H}'_{sk'}(x, \text{CRHF}(x,e)) \]

By the 2-smoothness property of $\mathcal{H}'$, when $x \notin \mathbb{L}$, the value $\mathcal{H}'_{sk'}(x,t)$ is statistically indistinguishable from random, even given $pk'$. Therefore, the probability that $A$ can generate such a $\pi$ is at most $1/|\Pi|$, which is negligible.
\end{proof}

Now we proceed with the hybrid games:

\begin{itemize}
    \item $\mathcal{G}_0$: The real CCA2 game.
    
    \item $\mathcal{G}_1$: Same as $\mathcal{G}_0$, but the challenger computes the challenge ciphertext's $e^*$ component using $sk$ instead of $pk$:
    \[ e^* = \mathcal{H}_{sk}(x^*) \cdot m_b \]
    instead of
    \[ e^* = \mathcal{H}_{pk}(x^*,w) \cdot m_b \]
    
    By the correctness property of HPS, these are identical when $x^* \in \mathbb{L}$, so:
    \[ \Pr[\mathcal{G}_0 = 1] = \Pr[\mathcal{G}_1 = 1] \]

    \item $\mathcal{G}_2$: Same as $\mathcal{G}_1$, but now the challenger samples $x^* \gets \mathbb{X} \setminus \mathbb{L}$ instead of from $\mathbb{L}$.
    
    By the hardness of distinguishing elements in $\mathbb{L}$ from elements in $\mathbb{X} \setminus \mathbb{L}$ (which follows from the DDH assumption in our concrete construction):
    \[ |\Pr[\mathcal{G}_1 = 1] - \Pr[\mathcal{G}_2 = 1]| \leq \epsilon_{\text{DDH}} \]

    \item $\mathcal{G}_3$: Same as $\mathcal{G}_2$, but replace $\mathcal{H}_{sk}(x^*)$ with a uniform random value $u \gets \Pi$ in computing $e^*$:
    \[ e^* = u \cdot m_b \]
    
    By the smoothness property of $\mathcal{H}$, since $x^* \notin \mathbb{L}$:
    \[ |\Pr[\mathcal{G}_2 = 1] - \Pr[\mathcal{G}_3 = 1]| \leq \epsilon_{\text{smooth}} \]

    \item $\mathcal{G}_4$: Same as $\mathcal{G}_3$, but replace $e^*$ with a uniform random value in $G$. 
    
    Since $u$ is uniform in $\Pi$ and independent of $m_b$, $e^*$ is uniform in $G$ and independent of $m_b$, so:
    \[ \Pr[\mathcal{G}_3 = 1] = \Pr[\mathcal{G}_4 = 1] \]
\end{itemize}

In $\mathcal{G}_4$, the challenge ciphertext is independent of $m_b$, so $A$ has no advantage. Therefore:
\[ \text{Adv}^{\text{CCA2}}_A \leq \epsilon_{\text{DDH}} + \epsilon_{\text{smooth}} \]

To complete the proof, we need to show that the decryption oracle queries in all games can be answered properly. For any decryption query $(x,e,\pi)$:

\begin{enumerate}
    \item If $(x,e) = (x^*,e^*)$ but $\pi \neq \pi^*$: By 2-smoothness of $\mathcal{H}'$, generating a valid $\pi$ is infeasible.
    \item If $(x,e) \neq (x^*,e^*)$ but $\text{CRHF}(x,e) = \text{CRHF}(x^*,e^*)$: This breaks collision resistance of CRHF.
    \item If $\text{CRHF}(x,e) \neq \text{CRHF}(x^*,e^*)$: By 2-smoothness of $\mathcal{H}'$, any $\pi$ that verifies must correspond to a witness $w$ such that $(x,w) \in \mathbb{R}$, meaning $x \in \mathbb{L}$. Therefore, the decryption oracle can be simulated properly.
\end{enumerate}

Thus, all hybrid games are indistinguishable to $A$, and the scheme is CCA2-secure.
\end{proof}

%\chapter{Advanced Encryption Schemes}
\section{Identity-Based Encryption}

We introduce Bilinear Maps and two of its applications: NIKE, Non-Interactive Key Exchange; and IBE, Identity Based Encryption.


\section{Diffie-Hellman Key Exchange}

\begin{figure}
\label{fig:dh}
\centering
  \includegraphics[width=0.7\textwidth]{Old Scribe Notes/fig1.pdf}
\caption{Diffie-Hellman Key Exchange}
\end{figure}


Fig \ref{fig:dh} illustrates Diffie-Hellman key exchange. Alice and Bob each has a private key ($a$ and $b$ respectively), and they want to build a shared key for symmetric encryption communication. They can only communicate over a insecure link, which is eavesdropped by Eve.
So Alice generates a public key $A$ and Bob generates a public key $B$, and they send their public key to each other at the same time. Then Alice generates the shared key $K$ from $a$ and $B$, and likewise, Bob generates the shared key $K$ from $b$ and $A$.
And we have $\forall$ PPT Eve, $Pr[k=Eve(A,B)]=neg(k)$, where $k$ is the length of $a$.


\subsection{Discussion 1}

Assume that $\forall (g, p)$, and $a_1,b_1 \stackrel{\$}{\gets} Z^*_p$, and $a_2,b_2,r \stackrel{\$}{\gets}Z^*_p$, we have $(g^{a_1}, g^{b_1}, g^{a_1b_1}) \stackrel{c}{\simeq} (g^{a_2}, g^{b_2}, g^r)$. How to apply this to Diffie-Hellman Key Exchange?


Make $A=g^a$, $B=g^b$, $K=A^b=g^{ab}$, and $K=B^a=g^{ab}$.


\subsection{Discussion 2}


How does Diffie-Hellman Key Exchange imply Public Key Encryption?


Alice
$pk = A$, $sk = a$, $Enc(pk, m \in \{0, 1\})$.

Bob
$b,r \gets Z^*_p$
$(g^b, mA^b+(1-m)g^r)$

Alice $Dec(sk, (c_1, c_2))$

$c_1^a \stackrel{?}{=} c_2$




\section{Bilinear Maps}

\begin{definition}{Bilinear Maps}

Bilinear Maps is $(G,P,G_T,g,e)$, where $e$ is an efficient function $G \times G \to G_T$ such that

\begin{itemize}
\item if $g$ is generator of $G$, then $e(g, g)$ is the generator of $G_T$.
\item $\forall a,b \in Z_p$, we have $e(g^a, g^b) = e(g, g)^{ab} = e(g^b, g^a)$.
\end{itemize}

\end{definition}

\subsection{Discussion 1}


How does Bilinear Maps apply to Diffie-Hellman?

Make $A=g^a$, $B=g^b$, and $T=g^{ab}$, then Diffie-Hellman has $e(A, B)=e(g, T)$.


\section{Tripartite Diffie-Hellman}

\begin{figure}
\label{fig:3dh}
\centering
  \includegraphics[width=0.7\textwidth]{Old Scribe Notes/fig2.pdf}
\caption{Tripartite Diffie-Hellman Key Exchange}
\end{figure}

Fig \ref{fig:3dh} illustrates Tripartite Diffie-Hellman key exchange. $a$, $b$, and $c$ are private key of Alice, Bob, and Carol, respectively.
They use $g^a$, $g^b$, $g^c$ as public key, and the shared key $K=e(g,g)^{abc}$.
Formally, we have
$$a,b,c \stackrel{\$}{\gets} Z^*_p, r \stackrel{\$}{\gets} Z^*_p$$
$$A=g^a, B=g^b, C=g^c$$
$$K=e(g,g)^{abc}$$



\section{IBE: Identity-Based Encryption}
When two parties communicate secure messages through a public key infrastructure, they need to go through a time-consuming and error-prone process to get each other's key and verify each other's identity through a Certificate Authority. 
Identity-based cryptography (IBC) seeks to reduce these barriers by requiring no preparation on the part of the message recipient, therefore saving the initial round trip. 
Identity based encryption can also be used to construct CCA-secure public key encryption and digital signatures. 

IBE contains four steps: \emph{Setup}, \emph{KeyGen}, \emph{Enc}, and \emph{Dec}. We illustrate it in Figure \ref{fig:ibe}.
IBE relies on a trusted third party called Private Key Generator(PKG). 
In first step, PKG gets a Master Public Key ($mpk$) and Master Signing Key ($msk$) from $Gen(1^n)$. 
Then a user with an ID (in this example, ``Mike''), sends his ID to the PKG. 
The PKG generates the Signing Key of Mike with $KeyGen(msk, id)$ ans sends it back. 
Another user, Alice, wants to send an encrypted message to Mike. 
She only has $mpk$ and Mike's ID. 
So she encrypts the message with $c=Enc(mpk, id=Mike, m)$, and sends the encrypted message $c$ to Mike. 
Mike decodes $c$ with $m=Dec(c, sk_{Mike})$. 
Notice that Alice never need to know Mike's public key. 
She only needs to remember MPK and other people's IDs.

\begin{figure}
\label{fig:ibe}
\centering
  \includegraphics[width=0.7\textwidth]{Old Scribe Notes/fig3.pdf}
\caption{Identity-Based Encryption}
\end{figure}

Then we define IBE formally, 

\begin{definition}[Identity-Based Encryption]
    An \textbf{identity based encryption scheme} $\mathcal{E}_{id} = (G,K,E,D)$ is a tuple of four efficient algorithms: a \textbf{setup algorithm} $G$, a \textbf{key generation algorithm} $K$, an \textbf{encryption algorithm} $E$, and a \textbf{decryption algorithm} $D$.
    \begin{itemize}
        \item $G$ is a probabilistic algorithm invoked as $(mpk, msk)\stackrel{\$}{\gets} G(1^n)$, where $mpk$ is called the \textbf{master public key} and $msk$ is called the \textbf{master secret key} for the IBE scheme.
        \item $K$ is a probabilistic algorithm invoked as $sk_{id}\stackrel{\$}{\gets}K(msk,id )$, where $msk$ is the master secret key (as output by $S$), $id \in \mathcal{ID}$ is an identity, and $sk_{id}$ is a secret key for id.
        \item $E$ is a probabilistic algorithm invoked as $c\stackrel{\$}{\gets}E(mpk,id, m)$.
        \item $D$ is a deterministic algorithm invoked as $m\gets D(sk_{id}, c)$. Here $m$ is either a message or a special reject value $\bot$ (distinct from all messages).
    \end{itemize}
\end{definition}

As usual, we define the correctness of IBE to be the decryption undoes encryption, formally we have 
\begin{definition}[Correctness of IBE]
$\forall n, id, m$, we have
\[
Pr\begin{bmatrix}
   (mpk,msk) \gets G(1^n), \\[0.3em]
   sk_{id} \gets K(msk, ID), \\[0.3em]
   c \gets E(mpk, id, m), \\[0.3em]
   m \gets D(sk_{id}, c)
\end{bmatrix}
  =1 - \text{negl}(n)
\]
\end{definition}

Next we define the security of IBE scheme. 
The basic security definition considers an adversary who obtains the secret keys for a number of identities of its choice. 
The adversary should not be able to break semantic security for some other identity of its choice for which it does not have the secret key.

\begin{definition}[Security of IBE]
    For an IBE scheme $\mathcal{E}_{id}=(G,K,E,D)$ is secure if $\forall$ nuPPT $\mathcal{A}$,
    \[
    \Pr[\text{Exp}_{\pi,\mathcal{A}}^{IBE,CPA}(n)=1]=\text{negl}(n)
    \]
\end{definition}

We then define the experiment $\text{Exp}_{\pi,\mathcal{A}}^{IBE,CPA}$.
\begin{definition}[IBE-CPA Experiment]
We denote the experiment in the following order: 
\begin{enumerate}
    \item The challenger invokes $(mpk, msk)\stackrel{\$}{\gets} G(1^n)$ and send $mpk$ to adversary $\mathcal{A}$.
    \item $\mathcal{A}$ can make multiple key queries and generate desired ID $id^{*}$ and two message $(m_0, m_1)$. 
    \item Challenger random selects $b\stackrel{\$}{\gets}\{0,1\}$ and encrypt $c^{*}=E(mpk, id^{*},m_b)$ and send $c^{*}$ to $\mathcal{A}$.
    \item $\mathcal{A}$ can make more encryption queries based on $c^{*}$ and other id and message $(m_0,m_1)$. Note that throughout the process $\mathcal{A}$ can \textbf{never} make any query on $id^{*}$. 
    \item At the end, $\mathcal{A}$ generate $b'$ and send to challenger. 
    \item Challenger output $1$ if $b=b'$, $0$ otherwise. 
\end{enumerate}
\end{definition}

\subsection{IBE Construction from Pairing}
We then present a concrete IBE construction from pairing.
First, we will give the hardness assumption in this scheme called \textbf{bilinear Diffe-Hellman} assumption, or BDH. 
This assumption says that given random element $g_0^{\alpha},g_0^{\beta},g_0^{\gamma}\in\mathbb{G_0}$ and a few additional terms, the quantity $e(g0,g1)^{\alpha\beta\gamma}\in\mathbb{G}_T$ is computationally indistinguishable from a random element in GT.

\begin{definition}[Decisional bilinear Diffe-Hellman]
    Let $e: \mathbb{G}_0\times\mathbb{G}_1\rightarrow\mathbb{G}_T$ be a pairing where $\mathbb{G}_0,\mathbb{G}_1,\mathbb{G}_T$ are cyclic groups of prime order $q$ with generators $g_0 \in \mathbb{G}_0$ and $g_1 \in \mathbb{G}_1$. For a given adversary $\mathcal{A}$, the following distribution is distinguishable: 
    \[
    \{g_0^{\alpha},g_1^{\alpha},g_0^{\beta},g_1^{\gamma},e(g_0,g_1)^{\alpha\beta\gamma},  \alpha,\beta,\gamma\stackrel{\$}{\gets}\mathbb{Z}_q\} \approx^{c}
    \{g_0^{\alpha},g_1^{\alpha},g_0^{\beta},g_1^{\gamma},e(g_0,g_1)^{\delta},  \alpha,\beta,\gamma,\delta\stackrel{\$}{\gets}\mathbb{Z}_q\}
    \]
\end{definition}

Note that this assumption work even with $g_0=g_1$.

We then present our IBE construction. 
\begin{itemize}
    \item $G(1^n)$: 
    \[
    \alpha\stackrel{\$}{\gets}\mathbb{Z}_q, mpk\gets g^\alpha, msk\gets\alpha
    \]
    and output ($mpk,msk$)
    \item $K(msk=\alpha,id)$:
    \[
    sk_{id}\gets H(id)^\alpha
    \]
    where $H$ is a hash function $H:\{0,1\}^{*}\rightarrow\mathbb{G}$
    \item $E(mpk, id, m)$: 
    \[
    \beta\stackrel{\$}{\gets}\mathbb{Z}_q, c_1\gets g^{\beta}, c_2\gets e(mpk,H(id)^{\beta})\cdot m
    \]
    and output $(c_1,c_2)$.
    \item $D(sk_{id}, c=(c_1, c_2))$:
    \[
    m=\frac{c_2}{e(c_1,sk_{id})}
    \]
\end{itemize}
By the property of bilinear map, we can verify the correctness of this scheme, 
\begin{align*}
    m &= \frac{c_2}{e(c_1,sk_{id})} \\
      &= \frac{e(mpk,H(id)^{\beta})\cdot m}{e(c_1,sk_{id})} \\
      &= \frac{e(g^\alpha,H(id)^{\beta})\cdot m}{e(g^{\beta}, H(id)^{\alpha})} \\
      &= \frac{e(g,H(id))^{\alpha\beta}}{e(g, H(id))^{\alpha\beta}}\cdot m \\
      &= m
\end{align*}

We then prove the security property under random oracle model.
\begin{theorem}
    If decision BDH holds for e, H is modelled as a random oracle, then the above construction is a secure IBE scheme. 
\end{theorem}
\begin{proof}
    Let $\mathcal{A}$ be an adversary that breaks the IBE scheme, we can construct another adversary $\mathcal{B}$ such that it breaks the DBDH assumption.

    The adversary $\mathcal{B}$ works as follows:
    \begin{enumerate}
        \item $\mathcal{B}$ receives 5 elements as input $\{g^{\alpha},g^{\beta},g^{\gamma},z,  \alpha,\beta,\gamma\stackrel{\$}{\gets}\mathbb{Z}_q,z\in\mathbb{G}\}$, and $\mathcal{B}$ need to determine whether $z=e(g,g)^{\alpha\beta\gamma}$ or not. 
        \item $\mathcal{B}$ send IBE public parameter $mpk=g^{\alpha}$ to IBE adversary $\mathcal{A}$.
        \item Then $\mathcal{A}$ will make multiple $sk_{id}$ queries to $\mathcal{B}$. $\mathcal{B}$ responds them by (1) Choose $\rho\stackrel{\$}{\gets}\mathbb{Z}_q$ (2) Setting $H(id)=g^{\rho}$ (3) set the secret key be $sk_{id}=H(id)^{\alpha}=g^{\rho\cdot\alpha}$. One \textbf{exception} is that $\mathcal{B}$ will set a random $id'$ whose $H(id')=g^\beta$.
        \item After receiving the key query, $\mathcal{A}$ outputs $(id^{*}, m_0,m_1)$.  
        \item When $\mathcal{B}$ receives encryption query $(id^{*}, m_0, m_1)$, it first check if $\mathcal{A}$ have previously query $sk_{id^*}$ before, if yes, then abort. Otherwise, $\mathcal{B}$ choose $b\stackrel{\$}{\gets}\{0,1\}$ and encrypt $m_b$ using $(c_1=g^\gamma, c_2=e(msk,H(id')^\gamma)\cdot m_b)$ and send back to $\mathcal{A}$.
        \item $\mathcal{A}$ eventually output $b'$ to $\mathcal{B}$. And $\mathcal{B}$ output $1$ if $b'=b$ and $0$ otherwise. 
    \end{enumerate}
    Since $e(msk,H(id')^{\gamma})=e(g^\alpha,g^{\beta\gamma})=e(g,g)^{\alpha\beta\gamma}$, $\mathcal{B}$ can embedded the challenge to $\mathcal{A}$ and break DBDH. \qed
\end{proof}

\subsection{Digital Signature from IBE}
We can directly derive a secure signature scheme from IBE. 
Given a secure IBE $\mathcal{E}_{id}=(G,K,E,D)$ with id space $\mathcal{ID}$ and message space $\mathcal{M}_{IBE}$, we construct a secure digital signature scheme $\mathcal{S} = (G',S',V')$ as follows: 
\begin{itemize}
    \item $G'(1^n)$: run $(mpk, msk)\stackrel{\$}{\gets} G(1^n)$ and output $(mpk, msk)$ as the sign key pair, with $mpk$ as verification key and $msk$ as sign key.
    \item $S'(msk,m)$: Given message $m\in\mathcal{ID}$, compute $\sigma\stackrel{\$}{\gets}K(msk,m)$, output $\sigma$ as signature. 
    \item $V'(mpk,m,\sigma)$: Choose $r\stackrel{\$}{\gets}\mathcal{M}_{IBE}$, compute $c\stackrel{\$}{\gets}E(mpk,m,r)$, and accept if $D(\sigma,c) = r$. 
\end{itemize}

We have the following theorem,
\begin{theorem}
    Let $\mathcal{E}_{id}$ be a secure IBE with message space is super-poly. Then the derive signature scheme $\mathcal{S}$ is a secure digital signature scheme. 
\end{theorem}
\begin{proof}
    Let $\mathcal{A}$ be an adversary that breaks the digital signature scheme, we can construct another adversary $\mathcal{B}$ such that breaks the IBE security.

    The BIE adversary $\mathcal{B}$ is modelled as follows:
    \begin{enumerate}
        \item $\mathcal{B}$ receives $mpk$ from the challenger, and forward $mpk$ as a signature public key to $\mathcal{A}$. 
        \item $\mathcal{A}$ makes a series of signing queries $m_0,\dots,m_n$ to $\mathcal{B}$. $\mathcal{B}$ responds by issuing the corresponding key query to the challenger and forwarding the answer back the response to $\mathcal{A}$. 
        \item $\mathcal{A}$ will output a signature forgery $(m,\sigma)$ which it didn't issue the sign query $m$. 
        \item $\mathcal{B}$ then choose two random message $t_0,t_1\stackrel{\$}{\gets}\mathcal{M}_{IBE}$ and issue encryption query with the identity $m$.
        \item $\mathcal{B} $gets back the ciphertext $c\stackrel{\$}{\gets}E(mpk,m,t_b)$ for $b\in{0,1}$. Then it runs $t'\gets D(\sigma,c)$ and output $b'=t'$. 
    \end{enumerate}
    We observe that 
    \begin{itemize}
        \item when $b = 1$, then $c\stackrel{\$}{\gets}E(mpk, m,t_1)$, and $\mathcal{B}$ outputs 1 with probability the same as the probability of $\mathcal{A}$ breaks digital signature, we note as $SIGadv[\mathcal{A,S}]$
        \item when $b = 0$, then $c\stackrel{\$}{\gets}E(mpk, m,t_0)$, and $\mathcal{B}$ outputs 1 with probability $1/|\mathcal{M}_{IBE}|$ since $\mathcal{B}$ can only make random guess in the message space. 
    \end{itemize}
    We then have the probability of $\mathcal{B}$ break IBE scheme
    \[
        \frac{1}{2}(SIGadv[\mathcal{A,S}] + 1/|\mathcal{M}_{IBE}|)
    \]
    which is not negligible if $SIGadv[\mathcal{A,S}]$ is not negligible. \qed

\end{proof}

%\section{Fully Homomorphic Encryption}
So far, we've seen private and public key encryption and different security properties (CPA, CCA). We've also seen some advanced encrpytion schemes like Identity-Based Encryption (IBE) that allow us to encrypt to an identity rather than a public key. 

Consider an example where Alice has a complex tax return to fill out and decides to use a tax return preparation service. The current pipeline is as follows:
\begin{enumerate}
    \item Alice sends her tax forms (W2, 1099, etc) to the service.
    \item The service prepares the tax return and sends it back to Alice.
    \item Alice sends the tax return to the IRS.
\end{enumerate}
However, in this process, the service has access to all of Alice's tax information, which is a privacy concern. 

Consider an alternate scenario where Alice encrypts her tax forms before sending them to the service. It would be ideal if the service could still prepare the tax return without decrypting the forms by performing operations on the encrypted data. In this case, the service learns nothing about Alice's tax information but is still able to prepare the tax return. This is the idea behind Fully Homomorphic Encryption (FHE). FHE was first presented in [Gentry09]~\cite{STOC:Gentry09} and has since been improved upon. We will present the construction from [GSW13]~\cite{C:GenSahWat13}.

FHE can be defined in either private or public key settings. The below construction is defined in the private key setting for message space $\mathbb{Z}_2$.

\begin{definition}[Fully Homomorphic Encryption (FHE)]
    A FHE scheme for message space $\mathbb{Z}_2$ and circuit class $\mathcal{C}$ is a tuple of algorithms $(\mathsf{Gen}, \mathsf{Enc}, \mathsf{Dec}, \mathsf{Eval})$ such that:
    \begin{itemize}
        \item $\mathsf{Gen}(1^\lambda) \rightarrow (\ek, \sk)$: The key generation algorithm takes a security parameter $\lambda$ and outputs a secret key $\sk$ and evaluation key $\ek$.
        \item $\mathsf{Enc}(\sk, m) \rightarrow c$: The encryption algorithm takes a public key $\sk$ and message $m$ and outputs a ciphertext $c$.
        \item $\mathsf{Dec}(\sk, c) \rightarrow m$: The decryption algorithm takes a secret key $\sk$ and ciphertext $c$ and outputs a message $m$.
        \item $\mathsf{Eval}(\ek, F, c_1, \ldots, c_l) \rightarrow c$: The evaluation algorithm takes an evaluation key $\ek$, a circuit $F \in \mathcal{C}$, and $l$ ciphertexts $c_1, \ldots, c_l$ and outputs a ciphertext $\tilde{c}$.
    \end{itemize}
\end{definition}

A FHE scheme satisfies the following properties:
\begin{itemize}
    \item \textbf{Correctness}: $\forall n \in \mathbb{N},\, \forall F \in \mathcal{C},\, \forall (\mu_1, \mu_2, \dots, \mu_l) \in \mathbb{Z}_2^l$,
    \begin{align*}
        \Pr[\mathsf{Dec}(\sk, \mathsf{Eval}(\ek, F, \mathsf{Enc}(\sk, \mu_1), \dots, \mathsf{Enc}(\sk, \mu_l))) &= F(\mu_1, \mu_2, \dots, \mu_l)] \\
        &= 1 - \text{negl}(\lambda)
    \end{align*}

    \item \textbf{Security}: The following two distributions are computationally indistinguishable:
    \begin{align*}
        \{(\ek, \mathsf{ct}_0): \mathsf{ct}_0 \gets \mathsf{Enc}(\sk, 0), (\ek, \sk) \gets \mathsf{Gen}(1^\lambda)\} \\
        \{(\ek, \mathsf{ct}_1): \mathsf{ct}_1 \gets \mathsf{Enc}(\sk, 1), (\ek, \sk) \gets \mathsf{Gen}(1^\lambda)\}
    \end{align*}

    \item \textbf{Compactness}: The size of the ciphertext $\mathsf{Eval}(\ek, F, c_1, \ldots, c_l)$ is equal to a fresh encryption of the output of the circuit $F$ on the plaintexts $\mu_1, \ldots, \mu_l$:
    \begin{align*}
        \forall i,\, c_i &= \mathsf{Enc}(\sk, \mu_i) \\
        \lvert \mathsf{Eval}(\ek, F, c_1, \ldots, c_l) \rvert &= \lvert \mathsf{Enc}(\sk, F(\mu_1, \ldots, \mu_l)) \rvert
    \end{align*}
\end{itemize}

The construction of FHE is based on the Learning With Errors (LWE) problem. The high-level construction is done in two steps:
\begin{enumerate}
    \item \textbf{Leveled FHE}: We first construct a leveled FHE scheme that can evaluate arbitrary circuits of bounded depth.
    \item \textbf{Bootstrapping}: We then use the leveled FHE scheme to construct a fully homomorphic encryption scheme.
\end{enumerate}

\subsection{Leveled FHE}
To present some intuition for how we get homomorphic properties, consider the following construction. Let $C \in \mathbb{Z}_q^{l \times l}$ be a matrix and $v \in \mathbb{Z}_q^l$ be an eigenvector of this matrix. The eigenvalue is chosen as the message being encrypted. \\
Given this, we can easily perform operations on ciphertexts that correspond to operations on the underlying plaintexts.
\begin{itemize}
    \item \textbf{Addition}: Given two ciphertexts $C_1$ and $C_2$, we have $(C_1 + C_2) v = C_1 v + C_2 v = (m_1 + m_2) v$.
    \item \textbf{Multiplication}: Given two ciphertexts $C_1$ and $C_2$, we have $(C_1 \cdot C_2) v = C_1 (C_2 v) = C_1 (m_2 v) = m_1 m_2 v$.
\end{itemize}
Note that this is not a secure construction as presented, since with enough samples we can solve a linear system to obtain the secret key. However, this gives us some intuition for how we can perform operations on encrypted data. To make this construction secure, we need to add noise to the ciphertexts (which is where LWE comes in). \\
However, a naive way of doing this does not work; suppose we have that $Cv = m v + e$ where $e$ is the (small) noise term. Then, even a single multiplication gives us $C_1 C_2 v = m_1 m_2 v + C_1 e_2$, where the noise term is no longer guaranteed to be small since $C$ has no such guarantees.

The construction of the leveled FHE scheme is as follows. We use the LWE problem with parameters $(n, m, q, \chi)$ where $n$ is the dimension of the secret key, $m$ is the dimension of the public key, $q$ is the modulus, and $\chi$ is the noise distribution. The scheme is defined for message space $\mathbb{Z}_2$. Additionally, set $l = (n+1) \log q$.
\begin{itemize}
    \item $\mathsf{Gen}(1^\lambda) \rightarrow (\ek, \sk)$: Sample $s' \gets \mathbb{Z}_q^{n}$ and set $s =  
    \begin{bmatrix}
        -s' \\
        1
    \end{bmatrix} \in \mathbb{Z}_q^{n+1}$. 

    \item $\mathsf{Enc}(\sk \in \mathbb{Z}_q^{n+1}, m \in \mathbb{Z}_2) \rightarrow C \in \mathbb{Z}_q^{l \times (n+1)}$: Sample $A \gets \mathbb{Z}_q^{l \times n}$ and $e \gets \chi^{l}$. Define $B = A \| As' + e$ and $C = B + m \cdot G$ for a fixed gadget~\cite{EC:MicPei12} matrix $G \in \mathbb{Z}_q^{l \times (n+1)}$.
    
    \begin{itemize}
        \item Note that by choice of $B$, we have that $Bs = A(-s') + A(s') + e = e$ is an encryption of zero (with noise). Similarly, $Cs = Bs + mGs = e + mGs$. 
        \item $G$ is a block matrix containing $(n+1)$ block column vectors of size $\log q$ each. Each vector is $g = (1, 2, 4, \ldots, 2^{\log q - 1})$. Concisely, we can define $G = I_{n+1} \otimes g$ where $\otimes$ is the Kronecker product.
        \item We also define a $\mathsf{Flatten}$ operation on the ciphertext that converts $C \in \mathbb{Z}_q^{l \times (n+1)}$ to $\tilde{C} \in \mathbb{Z}_q^{l \times l}$ by bit decomposing each element of $C$ and replacing the element with its bit vector. \\
        This ensures that each element of this matrix is a bit. Looking ahead, this allows us to multiply ciphertexts without too much noise growth.
    \end{itemize}
    
\end{itemize}
%% Add Eval + and x, noting the noise growth of each 
\begin{itemize}
    \item $\mathsf{Dec}(s, C)$: Compute $v = Cs$. If $||v||_\infty < q/4$, i.e. each entry of $v$ is less than $q/4$, output $0$, else output $1$. (Note the exact choice of threshold is somewhat arbitrary.)
    
    \item $\mathsf{Eval}(+, C_1, C_2)$: Output $C = C_1 + C_2$. 

    \begin{itemize}
        \item Notice that $Cs = C_1s + C_2s = (m_1 Gs + e_1) + (m_2 G + e_2) = (m_1 + m_2) Gs + (e_1 + e_2)$. 

        \item Thus if the original ciphertexts errors $||e_1||_{\infty}, ||e_2||_{\infty} \leq B$, then the new cipertext error is bounded by $2B$. 
    \end{itemize} 

    \item $\mathsf{Eval}(+, C_1, C_2)$: Output $C = \mathsf{Flatten}(C_1) \times C_2$. 

    \begin{itemize}
        \item We utilize the $\mathsf{Flatten}$ operation so that the dimensions match for matrix multiplication. 

        \item Notice that 
        \begin{align*}
            Cs & = \mathsf{Flatten}(C_1) \times C_2 s \\
            & = \mathsf{Flatten}(C_1) \times (m_2 G s + e_2) \\ 
            & = m_2 (\mathsf{Flatten}(C_1) G) s + \mathsf{Flatten}(C_1) e_2 \\ 
            & = m_2 (C_1 s) + \mathsf{Flatten}(C_1) e_2 \\
            & = m_2 (m_1 Gs + e_1) + \mathsf{Flatten}(C_1) e_2 \\ 
            & = (m_1 m_2) G s + (m_2 e_1 + \mathsf{Flatten}(C_1) e_2). 
        \end{align*} 

        \item Since each entry of $\mathsf{Flatten}(C_1)$ is in $\{0,1\}$, each entry of $\mathsf{Flatten}(C_1) e_2$ is a subset sum of $e_2 \in \mathbb{Z}_q^{\ell}$. Thus if the original ciphertexts errors $||e_1||_{\infty}, ||e_2||_{\infty} \leq B$, then the new ciphertext error is bounded by $(1+\ell)B$. 
    \end{itemize}
\end{itemize}

% Analyze overall noise growth 

Notice that as $\mathsf{Eval}$ is applied iteratively to ciphertexts in order to implement an arithmetic circuit of depth $d$, the noise will stay bounded by $(\ell+1)^d B$. 
As long as the error stays below the threshold used by $\mathsf{Dec}$, i.e. $(\ell+1)^d B << q/4$, then correctness will hold. 
Recall that $\ell = (n+1) \log(q)$ where $q$ is typically exponential in $n$. 
Then the error will be manageable as long as $d$ is sublinear in $n$, e.g. $d = n^{0.99}$. 

Further, notice that since swapping ordering of $C_1$ vs $C_2$ changes the noise growth of multiplication from $m_2 e_1 + \mathsf{Flatten}(C_1) e_2$ to $m_1 e_2 + \mathsf{Flatten}(C_2) e_1$, we can optimize based on which ciphertext started with more noise than the other. 
This can be leveraged to get polynomial noise growth instead of exponential. 

% Add Eval for NAND 

Now, we have cheated slightly in the above---notice that our construction for $\mathsf{Eval}(+, C_1, C_2)$ implements addition mod $q$, but we actually wanted addition mod 2. 
One way to address this is to simply implement NAND, which on its own is a complete gate set that can implement any circuit: 
\begin{itemize}
    \item $\mathsf{Eval}(\text{NAND}, C_1, C_2)$: Output $C = I - \mathsf{Flatten}(C_1) \times C_2$. 
\end{itemize}

Altogether, we have achieved \emph{leveled} FHE. 

% Present bootstrapping and how it ensures noise doesn't grow too large 

\subsection{Bootstrapping} 

The goal of bootstrapping is to reset the ciphertext's noise back to a lower level after it has built up too much. 
This is done by letting the server re-encrypt the ciphertext \emph{without} using the secret key $s$. 
How is this possible? 
Simply evaluate $\mathsf{Dec}$ homomorphically! 
In particular, let $P_{\mathsf{Dec}, C}$ be a circuit which on input $s$ outputs $m = \mathsf{Dec}(s, C)$. 
Let $s_1, s_2, \ldots, s_{\ell} \in {0,1}$ be $s$ written in binary. 
We will set the evaluation key $ek = (ek_1, \ldots, ek_{\ell})$ for $ek_i = \mathsf{Enc}(s, s_i)$, i.e. it gives us an encrypted copy of the secret key. 
Then whenever we need to reduce the noise of a ciphertext $C$, we can compute $P_{\mathsf{Dec}, C}$ and run $\mathsf{Eval}(P_{\mathsf{Dec}, C}, ek_1 \ldots ek_\ell)$ to get a fresh ciphertext $\hat{C}$. 
Notice that the noise of $\hat{C}$ only depends on the noise of the input ciphertexts $ek_i$, which are fresh, and the depth $d_P$ of the circuit $P_{\mathsf{Dec}, C}$, which is independent of the noise of $C$. 
Thus $\hat{C}$ will have noise bounded by $(\ell + 1)^{d_P} B$. 
As long as our leveled scheme supports circuits of depth $d_p+1$, we can achieve arbitrary depth by bootstrapping after each operation. 
Of course, we'd like to avoid this as much as possible since doing the decryption operation homomorphically is expensive. 

% Circular security issue 

In order for this approach to work, security needs to be maintained even though we're releasing encryptions of the secret key. 
This is known as \emph{circular security}. 
There are no known attacks on circular security for GSW and other commonly used leveled FHE constructions. 
However, they have not been proven to be circularly secure, and there are encryption schemes for which circular security is known to not hold for certain cycle lengths. 
An alternative is using a new secret key to encrypt the old secret key. 

% Construction of CCA-2 secure IBE 

\section{CCA-2 Secure Encryption from IBE} 

Finally, we will see how to utilize CPA secure IBE to construct not only CCA-1 but CCA-2 secure encryption. 
Let $(G, K, E, D)$ be a CPA secure IBE scheme and $(\mathsf{Gen}_{sig}, \mathsf{Sign}, \mathsf{Verify})$ be a digital signature scheme. 
Then construct a CCA-2 secure encryption scheme as follows: 
\begin{itemize}
    \item $\mathsf{KeyGen}(1^n)$: 
    \begin{enumerate}
        \item Compute $(pk, sk) \leftarrow G(1^n)$. 
        
        \item Output $pp = pk$ and $msk = sk$. 
    \end{enumerate}

    \item $\mathsf{Enc}(pk, m)$: 
    \begin{enumerate}
        \item \textbf{Compute $(vk_s, sk_s) \leftarrow Gen_{sig}(1^n)$.}
        
        \item \textbf{Let $id = vk_s$.}

        \item Let $c = E(pp, id, m)$. 

        \item \textbf{Compute $\sigma = \mathsf{Sign}(sk_s, c)$.} 

        \item \textbf{Output $(id, c, \sigma)$.} 
    \end{enumerate}

    \item $\mathsf{Dec}(sk, (id,c,\sigma))$: 
    \begin{enumerate}
        \item \textbf{If $\mathsf{Verify}(id, c, \sigma) = 0$ then abort.} 

        \item Compute $sk_{id} \leftarrow K(sk, id)$. 

        \item Output $D(sk_{id}, c)$. 
    \end{enumerate}
\end{itemize}
The bolded lines are the ones that have been changed compared to the CCA-1 construction. 
The changes restrict you to only generating ciphertexts for $id$'s you sampled yourself. 
Note that for this construction we need slightly stronger forgery protection for the digital signature scheme than we've considered previously. 
Namely, in the forgery security game the attacker is now allowed to output $(m^*, \sigma^*)$ as long as the tuple is fresh, even if $m^*$ on its own is not. 
Notice that this stronger notion holds for any deterministic signature scheme, including the BLS scheme we saw. 
It is also sufficient for our purposes here to utilize a one-time signature scheme. 
\chapter{Proving Computation Integrity}
\section{Zero-Knowledge Proofs}
Traditional Euclidean style proofs allow us to prove veracity of statements to others. However, such proof systems have two shortcomings: (1) the running time of the verifier needs to grow with the length of the proof, and (2) the proof itself needs to be disclosed to the verifier. In this chapter, we will provide methods enabling provers to prove veracity of statements of their choice to verifiers while avoiding the aforementioned limitations. In realizing such methods we will allow the prover and verifier to be probabilistic and also allow them to interact with each other.\footnote{Formally, they can be modeled as interactive PPT Turing Machines.}

\section{Interactive Proofs}
\begin{definition} {\normalfont\textbf{(Interactive Proof System)}} For a language L we have an \textit{interactive proof system} if $\exists$ a pair of algorithms (or better, interacting machines) $(\mathcal{P},\mathcal{V})$, where $\mathcal{V}$ \DIFdelbegin \DIFdel{is polynomial in \mbox{%DIFAUXCMD
$|x|$
}%DIFAUXCMD
}\DIFdelend \DIFaddbegin \DIFadd{runs in polynomial time in its input length}\DIFaddend , and both can flip coins, such that:
		\begin{itemize}
			\item Completeness: $\forall x\in L$
		$$\Pr_{\mathcal{P},\mathcal{V}} \left[Output_{\mathcal{V}}(\mathcal{P}(x) \leftrightarrow \mathcal{V}(x))=1\right]=1,$$
			\item Soundness: $\forall x\notin L$, $\forall \mathcal{P}^*$ \DIFaddbegin \DIFadd{(unbounded)
		}\DIFaddend $$\Pr_{\mathcal{V}} \left[Output_{\mathcal{V}}(\mathcal{P}^*(x) \leftrightarrow \mathcal{V}(x))=1\right]<\DIFdelbegin \DIFdel{neg}\DIFdelend \DIFaddbegin \DIFadd{\mathsf{negl}}\DIFaddend (|x|),$$
		\end{itemize} where $Output_{\mathcal{V}}(\mathcal{P}(x) \leftrightarrow \mathcal{V}(x))$ denotes the output of $\mathcal{V}$ in the interaction between $\mathcal{P}$ and $\mathcal{V}$ where both parties get $x$ as input.
		We stress that $\mathcal{P}$ and $\mathcal{P}^*$ can be computationally unbounded. 
  \end{definition}
\DIFaddbegin \DIFadd{We can also consider other variants of this definition, e.g. imperfect completeness.
}\DIFaddend 

\DIFdelbegin \paragraph{\DIFdel{Interactive Proof for Graph Non-Isomorphism (GNI).}} %DIFAUXCMD
\addtocounter{paragraph}{-1}%DIFAUXCMD
\DIFdelend \DIFaddbegin \DIFadd{To understand the above definition, let's consider two languages over a pair of graphs \mbox{%DIFAUXCMD
$G_0$
}%DIFAUXCMD
and \mbox{%DIFAUXCMD
$G_1$
}%DIFAUXCMD
: 
}\begin{enumerate}
	\item \DIFadd{Graph Isomorphism (GI): }\DIFaddend We say that two graphs $G_0$ and $G_1$ are isomorphic, denoted $G_0 \cong G_1$, if $\exists$ an isomorphism $f: V(G_0) \rightarrow V(G_1)$ s.t. $(u,v)\in E(G_0)$ iff $(f(u),f(v))\in E(G_1)$, where $V(G)$ and $E(G)$ are the vertex and edge sets of some graph $G$. \DIFaddbegin \DIFadd{Let \mbox{%DIFAUXCMD
$GI=\lbrace(G_0,G_1)|\  G_0\cong G_1\rbrace$
}%DIFAUXCMD
be the language that consists of pairs of graphs that are isomorphic.
	}\item \DIFadd{Graph Non-Isomorphism (GNI): }\DIFaddend On the other hand, $G_0$ and $G_1$ are said to be non-isomorphic, $G_0 \ncong G_1$, if $\nexists$ any such $f$, and \DIFaddbegin \DIFadd{let }\DIFaddend $GNI=\lbrace(G_0,G_1)|\  G_0\ncong G_1\rbrace$ be the language that consists of pairs of graphs that are not isomorphic.
\DIFaddbegin \end{enumerate}
 \DIFaddend 

\DIFdelbegin \DIFdel{GNI is }\DIFdelend \DIFaddbegin \paragraph{\DIFadd{Trivial Case of Graph Isomorphism (GI).}} \DIFadd{A prover can easily prove to a verifier that two graphs are isomorphic by directly providing the isomorphism \mbox{%DIFAUXCMD
$f$
}%DIFAUXCMD
between them. The verifier can confirm the isomorphism in time polynomial in the size of the graphs (i.e., its input); hence we have perfect completeness. If the graphs are not isomorphic, no isomorphism exists, and the verifier always rejects; we have perfect soundness too. This proof was trivial, and we didn't even require (back-and-forth) interaction. We now look at a more interesting case of GNI. Moreover, looking ahead, we will see more interesting properties that we can ask of proof systems, like zero-knowledge, where this trivial proof system terribly fails, and we will revisit the GI problem to see how we can prove it with zero-knowledge.
}

\paragraph{\DIFadd{Interactive Proof for Graph Non-Isomorphism (GNI).}}  \DIFadd{Unlike the case of GI, for GNI, there is no succinct (e.g., linear in the size of graphs) information that the prover can provide, and consequently, no ``efficient'' (polynomial time in the graphs) verification that the verifier can do. This is where the }{\em \DIFadd{power of interaction}} \DIFadd{comes in. In other words, since GNI is }\DIFaddend not believed to have short proofs\DIFdelbegin \DIFdel{so an interactive }\DIFdelend \DIFaddbegin \DIFadd{, an }{\em \DIFadd{interactive}} \DIFaddend proof could offer \DIFdelbegin \DIFdel{a }\DIFdelend \DIFaddbegin \DIFadd{the }\DIFaddend prover a mechanism to prove to a polynomially bounded verifier that two graphs are non-isomorphic. \DIFaddbegin \DIFadd{We will now describe an interactive proof system for GNI.
}\DIFaddend 

The intuition \DIFdelbegin \DIFdel{behind a protocol to accomplish the above task }\DIFdelend is simple. Consider a verifier that randomly \DIFdelbegin \DIFdel{rename }\DIFdelend \DIFaddbegin \DIFadd{renames }\DIFaddend the vertices of one of the graphs and give it to the prover. Can the prover\DIFaddbegin \DIFadd{, }\DIFaddend given the relabeled graph\DIFaddbegin \DIFadd{, }\DIFaddend figure out which graph did the verifier start with?  If $G_0$ and $G_1$ were not isomorphic\DIFdelbegin \DIFdel{then an unbounded }\DIFdelend \DIFaddbegin \DIFadd{, then an unbounded-time }\DIFaddend prover can figure this out. However, in case $G_0$ and $G_1$ {are} isomorphic\DIFdelbegin \DIFdel{then the distribution }\DIFdelend \DIFaddbegin \DIFadd{, then the distributions }\DIFaddend resulting form random relabelings of $G_0$ and $G_1$ are actually identical. Therefore, even an unbounded prover has no way of distinguishing which graph the verifier started with. So the prover has only a $\frac12$ probability of guessing which graph the verifier started with. Note that by repeating this process we can reduce the success probability of a cheating prover to negligible\DIFaddbegin \footnote{\DIFadd{This strategy is called soundness amplification by ``sequential'' repetition. Later, we might cover proof systems where we additionally consider ``parallel'' repetition to achieve different security properties.}}\DIFaddend . More formally\DIFaddbegin \DIFadd{, given a claim \mbox{%DIFAUXCMD
$(G_0,G_1)\in GNI$
}%DIFAUXCMD
, we define the following interactive proof system}\DIFaddend :

%		If they consistently answer correctly, however, it would be hard to remain skeptical against $G_0 \ncong G_1$ as they beat the odds to almost impossible limits.  And so this interaction can ``prove" very strongly to the verifier that $(G_0,G_1)\in$ GNI.  Consider the protocol we can define from this:

		\begin{center}
			\includegraphics[scale=.51094]{Old Scribe Notes/GNI_IP_Protocol.png}
		\end{center}

		\begin{itemize}
			\item Completeness: If $(G_0,G_1)\in$ GNI, then the unbounded $\mathcal{P}$ can distinguish isomorphism of $G_0$ against those of $G_1$ and can always return the correct $b'$.  Thus, $\mathcal{V}$ will always output 1 for this case.
			\item Soundness: If $(G_0,G_1)\notin$ GNI, then it is equiprobable that $H$ is a random isomorphism of $G_0$ as it is \DIFaddbegin \DIFadd{of }\DIFaddend $G_1$\DIFaddbegin \DIFadd{, }\DIFaddend and so $\mathcal{P}$'s guess for $b'$ can be correct only with a probability $\frac{1}{2}$\DIFaddbegin \footnote{\DIFadd{A curious reader might notice that the challenge bit \mbox{%DIFAUXCMD
$b$
}%DIFAUXCMD
sampled by \mbox{%DIFAUXCMD
$\mathcal{V}$
}%DIFAUXCMD
is information-theoretically hidden from \mbox{%DIFAUXCMD
$\mathcal{P}$
}%DIFAUXCMD
(hidden in \mbox{%DIFAUXCMD
$H$
}%DIFAUXCMD
) when \mbox{%DIFAUXCMD
$\mathcal{P}$
}%DIFAUXCMD
's claim is false. This is similar to what we saw in Hash Proof Systems before.}}\DIFaddend .  Repeating this protocol $k$ times\DIFaddbegin \DIFadd{, with fresh verifier randomness each time, }\DIFaddend means the probability of guessing the correct $b'$ for all $k$ interactions is $\frac{1}{2^k}$.  And so the probability of $\mathcal{V}$ outputting \DIFdelbegin \DIFdel{0 }\DIFdelend \DIFaddbegin \DIFadd{\mbox{%DIFAUXCMD
$0$
}%DIFAUXCMD
}\DIFaddend (e.g. rejecting $\mathcal{P}$'s proof at the first sign of falter) is $1-\frac{1}{2^k}$.  
\DIFaddbegin 

		\DIFaddend \end{itemize}

		%The interaction between prover and verifier captures the notion of a proof system for GNI, a problem previously not known to have an efficient method of proof.  By interacting, we can prove what seemed impossible to efficiently prove before!
		\DIFaddbegin \DIFadd{To conclude, the interactive proof system we described above enabled something that wasn't possible without interaction. 
}\DIFaddend 

\section{Zero Knowledge Proofs}
\DIFaddbegin \DIFadd{We saw a crucial difference between GI and GNI: in GI, the prover already holds a succinct proof to back its claim, we call this a ``witness'', while in GNI, no such succinct proof exists (i.e., there is nothing that the prover can directly send to the verifier to back its claim). From this point onwards, we exclusively focus on the languages of the first kind, i.e., where a witness for the claim exists; these languages cover a vast majority of the use-cases of verifiable computation, and are formalized as follows:
}\DIFaddend 

\begin{definition} {\normalfont\textbf{(NP-Verifier)}} A language L has an NP-verifier if $\exists$ a verifier $\mathcal{V}$ that is polynomial time in $|x|$ such that:
		\begin{itemize}
			\item Completeness: $\forall x\in L,\ \exists\ a\ proof\ \pi\ s.t.\ \mathcal{V}(x,\pi)=1$
			\item Soundness: $\forall x \notin L$\DIFaddbegin \DIFadd{, and }\DIFaddend $\forall$ purported proof $\pi$\DIFaddbegin \DIFadd{, }\DIFaddend we have $\mathcal{V}(x,\pi)=0$
		\end{itemize}
  \end{definition}

		That is, the conventional idea of a proof is formalized in terms of what a computer can efficiently verify.\DIFdelbegin \DIFdel{So a set of statements considered true (e. g. in a language \mbox{%DIFAUXCMD
$L$
}%DIFAUXCMD
) is complete and sound if a proof can be written down that can be ``easily" and rigorously verified if and only if a statement is in the language. }\DIFdelend \DIFaddbegin \smallskip %DIF > A language $L$ has an NP-verifier if for all claims of statements $\in L$, a proof can be written down that can be ``easily'' and ``rigorously'' verified if and only if a statement is in the language.
\DIFaddend 

	\DIFdelbegin %DIFDELCMD < \bigskip
%DIFDELCMD < %%%
\DIFdelend \noindent \DIFdelbegin \textbf{\DIFdel{Efficient Provers.}}
		%DIFAUXCMD
\DIFdel{Unfortunately (fortunately?), there aren't real-life instances of all-powerful provers that we know of.  And for cryptography we must make more reasonable assumptions about the provers. In this case we will assume provers are also bounded to be }\emph{\DIFdel{efficient}}%DIFAUXCMD
\DIFdelend \DIFaddbegin \textit{\DIFadd{Keeping the witness private}}\DIFadd{. The goal of a proof system is for the verifier to learn if the prover's claim is valid or not. Let's focus on what a verifier actually learns at the end of its interaction with the prover. In the trivial GI proof system we saw above, the verifier learns the entire isomorphism --- in other words, the verifier learns }{\em \DIFadd{everything}} \DIFadd{that the prover knew. 
	This is too much leakage. Imagine the prover holding some secret or valuable information (e.g., its secret key) which is leaked to the verifier. This is not desirable. We want the verifier to learn only the validity of the claim, and nothing more. This is where the notion of zero-knowledge comes in.
	For a proof system for a language with an NP-verifier, this translates to the verifier not learning the witness from the prover}\DIFaddend .

	\DIFdelbegin \DIFdel{Previously, if a prover wanted to prove that two graphs, \mbox{%DIFAUXCMD
$G_0$
}%DIFAUXCMD
and \mbox{%DIFAUXCMD
$G_1$
}%DIFAUXCMD
were isomorphic, it would use its all-powerfulness to find the isomorphic mapping between the two graphs and give it to the verifier to complete the proof. 
	But now, being computationally bounded, the prover is in the same boat as the verifier and can find a proof no better than the verifier can.
	In order for the prover to be able to prove something that the verifier cannot find out on their own, the prover must have some extra information.
	If, for example, the prover simply knew the isomorphism between the graphs, this would be the sufficient extra information it needs to enact the proof .  That's a rather boring proof though. We have interaction now!  Can't we do something fancier?
		}\DIFdelend \DIFaddbegin \DIFadd{We now revisit the GI problem}\footnote{\DIFadd{GI is not NP-complete.}} \DIFadd{for which an NP-verifier exists, as we saw earlier. Later, we will consider NP-complete languages like graph 3-coloring --- giving us proof systems for all of NP. }\smallskip
\DIFaddend 

	\DIFdelbegin \DIFdel{What if the prover wanted to prove that two graphs were isomorphic but didn't want to fully reveal the isomorphism that they know.  If they're lying and don't know an isomorphism is their a way we can exploit them again?
		}\DIFdelend \DIFaddbegin \noindent \textit{\DIFadd{Hiding witness for Graph Isomorphism}}\DIFadd{. We will build the ideas for our proof system with zero-knowledge gradually by iterating through a series of straw-man approaches. On the way, we will formally define zero knowledge. 
}\DIFaddend 

	When $G_0$ and $G_1$ are isomorphic, the isomorphism between them would be a \textit{witness}, $w$, to that fact, that can be used in the proof.  \DIFdelbegin \DIFdel{Unfortunately, the prover is being stubborn and won't just tell us that }\DIFdelend \DIFaddbegin \DIFadd{The prover doesn't want to reveal the }\DIFaddend isomorphism, $w:V(G_0)\rightarrow V(G_1)$, that they claim to have.  The prover is comfortable however giving us a ``scrambled\DIFdelbegin \DIFdel{" }\DIFdelend \DIFaddbegin \DIFadd{'' }\DIFaddend version, $\phi$, of $w$ as long as it doesn't leak any information about their precious $w$.  For example, the prover is willing to divulge $\phi = \pi \circ w$ where $\pi$ is a privately chosen random permutation of $|V|=|V(G_0)|=|V(G_1)|$ vertices.  Since $\pi$ renames vertices completely randomly, it scrambles what $w$ is doing entirely and $\phi$ is just a random permutation of $|V|$ elements.  At this point, we might be a little annoyed at the prover since we could have just created a random permutation on our own.  \DIFdelbegin \DIFdel{This might give us an idea on how to gain a little more information however, even though we gained none here:
		}\DIFdelend \DIFaddbegin \DIFadd{Let's look at why this is still a good starting point.
		}\DIFaddend 

	If we want to be convinced that $\phi$ really is of the form $\pi \circ w$, thus containing $w$ in its definition, and isn't just a completely random permuation, we can note that if it is of that form then $\phi(G_0)=\pi(w(G_0))=\pi(G_1)$ (since $w$ being an isomorphism implies that $w(G_0)=G_1$).  Note that we started with a mapping on input $G_0$ and ended with a mapping on input $G_1$.  With an isormphism, one could get from one graph to the other seamlessly; if the prover \textit{really} has the isomorphism it claims to have, then it should have no problem displaying this ability.  So, what if we force the prover to give us $H=\pi (G_1)$ just after randomly choosing its $\pi$ and then let it show us its ability to go from $G_1$ to $G_0$ with ease: give us a $\phi$ so that $\phi(G_0)=\pi(G_1)=H$.  The only way the prover can give a mapping that jumps from $G_0$ to $G_1$ \DIFdelbegin \DIFdel{in such a way }\DIFdelend is if they know an isomorphism; \DIFaddbegin \DIFadd{in fact, }\DIFaddend if the prover could find a $\phi$ efficiently but did \textit{not} know an isomorphism then they would have been able to see that $\pi^{-1}(\phi(G_0))=G_1$ and thus have $\pi^{-1}\circ\phi$ as an isomorphism from $G_0$ to $G_1$, which would contradict the assumed hardness of finding isomorphisms in the GI problem. So by forcing the prover to give us $H$\DIFaddbegin \DIFadd{, }\DIFaddend as we've defined\DIFaddbegin \DIFadd{, }\DIFaddend and to produce a $\phi$ so that $\phi(G_0)=H$, we've found a way to expose provers that don't really have an isomorphism and we can then be convinced that they really do know $w$ when they pass our test.  \DIFdelbegin \DIFdel{And }\DIFdelend \DIFaddbegin \DIFadd{Importantly, }\DIFaddend the prover didn't directly tell us $w$, so \DIFdelbegin \DIFdel{they may be able to salvage some secrecy!
		}\DIFdelend \DIFaddbegin \DIFadd{we are headed in the right direction.
		}\DIFaddend 

	But not everything is airtight about this interaction.  Why, for instance, would the prover be willing to provide $H=\pi(G_1)$ when they're trying to divulge as little information as possible?  The prover was comfortable giving us $\phi$ since we could have just simulated the process of getting a completely random permutation of vertices ourselves, but couldn't the additional information of $H$ reveal information about $w$?  At this point, \DIFdelbegin \DIFdel{the annoyed feeling may return as we realize that, }\DIFdelend \DIFaddbegin \DIFadd{if we look closely, we realize that }\DIFaddend $H=\pi(G_1)=\pi'(G_0)$, for some $\pi'$, is just a random isomorphic copy of $G_0$ \textit{and} $G_1$ as long as $G_0 \cong G_1$; we could have just chosen a random $\pi'$, set $H=\pi'(G_0)$, and let $\phi=\pi'$ and would have created our very own random isomorphic copy, $H$, of $G_1$ that satisfies our test condition $H=\phi(G_0)$\DIFaddbegin \DIFadd{, }\DIFaddend just like what we got from our interaction with the prover. \DIFdelbegin \DIFdel{We couldn't have gained any new information from the prover because we could have run the whole teston our own!
		}%DIFDELCMD < 

%DIFDELCMD < 		%%%
\DIFdel{Well, something must be wrong; we couldn't have been convinced of something without gaining }\textit{\DIFdel{any}} %DIFAUXCMD
\DIFdel{new information}\DIFdelend \DIFaddbegin \DIFadd{To our annoyance, the prover can easily fool this test}\DIFaddend . Indeed, the test has a hole in it: how can we force the prover to give us $H=\pi(G_1)$ like we asked?  If the prover is lying and it knows our test condition is to verify that $H=\phi(G_0)$, the prover might just cheat and give us $H=\pi(G_0)$ so it doesn't have to use knowledge of $w$ to switch from $G_1$ to $G_0$.  And, in fact, by doing this and sending $\phi=\pi$, the prover would fool us!

	To keep the prover on their toes, though, we can randomly switch whether \DIFdelbegin \DIFdel{or not }\DIFdelend we want $H$ to equal $\phi(G_0)$ or $\phi(G_1)$.  \DIFdelbegin \DIFdel{If, in }\DIFdelend \DIFaddbegin \DIFadd{In }\DIFaddend our interaction, the prover must first provide \DIFdelbegin \DIFdel{their }\DIFdelend $H=\pi(G_1)$ before we let them know which we want\DIFdelbegin \DIFdel{, they then lock themselves }\DIFdelend \DIFaddbegin \DIFadd{. By sending \mbox{%DIFAUXCMD
$H$
}%DIFAUXCMD
, the prover locks itself }\DIFaddend into a commitment to either $G_0$ or $G_1$ \DIFdelbegin \DIFdel{depending on whether they're trying to cheat or not, respectively. They only have }\DIFdelend \DIFaddbegin \DIFadd{if it is cheating, but if not, then it can easily move between the two graphs. A prover only has }\DIFaddend a $50\%$ chance of committing to the same case we want on a given round and so, if they don't have $w$ to deftly switch between $G_0$ and $G_1$ to always answer correctly, they again have to be an extremely lucky guesser if they're trying to lie.

	\DIFdelbegin \DIFdel{Again}\DIFdelend \DIFaddbegin \DIFadd{Therefore}\DIFaddend , we've created an interactive scheme that can catch dishonest provers with probability 1-$\frac{1}{2^k}$ and where we always believe honest provers!

		\begin{center}
			\includegraphics[scale=.51094]{Old Scribe Notes/GI_ZK_Protocol.png}
		\end{center}

		\begin{itemize}
			\item Completeness: If \DIFdelbegin \DIFdel{\mbox{%DIFAUXCMD
$(G_0,G_1)\in$
}%DIFAUXCMD
GI }\DIFdelend \DIFaddbegin \DIFadd{\mbox{%DIFAUXCMD
$(G_0,G_1)\in GI$
}%DIFAUXCMD
}\DIFaddend and $\mathcal{P}$ knows $w$, then whether $\mathcal{V}$ chooses $b=0$ or 1, $\mathcal{P}$ can always give the correct $\phi$ which, by definition, will always result in $H=\phi(G_b)$ and so $\mathcal{V}$ will always output 1.
			\item Soundness: If \DIFdelbegin \DIFdel{\mbox{%DIFAUXCMD
$(G_0,G_1)\notin$
}%DIFAUXCMD
GI}\DIFdelend \DIFaddbegin \DIFadd{\mbox{%DIFAUXCMD
$(G_0,G_1)\notin GI$
}%DIFAUXCMD
}\DIFaddend , then $\mathcal{P}$ can only cheat, as discussed earlier, if the original $H$ it commits to ends up being $\pi(G_b)$ for the $b$ that is randomly chosen at the next step.  Since $b$ isn't even chosen yet, this can only happen by chance with probability $\frac{1}{2}$.  And so the probability $\mathcal{V}$ outputs \DIFdelbegin \DIFdel{0 }\DIFdelend \DIFaddbegin \DIFadd{\mbox{%DIFAUXCMD
$0$
}%DIFAUXCMD
}\DIFaddend is $1-\frac{1}{2^k}$ for $k$ rounds.
		\end{itemize}

		\DIFdelbegin \DIFdel{And so, again, we 've correctly captured the idea of a proof by having this interaction. But there's a strange feeling that may be lingering around us.
		..
		}\DIFdelend \DIFaddbegin \DIFadd{We have just shown that what we have so far is an interactive proof system. We now think of how the notion of zero-knowledge can be formalized here.
		}\DIFaddend 

		As a verifier, we've seen some things in interacting with the prover.  Surely, clever folks like ourselves must be able to glean \textit{some} information about $w$ after seeing enough to thoroughly convince us that the prover knows $w$.  We've first seen $H$, and we've also seen the random $b$ that we chose, along with $\phi$ at the end;  this is our whole view of information during the interaction.  But we're more bewildered than annoyed this time when we realize we could have always just chosen $b$ and $\phi$ randomly and set $H=\phi(G_b)$ on our own.  Again, everything checks out when $G_0 \cong G_1$ and we could have produced everything that we saw during the interaction before it even began.  That is, the distribution of the random variable triple ($H$, $b$, $\phi$) is identical whether it is what we saw from the prover during the interaction or it is yielded from the solitary process we just described.  We've just constructed a complete interactive proof system that entirely convinces us of the prover's knowledge of $w$, yet we could have simulated the whole experience on our own!  We couldn't have \DIFdelbegin \DIFdel{gain }\DIFdelend \DIFaddbegin \DIFadd{gained }\DIFaddend any knowledge about $w$ since we didn't see anything we couldn't have manufactured on \DIFaddbegin \DIFadd{our }\DIFaddend own, yet we are entirely convinced that $(G_0,G_1)\in$ GI and that $\mathcal{P}$ knows $w$!  And so the prover has proven something to us yet has given us absolutely zero additional knowledge!

		This may feel very surprising or as if you've been swindled by a fast talker, and it very much should feel this way; it was certainly an amazing research discovery!  But this is true, and it can be made rigorous\DIFdelbegin \DIFdel{:
		}\DIFdelend \DIFaddbegin \DIFadd{, as we do next.
		}\DIFaddend 

		We should first be sure what we want out of this new proof system.  We of course want it to be complete and sound so that we accept proofs iff they're true.  But we also want the verifier to gain zero knowledge from the interaction; that is, the verifier should have been able to simulate the whole experience on its own without the verifier.
		Finally, we would also like all witnesses to a true statement to each be sufficient to prove the veracity of that statement and so we let $R$ be the relation s.t. $x \in L$ iff $\exists$ a witness $w$ s.t. $(x,w)\in R$.  We can then gather all witness by defining $R(x)$ to be the set of all such witnesses. \DIFaddbegin \DIFadd{We will first look at a weaker notion of zero-knowledge, called }\textit{\DIFadd{Honest Verifier Zero Knowledge}} \DIFadd{(HVZK), where we only require that an }{\em \DIFadd{honest}} \DIFadd{verifier (follows the protocol steps) does not learn anything from the prover. We will then move on to the stronger notion of }\textit{\DIFadd{Zero Knowledge}} \DIFadd{(ZK), where we extend this to all verifiers, including malicious verifiers.
		}\DIFaddend 

		\begin{definition} {\normalfont\textbf{(Honest Verifier Zero Knowledge Proof [HVZK])}} 
			\DIFdelbegin \DIFdel{For a language L we have a }\DIFdelend \DIFaddbegin \DIFadd{\mbox{%DIFAUXCMD
$(\mathcal{P},\mathcal{V})$
}%DIFAUXCMD
is a }\DIFaddend (perfect) \DIFdelbegin \textit{\DIFdel{HVZK proof system}} %DIFAUXCMD
\DIFdelend \DIFaddbegin \DIFadd{HVZK proof system for a language \mbox{%DIFAUXCMD
$L$
}%DIFAUXCMD
}\DIFaddend w.r.t. witness relation $R$ if 
			$\exists$ \DIFdelbegin \DIFdel{an interactive proof system, \mbox{%DIFAUXCMD
$(\mathcal{P},\mathcal{V})$
}%DIFAUXCMD
s.t. \mbox{%DIFAUXCMD
$\exists$
}%DIFAUXCMD
}\DIFdelend a PPT machine $\mathcal{S}$ (called the simulator) s.t. $\forall x \in L$, $\forall w\in R(x)$\DIFaddbegin \DIFadd{, }\DIFaddend the following distributions are \DIFdelbegin \DIFdel{identical:
		}\DIFdelend \DIFaddbegin \DIFadd{(identical) indistinguishable:
		}\DIFaddend $$\DIFaddbegin \DIFadd{\{}\DIFaddend View_{\mathcal{V}}(\mathcal{P}(x,w) \leftrightarrow \mathcal{V}(x))\DIFaddbegin \DIFadd{\} \approx \{\mathcal{S}(x)\}}\DIFaddend $$
		\DIFdelbegin \begin{displaymath}\DIFdel{\mathcal{S}(x)}\end{displaymath}%DIFAUXCMD
%DIFDELCMD < 		%%%
\DIFdelend where $View_{\mathcal{V}}(\mathcal{P}(x,w) \leftrightarrow \mathcal{V}(x))$ is the random coins of $\mathcal{V}$ and all the messages $\mathcal{V}$ saw.
  \end{definition}

\begin{remark}
In the above definition, $View_{\mathcal{V}}(\mathcal{P}(x,w) \leftrightarrow \mathcal{V}(x))$ contains both the random coins of $\mathcal{V}$ and all the messages that $\mathcal{V}$ saw, because they together constitute the view of $\mathcal{V}$, and they are correlated. If the random coins of $\mathcal{V}$ are not included in the definition of $View_{\mathcal{V}}(\mathcal{P}(x,w) \leftrightarrow \mathcal{V}(x))$, then even if $\mathcal{S}$ can generate all messages that $\mathcal{V}$ saw with the same distribution as in the real execution, the verifier may still be able to distinguish the two views using its random coins.
\end{remark}

\DIFaddbegin \begin{remark}
\DIFadd{In the above definition, the order of quantifiers is quite important. We cannot change it to: \mbox{%DIFAUXCMD
$\forall x \in L$
}%DIFAUXCMD
, \mbox{%DIFAUXCMD
$\forall w\in R(x)$
}%DIFAUXCMD
, \mbox{%DIFAUXCMD
$\exists$
}%DIFAUXCMD
a PPT machine \mbox{%DIFAUXCMD
$\mathcal{S}$
}%DIFAUXCMD
. This is because the definition would be trivially satisfied by hardcoding the witness \mbox{%DIFAUXCMD
$w$
}%DIFAUXCMD
in the simulator \mbox{%DIFAUXCMD
$\mathcal{S}$
}%DIFAUXCMD
.
}\end{remark}

\DIFadd{To prove HVZK property of the GI proof system we described earlier, we now construct a simulator \mbox{%DIFAUXCMD
$\mathcal{S}$
}%DIFAUXCMD
, with input \mbox{%DIFAUXCMD
$G_0, G_1$
}%DIFAUXCMD
, as follows:
}\begin{enumerate}
	\item \DIFadd{Sample \mbox{%DIFAUXCMD
$b\in\{0,1\}$
}%DIFAUXCMD
uniformly at random.
	}\item \DIFadd{Sample a random permutation \mbox{%DIFAUXCMD
$\sigma$
}%DIFAUXCMD
of the vertices.
	}\item \DIFadd{Set \mbox{%DIFAUXCMD
$H \gets \sigma(G_b)$
}%DIFAUXCMD
.
	}\item \DIFadd{Output \mbox{%DIFAUXCMD
$(H, b, \sigma)$
}%DIFAUXCMD
.
}\end{enumerate}
\DIFadd{It is straightforward to see that this simulator produces the same distribution as the real interaction between the prover and the verifier. This is because \mbox{%DIFAUXCMD
$H = \sigma(G_b) = \sigma'(G_{1-b})$
}%DIFAUXCMD
, i.e., \mbox{%DIFAUXCMD
$H$
}%DIFAUXCMD
is a random permutation of both \mbox{%DIFAUXCMD
$G_0$
}%DIFAUXCMD
amd \mbox{%DIFAUXCMD
$G_1$
}%DIFAUXCMD
. 
	}

\DIFadd{To recap: }\DIFaddend There is an interesting progression of the requirements of a proof system: Completeness, Soundness, and the Zero Knowledge property.  Completeness first cares that a prover-verifier pair exist and can capture all true things as a team that works together; they both honestly obey the protocol trying prove true statements.  Soundness, however, assumes that the prover is a liar and cares about having a strong enough verifier that can stand up to any type of prover and not be misled.  Finally, Zero Knowledge assumes that the verifier is hoping to glean information from the proof to learn the prover's secrets and this requirement makes sure the prover is clever enough that it gives no information away in its proof. \DIFdelbegin %DIFDELCMD < 

%DIFDELCMD < 		%%%
\DIFdelend Unlike the soundness' \DIFdelbegin \DIFdel{requirment }\DIFdelend \DIFaddbegin \DIFadd{requirement }\DIFaddend for a verifier to combat \textit{all} malicious provers, HVZK is only concerned with the verifier in the original prover-verifier pair that follows the set protocol. Verifiers that stray from the protocol or cheat, however, are captured in the natural generalization to Zero Knowledge proofs.
%DIF < 		These are mostly discussed (including auxiliary inputs) in the next class, although the first definition is given below:
%DIF < 		
%DIF < 		{\definition {\normalfont\textbf{(Zero Knowledge Proof [ZK])}} For a language L we have a (perfect) \textit{ZK proof system} w.r.t. witness relation $R$ if $\exists$ an interactive proof system, $(\mathcal{P},\mathcal{V})$ s.t. $\exists$ a PPT machine $\mathcal{S}$ (called the simulator) s.t. $\forall x \in L$, $\forall w\in R(x)$, $\forall \mathcal{V}^*$, the following distributions are identical:
%DIF < 		$$View_{\mathcal{V}^*}(\mathcal{P}(x,w) \leftarrow \mathcal{V}^*(x))$$
%DIF < 		$$\mathcal{S}^{\mathcal{V}^*}(x)$$
%DIF < 		where $\mathcal{S}^{\mathcal{V}^*}(x)$ is the simulator with oracle access to $\mathcal{V}^*$.}

%DIF <  !TEX root = collection.tex
\DIFaddbegin \section{\DIFadd{Zero-Knowledge for Graph Isomorphism}}
\DIFaddend 

\DIFaddbegin \DIFadd{In this section, we construct our final zero-knowledge interactive proof system for GI where we don't have to assume an honest verifier for zero knowledge to hold. The proof system construction is exactly the same as the one we saw earlier. What changes is the definition of zero knowledge, and therefore, the simulator. 
}

\begin{definition} {\normalfont\textbf{\DIFadd{(Zero Knowledge Proof }[\DIFadd{ZK}]\DIFadd{)}}} 
	\DIFadd{\mbox{%DIFAUXCMD
$(\mathcal{P},\mathcal{V})$
}%DIFAUXCMD
is a (perfect) ZK proof system for a language \mbox{%DIFAUXCMD
$L$
}%DIFAUXCMD
w.r.t. witness relation \mbox{%DIFAUXCMD
$R$
}%DIFAUXCMD
if \mbox{%DIFAUXCMD
$\forall$
}%DIFAUXCMD
PPT machines \mbox{%DIFAUXCMD
$\mathcal{V}^*$
}%DIFAUXCMD
,
	\mbox{%DIFAUXCMD
$\exists$
}%DIFAUXCMD
a PPT machine \mbox{%DIFAUXCMD
$\mathcal{S}$
}%DIFAUXCMD
(called the simulator) s.t. \mbox{%DIFAUXCMD
$\forall x \in L$
}%DIFAUXCMD
, \mbox{%DIFAUXCMD
$\forall w\in R(x)$
}%DIFAUXCMD
, the following distributions are (identical) indistinguishable:
}$$\DIFadd{\{View_{\mathcal{V^*}}(\mathcal{P}(x,w) \leftrightarrow \mathcal{V^*}(x))\} \approx \{\mathcal{S}(x)\}}$$
\DIFadd{where \mbox{%DIFAUXCMD
$View_{\mathcal{V^*}}(\mathcal{P}(x,w) \leftrightarrow \mathcal{V^*}(x))$
}%DIFAUXCMD
is the random coins of \mbox{%DIFAUXCMD
$\mathcal{V^*}$
}%DIFAUXCMD
and all the messages \mbox{%DIFAUXCMD
$\mathcal{V^*}$
}%DIFAUXCMD
saw.
}\end{definition}
\begin{remark}
	\DIFadd{Note that the order of quantifiers matters again. The definition would be stronger if we switch the order to: \mbox{%DIFAUXCMD
$\exists$
}%DIFAUXCMD
a PPT machine \mbox{%DIFAUXCMD
$\mathcal{S}$
}%DIFAUXCMD
(called the simulator) s.t. \mbox{%DIFAUXCMD
$\forall$
}%DIFAUXCMD
PPT machines \mbox{%DIFAUXCMD
$\mathcal{V}^*$
}%DIFAUXCMD
. This is because the same simulator would need to work for all possible efficient verifiers. Interestingly, the simulator we construct below for GI satisfies this stronger definition too. In fact, most simulators we know work for all verifiers (i.e. black-box simulators). It wasn't until 2008 that Boaz Barak showed that we can also construct non-black-box simulators. 
}\end{remark}

\DIFadd{Recall our protocol for graph isomorphism: the interaction is \mbox{%DIFAUXCMD
$P(x,w) \leftrightarrow V(x)$
}%DIFAUXCMD
where \mbox{%DIFAUXCMD
$x$
}%DIFAUXCMD
represents graphs \mbox{%DIFAUXCMD
$G_0 = (V, E_0)$
}%DIFAUXCMD
and \mbox{%DIFAUXCMD
$G_1 = (V, E_1)$
}%DIFAUXCMD
and \mbox{%DIFAUXCMD
$w$
}%DIFAUXCMD
represents a permutation on \mbox{%DIFAUXCMD
$V$
}%DIFAUXCMD
such that \mbox{%DIFAUXCMD
$w (G_0) = G_1$
}%DIFAUXCMD
.
}

\begin{enumerate}
\item \DIFadd{\mbox{%DIFAUXCMD
$\mathcal{P}$
}%DIFAUXCMD
samples a random permutation \mbox{%DIFAUXCMD
$\sigma: V \to V$
}%DIFAUXCMD
and sends the graph \mbox{%DIFAUXCMD
$H = \sigma(G_1)$
}%DIFAUXCMD
to \mbox{%DIFAUXCMD
$V$
}%DIFAUXCMD
.
}

\item \DIFadd{\mbox{%DIFAUXCMD
$\mathcal{V}$
}%DIFAUXCMD
samples a random bit \mbox{%DIFAUXCMD
$b$
}%DIFAUXCMD
and sends it to \mbox{%DIFAUXCMD
$\mathcal{P}$
}%DIFAUXCMD
.
}

\item \DIFadd{If \mbox{%DIFAUXCMD
$b = 1$
}%DIFAUXCMD
, then \mbox{%DIFAUXCMD
$\mathcal{P}$
}%DIFAUXCMD
defines a permutation \mbox{%DIFAUXCMD
$\tau$
}%DIFAUXCMD
to be \mbox{%DIFAUXCMD
$\sigma$
}%DIFAUXCMD
. If \mbox{%DIFAUXCMD
$b = 0$
}%DIFAUXCMD
, then instead \mbox{%DIFAUXCMD
$\tau = \sigma \circ w$
}%DIFAUXCMD
. \mbox{%DIFAUXCMD
$\mathcal{P}$
}%DIFAUXCMD
then sends \mbox{%DIFAUXCMD
$\tau$
}%DIFAUXCMD
to \mbox{%DIFAUXCMD
$V$
}%DIFAUXCMD
.
}

\item \DIFadd{\mbox{%DIFAUXCMD
$\mathcal{V}$
}%DIFAUXCMD
verifies that \mbox{%DIFAUXCMD
$\tau(G_b) = H$
}%DIFAUXCMD
and accepts if so.
}

\end{enumerate}

\DIFadd{The reason the simulator for HVZK doesn't work anymore is because a malicious verifier \mbox{%DIFAUXCMD
$\mathcal{V^*}$
}%DIFAUXCMD
could pick its bit \mbox{%DIFAUXCMD
$b$
}%DIFAUXCMD
from a biased distribution (e.g. \mbox{%DIFAUXCMD
$b$
}%DIFAUXCMD
can be a function of \mbox{%DIFAUXCMD
$H$
}%DIFAUXCMD
seen by \mbox{%DIFAUXCMD
$\mathcal{V^*}$
}%DIFAUXCMD
).
}

\DIFadd{For zero knowledge, consider the following simulator}\footnote{\DIFadd{The simulator satisfies a stronger ZK property where the same simulator works for all \mbox{%DIFAUXCMD
$\mathcal{V^*}$
}%DIFAUXCMD
. Refer to the remark above for more details.}} \DIFadd{\mbox{%DIFAUXCMD
$S$
}%DIFAUXCMD
with input \mbox{%DIFAUXCMD
$G_0$
}%DIFAUXCMD
and \mbox{%DIFAUXCMD
$G_1$
}%DIFAUXCMD
(with vertex set \mbox{%DIFAUXCMD
$V$
}%DIFAUXCMD
) and verifier \mbox{%DIFAUXCMD
$V^*$
}%DIFAUXCMD
:
}

\begin{enumerate}
\item \DIFadd{For \mbox{%DIFAUXCMD
$i = 1\dots T$
}%DIFAUXCMD
; \mbox{%DIFAUXCMD
$T=\mathsf{poly}(n)$
}%DIFAUXCMD
:
}\begin{enumerate}
	\item \DIFadd{Sample a bit \mbox{%DIFAUXCMD
$b$
}%DIFAUXCMD
uniformly at random.
}

	\item \DIFadd{Sample a permutation \mbox{%DIFAUXCMD
$\sigma: V \to V$
}%DIFAUXCMD
uniformly at random
	}

	\item \DIFadd{Send \mbox{%DIFAUXCMD
$H = \sigma (G_b)$
}%DIFAUXCMD
to \mbox{%DIFAUXCMD
$\mathcal{V^*}$
}%DIFAUXCMD
.
}

	\item \DIFadd{Receive \mbox{%DIFAUXCMD
$b'$
}%DIFAUXCMD
from \mbox{%DIFAUXCMD
$\mathcal{V^*}$
}%DIFAUXCMD
.
}

	\item \DIFadd{If \mbox{%DIFAUXCMD
$b=b'$
}%DIFAUXCMD
, then output \mbox{%DIFAUXCMD
$(H, b, \sigma)$
}%DIFAUXCMD
and terminate. Otherwise, continue the loop.
}\end{enumerate}
\item \DIFadd{Output \mbox{%DIFAUXCMD
$\bot$
}%DIFAUXCMD
.
}

\end{enumerate}

\DIFadd{We construct a sequence of hybrids to prove zero-knowledge. Let \mbox{%DIFAUXCMD
$H_0$
}%DIFAUXCMD
define the interaction between \mbox{%DIFAUXCMD
$\mathcal{P}$
}%DIFAUXCMD
and \mbox{%DIFAUXCMD
$\mathcal{V^*}$
}%DIFAUXCMD
: \mbox{%DIFAUXCMD
$\mathcal{P}(x=(G_0, G_1), w)\leftrightarrow \mathcal{V^*}(x)$
}%DIFAUXCMD
. Define \mbox{%DIFAUXCMD
$H_1$
}%DIFAUXCMD
as follows: 
}\begin{enumerate}
	\item \DIFadd{For \mbox{%DIFAUXCMD
$i = 1\dots T$
}%DIFAUXCMD
:
}\begin{enumerate}
	\item \DIFadd{Sample a bit \mbox{%DIFAUXCMD
$b^*$
}%DIFAUXCMD
uniformly at random.
}

	\item \DIFadd{Run \mbox{%DIFAUXCMD
$H_0$
}%DIFAUXCMD
, i.e., \mbox{%DIFAUXCMD
$\mathcal{P}(x=(G_0, G_1), w)\leftrightarrow \mathcal{V^*}(x)$
}%DIFAUXCMD
.
	}

	\item \DIFadd{If \mbox{%DIFAUXCMD
$b^*=0$
}%DIFAUXCMD
, output \mbox{%DIFAUXCMD
$View_{\mathcal{V^*}}(\mathcal{P}(x,w) \leftrightarrow \mathcal{V^*}(x))$
}%DIFAUXCMD
.
	}

	\item \DIFadd{If \mbox{%DIFAUXCMD
$b^*=1$
}%DIFAUXCMD
, continue with the loop.
}\end{enumerate}
\item \DIFadd{Output \mbox{%DIFAUXCMD
$\bot$
}%DIFAUXCMD
.
}\end{enumerate}

\DIFadd{The hybrid \mbox{%DIFAUXCMD
$H_1$
}%DIFAUXCMD
produces identical distribution as \mbox{%DIFAUXCMD
$H_0$
}%DIFAUXCMD
except if \mbox{%DIFAUXCMD
$b^*$
}%DIFAUXCMD
in all the \mbox{%DIFAUXCMD
$T$
}%DIFAUXCMD
iterations is \mbox{%DIFAUXCMD
$0$
}%DIFAUXCMD
; note that \mbox{%DIFAUXCMD
$\mathcal{P}(x=(G_0, G_1), w)\leftrightarrow \mathcal{V^*}(x)$
}%DIFAUXCMD
doesn't depend on \mbox{%DIFAUXCMD
$b^*$
}%DIFAUXCMD
. This happens with probability at most \mbox{%DIFAUXCMD
$1/2^T$
}%DIFAUXCMD
. Next, we define \mbox{%DIFAUXCMD
$H_2$
}%DIFAUXCMD
where we change the logic of which transcript is thrown away. In each iteration, if \mbox{%DIFAUXCMD
$b^* = b$
}%DIFAUXCMD
, where \mbox{%DIFAUXCMD
$b$
}%DIFAUXCMD
is part of the transcript \mbox{%DIFAUXCMD
$\mathcal{P}(x=(G_0, G_1), w)\leftrightarrow \mathcal{V^*}(x)$
}%DIFAUXCMD
, we output the transcript; otherwise, we continue with the loop. Note again that \mbox{%DIFAUXCMD
$b^*$
}%DIFAUXCMD
is still not used in any of the transcript (interaction). Therefore, distribution produced by \mbox{%DIFAUXCMD
$H_2$
}%DIFAUXCMD
is identical to \mbox{%DIFAUXCMD
$H_1$
}%DIFAUXCMD
because the interaction hasn't changed at all. Finally, we construct our last hybrid \mbox{%DIFAUXCMD
$H_3$
}%DIFAUXCMD
where the simulator \mbox{%DIFAUXCMD
$S$
}%DIFAUXCMD
is run. The distribution generated by \mbox{%DIFAUXCMD
$H_3$
}%DIFAUXCMD
is identical to \mbox{%DIFAUXCMD
$H_2$
}%DIFAUXCMD
by same argument we used for the HVZK simulator. }\bigskip


\noindent\textbf{\DIFadd{Efficient Provers.}} \DIFadd{So far, we have considered unbounded provers, but unfortunately (fortunately?), there aren't real-life instances of all-powerful provers that we know of. And for cryptography we must make more reasonable assumptions about the provers.  We will now assume provers are also bounded to be }\emph{\DIFadd{efficient}}\DIFadd{. Note that if the prover in our GI proof system already holds the isomorphism, then generating the proof only takes polynomial time, and it is easy to see that it satisfies the definition below. }\smallskip

\DIFaddend \begin{definition}[Efficient Prover Zero-Knowledge Proof]
We say $(P, V)$ is an efficient prover zero-knowledge proof system for a language $L$ and relation $R_L$ if \begin{enumerate}

\item The prover $P$ runs in polynomial time.

\item The protocol is \emph{complete}. That is, for every $x \in L$ there exists a witness $w \in R_L (x)$ such that $$\Pr [P(x,w) \leftrightarrow V(x) \ \emph{accepts}] = 1.$$

\item The protocol is \emph{sound} against unbounded provers. That is, for $\forall x \notin L$, we have $$\Pr [P^*(x,w) \leftrightarrow V(x) \ \emph{rejects}] \geq 1/2$$ for any prover $P^*$ of arbitrary computation power and any witness $w$.

\item There exists an expected polynomial time probabilistic machine $S$ (a simulator) such that for all PPT $V^*$, for all $x \in L, w \in R_L (x), z \in \{ 0, 1 \}^*$ we have $$\{ View_{V^*} (P(x,w) \leftrightarrow V^* (x,z)) \} \simeq_c \{ S^{V^*} (x,z) \} $$ \end{enumerate}

\end{definition}

The soundness probability can be amplified to be greater than any $1 - 1/2^k$, for arbitrary $k > 0$, by repeating the proof $k$ times. More precisely, we construct an efficient prover zero-knowledge proof system $(\tilde P, \tilde V)$ which repeats $(P,V)$ independently for $k$ times, and $\tilde V$ accepts if and only if $V$ accepts in all the executions.

It is easy to see that $\tilde P$ runs in polynomial time and that the protocol is complete.
Moreover, it has the following soundness guarantee:
for $\forall x \notin L$,
\begin{align*}
& \Pr \left[\tilde P^*(x,w) \leftrightarrow \tilde V(x) \ \text{rejects}\right]\\
= & 1- \Pr \left[\forall 1\leq i\leq k, P^*_i(x,w) \leftrightarrow V(x) \ \text{accepts}\right] \\
= & 1- \prod_{i=1}^k \Pr \left[P^*_i(x,w) \leftrightarrow V(x) \ \text{accepts}\right] \\
\geq& 1-\frac{1}{2^k}
\end{align*}
for any prover $\tilde P^*=(P^*_1, \cdots, P^*_k)$ of arbitrary computation power and any witness $w$.

Finally, it  is zero-knowledge, namely, there exists an expected PPT $\tilde S$ such that for all PPT $\tilde V^*$, and for all $x \in L, w \in R_L (x), z \in \{ 0, 1 \}^*$,
$$\left\{ View_{\tilde V^*} (\tilde P(x,w) \leftrightarrow \tilde V^* (x,z)) \right\} \simeq_c \left\{ \tilde S^{\tilde V^*} (x,z) \right\}.$$
The construction of $\tilde S$ is repeating $S$ for $k$ times. We prove by hybrid argument that the above two distributions are indistinguishable. $H_i$ is defined to be the output of repeating $S$ for the first $i$ executions with $\tilde V^*$ and repeating $P$ for the rest $k-i$ executions. Then $H_0$ is the left distribution and $H_k$ is the right one. Any attacher that can distinguish the above two distributions leads to an attacker that can distinguish $H_{i-1}$ and $H_{i}$ for some $1\leq i \leq k$, which violates the zero-knowledge property of the original proof system $(P,V)$.

\DIFdelbegin %DIFDELCMD < \bigskip
%DIFDELCMD <     %%%
\DIFdel{The }\DIFdelend \DIFaddbegin \DIFadd{Similar to what we saw in previous definitions of zero-knowlege, the }\DIFaddend order of the quantifiers in item 4 matters.
If we quantify over $x$ and $w$ before quantifying over the simulator,
then we could hard-code  $x$ and $w$ into our simulator. That is, for all $x \in L, w \in R_L (x)$, there exists an expected polynomial time probabilistic machine $S_{x,w}$ such that for all PPT $V^*$ and $z \in \{ 0, 1 \}^*$,
$$\{ View_{V^*} (P(x,w) \leftrightarrow V^* (x,z)) \} \simeq_c \{ S_{x,w}^{V^*} (x,z) \} $$
Since we would like our simulator to be universal,  this is not acceptable.

If we quantify first over the verifier $V^*$ and then over simulators $S$, then this variant is considered as \emph{non-black-box zero-knowledge}. Our standard definition is considered as \emph{black-box zero-knowledge}. There  also exist variants that use statistical indistinguishability rather than computational indistinguishability.

The $z$ in item 4 is considered as \emph{auxiliary input}. The auxiliary input is crucial for the above argument of soundness amplification.

We will discuss the importance of requiring expected polynomial time in the next section. \DIFaddbegin \bigskip
%DIF > 		These are mostly discussed (including auxiliary inputs) in the next class, although the first definition is given below:
%DIF > 		
%DIF > 		{\definition {\normalfont\textbf{(Zero Knowledge Proof [ZK])}} For a language L we have a (perfect) \textit{ZK proof system} w.r.t. witness relation $R$ if $\exists$ an interactive proof system, $(\mathcal{P},\mathcal{V})$ s.t. $\exists$ a PPT machine $\mathcal{S}$ (called the simulator) s.t. $\forall x \in L$, $\forall w\in R(x)$, $\forall \mathcal{V}^*$, the following distributions are identical:
%DIF > 		$$View_{\mathcal{V}^*}(\mathcal{P}(x,w) \leftarrow \mathcal{V}^*(x))$$
%DIF > 		$$\mathcal{S}^{\mathcal{V}^*}(x)$$
%DIF > 		where $\mathcal{S}^{\mathcal{V}^*}(x)$ is the simulator with oracle access to $\mathcal{V}^*$.}
\DIFaddend 


\DIFdelbegin \section{\DIFdel{Graph Isomorphism}}
    %DIFAUXCMD
\addtocounter{section}{-1}%DIFAUXCMD
\DIFdelend %DIF >  !TEX root = collection.tex

\DIFdelbegin \DIFdel{Recall our protocol for graph isomorphism: the interaction is \mbox{%DIFAUXCMD
$P(x,w) \leftrightarrow V(x)$
}%DIFAUXCMD
where \mbox{%DIFAUXCMD
$x$
}%DIFAUXCMD
represents graphs \mbox{%DIFAUXCMD
$G_0 = (V, E_0)$
}%DIFAUXCMD
and \mbox{%DIFAUXCMD
$G_1 = (V, E_1)$
}%DIFAUXCMD
and \mbox{%DIFAUXCMD
$w$
}%DIFAUXCMD
represents a permutation \mbox{%DIFAUXCMD
$\pi$
}%DIFAUXCMD
on \mbox{%DIFAUXCMD
$V$
}%DIFAUXCMD
such that \mbox{%DIFAUXCMD
$\pi (G_0) = G_1$
}%DIFAUXCMD
.
    }%DIFDELCMD < 

%DIFDELCMD <     \begin{enumerate}
\begin{enumerate}%DIFAUXCMD
%DIFDELCMD <     \item %%%
\item%DIFAUXCMD
\DIFdel{\mbox{%DIFAUXCMD
$P$
}%DIFAUXCMD
samples a random permutation \mbox{%DIFAUXCMD
$\sigma: V \to V$
}%DIFAUXCMD
and sends the graph \mbox{%DIFAUXCMD
$H = \sigma(G_1)$
}%DIFAUXCMD
to \mbox{%DIFAUXCMD
$V$
}%DIFAUXCMD
.
    }%DIFDELCMD < 

%DIFDELCMD <     \item %%%
\item%DIFAUXCMD
\DIFdel{\mbox{%DIFAUXCMD
$V$
}%DIFAUXCMD
samples a random bit \mbox{%DIFAUXCMD
$b$
}%DIFAUXCMD
and sends it to \mbox{%DIFAUXCMD
$P$
}%DIFAUXCMD
.
    }%DIFDELCMD < 

%DIFDELCMD <     \item %%%
\item%DIFAUXCMD
\DIFdel{If \mbox{%DIFAUXCMD
$b = 1$
}%DIFAUXCMD
, then \mbox{%DIFAUXCMD
$P$
}%DIFAUXCMD
defines a permutation \mbox{%DIFAUXCMD
$\tau$
}%DIFAUXCMD
to be \mbox{%DIFAUXCMD
$\sigma$
}%DIFAUXCMD
. If \mbox{%DIFAUXCMD
$b = 0$
}%DIFAUXCMD
, then instead \mbox{%DIFAUXCMD
$\tau = \sigma \circ \pi$
}%DIFAUXCMD
. \mbox{%DIFAUXCMD
$P$
}%DIFAUXCMD
then sends \mbox{%DIFAUXCMD
$\tau$
}%DIFAUXCMD
to \mbox{%DIFAUXCMD
$V$
}%DIFAUXCMD
.
    }%DIFDELCMD < 

%DIFDELCMD <     \item %%%
\item%DIFAUXCMD
\DIFdel{\mbox{%DIFAUXCMD
$V$
}%DIFAUXCMD
verifies that \mbox{%DIFAUXCMD
$\tau(G_b) = H$
}%DIFAUXCMD
and accepts if so.
    }%DIFDELCMD < 


\end{enumerate}%DIFAUXCMD
%DIFDELCMD <     \end{enumerate}
%DIFDELCMD <     

%DIFDELCMD <     %%%
\DIFdel{We will show that this is an efficient prover zero-knowledge proof system. It is clear that if \mbox{%DIFAUXCMD
$G_0$
}%DIFAUXCMD
and \mbox{%DIFAUXCMD
$G_1$
}%DIFAUXCMD
are isomorphic, then this protocol will succeed with probability 1.
    }%DIFDELCMD < 

%DIFDELCMD <     %%%
\DIFdel{For soundness, observe that if \mbox{%DIFAUXCMD
$G_0$
}%DIFAUXCMD
is not isomorphic to \mbox{%DIFAUXCMD
$G_1$
}%DIFAUXCMD
, then the graph \mbox{%DIFAUXCMD
$H$
}%DIFAUXCMD
that \mbox{%DIFAUXCMD
$P$
}%DIFAUXCMD
sends to \mbox{%DIFAUXCMD
$V$
}%DIFAUXCMD
in step 1 of the protocol can be isomorphic to at most one of \mbox{%DIFAUXCMD
$G_0$
}%DIFAUXCMD
or \mbox{%DIFAUXCMD
$G_1$
}%DIFAUXCMD
. Since \mbox{%DIFAUXCMD
$V$
}%DIFAUXCMD
samples a bit \mbox{%DIFAUXCMD
$b$
}%DIFAUXCMD
uniformly at random in step 2, then there is a probability of at most 1/2 that \mbox{%DIFAUXCMD
$P$
}%DIFAUXCMD
can produce a valid isomorphism in step 3.
    }%DIFDELCMD < 

%DIFDELCMD <     %%%
\DIFdel{For zero knowledge, consider the following simulator \mbox{%DIFAUXCMD
$S$
}%DIFAUXCMD
with input \mbox{%DIFAUXCMD
$G_0$
}%DIFAUXCMD
and \mbox{%DIFAUXCMD
$G_1$
}%DIFAUXCMD
(with vertex set \mbox{%DIFAUXCMD
$V$
}%DIFAUXCMD
) and verifier \mbox{%DIFAUXCMD
$V^*$
}%DIFAUXCMD
:
    }%DIFDELCMD < 

%DIFDELCMD <     \begin{enumerate}
\begin{enumerate}%DIFAUXCMD
%DIFDELCMD <     \item %%%
\item%DIFAUXCMD
\DIFdel{Guess a bit \mbox{%DIFAUXCMD
$b$
}%DIFAUXCMD
uniformly at random.
    }%DIFDELCMD < 

%DIFDELCMD <     \item %%%
\item%DIFAUXCMD
\DIFdel{Sample a permutation \mbox{%DIFAUXCMD
$\pi: V \to V$
}%DIFAUXCMD
uniformly at random and send \mbox{%DIFAUXCMD
$\pi (G_b)$
}%DIFAUXCMD
to \mbox{%DIFAUXCMD
$V^*$
}%DIFAUXCMD
.
    }%DIFDELCMD < 

%DIFDELCMD <     \item %%%
\item%DIFAUXCMD
\DIFdel{Receive \mbox{%DIFAUXCMD
$b'$
}%DIFAUXCMD
form \mbox{%DIFAUXCMD
$V^*$
}%DIFAUXCMD
.
    }%DIFDELCMD < 

%DIFDELCMD <     \item %%%
\item%DIFAUXCMD
\DIFdel{If \mbox{%DIFAUXCMD
$b=b'$
}%DIFAUXCMD
, then output \mbox{%DIFAUXCMD
$(\pi (G_b), b, \pi)$
}%DIFAUXCMD
and terminate. Otherwise, restart at step 1.
    }%DIFDELCMD < 


\end{enumerate}%DIFAUXCMD
%DIFDELCMD <     \end{enumerate}
%DIFDELCMD <     

%DIFDELCMD <     %%%
\DIFdel{Note that if \mbox{%DIFAUXCMD
$G_0 \simeq G_1$
}%DIFAUXCMD
, then \mbox{%DIFAUXCMD
$\pi(G_b)$
}%DIFAUXCMD
is statistically independent of \mbox{%DIFAUXCMD
$b$
}%DIFAUXCMD
because \mbox{%DIFAUXCMD
$b$
}%DIFAUXCMD
and \mbox{%DIFAUXCMD
$\pi$
}%DIFAUXCMD
are sampled uniformly. Thus, with probability 1/2, \mbox{%DIFAUXCMD
$V^*$
}%DIFAUXCMD
will output \mbox{%DIFAUXCMD
$b$
}%DIFAUXCMD
so on average, two attempts will be needed before \mbox{%DIFAUXCMD
$S$
}%DIFAUXCMD
terminates. It follows that \mbox{%DIFAUXCMD
$S$
}%DIFAUXCMD
will terminate in }\emph{\DIFdel{expected}} %DIFAUXCMD
\DIFdel{polynomial time.
    }%DIFDELCMD < 

%DIFDELCMD <     %%%
\DIFdel{Since \mbox{%DIFAUXCMD
$b$
}%DIFAUXCMD
is sampled uniformly at random, \mbox{%DIFAUXCMD
$\pi (G_b)$
}%DIFAUXCMD
is uniformly distributed with all graphs of the form \mbox{%DIFAUXCMD
$\sigma (G_1)$
}%DIFAUXCMD
where \mbox{%DIFAUXCMD
$\sigma$
}%DIFAUXCMD
is sampled uniformly at random from permutations on \mbox{%DIFAUXCMD
$V$
}%DIFAUXCMD
. Thus, the output \mbox{%DIFAUXCMD
$\pi(G_b)$
}%DIFAUXCMD
in our simulator will be identically distributed with the output \mbox{%DIFAUXCMD
$H$
}%DIFAUXCMD
in our graph isomorphism protocol.
    }%DIFDELCMD < 

%DIFDELCMD <     %%%
\DIFdel{In step 3 of our graph isomorphism protocol, note that \mbox{%DIFAUXCMD
$\tau$
}%DIFAUXCMD
is distributed uniformly at random. This is because composing a uniformly random permutation with a fixed permutation will not change its distribution. Thus \mbox{%DIFAUXCMD
$\tau$
}%DIFAUXCMD
will be identically distributed with \mbox{%DIFAUXCMD
$\pi$
}%DIFAUXCMD
in our simulator. It follows that the transcripts outputted by our simulator will be identically distributed with the transcripts produced by the graph isomorphism protocol.
    }%DIFDELCMD < 

%DIFDELCMD <     %%%
\DIFdelend \section{Zero-Knowledge for NP}

An $n$-coloring of a graph $G = (A, E)$ is a function $c: A \to \{1, \ldots, n \}$ such that if $(i, j) \in E$, then $c(i) \neq c(j)$. So we want to paint each vertex of a graph a certain color so that the endpoints of any edge are colored differently.

In the graph 3-coloring problem (3COL), we are given a graph and asked if there exists a 3-coloring. In this section, we will provide a computational zero knowledge proof for 3COL. It is a fact that 3COL is NP-complete, so any problem in NP has a polynomial time reduction to 3COL. Thus, by giving a zero knowledge proof for 3COL, we will show that there are zero knowledge proofs for all of NP.

We will first give a high-level description of a zero-knowledge protocol for 3COL. Suppose a prover $P$ wants to convince a verifier $V$ that his graph $G$ is 3-colorable without revealing what the coloring $c$ actually is. If the three colors we use are red, green, and blue, then note that if we colored all the red vertices blue, all the green vertices red, and all the blue vertices green, we would still have a valid 3-coloring. In fact, if $\phi$ was any permutation on the color set of red, green, and blue, then $\phi \circ c$ would be a valid 3-coloring of $G$.

$P$ asks $V$ to leave the room and then samples a random permutation $\phi$ of the three colors. He colors the vertices of $G$ according to $\phi \circ c$, then covers all the vertices with cups. At this point, $P$ invites $V$ back into the room. $V$ is allowed to pick one edge and then uncover the two endpoints of the edge. If the colors on the two endpoints are the same, then $V$ rejects $P$'s claim that the graph is 3-colorable.

If the colors on the two endpoints are different, then $V$ leaves the room again, $P$ samples $\phi$ randomly, and the process repeats itself. Certainly if $G$ is actually 3-colorable, then $V$ will never reject the claim. If $G$ is not 3-colorable, then there will always be an edge with endpoints that are colored identically and $V$ will eventually uncover such an edge.

Note that $V$ does not gain any information on the coloring because it is masked by a (possibly) different random permutation every time $V$ uncovers an edge. Of course this protocol depends on $P$ not being able to quickly recolor the endpoints of an edge after removing the cups. This is why we need commitment schemes.

\subsection{Commitment Schemes}

We want to construct a protocol between a sender and a receiver where the sender sends a bit to the receiver, but the receiver will not know the value of this bit until the sender chooses to "open" the data that he sent. Of course, this protocol is no good unless the receiver can be sure that the sender was not able to change the value of his bit in between when the receiver first obtained the data and when the sender chose to open it.

\begin{definition}
A \emph{commitment scheme} is a PPT machine $C$ taking input $(b,r)$ that satisfies two properties: \begin{itemize}
\item (perfect binding) For all $r, s$, we have $C(0,r) \neq C(1,s)$.

\item (computational hiding) $\{ C(0, U_n) \} \simeq_c \{ C(1, U_n) \}$

\end{itemize}
\end{definition}

So for the sender to "open" the data, he just has to send his value of $r$ to the receiver. We say that $r$ is a \emph{decommitment} for $C(x,r)$. Why do we require perfect binding instead of just statistical binding? If there existed even a single pair $r, s$ where $C(0,r) = C(1,s)$, then the sender could cheat. If he wished to reveal a bit value of 0 then he could just offer $r$ and if he wished to reveal a bit value of 1 then he could just offer $s$.

We can use injective one-way functions to construct commitment schemes.

\begin{theorem}
If injective one-way functions exist, then so do commitment schemes.
\end{theorem}
\proof{We can let $f$ be an injective one-way function. Recall from Lecture 3 that $f' (x, r) := (f(x), r)$ will also be an injective one-way function with hard-core bit $B(x,r) := \langle x, r \rangle$. We claim that $C(b,x,r) := (f'(x,r), b \oplus B(x,r))$ is a commitment scheme.

If $(x,r)  \neq (y,s)$ then $C(0,x,r) \neq C(0,y,s)$ because $f'$ is injective. Since $C(0,x,r) = (f'(x,r), B(x,r)) \neq (f'(x,r), \overline{B(x,r)}) = C(1,x,r)$, then $C$ satisfies perfect binding.

Suppose $D$ can distinguish $C(0, U_n)$ from $C(1, U_n)$. Then we can distinguish $B(x,r)$ from $\overline{B(x,r)}$ given $f'(x,r)$ which contradicts the fact that $B(x,r)$ is a hard-core bit for $f'(x,r)$. Thus, $C$ has the computational hiding property.}
\qed

\medskip
We can extend the definition of commitment schemes to hold for messages longer than a single bit. These commitment schemes will work by taking our commitment schemes for bits and concatenating them together. For the extended definition, we require that for any two messages $m_0$ and $m_1$ of the same length, the ensembles $\{ C(m_0, U_n) \}$ and $\{ C(m_1, U_n) \}$ are computationally indistinguishable.

\subsection{3COL Protocol}

Below we describe the protocol $P(x,z) \leftrightarrow V(x)$, where $x$ describes a graph $G = (\{1, \ldots, n \}, E)$ and $z$ describes a 3-coloring $c$:

\begin{enumerate}
\item $P$ picks a random permutation $\pi : \{ 1, 2, 3 \} \to \{ 1, 2, 3 \}$ and defines the 3-coloring $\beta := \pi \circ c$ of $G$. Using a commitment scheme $C$ for the messages $\{ 1, 2, 3 \}$, $P$ defines $\alpha_i = C(\beta(i), U_n)$ for each $i \in V$. $P$ sends $\alpha_1, \alpha_2, \ldots, \alpha_n$ to $V$.

\item $V$ uniformly samples an edge $e = (i, j) \in E$ and sends it to $P$.

\item $P$ opens $\alpha_i$ and $\alpha_j$.

\item $V$ will accept only if it received valid decommitments for $\alpha_i$ and $\alpha_j$, and if $\beta(i)$ and $\beta(j)$ are distinct and valid colors.

\end{enumerate}

It is clear that this protocol is PPT. If $G$ is not 3-colorable, then there will be at least a $1/|E|$ probability that $V$ will reject $P$'s claim in step 4. Since $|E| \leq n^2$ we can repeat the protocol polynomially many times to increase the rejection probability to at least 1/2.

We will now show that this protocol is zero-knowledge. We describe a simulator $S$ below, given a verifier $V^*$: \begin{enumerate}
\item Sample an edge $e = (i, j) \in E$ uniformly at random.

\item Assign $c_i$ and $c_j$ to have distinct values from $\{ 1, 2, 3 \}$ and do so uniformly at random. Set $c_k := 1$ for all $k \neq i, j$.

\item Compute $n$ random keys $r_1, \ldots, r_n$ and set $\alpha_i = C(c_i, r_i)$ for all $i$.

\item Let $e' \in E$ be the response of $V^*$ upon receiving $\alpha_1, \ldots, \alpha_n$.

\item If $e' \neq e$, then terminate and go back to step 1. Otherwise, proceed. If $S$ returns to step 1 more than $2n |E|$ times, then output $\sf{fail}$ and halt the program.

\item Print $\alpha_1, \ldots, \alpha_n, e$, send $r_i$ and $r_j$ to $V^*$ and then print whatever $V^*$ responds with.
\end{enumerate}

By construction, $S$ will run in polynomial time. However, sometimes it may output a $\sf{fail}$ message. We will show that this occurs with negligible probability.

Suppose that for infinitely many graphs $G$, $V^*$ outputs $e' = e$ in step 4 with probability less than $1/2|E|$. If this is true, then it is possible for us to break the commitment scheme $C$ that we use in $S$. Consider a modified version of $S$ called $\tilde{S}$, where in step 2 we set $c_i = 1$ for all $i$. Note that in this case, $V^*$ cannot distinguish between any of the edges so the probability that it returns $e' = e$ is $1/|E|$.

If we gave $V^*$ a set of commitments $\alpha_k = C(1, r_k)$ for random keys $r_k$, then we would be in the setting of $\tilde{S}$. If we gave $V^*$ the commitments $\alpha_k$ but with two of the values set to $C(c, r)$ and $C(c', r')$ where $c, c'$ are distinct random values from $\{ 1, 2, 3 \}$ and $r, r'$ are random keys, then we are in the setting of $S$. This implies that it possible to distinguish between these two commitment settings with a probability of at least $1/2|E|$ which is non-negligible. It follows that $V^*$ outputs $e' = e$ with probability less than $1/2|E|$ for only finitely many graphs $G$.

Thus, the probability that $S$ outputs $\sf{fail}$ in the end is less than $(1 - 1/2|E|)^{2n|E|} < 1/e^n$ which is negligible.

Now we need to argue that the transcripts generated by $S$ are computationally indistinguishable from the transcripts generated by $P \leftrightarrow V^*$. Again, we consider a modified version of $S$, called $S'$, given a 3-coloring of its input $G$ as auxiliary input. In step 2 of the simulation, $S'$ will choose a random permutation of the colors in its valid 3-coloring for the values of $c_i$ rather than setting all but two values $c_i$ and $c_j$ equal to 1. Note that this is how our protocol between $P$ and $V$ behaves.

Observe that $P \leftrightarrow V^*$ is computationally indistinguishable from $S'$ because $S'$ outputs $\sf{fail}$ with negligible probability. Thus, it suffices to show that $S$ and $S'$ are computationally indistinguishable. Again, we will suppose otherwise and argue that as a result we can distinguish commitments.

We consider two messages $m_0$ and $m_1$ of the same length where $m_0$ consists of $n-2$ instances of the message $1$ and two committed colors $c_i$ and $c_j$ (for a random edge $(i, j) \in E$) and $m_1$ consists of a committed random 3-coloring of $G$ (with a random edge $(i, j) \in E$) chosen. Observe that by feeding the former message to $V^*$ we are in the setting of $S'$ and by feeding the latter message to $V^*$ we are in the setting of $S$. If we could distinguish those two settings, then we could distinguish the commitments for $m_0$ and $m_1$. This contradiction completes our argument that our 3-coloring protocol is zero-knowledge.
%DIF < % !TEX root = collection.tex

\section{Zero-Knowledge for NP}

\noindent\textbf{Efficient Provers.} Now we shift our focus to zero-knowledge proofs for NP-complete languages. So far, we have considered unbounded provers, but unfortunately (fortunately?), there aren't real-life instances of all-powerful provers that we know of. And for cryptography we must make more reasonable assumptions about the provers.  We will now assume provers are also bounded to be \emph{efficient}. \smallskip

\begin{definition}[Efficient Prover Zero-Knowledge Proof]
We say $(P, V)$ is an efficient prover zero-knowledge proof system for a language $L$ and relation $R_L$ if \begin{enumerate}

\item The prover $P$ runs in polynomial time.

\item The protocol is \emph{complete}. That is, for every $x \in L$ there exists a witness $w \in R_L (x)$ such that $$\Pr [P(x,w) \leftrightarrow V(x) \ \emph{accepts}] = 1.$$

\item The protocol is \emph{sound} against unbounded provers. That is, for $\forall x \notin L$, we have $$\Pr [P^*(x,w) \leftrightarrow V(x) \ \emph{rejects}] \geq 1/2$$ for any prover $P^*$ of arbitrary computation power and any witness $w$.

\item There exists an expected polynomial time probabilistic machine $S$ (a simulator) such that for all PPT $V^*$, for all $x \in L, w \in R_L (x), z \in \{ 0, 1 \}^*$ we have $$\{ View_{V^*} (P(x,w) \leftrightarrow V^* (x,z)) \} \simeq_c \{ S^{V^*} (x,z) \} $$ \end{enumerate}

\end{definition}

The soundness probability can be amplified to be greater than any $1 - 1/2^k$, for arbitrary $k > 0$, by repeating the proof $k$ times. More precisely, we construct an efficient prover zero-knowledge proof system $(\tilde P, \tilde V)$ which repeats $(P,V)$ independently for $k$ times, and $\tilde V$ accepts if and only if $V$ accepts in all the executions.

It is easy to see that $\tilde P$ runs in polynomial time and that the protocol is complete.
Moreover, it has the following soundness guarantee:
for $\forall x \notin L$,
\begin{align*}
& \Pr \left[\tilde P^*(x,w) \leftrightarrow \tilde V(x) \ \text{rejects}\right]\\
= & 1- \Pr \left[\forall 1\leq i\leq k, P^*_i(x,w) \leftrightarrow V(x) \ \text{accepts}\right] \\
= & 1- \prod_{i=1}^k \Pr \left[P^*_i(x,w) \leftrightarrow V(x) \ \text{accepts}\right] \\
\geq& 1-\frac{1}{2^k}
\end{align*}
for any prover $\tilde P^*=(P^*_1, \cdots, P^*_k)$ of arbitrary computation power and any witness $w$.

Finally, it  is zero-knowledge, namely, there exists an expected PPT $\tilde S$ such that for all PPT $\tilde V^*$, and for all $x \in L, w \in R_L (x), z \in \{ 0, 1 \}^*$,
$$\left\{ View_{\tilde V^*} (\tilde P(x,w) \leftrightarrow \tilde V^* (x,z)) \right\} \simeq_c \left\{ \tilde S^{\tilde V^*} (x,z) \right\}.$$
The construction of $\tilde S$ is repeating $S$ for $k$ times. We prove by hybrid argument that the above two distributions are indistinguishable. $H_i$ is defined to be the output of repeating $S$ for the first $i$ executions with $\tilde V^*$ and repeating $P$ for the rest $k-i$ executions. Then $H_0$ is the left distribution and $H_k$ is the right one. Any attacher that can distinguish the above two distributions leads to an attacker that can distinguish $H_{i-1}$ and $H_{i}$ for some $1\leq i \leq k$, which violates the zero-knowledge property of the original proof system $(P,V)$.


\bigskip
The order of the quantifiers in item 4 matters.
If we quantify over $x$ and $w$ before quantifying over the simulator,
then we could hard-code  $x$ and $w$ into our simulator. That is, for all $x \in L, w \in R_L (x)$, there exists an expected polynomial time probabilistic machine $S_{x,w}$ such that for all PPT $V^*$ and $z \in \{ 0, 1 \}^*$,
$$\{ View_{V^*} (P(x,w) \leftrightarrow V^* (x,z)) \} \simeq_c \{ S_{x,w}^{V^*} (x,z) \} $$
Since we would like our simulator to be universal,  this is not acceptable.

If we quantify first over the verifier $V^*$ and then over simulators $S$, then this variant is considered as \emph{non-black-box zero-knowledge}. Our standard definition is considered as \emph{black-box zero-knowledge}. There  also exist variants that use statistical indistinguishability rather than computational indistinguishability.

The $z$ in item 4 is considered as \emph{auxiliary input}. The auxiliary input is crucial for the above argument of soundness amplification.

We will discuss the importance of requiring expected polynomial time in the next section. \bigskip



\noindent \textit{Graph 3-Coloring}. An $n$-coloring of a graph $G = (A, E)$ is a function $c: A \to \{1, \ldots, n \}$ such that if $(i, j) \in E$, then $c(i) \neq c(j)$. So we want to paint each vertex of a graph a certain color so that the endpoints of any edge are colored differently.

In the graph 3-coloring problem (3COL), we are given a graph and asked if there exists a 3-coloring. In this section, we will provide a computational zero knowledge proof for 3COL. It is a fact that 3COL is NP-complete, so any problem in NP has a polynomial time reduction to 3COL. Thus, by giving a zero knowledge proof for 3COL, we will show that there are zero knowledge proofs for all of NP.

We will first give a high-level description of a zero-knowledge protocol for 3COL. Suppose a prover $P$ wants to convince a verifier $V$ that his graph $G$ is 3-colorable without revealing what the coloring $c$ actually is. If the three colors we use are red, green, and blue, then note that if we colored all the red vertices blue, all the green vertices red, and all the blue vertices green, we would still have a valid 3-coloring. In fact, if $\phi$ was any permutation on the color set of red, green, and blue, then $\phi \circ c$ would be a valid 3-coloring of $G$.

$P$ asks $V$ to leave the room and then samples a random permutation $\phi$ of the three colors. He colors the vertices of $G$ according to $\phi \circ c$, then covers all the vertices with cups. At this point, $P$ invites $V$ back into the room. $V$ is allowed to pick one edge and then uncover the two endpoints of the edge. If the colors on the two endpoints are the same, then $V$ rejects $P$'s claim that the graph is 3-colorable.

If the colors on the two endpoints are different, then $V$ leaves the room again, $P$ samples $\phi$ randomly, and the process repeats itself. Certainly if $G$ is actually 3-colorable, then $V$ will never reject the claim. If $G$ is not 3-colorable, then there will always be an edge with endpoints that are colored identically and $V$ will eventually uncover such an edge.

Note that $V$ does not gain any information on the coloring because it is masked by a (possibly) different random permutation every time $V$ uncovers an edge. Of course this protocol depends on $P$ not being able to quickly recolor the endpoints of an edge after removing the cups. This is why we need commitment schemes.

\subsection{Commitment Schemes}

We want to construct a protocol between a sender and a receiver where the sender sends a bit to the receiver, but the receiver will not know the value of this bit until the sender chooses to "open" the data that he sent. Of course, this protocol is no good unless the receiver can be sure that the sender was not able to change the value of his bit in between when the receiver first obtained the data and when the sender chose to open it.

\begin{definition}
A \emph{commitment scheme} is a PPT machine $C$ taking input $(b,r)$ that satisfies two properties: \begin{itemize}
\item (perfect binding) For all $r, s$, we have $C(0,r) \neq C(1,s)$.

\item (computational hiding) $\{ C(0, U_n) \} \simeq_c \{ C(1, U_n) \}$

\end{itemize}
\end{definition}

So for the sender to "open" the data, he just has to send his value of $r$ to the receiver. We say that $r$ is a \emph{decommitment} for $C(x,r)$. Why do we require perfect binding instead of just statistical binding? If there existed even a single pair $r, s$ where $C(0,r) = C(1,s)$, then the sender could cheat. If he wished to reveal a bit value of 0 then he could just offer $r$ and if he wished to reveal a bit value of 1 then he could just offer $s$.

We can use injective one-way functions to construct commitment schemes.

\begin{theorem}
If injective one-way functions exist, then so do commitment schemes.
\end{theorem}
\proof{We can let $f$ be an injective one-way function. Recall from Lecture 3 that $f' (x, r) := (f(x), r)$ will also be an injective one-way function with hard-core bit $B(x,r) := \langle x, r \rangle$. We claim that $C(b,x,r) := (f'(x,r), b \oplus B(x,r))$ is a commitment scheme.

If $(x,r)  \neq (y,s)$ then $C(0,x,r) \neq C(0,y,s)$ because $f'$ is injective. Since $C(0,x,r) = (f'(x,r), B(x,r)) \neq (f'(x,r), \overline{B(x,r)}) = C(1,x,r)$, then $C$ satisfies perfect binding.

Suppose $D$ can distinguish $C(0, U_n)$ from $C(1, U_n)$. Then we can distinguish $B(x,r)$ from $\overline{B(x,r)}$ given $f'(x,r)$ which contradicts the fact that $B(x,r)$ is a hard-core bit for $f'(x,r)$. Thus, $C$ has the computational hiding property.}
\qed

\medskip
We can extend the definition of commitment schemes to hold for messages longer than a single bit. These commitment schemes will work by taking our commitment schemes for bits and concatenating them together. For the extended definition, we require that for any two messages $m_0$ and $m_1$ of the same length, the ensembles $\{ C(m_0, U_n) \}$ and $\{ C(m_1, U_n) \}$ are computationally indistinguishable.

\subsection{3COL Protocol}

Below we describe the protocol $P(x,z) \leftrightarrow V(x)$, where $x$ describes a graph $G = (\{1, \ldots, n \}, E)$ and $z$ describes a 3-coloring $c$:

\begin{enumerate}
\item $P$ picks a random permutation $\pi : \{ 1, 2, 3 \} \to \{ 1, 2, 3 \}$ and defines the 3-coloring $\beta := \pi \circ c$ of $G$. Using a commitment scheme $C$ for the messages $\{ 1, 2, 3 \}$, $P$ defines $\alpha_i = C(\beta(i), U_n)$ for each $i \in V$. $P$ sends $\alpha_1, \alpha_2, \ldots, \alpha_n$ to $V$.

\item $V$ uniformly samples an edge $e = (i, j) \in E$ and sends it to $P$.

\item $P$ opens $\alpha_i$ and $\alpha_j$.

\item $V$ will accept only if it received valid decommitments for $\alpha_i$ and $\alpha_j$, and if $\beta(i)$ and $\beta(j)$ are distinct and valid colors.

\end{enumerate}

It is clear that this protocol is PPT. If $G$ is not 3-colorable, then there will be at least a $1/|E|$ probability that $V$ will reject $P$'s claim in step 4. Since $|E| \leq n^2$ we can repeat the protocol polynomially many times to increase the rejection probability to at least 1/2.

We will now show that this protocol is zero-knowledge. We describe a simulator $S$ below, given a verifier $V^*$: \begin{enumerate}
\item Sample an edge $e = (i, j) \in E$ uniformly at random.

\item Assign $c_i$ and $c_j$ to have distinct values from $\{ 1, 2, 3 \}$ and do so uniformly at random. Set $c_k := 1$ for all $k \neq i, j$.

\item Compute $n$ random keys $r_1, \ldots, r_n$ and set $\alpha_i = C(c_i, r_i)$ for all $i$.

\item Let $e' \in E$ be the response of $V^*$ upon receiving $\alpha_1, \ldots, \alpha_n$.

\item If $e' \neq e$, then terminate and go back to step 1. Otherwise, proceed. If $S$ returns to step 1 more than $2n |E|$ times, then output $\sf{fail}$ and halt the program.

\item Print $\alpha_1, \ldots, \alpha_n, e$, send $r_i$ and $r_j$ to $V^*$ and then print whatever $V^*$ responds with.
\end{enumerate}

By construction, $S$ will run in polynomial time. However, sometimes it may output a $\sf{fail}$ message. We will show that this occurs with negligible probability.

Suppose that for infinitely many graphs $G$, $V^*$ outputs $e' = e$ in step 4 with probability less than $1/2|E|$. If this is true, then it is possible for us to break the commitment scheme $C$ that we use in $S$. Consider a modified version of $S$ called $\tilde{S}$, where in step 2 we set $c_i = 1$ for all $i$. Note that in this case, $V^*$ cannot distinguish between any of the edges so the probability that it returns $e' = e$ is $1/|E|$.

If we gave $V^*$ a set of commitments $\alpha_k = C(1, r_k)$ for random keys $r_k$, then we would be in the setting of $\tilde{S}$. If we gave $V^*$ the commitments $\alpha_k$ but with two of the values set to $C(c, r)$ and $C(c', r')$ where $c, c'$ are distinct random values from $\{ 1, 2, 3 \}$ and $r, r'$ are random keys, then we are in the setting of $S$. This implies that it possible to distinguish between these two commitment settings with a probability of at least $1/2|E|$ which is non-negligible. It follows that $V^*$ outputs $e' = e$ with probability less than $1/2|E|$ for only finitely many graphs $G$.

Thus, the probability that $S$ outputs $\sf{fail}$ in the end is less than $(1 - 1/2|E|)^{2n|E|} < 1/e^n$ which is negligible.

Now we need to argue that the transcripts generated by $S$ are computationally indistinguishable from the transcripts generated by $P \leftrightarrow V^*$. Again, we consider a modified version of $S$, called $S'$, given a 3-coloring of its input $G$ as auxiliary input. In step 2 of the simulation, $S'$ will choose a random permutation of the colors in its valid 3-coloring for the values of $c_i$ rather than setting all but two values $c_i$ and $c_j$ equal to 1. Note that this is how our protocol between $P$ and $V$ behaves.

Observe that $P \leftrightarrow V^*$ is computationally indistinguishable from $S'$ because $S'$ outputs $\sf{fail}$ with negligible probability. Thus, it suffices to show that $S$ and $S'$ are computationally indistinguishable. Again, we will suppose otherwise and argue that as a result we can distinguish commitments.

We consider two messages $m_0$ and $m_1$ of the same length where $m_0$ consists of $n-2$ instances of the message $1$ and two committed colors $c_i$ and $c_j$ (for a random edge $(i, j) \in E$) and $m_1$ consists of a committed random 3-coloring of $G$ (with a random edge $(i, j) \in E$) chosen. Observe that by feeding the former message to $V^*$ we are in the setting of $S'$ and by feeding the latter message to $V^*$ we are in the setting of $S$. If we could distinguish those two settings, then we could distinguish the commitments for $m_0$ and $m_1$. This contradiction completes our argument that our 3-coloring protocol is zero-knowledge.















\DIFaddbegin 















\DIFaddend %
%\newcommand{\st}{~\text{s.t.}~}
\newcommand{\rgets}{\overset{\$}{\gets}}
%\newcommand{\ind}{\overset{c}{\approx}}
\section{NIZK Proof Systems}
We now consider a different class of Zero-Knowledge proof systems, where no
interaction is required: The Prover simply sends one message to the Verifier,
and the Verifier either accepts or rejects. Clearly for this class to be
interesting, we must have some additional structure:
both the Prover and Verifier additionally have access to a common random public string
$\sigma$ (trusted to be random by both). For example, they could derive $\sigma$
by looking at sunspot patterns. 

\section{Definitions}

\begin{definition}[NIZK Proof System]
    A \emph{NIZK proof system} for input $x$ in language $L$, with witness $\omega$, is a set of
efficient (PPT) algorithms $(K, P, V)$ such that:
\begin{enumerate}
    \item Key Generation: $\sigma \gets K(1^k)$ generates the random public string.
    \item Prover: $\pi \gets P(\sigma, x, \omega)$ produces the proof.
    \item Verifier: $V(\sigma, x, \pi)$ outputs $\{0, 1\}$ to accept/reject the proof.
\end{enumerate}
Which satisfies the completeness, soundness, and zero-knowledge properties below.
\end{definition}
Note: We will assume throughout that $x$ is of polynomially-bounded length, i.e., we are
considering the language $L \cap \{0, 1\}^{P(k)}$.


\medskip
\noindent\textbf{Completeness.} $\forall x \in L, \forall \omega \in R_L(x)$:
    $$\Pr[\sigma \gets K(1^k), \pi \gets P(\sigma, x, \omega) : V(\sigma, x,
    \pi) = 1] = 1.$$
There is an alternate definition of Statistical Correctness, where the probability above is $1 - \mathsf{negl}(n)$ instead of $1$. For this explanation, though, Completeness will be used.

\medskip
\noindent\textbf{Non-Adaptive Soundness.} $\forall x \not\in L$:
    $$\Pr[\sigma \gets K(1^k): \exists~\pi \st V(\sigma, x, \pi)
    = 1] = \mathsf{negl}(k).$$
If the value of $\sigma$ that is picked is a ``bad" value, then there does not exist a proof $\pi$ for $x$ and $\sigma$. The above definition is ``non-adaptive", because it does not allow a cheating
prover to decide which statement to prove after seeing the randomness $\sigma$.
We may also consider the stronger notion of ``adaptive soundness", where the
prover is allowed to decide $x$ after seeing $\sigma$:

\medskip
\noindent\textbf{Adaptive Soundness.}
    $$\Pr[\sigma \gets K(1^k): \exists~ (x, \pi) \st  x \not\in L, V(\sigma, x, \pi)
    = 1] = \mathsf{negl}(k).$$
    
\medskip
\noindent\textbf{(Non-Adaptive) Zero-Knowledge.}
    There exists a PPT simulator $S$ such that $\forall x \in L, \omega \in
    R_L(x)$, the two distributions are computationally indistinguishable:

\begin{minipage}{0.5\textwidth}
    \begin{enumerate}[itemsep=-3pt]
        \item $\sigma \gets K(1^k)$
        \item $\pi \gets P(\sigma, x, \omega)$
        \item Output $(\sigma, \pi)$
    \end{enumerate}
\end{minipage}
\begin{minipage}{0.5\textwidth}
    \begin{enumerate}
        \item $(\sigma, \pi) \gets S(1^k, x)$
        \item Output $(\sigma, \pi)$
    \end{enumerate}
\end{minipage}

\medskip
That is, the simulator is allowed to generate the distribution of randomness
$\sigma$ together with $\pi$. Note that if we did not allow $S$ to produce
$\sigma$, this definition would be trivial (a verifier could convince himself by
running the simulator, instead of interacting with $P$). Allowing $S$ to
generate $\sigma$ still keeps the definition zero-knowledge (since a verifier sees both $(\sigma,
\pi)$ together), but puts $P$ and $S$ on unequal footing.

We could also consider the adaptive counterpart, where a cheating verifier can
choose $x$ after seeing $\sigma$:

\medskip
\noindent\textbf{(Adaptive) Zero-Knowledge.}
    There exists a PPT simulator split into two stages $S_1, S_2$ such that
    for all PPT attackers $\ma$,
    the two distributions are computationally indistinguishable:

\medskip
\begin{minipage}{0.5\textwidth}
    \begin{enumerate}[itemsep=0pt]
        \item $\sigma \gets K(1^k)$
        \item $(x, \omega) \gets \ma(\sigma)$, s.t. $(x, \omega) \in R_L$
        \item $\pi \gets P(\sigma, x, \omega)$
        \item Output $(\sigma, x, \pi)$
    \end{enumerate}
\end{minipage}
\begin{minipage}{0.5\textwidth}
    \begin{enumerate}[itemsep=0pt]
        \item $(\sigma, \tau) \gets S_1(1^k)$
        \item $(x, \omega) \gets \ma(\sigma)$
        \item $\pi \gets S_2(\sigma, x, \tau)$
        \item Output $(\sigma, x, \pi)$
    \end{enumerate}
\end{minipage}

\medskip
\noindent where $\tau$ should be thought of as local state stored by the simulator (passed
    between stages).

\bigskip
Now we show that adaptive soundness is not harder to achieve than non-adaptive soundness.
\begin{theorem}\label{thm:amplify-soundness}
    Given a NIZK $(K, P, V)$ that is \emph{non-adaptively sound}, we can
    construct a NIZK $(K', P', V')$ that is \emph{adaptively sound}.
\end{theorem}
\proof
For $x_0 \not\in L$, let us call a particular $\sigma$ ``bad for $x_0$" if 
there exists a false proof for $x_0$ using randomness $\sigma$:
$\exists~ \pi \st V(\sigma, x_0, \pi) = 1$.
By non-adaptive soundness of $(K, P, V)$, we have
$\Pr_\sigma[\sigma \text{ bad for } x_0] = \mathsf{negl}(k)$.

Now we construct a new NIZK $(K',P',V')$ by repeating $(K,P,V)$ polynomially-many times
(using fresh randomness, and $V'$ accepts if and only if $V$ accepts in each iteration).
We can ensure that $\mathsf{negl}(k) \leq 2^{-2P(k)}$.
Now by union bound:
$$
\Pr[\sigma \gets K'(1^k): \exists~ (x, \pi) \st V'(\sigma, x, \pi) = 1] \leq
2^{P(k)}\cdot \Pr_\sigma[\sigma \text{ bad for } x_0] \leq 2^{-P(k)}.$$
So this new NIZK is adaptively-sound. \qed

\section{NIZKs from Bilinear Maps}


\input{lec20-F24-bonus}
%
In this proof we assumed that the random string $r$ comes from a very specific distribution
that corresponds to cycle matrices.
Now we need to show that the general problem (where $r$ comes from a
random uniform distribution of $\ell$ bits) can be reduced into this previous scenario.

We proceed as follows.
Let the length of the random string be
$\ell=\left\lceil 3\cdot \log_2 n\right\rceil \cdot n^4$.
We view the random string $r$ as $n^4$ blocks of $\left\lceil 3\cdot \log_2 n\right\rceil$
bits and we generate a random string $r'$ of length $n^4$ such that each bit in $r'$
is 1 if and only if all the bits in the corresponding block of $r$ are equal to 1.
This way, the probability that the $i$-th bit of $r'$ equals 1 is $\Pr[r'_i=1]\approx\frac{1}{n^3}$ for every $i$.

Then we create an $n^2\times n^2$ matrix $M$ whose entries are given by the bits of $r'$.
Let $x$ be the number of one entries in the matrix $M$.
The expected value for $x$ is $\frac{n^4}{n^3}=n$.
And the probability that $x$ is exactly $n$ is noticeable. To prove that, we can use
Chebyshev's inequality:
$$\Pr[|x-n|\geq n]\leq\frac{\sigma^2}{n^2}=
\frac{n^4\cdot \frac{1}{n^3}\cdot\left(1-\frac{1}{n^3}\right)}{n^2}<\frac{1}{n}.$$
So we have $\Pr[1\leq x\leq 2n-1]>\frac{n-1}{n}$.
And the probability $\Pr[x=k]$ is maximal for $k=n$, so we conclude that
$\Pr[x=n]>\frac{n-1}{n(2n-1)}>\frac{1}{3n}$.

Now suppose that this event ($x=n$) occurred and we have exactly $n$ entries equal to 1
in matrix $M$. What is the probability that those $n$ entries are all in different rows
and are all in different columns?

We can think about the problem this way: after $k$ one entries have been added to the matrix,
the probability that a new entry will be in a different row and different column is given by
$\left(1-\frac{k}{n^2}\right)^2$. Multiplying all these values we get

\begin{align*}
\Pr[\text{no collision}] &\geq \left(1-\frac{1}{n^2}\right)^2 \cdot \left(1-\frac{2}{n^2}\right)^2
\cdots \left(1-\frac{n-1}{n^2}\right)^2 \\
& > 1 - 2\left(\frac{1}{n^2} + \frac{2}{n^2} +\cdots + \frac{n-1}{n^2}\right)
= 1 - \frac{n-1}{n} = \frac{1}{n}.
\end{align*}

Now assume that this event happened: the matrix $M$ has exactly $n$ entries equal to 1
and they are all in different rows and different columns.
Then we can define a new $n\times n$ matrix $M_c$ by selecting only those $n$ rows
and $n$ columns of $M$. By construction, $M_c$ is a permutation matrix.
The probability that $M_c$ is a cycle matrix is $\frac{(n-1)!}{n!}=\frac{1}{n}$.
An example is shown in Figures~\ref{fig:n2}~and~\ref{fig:n}.

\begin{figure}[ht]
	\centering
		\includegraphics[height=8cm]{Old Scribe Notes/n2.png}
	\caption{Matrix $M$ which is $n^2\times n^2$ for $n=8$.}
	\label{fig:n2}
\end{figure}

\begin{figure}[ht]
	\centering
		\includegraphics[height=4cm]{Old Scribe Notes/n.png}
	\caption{Matrix $M_c$ which is $n\times n$ for $n=8$. The construction worked,
	         because $M_c$ is a cycle matrix.}
	\label{fig:n}
\end{figure}


Now let's join all those probabilities. The probability that $M_c$ is a cycle matrix is at least
$$\frac{1}{3n}\cdot \frac{1}{n}\cdot \frac{1}{n} > \frac{1}{3n^3}.$$

If we repeat this process $n^4$ times, then the probability that $M_c$ is a cycle matrix in at least one iteration is at least
$$1-\left(1-\frac{1}{3n^3}\right)^{n^4}\approx 1-e^{-\frac{n}{3}} = 1-\mathsf{negl}(n).$$


\bigskip
The proof system works as follows. Given a random string $r$, the prover $P$ tries
to execute the construction above to obtain a cycle matrix.
If the construction fails, the prover simply reveals all the bits in the string $r$
to the verifier, who checks that the constructions indeed fails.
If the construction succeeds, the prover reveals all the entries in the random string $r$
that correspond to values in the matrix $M$ which are not used in matrix $M_c$.
The verifier will check that all these values for matrix $M$ are indeed 0.

Then the prover proceeds as in the previous scenario using matrix $M_c$: he
reveals the transformation $\phi$ and opens all the non-edges.

This process is repeated $n^4$ times. Or, equivalently, a big string of length
$\left\lceil 3\cdot \log_2 n\right\rceil \cdot n^4\cdot n^4$ is used and they are all
executed together. This produces a zero knowledge proof.

\textit{Completeness:} if $P$ knows the Hamiltonian cycle of $G$,
then he will be able to find a suitable transformation $\phi$ whenever a cycle graph is
generated by the construction.

\textit{Soundness:} if $P$ is lying and trying to prove a false statement, then he will
get caught with very high probability. If any of the $n^4$ iterations produces a cycle
graph, then $P$ will be caught. So the probability that he will be caught is
$1-e^{-\frac{n}{3}} = 1-\mathsf{negl}(n)$.

\textit{Zero Knowledge:} again $V$ cannot get any information if the construction succeeds.
And if the construction doesn't succeed, all $V$ gets is the random string $r$, which also
doesn't give any information.
\qed


\begin{theorem}
For any language $L$ in $NP$, there is a non-interactive zero-knowledge (NIZK) proof
in the hidden-bit model (HBM) for the language $L$.
\end{theorem}
\proof
The language $L^*$ of Hamiltonian graphs is $NP$-complete. So any problem in $L$ can
be reduced to a problem in $L^*$. More precisely, there is a polynomial-time function
$f$ such that
$$x\in L \Longleftrightarrow f(x)\in L^*.$$
So given an input $x$, the prover can simply calculate $f(x)$ and
produce a NIZK proof in the hidden-bit model for the fact that $f(x)\in L^*$.
Then the verifier just needs to calculate $f(x)$ and check if the proof for the fact
$f(x)\in L^*$ is correct.
\qed

\begin{theorem}\label{the:NIZK_NP}
For any language $L$ in $NP$, there is a non-interactive zero-knowledge (NIZK) proof
in the common reference string (CRS) model for the language $L$.
\end{theorem}
\proof
In Theorem~\ref{thm:NIZK-amplify} it was shown that any NIZK proof in the hidden-bit model can
be converted into a NIZK proof in the standard (common reference string) model by using
a trapdoor permutation.
\qed

%\input{lec22-F24}
%\section*{Exercises}
\begin{exercise}[Leaky ZK proof] Formally define:
\begin{enumerate}
  \item 
What it means for an  interactive proof $(P,V)$ to be \textbf{first-bit leaky} zero-knowledge, where we require that the protocol doesn't leak anything more than the first bit of the witness.

\item What it means for an  interactive proof $(P,V)$ to be \textbf{one-bit leaky} zero-knowledge, where we require that the protocol doesn't leak anything more than one bit that is an arbitrary adversarial chosen function of the witness.
    \end{enumerate}
\end{exercise}

\begin{exercise}[Proving OR of two statements] Give a statistical zero-knowledge proof system $\Pi = (P,V)$ (with efficient prover) for the following language.
    \[ L = \left\{((G_0,G_1),(G_0',G_1'))\left| G_0 \simeq G_1 \bigvee G_0' \simeq G_1'\right.\right\}\]\\
    \textbf{Caution:} Make sure the verifier doesn't learn which of the two pairs of graphs is isomorphic.
\end{exercise}

\begin{exercise} [ZK implies WI] Let $L \in NP$ and let $(P,V)$ be an interactive proof system for $L$. We say that $(P,V)$ is \emph{witness indistinguishable (WI)} if for all PPT $V^*$, for all $x \in L$, distinct witnesses $w_1, w_2 \in R_L(x)$ and  auxiliary input $z\in \binset{*}$, the following two views are computationally indistinguishable:
\[View_{V^*} \left(P(x,w_1) \leftrightarrow V^*(x,z) \right) \simeq_c View_{V^*} \left(P(x,w_2) \leftrightarrow V^*(x,z) \right).\]
\begin{enumerate}
\item Show that if $(P,V)$ is an efficient prover zero-knowledge proof system, then it is also witness indistinguishable.

\item Assume $(P,V)$ is an efficient prover zero-knowledge proof system. We have seen in the exercise that $(P,V)$ is also witness indistinguishable. Define $(\tilde P, \tilde V)$ to repeat $(P,V)$ independently for $k$ times \emph{in parallel} ($k$ is a polynomial), and $\tilde V$ accepts if and only if $V$ accepts in all the executions. Prove that $(\tilde P, \tilde V)$ is still witness indistinguishable.
\end{enumerate}    
\end{exercise}

\begin{exercise}
\textbf{Multi-statement NIZK.} The NIZK proof system we constructed in class required a fresh common random string (CRS) for each statement proved. In various settings we would like to reuse the same random string to prove multiple theorem statements while still preserving the zero-knowledge property.
    
    A multi-statement NIZK proof system $(K,P,V)$ for a language $L$ with corresponding relation $R$ is a NIZK proof system for $L$ with a stronger zero-knowledge property, defined as follows: $\exists$ a PPT machine $\mathcal{S} = (\mathcal{S}_1,\mathcal{S}_2)$ such that $\forall$ PPT machines $A_1$ and $A_2$ we have that:
    \[\left|\Pr\left[\begin{split}\sigma \gets K(1^\kappa),\\ (\{x_i,w_i\}_{i \in [q]},\textsf{state}) \gets A_1(\sigma),\\ \text{ such that } \forall i \in [q], (x_i,w_i)\in R\\\forall i \in [q],  \pi_i \gets P(\sigma, x_i,w_i);\\
    A_2(\textsf{state}, \{\pi_i\}_{i \in [q]}) =1\end{split}\right]
    -
    \Pr\left[\begin{split}(\sigma,\tau) \gets \mathcal{S}_1(1^\kappa),\\ (\{x_i,w_i\}_{i \in [q]},\textsf{state}) \gets A_1(\sigma),\\\text{ such that } \forall i \in [q], (x_i,w_i)\in R\\\forall i \in [q],  \pi_i \gets \mathcal{S}_2(\sigma, x_i,\tau);\\ A_2(\textsf{state}, \{\pi_i\}_{i \in [q]})=1\end{split}\right]\right|
    \leq \textsf{negl}(\kappa).
    \]
    
    Assuming that a single statement NIZK proof system $(K,P,V)$ for NP exists, construct a multi-statement NIZK proof system $(K',P',V')$ for NP.\\
\textbf{Hint:} Let $g: \{0,1\}^\kappa \rightarrow \{0,1\}^{2\kappa}$ be a length doubling PRG. Let $K'$ output the output of $K$ along with $y$, a random $2\kappa$ bit string. To prove $x \in L$ the prover $P'$ proves that $\exists (w,s)$ such that either $(x,w)\in R$ or $y = g(s)$.
\end{exercise}






%\newcommand{\proofsketch}{\smallskip\noindent{\bf Proof sketch. }}
\algrenewcommand\algorithmicfunction{\textbf{Machine}}
\newcommand{\bits}{\set{0,1}}
\newcommand{\Ex}{\mathbb{E}}

\renewcommand{\O}{\ensuremath{\mathcal{O}}}
\newcommand{\To}{\rightarrow}
\newcommand{\e}{\epsilon}
\newcommand{\R}{\mathbb{R}}
\newcommand{\N}{\mathbb{N}}
\newcommand{\Z}{\mathbb{Z}}
\newcommand{\logAnd}{\wedge}

\newcommand{\indis}{\mathrel{\overset{\makebox[0pt]{\mbox{\normalfont\tiny c}}}{\approx}}}
\newcommand{\allindis}{\mathrel{\overset{\makebox[0pt]{\mbox{\normalfont\tiny p/s/c}}}{\approx}}}

\newcommand{\cclass}[1]{\mathsf{#1}}
\renewcommand{\P}{\cclass{P}}
\newcommand{\NP}{\cclass{NP}}
\newcommand{\Time}{\cclass{Time}}
\newcommand{\BPP}{\cclass{BPP}}
\newcommand{\Size}{\cclass{Size}}
\newcommand{\Ppoly}{\cclass{P_{/poly}}}
\newcommand{\CSAT}{\ensuremath{\mathsf{CSAT}}}
\newcommand{\SAT}{\ensuremath{\mathsf{3SAT}}}
\newcommand{\IS}{\mathsf{INDSET}}

\newcommand{\inp}{\mathsf{in}}
\newcommand{\outp}{\mathsf{out}}

\newcommand{\Param}{\kappa}
\newcommand{\Adv}{\mathsf{Adv}}
\newcommand{\Supp}{\mathsf{Supp}}

\newcommand{\PRG}{\mathsf{G}}
\renewcommand{\Enc}{\mathsf{Enc}}
\renewcommand{\Dec}{\mathsf{Dec}}
\renewcommand{\sk}{\mathsf{sk}}
\newcommand{\sfC}{\mathsf{C}}
\newcommand{\sfR}{\mathsf{R}}

\newcommand{\eqdef}{\stackrel{\text{\tiny def}}{=}}

\newcommand{\cF}{\mathcal{F}}

\newcommand{\angles}[1]{\langle #1 \rangle}
\newcommand{\iprod}[1]{\angles{#1}}

\newcommand{\Com}{\mathsf{Com}}

% Real vs. Ideal
\newcommand{\RealAdv}{\mathcal{A}}
\newcommand{\IdealAdv}{\mathcal{S}}
\newcommand{\RealVar}{\mathsf{Real}}
\newcommand{\IdealVar}{\mathsf{Ideal}}

% Participating parties
\newcommand{\PartyA}{P_1}
\newcommand{\PartyB}{P_2}
\newcommand{\InputA}{x_1}
\newcommand{\InputB}{x_2}

% Garbling Schemes
\newcommand{\Garble}{\mathsf{Garble}}
\newcommand{\Cir}{C}
\newcommand{\GCir}{\widetilde{C}}
\newcommand{\Lab}{\mathsf{lab}}

% Proof
\newcommand{\Sim}{\mathsf{Sim}}

% GMW

% Misc
\newcommand{\out}{\mathsf{out}}
\newcommand{\Assign}{:=}

\section{zkSNARKs}

In this section, we will overview the fundamentals of zkSNARKs, or Zero-Knowledge Succinct Non-interactive ARguments of Knowledge.

\subsection{Preliminaries}

Before diving into zkSNARKs, let's understand the basic framework we're working with. In cryptography, we often deal with statements of the form ``I know a secret value that satisfies some property.'' For instance, we might want to prove that we know the private key corresponding to a public key, or that we know a solution to a Sudoku puzzle, or that we know a valid password for an account.

To formalize these statements, we use what's called a binary relation $R$. This relation takes two inputs: the public statement $x$ (like a Sudoku puzzle) and the secret witness $w$ (like the solution). $R$ is an efficiently computable binary relation that outputs 1 if the witness $w$ is valid for statement $x$, and 0 otherwise.

For an NP language $\mathcal{L}$, we can say that $x \in \mathcal{L}$ if and only if there exists a witness $w$ such that $R(x, w) = 1$. Conversely, $x \notin \mathcal{L}$ if and only if there does not exist any witness $w$ such that $R(x, w) = 1$. A crucial property is that while finding a valid witness $w$ may be computationally hard (like solving a Sudoku puzzle), verifying the relation $R(x,w)=1$ is always efficient (like checking if a Sudoku solution is valid). This verification can be done in polynomial time.

This framework allows us to express a wide variety of practical problems where we want to prove knowledge of a solution without revealing the solution itself, which is exactly what zkSNARKs will help us achieve.

\subsection{Properties of zkSNARKs}

We now introduce the properties of zkSNARKs. We seek to use zkSNARKs to prove that $x \in L \iff R(x, w) = 1$. Informally, if a prover has $(x, w)$, we must send a proof $\pi$ such that the verifier, who has the instance $x$, can efficiently check that $R(x, w) = 1$. For the purposes of this lecture, we are focusing on non-interactive proof systems. This means that there is no back-and-forth communication between the prover and the verifier, and the verifier can verify the prover's statement in one shot.

Two of the properties of zkSNARKs are common to all proof systems: Correctness ensures that if the statement is true, the prover should always be able to convince the verifier. Soundness ensures that no cheating prover can convince the verifier. However, a correct and sound proof system is not enough for a zkSNARK. Indeed, we could simply achieve this by sending the witness $w$ from the prover to the verifier. We now discuss two properties that are unique to zkSNARKs: Succinctness and Zero knowledge. Succinctness requires that the proof $\pi$ sent by the prover is significantly smaller than the witness $w$. Specifically, the proof size $|\pi|$ must be bounded by $poly(\lambda, \log(|w|))$, where $\lambda$ is the security parameter. This means the proof grows only polylogarithmically with the witness size. Zero knowledge ensures that the proof should not reveal any information about the witness $w$ beyond what can be deduced from the statement being proven. This property is what differentiates zkSNARKs from SNARKs.

The succinctness properties make zkSNARKs incredibly relevant, even for practical applications where we do not care about hiding the witness $w$. This is because the proof size is exponentially more efficient than directly sending the witness $w$ from the prover to the verifier.

\paragraph{Succinctness example.} Let's consider a practical example to illustrate the power of SNARKs' succinctness property. Suppose we have a 1 TB hard disk and want to prove to a verifier that $Hash(hard\text{ }disk) = x$ for some known hash value $x$. We have two options: Without SNARKs, we would need to send the entire 1 TB hard disk to the verifier, who then computes the hash themselves. With SNARKs, we can generate a succinct proof $\pi$ (approximately 1 KB) that proves knowledge of the hard disk contents whose hash equals $x$. Even in scenarios where we do not need to hide the hard disk contents (i.e., zero-knowledge is not required), the SNARK approach is dramatically more efficient in terms of communication complexity: sending a 1 KB proof versus transferring 1 TB of data. This lecture will primarily focus on achieving succinctness, as adding zero-knowledge properties is a relatively straightforward extension once the basic SNARK construction is understood.

\paragraph{How to build SNARKs?} To construct SNARKs, we will follow these key steps. First, we need to find a ``SNARK-friendly'' representation model for NP languages. The example we will use is square span programs. Next, we must show that this model can capture all NP relations, using boolean circuits as an example. Then, we construct the SNARK using cryptographic techniques, specifically employing bilinear groups in our example construction. After that, we prove soundness. Finally, we add zero-knowledge properties. The final SNARK that emerged from this approach is known as Groth16. While we will not cover Groth16 directly, we will study the DFGK14 SNARK, which is conceptually simpler but employs the same fundamental cryptographic ideas.

\subsection{Square Span Programs}

We begin by introducing Square Span Programs (SSPs), which provide a ``SNARK-friendly'' representation for NP languages.

\begin{definition}[Square Span Program]
A square span program $Q$ over a field $\mathbb{F}$ consists of a size parameter $m \in \mathbb{N}$, a degree parameter $d \in \mathbb{N}$, and a set of polynomials $\{v_0(x), v_1(x), \dots, v_m(x), t(x)\}$. Each $v_i(x)$ is a polynomial over $\mathbb{F}$ of degree at most $d$, and $t(x)$ is a polynomial over $\mathbb{F}$ of degree exactly $d$.
\end{definition}

\begin{definition}[SSP Acceptance]
We say that a square span program $Q$ accepts an input $(a_1, \dots, a_\ell) \in \mathbb{F}^\ell$ if and only if there exist values $a_{\ell+1}, \dots, a_m \in \mathbb{F}$ such that $t(x)$ divides $(v_0(x) + \sum_{i=1}^m a_i v_i(x))^2 - 1$. In other words, there exists a polynomial $h(x)$ such that $(v_0(x) + \sum_{i=1}^m a_i v_i(x))^2 - 1 = h(x)t(x)$.
\end{definition}

The values $a_1, \dots, a_\ell$ represent the input to our computation, while $a_{\ell+1}, \dots, a_m$ serve as auxiliary values (similar to a witness in an NP relation). As we will see, this algebraic structure is particularly well-suited for constructing SNARKs. A key property of SSPs is their expressiveness: we can transform any boolean circuit into an equivalent square span program. This transformation will be our next focus.

\subsection{From Boolean Circuits to SSPs}

Consider the boolean circuit in Figure \ref{fig:ssp}. We will transform this circuit into an SSP.

\begin{figure}[h]
  \centering
  \includegraphics[width=0.4\textwidth]{figures/ssp.pdf}
  \caption{Boolean circuit example for SSP transformation \label{fig:ssp}} 
\end{figure}

First, we formalize the constraints for each gate type. For an XOR gate, the output $a_4$ of inputs $a_1, a_2$ must satisfy $a_1 + a_2 + a_4 \in \{0, 2\}$. For an OR gate, the output $a_5$ of inputs $a_2, a_3$ must satisfy $(1-a_2) + (1-a_3) - 2(1-a_5) \in \{0, 1\}$. For a NAND gate, the output $a_6$ of inputs $a_4, a_5$ must satisfy $a_4 + a_5 - 2(1-a_6) \in \{0, 1\}$.

For the circuit to be satisfiable, all gate constraints must be satisfied, the output constraint $3(1 - a_6) \in \{0, 2\}$ must hold, and all boolean value constraints $a_i \in \{0, 1\}$ for all $i \in \{1,\ldots,6\}$ must be met. To standardize these constraints, we multiply all constraints involving $\{0, 1\}$ by 2 to normalize ranges and combine constraints into a matrix form for $a_1,\ldots,a_6$. Finally, we seek to unify the constraints into a larger matrix that encompasses all the constraints in the whole circuit, where each column represents a single boolean algebra constraint. The resulting constraint matrix $M$ is below, with columns representing the XOR, OR, NAND, and output constraints, respectively:

\[ \begin{pmatrix}
1 & 0 & 0 & 0 & 2 & 0 & 0 & 0 & 0 & 0 \\
1 & -2 & 0 & 0 & 0 & 2 & 0 & 0 & 0 & 0 \\
0 & -2 & 0 & 0 & 0 & 0 & 2 & 0 & 0 & 0 \\
1 & 0 & 2 & 0 & 0 & 0 & 0 & 2 & 0 & 0 \\
0 & 4 & 2 & 0 & 0 & 0 & 0 & 0 & 2 & 0 \\
0 & 0 & 4 & -3 & 0 & 0 & 0 & 0 & 0 & 2
\end{pmatrix} \]

There is also a constant vector $\vec{\delta}$ associated with these constraints:

\[ \vec{\delta} = [0 \; 0 \; -4 \; 3 \; | \; 0 \; 0 \; 0 \; 0 \; 0 \; 0] \]

All constraints must evaluate to elements in $\{0, 2\}^{10}$. 

\paragraph{Generalizing to Arbitrary Circuits.}

Let's now see how we can transform any arbitrary boolean circuit into a Square Span Program. Consider a circuit with $m$ wires and $n$ gates. Our first task is to construct a matrix that captures all the constraints of our circuit. For each gate $k$ in our circuit, we create a column vector $\vec{v_k} = (v_{1k}, v_{2k}, \ldots, v_{mk})^T$ that encodes the gate's constraints. These constraints ensure the gate operates correctly—just as we saw with our XOR, OR, and NAND gates in the previous example.

After encoding all $n$ gates, we add a special column for the output constraint, which takes the form $(0,\ldots,0,-3)^T$. The $-3$ here is somewhat arbitrary; any field element different from 2 would work. We then augment our matrix with a diagonal matrix $D$ where each diagonal entry is 2. This diagonal matrix serves to enforce that all our variables are boolean, a crucial requirement for circuit satisfaction.

To complete our constraint system, we need a constant vector $\vec{\delta} = (\delta_1,\ldots,\delta_n,3,0,\ldots,0)$ where each $\delta_i$ represents the constant term for gate $i$. These constants are chosen from the set $\{0,2\}$, with the exception of the output constraint's constant which is 3.

Let's now see how we can transform any arbitrary boolean circuit into a Square Span Program. Consider a circuit with $m$ wires and $n$ gates. The complete constraint system can be written as:

\[ \begin{pmatrix} a_1 & a_2 & \cdots & a_m \end{pmatrix}
\begin{pmatrix}
v_{11} & v_{12} & \cdots & v_{1n} & 0 & 2 & 0 & \cdots & 0 \\
v_{21} & v_{22} & \cdots & v_{2n} & 0 & 0 & 2 & \cdots & 0 \\
v_{31} & v_{32} & \cdots & v_{3n} & 0 & 0 & 0 & \ddots & 0 \\
\vdots & \vdots & \ddots & \vdots & \vdots & \vdots & \vdots & \ddots & \vdots \\
v_{m1} & v_{m2} & \cdots & v_{mn} & -3 & 0 & 0 & \cdots & 2
\end{pmatrix} + \begin{pmatrix} \delta_1 & \delta_2 & \cdots & \delta_n & 3 & 0 & \cdots & 0 \end{pmatrix} \]

where $(a_1,\ldots,a_m)$ are the wire values (variables we solve for), the first $n$ columns $(v_{ij})$ represent the gate constraints, column $n+1$ is the output constraint $(0,\ldots,0,-3)^T$, the next $m$ columns form the diagonal matrix $D$ with $2$'s on the diagonal, and the constant vector $(\delta_1,\ldots,\delta_m)$ contains $\delta_i \in \{0,2\}$ for $i \leq n$ (gate constraints), $\delta_{n+1} = 3$ (output constraint), and $\delta_i = 0$ for $i > n+1$ (boolean constraints).

Next, we convert these discrete constraints into a polynomial system. We choose distinct field elements $x_1,\ldots,x_d$ (where $d$ is our total number of constraints) and use polynomial interpolation to create our SSP. For each row $i$ of our matrix, we construct a polynomial $v_i(x)$ that evaluates to the $(i,j)$ entry when evaluated at point $x_j$. Similarly, we create a polynomial $v_0(x)$ that interpolates our constant vector $\vec{\delta}$.

The target polynomial $t(x)$ is defined as the product of all linear terms:

\[ t(x) = \prod_{j=1}^d (x-x_j) \]

This polynomial is crucial because it ``zeros out'' exactly at our constraint points. The beauty of this construction is that it transforms our circuit satisfaction problem into an elegant polynomial divisibility question: the circuit is satisfiable if and only if there exist values $(a_1,\ldots,a_m)$ such that:

\[ \left(v_0(x) + \sum_{i=1}^m a_i v_i(x)\right)^2 - 1 = h(x)t(x) \]

for some polynomial $h(x)$.

After converting to a polynomial system, our constraint matrix becomes:

\[ \begin{pmatrix} 
a_1 \\ 
a_2 \\ 
\vdots \\ 
a_m 
\end{pmatrix}^T
\begin{pmatrix}
v_1(x_1) & v_1(x_2) & \cdots & v_1(x_n) & v_1(x_{n+1}) & v_1(x_{n+2}) & 0 & \cdots & 0 \\
v_2(x_1) & v_2(x_2) & \cdots & v_2(x_n) & v_2(x_{n+1}) & 0 & v_2(x_{n+2}) & \cdots & 0 \\
v_3(x_1) & v_3(x_2) & \cdots & v_3(x_n) & v_3(x_{n+1}) & 0 & 0 & \ddots & 0 \\
\vdots & \vdots & \ddots & \vdots & \vdots & \vdots & \vdots & \ddots & \vdots \\
v_m(x_1) & v_m(x_2) & \cdots & v_m(x_n) & v_m(x_{n+1}) & 0 & v_m(x_{n+2}) & \cdots & v_m(x_d)
\end{pmatrix} + 
\begin{pmatrix} v_0(x_1) \\ v_0(x_2) \\ \vdots \\ v_0(x_n) \\ v_0(x_{n+1}) \\ v_0(x_{n+2}) \\ 0 \\ \vdots \\ 0 \end{pmatrix}^T \]

where:
\begin{itemize}
    \item Each $v_i(x_j)$ evaluates to the corresponding matrix entry at point $x_j$
    \item The diagonal entries evaluate to 2 at their respective points
    \item The output constraint column evaluates to $(0,\ldots,0,-3)$ at point $x_{n+1}$
    \item $v_0(x)$ interpolates the constant vector $(\delta_1,\ldots,\delta_n,3,0,\ldots,0)$
\end{itemize}

For any boolean circuit $C$, we can construct an SSP instance $(v_0(x),\ldots,v_m(x),t(x))$ such that $C(x_1,\ldots,x_\ell) = 1$ if and only if there exist values $(a_{\ell+1},\ldots,a_m)$ making the above polynomial division possible.

The correctness of this transformation follows from our construction: each gate's constraints are captured at distinct evaluation points, the boolean nature of our variables is enforced by the diagonal matrix, and the polynomial division condition.




Now, for notational convenience, we define $v_0'(x) = v_0(x) - 1$. 

At each evaluation point $x_j$, our circuit constraints require:
\[ \left(\sum_{i=1}^m a_i v_i(x_j) + v_0'(x_j)\right)^2 = 1 \]

This set of point-wise constraints can be unified into a single polynomial constraint:
\[ \left(\sum_{i=1}^m a_i v_i(x) + v_0'(x)\right)^2 - 1 \equiv 0 \pmod{t(x)} \]

where $t(x) = \prod_{j=1}^d (x - x_j)$ is our target polynomial.

Equivalently, there must exist some polynomial $h(x)$ such that:
\[ \left(\sum_{i=1}^m a_i v_i(x) + v_0'(x)\right)^2 - 1 = h(x)t(x) \]

This is precisely the SSP satisfiability condition. Thus, we have shown that circuit satisfiability is equivalent to the existence of coefficients $a_i$ that satisfy this polynomial divisibility condition.

\subsection{From SSPs to SNARKs}

We'll use bilinear groups to construct our SNARK. Our construction begins with a bilinear group setup:
\[ (\mathbb{F}_p, G_1, G_2, G_T, e, g_1, g_2) \]

The common reference string (CRS) forms the foundation of our construction. It consists of powers of a secret value $\tau$:
\[ \text{CRS} = (bg, g_1^\tau, g_1^{\tau^2}, \dots, g_1^{\tau^d}, g_2^\tau, g_2^{\tau^2}, \dots, g_2^{\tau^d}, g_1^{B v_{\ell + 1}(\tau)}, \dots, g_1^{B v_m(\tau)}, h_2, h_2^B) \]

The core idea of our construction leverages the pairing operation $e$ to verify polynomial equations in the exponent:
\[ e(g_1^{[v_0(x) + \sum_{i=1}^m a_i v_i(x)]}, g_2^{[v_0(x) + \sum_{i=1}^m a_i v_i(x)]}) = e(g_1^{h(x)}, g_2^{t(x)}) \cdot e(g_1, g_2) \]

In our protocol, we distinguish between two types of values:
\begin{itemize}
    \item Statement: $(a_1, \dots, a_\ell)$ - the public inputs
    \item Witness: $(a_{\ell+1}, \dots, a_m)$ - the private values
\end{itemize}

For proof generation, the prover computes four crucial group elements:
\begin{align*}
    \pi_1 &= g_1^{\sum_{i=\ell+1}^m a_i v_i(z)} \\
    \pi_2 &= g_2^{\sum_{i=\ell+1}^m a_i v_i(z)} \\
    \pi_3 &= g_1^{h(z)} \\
    \pi_4 &= g_1^{B\sum_{i=1}^m a_i v_i(z)}
\end{align*}

The verification process consists of two steps. First, the verifier computes intermediate values:
\begin{align*}
    \pi_1' &= g_1^{v_0(z) + \sum_{i=1}^\ell a_i v_i(z)} \cdot \pi_1 \\
    \pi_2' &= g_2^{v_0(z) + \sum_{i=1}^\ell a_i v_i(z)} \cdot \pi_2
\end{align*}

Then, the verifier performs three critical pairing checks:
\begin{align*}
    e(\pi_1', \pi_2') &= e(\pi_3, g_2^{t(z)}) \cdot e(g_1, g_2) \\
    e(\pi_1, g_2) &= e(g_1, \pi_2) \\
    e(\pi_1, h_2^B) &= e(\pi_4, h_2)
\end{align*}

These pairing equations serve distinct purposes in ensuring the proof's validity. The first check confirms that the SSP is satisfied by verifying the polynomial equation in the exponent. The second check ensures consistency between the prover's elements in groups $G_1$ and $G_2$. The final check prevents malformed group elements by verifying they were properly constructed using the CRS values.

Due to the non-interactive nature of our protocol, we cannot rely on black-box extraction for our security proofs. Instead, we require a non-black-box extractor with access to the prover's code, which is acceptable within our security model.

\subsection{Soundness}

The soundness guarantees of SNARKs, including our SSP SNARK construction, differ from traditional cryptographic protocols. They necessarily rely on non-falsifiable assumptions or the random oracle model (ROM).

Non-falsifiable assumptions represent a unique class of cryptographic assumptions. Informally, given $(g, g^\alpha)$, these assumptions state that no adversary can output a pair $(g^x, (g^x)^\alpha)$ without actually ``knowing'' the exponent $x$. The term ``non-falsifiable'' stems from the fact that these assumptions make claims about what an adversary must know, rather than just what it can or cannot compute.

Our first attempt at formalizing soundness states that for every probabilistic polynomial-time (PPT) adversary $\mathcal{A}$, there exists a PPT extractor $E_\mathcal{A}$ such that:
\[ \Pr[(v_1, v_2) \gets \mathcal{A}(g, g^\alpha, z), x \gets E_\mathcal{A}(g, g^\alpha, z) : v_2 = v_1^\alpha \land v_1 \neq g^x] = negl(\lambda) \]

This probability involves three key components:
\begin{itemize}
    \item $z$ represents auxiliary information available to the adversary
    \item $\lambda$ denotes the security parameter
    \item $negl(\lambda)$ represents a negligible function
\end{itemize}

Recent research has revealed limitations in this initial formulation. Specifically, certain obfuscated programs, when provided as auxiliary input $z$, enable adversaries to produce valid pairs $(v_1, v_2)$ while preventing the extractor $E_\mathcal{A}$ from determining $x$ due to the complexity of reverse-engineering the obfuscated program.

To address these limitations, we've developed a more robust formulation for our SNARK construction. For all relations $R \in \mathcal{R}$, benign auxiliary information $z \in \mathcal{Z}$, and non-uniform adversaries $\mathcal{A}$, there exists a non-uniform PPT extractor $E_\mathcal{A}$ such that for all benign $z \in \mathcal{Z}$:
\[ \begin{aligned}
&\Pr[((v_1, v_2); c_0, \dots, c_d) \gets (\mathcal{A} \| E_\mathcal{A})(bg, g_1, g_2, \dots, g_2^{\tau^d}, z) : \\
&\quad e(v_1, g_2) = e(g_1, v_2) \land v_1 \neq g_1^{\sum_{i=0}^d c_i \tau^i}] = negl(\lambda)
\end{aligned} \]

This refined definition captures several crucial aspects: the adversary and extractor share access to the bilinear group setup and CRS, the extractor must produce coefficients $c_i$ explaining the adversary's output, the auxiliary information must be ``benign'' (excluding problematic cases like obfuscated programs), and the probability of adversarial success without the extractor finding a valid witness must be negligible.

While this non-falsifiable assumption is stronger than traditional cryptographic assumptions, it appears to be necessary for constructing efficient SNARKs using current techniques.

%%%% Paper Macros.

\let\counterwithout\relax
\let\counterwithin\relax

\let\proof\relax
\let\endproof\relax
\let\example\relax
\let\endexample\relax

%% Tikz configuration
\usetikzlibrary{shapes.geometric, arrows}

% Change thanks markers to numbers
\makeatletter
\let\@fnsymbol\@arabic
\makeatother

%% LLNCS hyperref requirement
\renewcommand\UrlFont{\color{blue}\rmfamily}

%% Eurocrypt requirement
\pagestyle{plain}


%% Allow align environment to span multiple pages
\allowdisplaybreaks

%% Theorem Environments

% Bold optional theorem titles
\makeatletter
\def\th@definition{%
  \thm@notefont{}% same as heading font
  \normalfont % body font
}
\makeatother

% No italics
% \theoremstyle{definition}

% Rename environments
\newtheorem{assumption}{Assumption}
% \newtheorem{construction}{Construction}
\newtheorem{attention}{Remark}
\newtheorem{notation}{Notation}

% Algorithm counter to construction
\makeatletter 
\renewcommand\thealgorithm{\theconstruction.\arabic{algorithm}} 
\@addtoreset{algorithm}{construction}
\makeatother

%% Annotations
\newcommand\todo[1]{\textcolor{red}{TODO: #1}}
\definecolor{comment-color}{RGB}{68, 59, 141}
\renewcommand{\algorithmiccomment}[1]{\textcolor{comment-color}{// #1}}
\newcommand\blue[1]{\textcolor{blue}{#1}}
\newcommand{\mathhl}[1]{\colorbox{yellow}{$\displaystyle #1$}}



%% Diagram Macros

\newcommand{\rightarr}[1]{\sendmessageright{length=2.5cm, top=\colorbox{white}{$#1$}, topstyle={yshift=-9}}}
\newcommand{\leftarr}[1]{\sendmessageleft{length=2.5cm, top=\colorbox{white}{$#1$}, topstyle={yshift=-9}}}

\newcommand{\cfbox}[2]{
    \colorlet{currentcolor}{.}
    {\color{#1}
    \fbox{\color{currentcolor}#2}}
}

\newcommand{\En}{\mathcal{K}}
\newcommand{\Gn}{\mathcal{G}}
\newcommand{\Po}{\mathcal{P}}
\newcommand{\V}{\mathcal{V}}


\newcommand{\niksdef}[6]{
For all expected polynomial-time adversaries 
$\mathcal{P}^*$ 
there exists an expected polynomial-time extractor
$\mathcal{E}$ such that
    \[
    \Pr_{\mathsf{r}}
    \left[
      \begin{array}{l}
        #6
      \end{array}
      \middle\vert
      \begin{array}{l}
        % Generator
        \mathsf{pp} \gets \mathcal{G}(1^{\lambda}, N),\\ 
        % Adversarial Statement, Proof
        (#1, #2, #4) \gets \mathcal{P}^*(\mathsf{pp}, \mathsf{r}),\\
        % Key Generator
        (\pk, \vk) \gets \En(\pp, #1),\\
        % Precondition
        #5,\\
        % Extractor
        #3 \gets \mathcal{E}(\pp, \mathsf{r})
      \end{array}
    \right]
    \approx 
    1
    \]
    where $\mathsf{r}$ denotes an arbitrarily long random tape.
}


% \newcommand{\pp}{\mathsf{pp}}
% \newcommand{\pk}{\mathsf{pk}}
% \newcommand{\vk}{\mathsf{vk}}

% \newcommand{\R}{\mathcal{R}}

\newcommand{\FP}{F'}
\newcommand{\io}{\mathsf{x}}
\newcommand{\fu}{\mathsf{u}}
\newcommand{\fw}{\mathsf{w}}
\newcommand{\acc}{\mathsf{U}}
\newcommand{\aw}{\mathsf{W}}
\newcommand{\trivi}{\fu_{\bot}}
\newcommand{\trivw}{\fw_{\bot}}
\newcommand{\fold}{\mathsf{NIFS}}
\newcommand{\snark}{\mathsf{SNARK}}
\newcommand{\RIVC}{\R_{\mathsf{IVC}}}
\newcommand{\Str}{\mathsf{s}}
\newcommand{\com}{\mathsf{com}}

% \newcommand{\Gen}{\mathsf{Gen}}
\newcommand{\Commit}{\mathsf{Commit}}

% \newcommand{\negl}[1]{\mathsf{negl}(#1)}

\section{Recursive Proofs of Knowledge}

In this lecture, 
we discuss recursive zero-knowledge succinct non-interactive
arguments of knowledge (zkSNARKs).
%
We assume familiarity with zkSNARKs.

\subsection{Introduction}

% Zero-Knowledge Proofs
Succinct non-interactive arguments of knowledge are short certificates that attest to the correct execution of a computation without revealing any secret inputs. 
%
Today, 
zero-knowledge proofs are being used 
to secure \emph{billions} of dollars worth of assets~\cite{zerocash, stark}.
%
Zero-knowledge proofs 
enable a new class of secure applications 
with enhanced integrity and privacy guarantees
such as verifiable databases~\cite{zkvsql, vsql, accountablestorage, integridb},
private voting protocols~\cite{privatevoting},
anonymous credentials~\cite{cinderella, dacreds},
and
private cryptocurrencies~\cite{zerocash, pinocchiocoin, stark}.

% Proving a Function
More technically, 
SNARKs 
(for circuit-satisfiability)
allow a prover to demonstrate that it knows a secret witness $w$
such that for prescribed circuit $F$ and prescribed input and output pair $(x, y)$ that $F(x, w) = y$.
%
% Proving Recursion
However, today, we will be interested in
proving \emph{recursive} computation. 
Without loss of generality, we are interested in proving \emph{tail}
recursion,
that is, we want to prove
(unbounded) recursive applications 
of a function $F$.
Unbounded recursion
in general allows us to implement more complex programming patterns such as 
$\mathsf{for}$ and $\mathsf{while}$, 
which are not bounded ahead of time.
This allows us to prove stateful computations with dynamic control flow.
%% Use cases
In practice, 
proving recursion allows us to recursively prove blockchain updates, verifiable delay functions, and even a universal machine,
where each recursive step is a single cycle of a CPU.


% Naive Solution
Historically, 
the best known approach to design a proof system for recursive applications of a function $F$ 
was to unroll the entire execution $F \circ F \circ \cdots \circ F$ into a monolithic arithmetic circuit,
and then use a standard proof system with succinct proofs for circuit satisfiability.
%
Unfortunately,
this would necessarily mean that the prover's space complexity would scale with the \emph{entire trace} of the computation.
%
Moreover, 
in the setting of preprocessed arguments of knowledge
(where the prover and verifier would process the circuit into a succinct key to use for multiple inputs)
this fixed the recursion-depth ahead of time,
which often does not reflect practice.

% Recursive Solution Overview
The first breakthrough was due to Valiant~\cite{valiant} in 2012, 
who proposed incrementally verifiable computation (IVC),
which reflected the recursive structure of the computation
into the proof itself:
%
Given a succinct proof $\pi_{i}$ attesting to $i$ steps of computation,
the prover can write a succinct proof $\pi_{i + 1}$ that attests to $i + 1$ steps
by proving the correct execution of an arithmetic circuit 
that runs the latest step of computation,
and checks $\pi_i$ 
(using the proof system's verifier).
This avoids having to fix the recursion depth ahead of time,
while ensuring that the prover's space complexity only scales with a single step of execution.

Although undoubtedly elegant, 
Valiant's technique introduces a subtle issue:
Proofs of knowledge must satisfy a stronger notion of soundness 
known as \emph{knowledge-soundness}.
%
Essentially,
a proof system is considered knowledge-sound 
if, 
for any successful prover with some secret input to the computation, 
there exists a corresponding \emph{extractor} that,
with at most polynomial overhead,
can retrieve this secret input given access to the ``source code'' of the prover.
%
This extractor-based definition becomes problematic in the recursive setting:
Recursive proofs require \emph{recursive extraction} 
in which the extractor for step $i$ 
plays the successful prover for the extractor at step $i - 1$. 
This incurs a polynomial blowup in the extractor for each successive recursive step. 
In particular, 
this results in a final extractor that runs in exponential-time with respect to the recursion-depth, 
which disqualifies it as a valid extractor. 
%
To account for this issue, 
Valiant's original technique 
(and modern techniques) 
can only provably guarantee logarithmic-depth recursion in standard models.

\subsection{Preliminaries}

We operate in the \textit{preprocessing model}, which means that a trusted party will be responsible for generating a prover and verifier key.

% Definition
\begin{definition}
   [Incrementally Verifiable Computation]\label{def:ivc}
   An 
   incrementally verifiable computation (IVC)  
   scheme is defined by
   PPT algorithms 
   $(\mathcal{G}, \mathcal{P}, \mathcal{V})$ 
   and deterministic $\En$
   denoting the generator, 
   the prover, 
   the verifier,
   and the encoder respectively,
   with the following interface
   \begin{itemize}
     \item $\mathcal{G}(1^\lambda, N) \to \pp$: 
     on input security parameter $\lambda$ and size bounds $N$, 
     samples public parameters $\pp$.
     \item $\En(\pp, F) \to (\pk, \vk)$: 
     on input public parameters $\pp$, 
     and polynomial-time function $F$,
     deterministically produces
     a prover key $\pk$ 
     and a verifier key $\vk$.
     \item $\mathcal{P}(\pk, (i, z_0, z_{i}), \omega_{i}, \pi_{i}) \to \pi_{i+1}$: 
     on input a prover key $\pk$, 
     a counter $i$, 
     an initial input $z_0$, 
     a claimed output after $i$ iterations $z_i$,
     a non-deterministic advice $\omega_i$,
     and an IVC proof $\pi_i$ attesting to $z_i$,
     produces a new proof $\pi_{i + 1}$ attesting to $z_{i + 1} = F(z_{i}, \omega_{i})$.
     \item $\mathcal{V}(\vk, (i, z_0, z_{i}), \pi_{i}) \to \{0, 1\}$: 
     on input a verifier key $\vk$,
     a counter $i$,
     an initial input $z_0$, 
     a claimed output after $i$ iterations $z_i$,
     and an IVC proof $\pi_i$ attesting to $z_i$,
     outputs $1$ if $\pi_i$ is accepting, and 
     $0$ otherwise.
   \end{itemize}
     An IVC scheme 
     $(\mathcal{G}, \En, \mathcal{P}, \mathcal{V})$
     satisfies the following requirements.
     \begin{enumerate}
     \item Perfect Completeness:    
     For any
     PPT adversary $\mathcal{A}$
     \begin{equation*}
     \Pr
     \left[
       \begin{array}{l}
         \mathcal{V}(\vk, (i + 1, z_0, z_{i + 1}), \pi_{i + 1}) = 1 
       \end{array}
       \middle\vert
       \begin{array}{l}
         \mathsf{pp} \gets \mathcal{G}(1^{\lambda}, N),\\
         (F, (i, z_0, z_i), (\omega_i, \pi_i)) \gets \mathcal{A}(\mathsf{pp}),\\
         (\pk, \vk) \gets \En(\pp, F),\\
         z_{i + 1} \gets F(z_{i}, \omega_{i}),\\
         \mathcal{V}(\vk, i, z_0, z_{i}, \pi_{i}) = 1,\\
         \pi_{i + 1} \gets \mathcal{P}(\pk, (i, z_0, z_i), (\omega_{i}, \pi_{i}))
       \end{array}
     \right] = 1
     \end{equation*}
     where $F$ is a polynomial-time computable function represented as an arithmetic circuit.
   \item Knowledge Soundness:
   Consider constant $n \in \mathbb{N}$.
   \niksdef
   % Structure
   {F} 
   % Statement
   {(z_0, z_i)}
   % Witness
   {(\omega_0, \ldots, \omega_{n - 1})}
   % Proof Format
   {\Pi}
   % Precondition
   {\mathcal{V}(\vk, (n, z_0, z), \Pi) = 1}
   % Postcondition
   {z_n = z \text{ where }\\ z_{i + 1} \gets F(z_i, \omega_i)\\
   \forall i \in \{0, \ldots, n - 1\}
   }
   Moreover, 
     $F$ is a polynomial-time computable function represented as an arithmetic circuit.
     \item Succinctness: 
     The size of an IVC proof $\pi$ is independent of the number of iterations $i$.
   \end{enumerate}
\end{definition}

\begin{definition}[Non-Interactive Argument of Knowledge]\label{def:nark}
   Consider a relation $\R$ over 
   public parameters, structure, instance, and witness tuples.
   A non-interactive argument of knowledge for $\R$ consists of PPT algorithms
   $(\Gn, \Po,\V)$ 
   and deterministic $\En$,
   denoting the generator, 
   the prover,    
   the verifier
   and the encoder respectively with the following interface.
   \begin{itemize}
       \item $\Gn(1^{\lambda}, N) \to \pp$: 
       On input security parameter $\lambda$,
       and length parameter $N$
       samples public parameters $\pp$.
       \item $\En(\pp, \Str) \to (\pk, \vk)$: 
       On input structure $\Str$, 
       representing common structure among instances,
       outputs the prover key $\pk$ and verifier key $\vk$.
       \item $\Po(\pk, u, w) \to \pi$: On input instance $u$ and
         witness $w$, outputs a proof $\pi$ proving that $(\pp, \Str, u, w) \in \R$.
       \item $\V(\vk, u, \pi) \to \{0, 1\}$: 
       Checks instance $u$ 
       given proof $\pi$.
   \end{itemize}
   An argument of knowledge satisfies \textit{completeness} if for any PPT adversary $\mathcal{A}$,
   \begin{align*}
   \Pr
   \left[
       \begin{array}{l}
       \V(\vk, u, \pi) = 1
       \end{array}
       \middle\vert
       \begin{array}{l}
       \pp \gets \Gn(1^{\lambda}, N),\\
       (\Str,(u, w)) \gets \mathcal{A}(\pp),\\
       (\pp, \Str, u, w) \in \R,\\
       (\pk, \vk) \gets \En(\pp, \Str),\\
       \pi \gets \Po(\pk, u, w)
       \end{array}
       \right]
   = 1.
   \end{align*}
   An argument of knowledge satisfies \textit{knowledge soundness} if for all PPT adversaries $\Po^*$ there exists a PPT extractor $\mathcal{E}$ such that for all randomness $\mathsf{r}$
   \[
   \Pr
   \left[
       \begin{array}{l}
       (\pp, \Str, u, w) \in \R
       \end{array}
       \middle\vert
       \begin{array}{l}
       \pp \gets \Gn(1^{\lambda}, N),\\
       (\Str,u,\pi) \gets \Po^*(\pp, \mathsf{r}),\\
       (\pk, \vk) \gets \En(\pp, \Str),\\
       \V(\vk, u, \pi) = 1,\\
       w \gets \mathcal{E}(\pp, \mathsf{r})
       \end{array}
       \right]
   \approx 1.
   \]
\end{definition}

\begin{definition}[Succinctness]
   A non-interactive argument system is succinct if the size of the proof $\pi$
   is polylogarithmic in the size of the witness $w$.
\end{definition}

\begin{definition}[Commitment Scheme]\label{def:commitment}
   A commitment scheme is defined by polynomial-time algorithm
   $\Gen : \mathbb{N}^2 \to P$
   that produces public parameters given the security parameter and size parameter, 
   a deterministic polynomial-time algorithm
   $\Commit : P \times M \times R \to C$
   that produces a commitment in $C$ given a public parameters, message, and randomness tuple
   such that 
   binding holds.
   That is, 
   for any $\mathsf{PPT}$ adversary $\mathcal{A}$,
   given
   $\pp \gets \Gen(\lambda, n)$,
   and given $((m_1, r_1), (m_2, r_2)) \gets \mathcal{A}(\pp)$
   we have that
   \[
       \Pr[(m_1, r_1) \neq (m_2, r_2) \land \Commit(\pp, m_1, r_1) = \Commit(\pp, m_2, r_2)] \approx 0.
   \]
   %
   The commitment scheme is deterministic if $\Commit$ does not use its randomness. 
 \end{definition}

 \begin{definition}[Circuit Satisfiability]
   We define the circuit satisfiability relation $\mathsf{CSAT}$
   over structure, instance, witness tuples
   as follows.
   \begin{equation*}
     \mathsf{CSAT}
     = 
     \left\{
     \begin{array}{l}
         % Statment, Witness
         (C, (x, y), w)
     \end{array}
     \middle\vert
     \begin{array}{l}
         C(x, w) = y
     \end{array}
     \right\}.
   \end{equation*}  
 \end{definition}

\subsection{Construction}

% Formal Construction

\begin{construction}[Incrementally Verifiable Computation]\label{cons:ivc}
   Given a 
   a succinct commitment scheme
   $(\Gen, \Commit)$
   and a
   succinct non-interactive argument of knowledge 
   $\snark$ for circuit-satisfiability
   we construct an IVC scheme as follows.
   
   %% Function F  
   Consider an arithmetic circuit $F$ 
   that takes non-deterministic input
   We begin by defining an augmented circuit
   $\FP$
   as follows,
   where all input arguments are taken as non-deterministic advice.
   %
   \begin{mdframed}[nobreak=true]
     \noindent \underline{$\FP(\io_i, \omega_i, \pi_i)$}:
     \begin{enumerate}
       \item Parse $(\vk_\snark, i, z_0, z_i) \gets \io_i$.
       \item  If $i = 0$:
       \begin{enumerate}
         \item Check that $z_0 = z_i$.
         \label{ivc:fp:base}
       \end{enumerate}
       \item Otherwise:
       \begin{enumerate} 
         \item Check that $\snark.\V(\vk_\snark, \io_i, \pi_i) = 1$.
         \label{ivc:fp:check:general}
       \end{enumerate}
       \item Output $\io_{i + 1} \gets (\vk_{\snark}, i + 1, z_0, F(z_i, \omega_i))$.
       \label{ivc:fp:output}
     \end{enumerate}  
   \end{mdframed} 
   %
   %
   Given the augmented circuit $\FP$,
   we define $(\Gn, \En, \Po, \V)$ as follows.
   \begin{mdframed}[nobreak=true]
     \underline{$\Gn(\lambda, N)$}:
     \begin{enumerate}
       \item Output $\pp \gets \snark.\mathcal{G}({\lambda}, N)$.
     \end{enumerate}
   \end{mdframed}
   %
   \begin{mdframed}[nobreak=true]
     \underline{$\En(\pp, F)$}:
     \begin{enumerate}
       
       \item Compute 
       $(\pk_\snark, \vk_\snark) \gets \snark.\En(\pp, \FP)$.
       \item Output $\pk \gets (\FP, \pk_\snark, \vk_\snark)$ and $\vk \gets \vk_\snark$.
     \end{enumerate}
   \end{mdframed}
   \begin{mdframed}[nobreak=true]
     \underline{$\Po(\pk, (i, z_0, z_i), (\omega_i, \pi_i))$}:
     \begin{enumerate} 
       \item Parse $(\FP, \pk_\snark, \vk_\snark) \gets \pk$.
       \item Compute
       $\io_{i + 1} \gets \FP(\vk_{\snark},
       (i, z_0, z_i), \omega_i, \pi_i)$.
       \item Let $\io_i \gets (\vk_{\snark}, i, z_0, z_i)$
       \label{ivc:prover:io}
       \item Output
       $
       \pi_{i + 1} \gets \snark.\Po(\pk_\snark, (\bot, \io_{i + 1}), (\io_i
       , \omega_i, \pi_i))
       $.
       \label{ivc:prover:proof}
     \end{enumerate} 
   \end{mdframed}
   %
   \begin{mdframed}[nobreak=true]
     \underline{$\V(\vk, (i, z_0, z_i), \pi_i)$}:
     \begin{enumerate}
       \item If $i = 0$: Check that $z_i = z_0$.
       \label{ivc:verifier:base}
       \item Otherwise:
       \begin{enumerate}
         \item Parse $\vk_{\snark} \gets \vk$.
         \item Let $\io_i \gets (\vk_\snark, i, z_0, z_i)$.
         \label{ivc:v:check:first}
         \item Check that
         $\snark.\V(\vk_\snark, (\bot, \io_i), \pi_i) = 1$.
         \label{ivc:v:check:second}
       \end{enumerate}
       
     \end{enumerate}
   \end{mdframed}
 \end{construction}

\begin{lemma}[Completeness]
 Construction~\ref{cons:ivc}
 is complete.
\end{lemma}
\newcommand{\proof}{\noindent{\bf Proof. }} %% To begin a proof write \proof

\begin{proof}
   % Adversary
 Consider arbitrary PPT adversary $\mathcal{A}$.
 % public parameters
 Suppose $\pp \gets \Gen(1^{\lambda}, N)$.
 % Structure, Instance, Witness
 Suppose that
 \[
 (F, (z_0, z_i, i), \pi_i) \gets \mathcal{A}(\pp).
 \]
 % Prover Key, Verifier Key
 Suppose that for
   $(\pk, \vk) \gets \En(\pp, F)$
 % Precondition
 we have that
 \begin{align}\label{ivc:completeness:precondition:verifier}
   \V(\vk, (z_0, z_{i}, i), \pi_{i}) = 1.
 \end{align}
 % Postcondition
 Then,
 given
 \begin{align*}
   z_{i + 1} \gets F(z_{i}, \omega_{i})
 \end{align*}
 and 
 \begin{align*}
   \pi_{i + 1} \gets \Po(\pk, (z_0, z_i, i), (\omega_i, \pi_i))
 \end{align*}
 we must show that
 \begin{align}\label{ivc:completeness:postcondition}
   \V(\vk, (z_0, z_{i+1}, N), \pi_{i + 1}) = 1
 \end{align}
 with probability $1$.
 %
 
 % Case i = 0
 Indeed, 
 consider the base case where $i = 0$.
 %
 Then, 
 by Precondition~\ref{ivc:completeness:precondition:verifier},
 by the verifier's check in the base case 
 (Step~\ref{ivc:verifier:base})
 we have that $z_0 = z_i$.
 %
 Therefore,
 $\Po$ can successfully compute $\io_{i + 1}$ 
 (Step~\ref{ivc:prover:io}),
 because the base case check of $\FP$ 
 (Step~\ref{ivc:fp:base}) passes.
 %
 Then,
 by construction of $\FP$
 (Step~\ref{ivc:fp:output})
 we have that 
 \[
   \io_{i + 1} = (\vk_{\snark}, i + 1, z_0, F(z_i, \omega_i))
 \]
 Moreover,
 by the completeness
 of $\snark$,
 we have that $\pi_{i + 1}$ generated by $\Po$ 
 (Step~\ref{ivc:prover:proof})
 is indeed satisfying.
 %
 Therefore, 
 both the checks of $\V$ in Steps~\ref{ivc:v:check:first} and~\ref{ivc:v:check:second}
 are passing.
 %
 As such,
 we have that postcondition~\ref{ivc:completeness:postcondition}
 holds.
 
 
 % Case i \geq 1
 Suppose instead that $i \geq 1$.
 %
 by Precondition~\ref{ivc:completeness:precondition:verifier},
 by the verifier's check in the general case 
 we have that
 \begin{align*}
   \snark.\V(\vk_\snark, (\bot, \io_i), \pi_i) = 1
 \end{align*}
 for $\io_i = (\vk_\snark, i, z_0, z_i)$.
 %
 Then,
 $\Po$ can successfully compute $\io_{i + 1}$ 
 (Step~\ref{ivc:prover:io}),
 as the SNARK verifier check in $\FP$
 (Step~\ref{ivc:fp:check:general})
 holds.
 %
 %
 Once again,
 by construction of $\FP$
 (Step~\ref{ivc:fp:output})
 we have that 
 \[
   \io_{i + 1} = (\vk_{\snark}, i + 1, z_0, F(z_i, \omega_i))
 \]
 Moreover,
 by the completeness
 of $\snark$,
 we have that $\pi_{i + 1}$ generated by $\Po$ 
 (Step~\ref{ivc:prover:proof})
 is indeed satisfying.
 %
 Therefore, 
 both the checks of $\V$ in Steps~\ref{ivc:v:check:first} and~\ref{ivc:v:check:second}
 are passing.
 %
 As such,
 we have that postcondition~\ref{ivc:completeness:postcondition}
 holds.
\end{proof}

\begin{lemma}[Knowledge-Soundness]
 Construction~\ref{cons:ivc}
 is knowledge-sound.
\end{lemma}

% Proof for ivc knowledge soundness

\begin{proof}
   % Premise
   Let $n$ be a global constant.
   %
   Consider a deterministic expected polynomial-time adversary $\Po^*$.
   %
   Let $\pp \gets \Gen(1^{\lambda}, N)$.
   %
   Suppose 
   on input $\pp$ and randomness $\mathsf{r}$,
   $\Po^*$
   outputs 
   polynomial-time function $F$,
   instance $(z_0, z)$,
   and IVC proof $\pi$. 
   %
   Let $(\pk, \vk) \gets \En(\pp, F)$.
   Suppose that
   % Precondition
   \begin{align*}
     \V(\vk, (z_0, z, n), \pi) = 1.
   \end{align*}
   % Postcondition
   We must construct an expected polynomial-time extractor $\mathcal{E}$
   that, 
   with input $(\pp, \mathsf{r})$,
   outputs $(\omega_0, \ldots, \omega_{n - 1})$ 
   such that by computing
   \begin{align*}
     z_{i + 1} \gets F(z_{i}, \omega_{i})
   \end{align*}
   we have that $z_n = z$ with probability 
   $1 - \negl{\lambda}$.
   
   % Overview
   We show inductively that we can construct an expected polynomial-time extractor $\mathcal{E}_i$ that outputs 
   $((z_i, \ldots, z_{n - 1}), (\omega_i, \ldots, \omega_{n - 1}), \pi_i)$ such that for all $j \in  \{i + 1, \ldots, n\}$,
   \begin{align*}
     z_j = F(z_{j - 1}, \omega_{j - 1})
   \end{align*}
   and 
   \begin{align}\label{eq:ih:2}
     \V(\vk, z_0, z_i, \pi_i) = 1
   \end{align}
   for $z_n = z$ with probability $1 - \negl{\lambda}$.
   %
   Then, 
   because when $i = 0$,
   $\V$ checks that $z_0 = z_i$ 
   the values $(\omega_0, \ldots, \omega_{n - 1})$ 
   retrieved by $\mathcal{E} = \mathcal{E}_0$
   are such that computing  
   $z_{i + 1} = F(z_i, \omega_i)$ for all $i \geq 1$ gives $z_n = z$.
   
   At a high level,
   to construct an extractor $\mathcal{E}_{i - 1}$,
   we first assume the existence of $\mathcal{E}_i$ that satisfies the inductive hypothesis. 
   We then use $\mathcal{E}_i$ to construct an adversary 
   for the underlying succinct non-interactive argument, which we denote as $\widetilde{\Po}_{i - 1}$.
   This in turn guarantees an extractor for the underlying non-interactive argument,
   which we denote as $\widetilde{\mathcal{E}}_{i - 1}$. 
   We then use $\widetilde{\mathcal{E}}_{i - 1}$ to construct $\mathcal{E}_{i - 1}$ that satisfies the inductive hypothesis.
   
   
   % Base case
   In the base case,
   let $\mathcal{E}_n(\pp, \mathsf{r})$ 
   output 
   $(\bot, \bot, \pi_n)$ 
   where $\pi_n$ is the output of 
   $\Po^*(\pp, \mathsf{r})$.
   By the premise,
   we have that $\pi_n$ is satisfying.
   As such,
   $\mathcal{E}_n$ succeeds with probability $1 - \negl{\lambda}$ in expected polynomial-time.
   
   % Inductive Step
   For $i \geq 1$, 
   suppose we can construct an expected polynomial-time extractor 
   $\mathcal{E}_i$ 
   that outputs
   $((z_i, \ldots, z_{n - 1}), (\omega_i, \ldots, \omega_{n - 1}))$, 
   and $\pi_i$ 
   that satisfies the inductive hypothesis.
   % Construct SNARK prover
   To construct an extractor $\mathcal{E}_{i - 1}$, 
   we first construct an adversary $\widetilde{\Po}_{i - 1}$
   for the underlying SNARK as follows.
   %
   %
   %
   \begin{mdframed}[nobreak=true]
     \noindent \underline{$\widetilde{\Po}_{i - 1}(\pp, \mathsf{r})$}: 
     \begin{enumerate}
       \item Let $(F, z_0) \gets \Po^*(\pp, \mathsf{r})$
       \item Let $((z_i, \ldots, z_{n - 1}), (\omega_i, \ldots, \omega_{n -
       1}), \pi_i) \gets \mathcal{E}_i(\pp, \mathsf{r})$.
       \item Let $\vk_{\snark} \gets \snark.\En(\pp, F)$.
       \item Let $\io_i \gets (\vk_\snark, (i, z_0, z_i))$.
       \item Output $(\FP, (\bot, \io_i), \pi_i)$.
     \end{enumerate}
   \end{mdframed}
   
   % SNARK prover success probability
   We now analyze the success probability of $\widetilde{\Po}_{i - 1}$. 
   By the inductive hypothesis,
   we have that 
   \[
   \V(\vk, z_0, z_i, \pi_i) = 1,
   \]
   where 
   $\pi_i \gets \mathcal{E}_i(\pp, \mathsf{r})$ 
   with probability 
   $1 - \negl{\lambda}$.
   Therefore, 
   by the the verifier's checks 
   we have that
   \[
   \snark.\V(\vk_\snark, (\bot, \io_i), \pi_i) = 1
   \]
   for $\io_i = (\vk_\snark, (i, z_0, z_i))$
   and $\vk_\snark \gets \snark.\En(\pp, \FP)$.
   Therefore,
   $\widetilde{\Po}_{i - 1}$
   succeeds in producing a satisfying proof $\pi_i$
   for structure $\FP$ and instance $\io_i$
   with probability $1 - \negl{\lambda}$.
 
   
   % Corresponding SNARK extractor
   Then,
   by the knowledge soundness of $\snark$
   there exists an 
   expected-polynomial-time
   extractor $\widetilde{\mathcal{E}}_{i - 1}$ that
   outputs 
   $(\io_{i - 1}, \omega_{i - 1}, \pi_{i - 1})$
   such that
   $\io_i = \FP(\io_{i - 1}, \omega_{i - 1}, \pi_{i - 1})$
   with probability $1 - \negl{\lambda}$.
   
   % Constructing E_{i - 1}
   Given $\widetilde{\mathcal{E}}_{i - 1}$, 
   we construct an expected polynomial time $\mathcal{E}_{i - 1}$ as follows.
 
 
   \begin{mdframed}[nobreak=true]
   \noindent \underline{$\mathcal{E}_{i - 1}(\pp, \mathsf{r})$}: 
   \begin{enumerate}
     \item Run $\widetilde{\Po}_{i - 1}(\pp, \mathsf{r})$
     to parse
     \[
     ((z_i, \ldots, z_{n - 1}), (\omega_i, \ldots, \omega_{n -
     1}), \pi_i)
     \]
     from its internal state.
     \item Compute $(\io_{i - 1}, \omega_{i - 1}, \pi_{i - 1}) \gets \widetilde{\mathcal{E}}_{i - 1}(\pp, \mathsf{r})$.
     \item Parse $(\vk_\snark, (i - 1, z_0, z_{i - 1})) \gets \io_{i - 1}$
     \item Output $((z_{i - 1}, \ldots, z_{n - 1}), (\omega_{i - 1}, \ldots, \omega_{n - 1}), \pi_{i - 1})$. 
   \end{enumerate}
 \end{mdframed}
 
  We now reason about the success probability of $\mathcal{E}_{i - 1}$.
   We first reason that the output $(z_{i - 1}, \ldots, z_{n - 1})$, and
   $(\omega_{i - 1}, \ldots, \omega_{n - 1})$ are valid.
   By the inductive hypothesis, 
   we already have that for all $j \in \{i + 1, \ldots, n\}$,
   \begin{align*}
     z_j = F(z_{j - 1}, \omega_{j - 1})
   \end{align*}
   with probability $1 - \negl{\lambda}$.
   Moreover,
   by the success probability of $\mathcal{E}_{i - 1}$,
   we have that
   \begin{align*}
     z_{i} = F(z_{i - 1}, \omega_{i - 1})
   \end{align*}
   and that
   \[
   \snark.\V(\vk_\snark, \io_{i - 1}, \pi_{i - 1}) = 1
   \]
   where $\io_{i - 1} = (\vk_\snark, (i - 1, z_0, z_i))$
   with probability $1 - \negl{\lambda}$.
   %
   Therefore we have that $\mathcal{E}_{i - 1}$
   succeeds with probability $1 - \negl{\lambda}$
   satisfying the inductive hypothesis.
 \end{proof} 
%\newcommand{\proofsketch}{\smallskip\noindent{\bf Proof sketch. }}
\algrenewcommand\algorithmicfunction{\textbf{Machine}}
\newcommand{\bits}{\set{0,1}}
\newcommand{\Ex}{\mathbb{E}}

\renewcommand{\O}{\ensuremath{\mathcal{O}}}
\newcommand{\To}{\rightarrow}
\newcommand{\e}{\epsilon}
% \newcommand{\R}{\mathbb{R}}
\newcommand{\N}{\mathbb{N}}
\newcommand{\Z}{\mathbb{Z}}
\newcommand{\logAnd}{\wedge}

\newcommand{\indis}{\mathrel{\overset{\makebox[0pt]{\mbox{\normalfont\tiny c}}}{\approx}}}
\newcommand{\allindis}{\mathrel{\overset{\makebox[0pt]{\mbox{\normalfont\tiny p/s/c}}}{\approx}}}

\newcommand{\cclass}[1]{\mathsf{#1}}
\renewcommand{\P}{\cclass{P}}
\newcommand{\NP}{\cclass{NP}}
\newcommand{\Time}{\cclass{Time}}
\newcommand{\BPP}{\cclass{BPP}}
\newcommand{\Size}{\cclass{Size}}
\newcommand{\Ppoly}{\cclass{P_{/poly}}}
\newcommand{\CSAT}{\ensuremath{\mathsf{CSAT}}}
\newcommand{\SAT}{\ensuremath{\mathsf{3SAT}}}
\newcommand{\IS}{\mathsf{INDSET}}



\newcommand{\inp}{\mathsf{in}}
\newcommand{\outp}{\mathsf{out}}

\newcommand{\Param}{\kappa}
\newcommand{\Adv}{\mathsf{Adv}}
\newcommand{\Supp}{\mathsf{Supp}}


\newcommand{\PRG}{\mathsf{G}}
\renewcommand{\Enc}{\mathsf{Enc}}
\renewcommand{\Dec}{\mathsf{Dec}}
\renewcommand{\sk}{\mathsf{sk}}
\newcommand{\sfC}{\mathsf{C}}
\newcommand{\sfR}{\mathsf{R}}

\newcommand{\eqdef}{\stackrel{\text{\tiny def}}{=}}

\newcommand{\cF}{\mathcal{F}}

\newcommand{\angles}[1]{\langle #1 \rangle}
\newcommand{\iprod}[1]{\angles{#1}}

\newcommand{\Com}{\mathsf{Com}}


% Real vs. Ideal
\newcommand{\RealAdv}{\mathcal{A}}
\newcommand{\IdealAdv}{\mathcal{S}}
\newcommand{\RealVar}{\mathsf{Real}}
\newcommand{\IdealVar}{\mathsf{Ideal}}

\newcommand{\RealView}[2]{\mathsf{Real}^{#1}_{#2}}
\newcommand{\IdealView}[2]{\mathsf{Ideal}^{#1}_{#2}}

% Participating parties
\newcommand{\PartyA}{P_1}
\newcommand{\PartyB}{P_2}
\newcommand{\InputA}{x_1}
\newcommand{\InputB}{x_2}


% Garbling Schemes
\newcommand{\Garble}{\mathsf{Garble}}
\newcommand{\Cir}{C}
\newcommand{\GCir}{\widetilde{C}}
\newcommand{\Lab}{\mathsf{lab}}

% Proof
\newcommand{\Sim}{\mathsf{Sim}}

% GMW

% Misc
\newcommand{\out}{\mathsf{out}}
\newcommand{\Assign}{:=}

\chapter{Secure Computation}

\section{Introduction}
Secure multiparty computation (or MPC) considers the problem of enabling mutually distrusting parties
to compute a joint public function on their private inputs without revealing
any extra information about these inputs beyond what is leaked by the output
of the computation.
For instance, there could be three parties $P_1, P_2, P_3$ with private inputs $x_1, x_2, x_3$ respectively, and they want to compute a public function $f$ without revealing anything beyond $f(x_1, x_2, x_3)$.
Note that this notion generalizes zero knowledge proofs, where the prover's input is the statement-witness pair $(x, w)$, the verifier's input is the statement $x$, and the public function $f$ is the verification algorithm for the corresponding relation which outputs $1$ if $w$ is a valid witness w.r.t. $x$.

This setting is well motivated, and captures many different applications.
Now, we look at some applications of MPC which will help build intuition and also highlight the challenge in defining security for MPC.
% Considering some of these applications will provide intuition about how security should be defined for secure computation:
\begin{description}
  \item[Voting:] Electronic voting can be thought of as a multi-party computation
	  between $n$ players: the voters. Their input is their choice $b \in \{0,1\}$
    (we restrict ourselves to the binary choice setting without loss of generality), and the function
    they wish to compute is the majority function.
    % Now consider what happens when only one user votes: their input is trivially revealed as the output of the computation. What does privacy of inputs mean in this scenario?

  \item[Yao's Millionaires' Problem:] Two millionaires want to know who is richer
    without revealing their actual wealth. The function they want to compute is
    $f(x_1, x_2) = x_1 > x_2$.
    % Let's say $x_2 = 1$M $+ 1$ and $f(x_1, x_2) = 0$. Then 
  
  \item[Private Contact Discovery:] Signal employs MPC to identify contacts from your phone who are also on Signal, all without exposing your contact list to Signal.
  Here the function $f$ is an intersection function over the user's contact list and Signal's member list. This specific application of MPC is commonly known as private set intersection (PSI) and has been the subject of extensive research.

  \item[Password Breaches:] Suppose your browser needs to verify whether your password has been compromised, but without disclosing it to a service that maintains a database of leaked passwords. This is also an instance of PSI.

  \item[Cryptographic Wallets:] Cryptographic wallets typically split their keys into multiple shards that are then given to multiple parties (e.g., the user's device and the wallet provider) so that compromising any one party does not reveal the keys. In principle, whenever these keys are needed, the client can get the shards from each party and perform the operation. However, the client becomes the single point of compromise for that period and that defeats the point of sharding. MPC enables the parties holding the shards to perform any operations on the keys without reconstructing it on a single device.

  \item[Searchable Encryption:] Searchable encryption schemes allow clients
    to store their data with a server, and subsequently grant servers tokens
    to conduct specific searches.
    % However, most schemes do not consider access pattern leakage. This leakage tells the server potentially valuable information about the underlying plaintext. How do we model all the different kinds information that is leaked?
\end{description}

Note that in all these applications, the output revealed to the parties itself can reveal a lot of information about the private inputs of the parties.
For instance, in the voting example, if only one user votes, their input is trivially revealed.
Or, in the case of the millionaires, if $x_1 = 10^6 + 1$ and $f(x_1, x_2) = 1$, then the first millionaire knows that $x_2 = 10^6$.
Similarly, most searchable encryption schemes do not consider access pattern leakage. This leakage tells the server potentially valuable information about the underlying plaintext.
How do we model all the different kinds information that is leaked?

From these examples we see that defining security of MPC is tricky because it is imperfect in that some leakage in inherent in the computation.
What we want to capture is that no party should learn anything about the private inputs of other parties beyond what is revealed by the output of the computation.
% We want to ensure that no party can learn anything more from the secure computation protocol than it can from just its input and the result of the computation.
To formalize this notion, we adopt the \textbf{real/ideal world paradigm}.



\section{Real/Ideal World Paradigm}
Suppose there are $n$ parties and each party $P_i$ has a private input $x_i$.
They want to compute a public function (represented by circuit $C$) on their inputs.
The goal is to do this securely: even if some parties are corrupted, no party $P_i$ should learn anything beyond $y_i$ where $(y_1, \ldots, y_n) = C(x_1, \ldots, x_n)$.

\paragraph{Real World.} In the real world, the $n$ parties interact with each other and participate in a protocol $\Pi$
to compute $C$.
This protocol can involve multiple rounds of interaction.
The real world adversary $\RealAdv$ can corrupt a subset of the parties.
The interaction is summarized in Figure~\ref{fig:real-world}.

\begin{marginfigure}
  \centering
  \includegraphics[width=\textwidth]{figures/real-world.pdf}
  \caption{Real world interaction.} \label{fig:real-world} 
\end{marginfigure}

\paragraph{Ideal World.} In the ideal world, an angel or trusted third party $\cF_{C}$ (parameterized by $C$) helps in the computation of $C$.
The computation in this world is secure by design:
each party $P_i$ sends its input $x_i$ to $\cF_C$ and receives the output $y_i$ of the computation $(y_1, \ldots, y_n) = C(x_1, \dotsc, x_n)$.
In this world, an ideal world adversary $\IdealAdv$ controls the parties corrupted by $\RealAdv$ in the real world, and as such, is responsible for sending their inputs to $\cF_C$.
The interaction is summarized in Figure~\ref{fig:ideal-world}.
% $\IdealAdv$ has access to the inputs of the corrupted parties and can arbitrarily choose the inputs sent to $\cF_C$.
\begin{marginfigure}
  \centering
  \includegraphics[width=\textwidth]{figures/ideal-world.pdf}
  \caption{Ideal world interaction.} \label{fig:ideal-world} 
\end{marginfigure}

% To model malicious adversaries, we need to modify the ideal world model as follows. 
% Some parties are honest, and each honest party $P_i$ simply sends $x_i$ to the angel. The other parties are corrupted and are under control of the adversary $\IdealAdv$. The adversary chooses an input $x_i'$ for each corrupted party $P_i$ (where possibly $x_i' \neq x_i$) and that party then sends $x_i'$ to the angel. The angel computes a function $f$ of the values she receives (for example, if only party 1 is honest, then the angel computes $f(x_1, x_2', x_3', \dotsc, x_n')$) in order to obtain a tuple $(y_1, \dotsc, y_n)$. 
% She then sends $y_i$ of corrupted parties to the adversary, who gets to decide whether or not honest parties will receive their response from the angel. The angel obliges. Each honest party $P_i$ then outputs $y_i$ if they receive $y_i$ from the angel and $\perp$ otherwise, and corrupted parties output whatever the adversary tells them to. 

\paragraph{Notation.}
Let $\lambda$ denote the security parameter. Let $\RealView{\RealAdv}{\Pi}(1^\lambda, x_1, \ldots, x_n, z)$ denote the joint distribution of the honest party outputs as well as the view of the corrupted parties (randomness, inputs, outputs, all messages seen during the protocol) in the real world, where $z$ is an auxiliary inputs available to $\RealAdv$.
Let $\IdealView{\IdealAdv}{\cF_C}(1^\lambda, x_1, \ldots, x_n, z)$ denote the joint distribution of the honest party outputs in the ideal world and the protocol transcript (or view) output by ideal adversary $\IdealAdv$ given inputs and outputs of corrupted parties.

\paragraph{Security Definition.} 
A protocol $\Pi$ securely realizes $C$ if there exists a PPT ideal adversary $\IdealAdv^{\RealAdv(\cdot)}$ (with oracle access to the real world adversary) such that $\forall$ non-uniform PPT adversaries $\RealAdv$ in the real world, $\forall$ private inputs $x_1, \ldots, x_n$, and $\forall$ auxiliary inputs $z \in \bit^{\ast}$, we have:
\[ \RealView{\RealAdv}{\Pi}(1^\lambda, x_1, \ldots, x_n, z) \approx_{c} \IdealView{\IdealAdv}{\cF_C}(1^\lambda, x_1, \ldots, x_n, z) \]

% there exists an PPT adversary $\IdealAdv$ in the ideal world such that for all tuples of bit strings $(x_1, \dotsc, x_n)$, we have
% \[ \mathrm{Real}_{\Pi, \RealAdv}(x_1, \dotsc x_n) \stackrel{c}{\simeq} \mathrm{Ideal}_{F,\IdealAdv}(x_1, \dotsc, x_n) \]
% where the left-hand side denotes the output distribution induced by $\Pi$ running with $\RealAdv$, and the right-hand side denotes the output distribution induced by running the ideal protocol $F$ with $\IdealAdv$. 
% The ideal protocol is either the original one described for semi-honest adversaries, or the modified one described for malicious adversaries. 





%We require that the views of the parties
%in each of the scenarios be identical, i.e.\ that a real-world execution of the
%protocol $\Pi$ should not leak any information not leaked by the ideal-world
%execution. Hence, the parties can only learn what they can infer from their
%inputs and the output $f(\InputA, \InputB)$. More formally, assuming $\RealAdv$
%corrupts one party (say $\PartyA$, wlog), we define random variables
%$\RealVar_{\Pi, \RealAdv}(\InputA, \InputB) = \RealAdv(\InputA, r_1, \text{messages
%sent in } \Pi)$ and $\IdealVar_{F, \IdealAdv}(\InputA, \InputB) = \IdealAdv(\InputA,
%f(\InputA,\InputB))$.  These random variables represent the views of the
%adversary in each of the two settings. Our definition of security thus requires
%that
%
%\begin{equation*}
%\RealVar_{\Pi, \RealAdv}(\InputA, \InputB) \indis \IdealVar_{F, \IdealAdv}(x_1, x_2).
%\end{equation*}

\paragraph{Setting.} So far, we have brushed over some important details of the setting.
Below we state these details explicitly:
\begin{enumerate}
  \item \textbf{Assumption:} The protocol could rely on cryptographic assumptions and be secure against computationally bounded adversaries, or it could be statistically secure and protect against unbounded adversaries.
  \item \textbf{Setup:} The parties in the protocol could have access to a common reference string (CRS) that is generated by the trusted party in the ideal world.
  \item \textbf{Communication Channel:} 
  The setting could assume private peer-to-peer (P2P) channels among the parties, a broadcast channel, a combination of both, or insecure channels which don't have privacy or integrity. In the statistical setting, we assume secure P2P channels because they require computational assumptions to set up.
  \item \textbf{Corruption Type:} We consider primarily two types of adversaries:
    \begin{itemize}
      \item \emph{Semi-honest adversaries:} Corrupted parties follow the protocol
        execution $\Pi$ honestly, but attempt to learn as much information as they
        can from the protocol transcript.
      \item \emph{Malicious adversaries:} Corrupted parties can deviate arbitrarily
        from the protocol $\Pi$.
    \end{itemize}
  % We assume that the communication channel between the involved parties is completely insecure, i.e., it does not preserve the privacy of the messages. However, we assume that it is reliable, which means that the adversary can drop messages, but if a message is delivered, then the receiver knows the origin.
  \item \textbf{Corruption Model:} We have different models of how and when the
    adversary can corrupt parties involved in the protocol:
    \begin{itemize}
      \item \emph{Static:} The adversary chooses which parties to corrupt before the protocol execution starts, and during the protocol, the malicious parties remain fixed.
      \item \emph{Adaptive:} The adversary can corrupt parties dynamically during the protocol execution and the state of that honest party corrupted is given to the adversary. 
      % \item \emph{Mobile:} Parties corrupted by the adversary can be ``uncorrupted'' at any time during the protocol execution at the adversary's discretion.
    \end{itemize}
  \item \textbf{Fairness:}
  The protocol we consider could be ``fair'', i.e., if one party gets their output, then all parties get their output, or they could be unfair where a corrupted party can abort the computation after learning the output before the honest parties can learn it.
  It is important to model this weakness of the protocol in the ideal functionality $\cF_C$.
  This is done by having the ideal functionality ask the ideal adversary if the honest parties should receive an output. If adversary disapproves, the ideal functionality sends the special abort output $\bot$ to the honest parties.
  % The protocols we consider are not ``fair'', i.e., the adversary can cause corrupted parties to abort arbitrarily. This can mean that one party does not get its share of the output of the computation.

  \item \textbf{Corruption Bound:} 
  The setting places some upper bound $t$ on the total parties $n$ that can be corrupted by the adversary (there is nothing to protect when all parties are corrupted). 
  Some well-known corruption settings are $t<n$, $t<n/2$, $t<n/3$, and $t \leq \log n$.
  % In some scenarios, we place upper bounds on the number of parties that the adversary can corrupt.

  \item \textbf{Standalone vs Concurrent Execution:}
  In some settings, protocols can be executed in isolation; only one instance of a particular protocol is ever executed at any given time. In other settings, many different protocols can be executed concurrently. This can compromise security.
\end{enumerate}

\section{A Simple and Efficient MPC Protocol}
This protocol is highly efficient and considers a very simple setting: $n=3$ parties, corruption bound $t=1$, unbounded and semi-honest adversary, statistical security, unfairness, and secure P2P channels.

\noindent\textbf{(2,3)-secret sharing scheme:}
A $(n,t)$-sharing splits a secret such that any $t$ shares are enough to reconstruct the secret, and any set of less than $t$ shares do not reveal anything about the secret.
Let $k \in \mathbb{Z}_p$ be a secret. A $(3,3)$-sharing is easy to construct: sample $k_1, k_2$ uniformly at random from $\mathbb{Z}_p$ and set $k_3 = k - k_1 - k_2 \mod p$.
Given $(3,3)$-shares, we can construct a $(2,3)$-sharing by considering a replicated secret sharing scheme: each party $P_i$ gets shares $k_i, k_{i+1 \mod 3}$. It is easy to see that the shares held by any two parties are sufficient to reconstruct the secret.

\noindent\textbf{Protocol:}
The protocol proceeds in the following phases:
\begin{description}
  \item[Input Sharing Phase:] 
  Each party $P_i$ shares their input $x_i$ using the $(2,3)$-sharing scheme, and sends the corresponding shares to the other parties.
  For instance, $P_1$ shares $x_1$ as $(x_{1,1}, x_{1,2}, x_{1,3})$, keeps $(x_{1,1}, x_{1,2})$, and sends $(x_{1,2}, x_{1,3})$ to $P_2$ and $(x_{1,3}, x_{1,1})$ to $P_3$.
  \item[Computation Phase:] 
  The computation phase maintains the invariant that given $(2,3)$-shares of the input to a gate (addition or multiplication), the computation produces $(2,3)$-shares of the output of the gate. That way, the parties can keep compute the gates all the way to the output of the circuit $C$.

  \textbf{Addition Gate}:
  Addition gates are simple: given $(2,3)$-shares of $\alpha$ and $\beta$, the parties can compute $(2,3)$-shares of $\alpha + \beta$ by simply summing up their shares locally.
  For instance, party $P_1$ can compute its shares of $\alpha+\beta$ by computing $(\alpha_1 + \beta_1, \alpha_2 + \beta_2)$. 

  \textbf{Multiplication Gate}:
  Multiplication gates are more involved. The product $\gamma = \alpha \cdot \beta$ can be represented in the form of shares as follows: $\gamma = \alpha \cdot \beta = (\alpha_1 + \alpha_2 + \alpha_3) \cdot (\beta_1 + \beta_2 + \beta_3)$.
  There are nine terms in this product, and note that all parties together can compute every term given their shares of $\alpha$ and $\beta$. 
  Thus, assuming each party computes the sum of three mutually exclusive terms (let's call it $\gamma_i$ for party $P_i$), they can locally compute $(3,3)$-shares of $\gamma$.
  To go from $(3,3)$-shares to $(2,3)$-shares, each party $P_i$ can send $\gamma_i$ to party $P_{i-1 \mod 3}$.

  We are not done yet. These shares represent partial sums and can reveal information about the secret. To prevent this, each party needs to additionally mask $\gamma_i$ before sending it.
  In particular, $P_i$ does the following: (i) mask $\gamma_i$ by adding a random value $r_i$ to it before sending it to $P_{i-1 \mod 3}$, (ii) subtract $r_i$ from $\gamma_{i+1 \mod 3}$ received from $P_{i+1 \mod 3}$, and (iii) additionally send $r_i$ to $P_{i+1 \mod 3}$, who also subtracts $r_i$ from $\gamma_{i+1 \mod 3}$.

  \item[Output Reconstruction Phase:] 
  Each party simply publishes its shares of the output of the circuit $C$.
\end{description}

\begin{theorem}
  The protocol described above securely realizes the circuit $C$ in the presence of a static semi-honest adversary that corrupts at most one party.
\end{theorem}
\begin{proof}
 Let $P_i$ be the corrupted party without loss of generality. Consider the following ideal world adversary $\IdealAdv$ that simulates the real world adversary $\RealAdv$:
  \begin{itemize}
    \item \emph{Input Sharing Phase:} $\IdealAdv$ knows the input $x_i$ of corrupted party $P_i$. It sends it to $\cF_C$ and receives the output $y_i = C(x_1, x_2, x_3)$. To simulate adversary's view perfectly during the input sharing phase, $\IdealAdv$ simply sends uniformly random strings on behalf of the honest parties.
    \item \emph{Computation Phase:} The view of corrupted party only consists of the messages transferred during the evaluation of multiplication gates. Note that the message received by a party is uniformly random due to the random masking, and thus, $\IdealAdv$ can again simulate the view perfectly by simply including uniformly random messages.
    \item \emph{Output Reconstruction Phase:} 
    The view of corrupted party during output reconstruction phase is the shares output by the honest parties. Note that $\IdealAdv$ knows the randomness of corrupted party $P_i$, its inputs as well as all the messages it saw during the protocol. Thus, it knows the exact shares held by $P_i$. Given the output $y_i$ for the corrupted party received from $\cF_C$ and the shares held by $P_i$, $\IdealAdv$ can infer what the shares output by honest parties should be. This completes the simulation.
  \end{itemize}
  It is easy to see that this transcript is identical to the real world transcript, and thus the protocol is secure.
\end{proof}

\section{Oblivious transfer}

\emph{Rabin's oblivious transfer} sets out to accomplish the following special task of two-party secure computation. The sender has a bit $s \in \{0,1\}$. She places the bit in a box. Then the box reveals the bit to the receiver with probability 1/2, and reveals $\perp$ to the receiver with probability 1/2. The sender cannot know whether the receiver received $s$ or $\perp$, and the receiver cannot have any information about $s$ if they receive $\perp$.

\subsection{1-out-of-2 oblivious transfer}
\emph{1-out-of-2 oblivious transfer} sets out to accomplish the following related task. The sender has two bits $s_0, s_1 \in \{0,1\}$ and the receiver has a bit $c \in \{0,1\}$. The sender places the pair $(s_0, s_1)$ into a box, and the receiver places $c$ into the same box. The box then reveals $s_c$ to the receiver, and reveals $\perp$ to the sender (in order to inform the sender that the receiver has placed his bit $c$ into the box and has been shown $s_c$). The sender cannot know which of her bits the receiver received, and the receiver cannot know anything about $s_{1-c}$.

\begin{lemma}
A system implementing 1-out-of-2 oblivious transfer can be used to implement Rabin's oblivious transfer.
\end{lemma}

\proof
The sender has a bit $s$. She randomly samples a bit $b \in \{0,1\}$ and $r \in \{0,1\}$, and the receiver randomly samples a bit $c \in \{0,1\}$. If $b = 0$, the sender defines $s_0 = s$ and $s_1 = r$, and otherwise, if $b = 1$, she defines $s_0 = r$ and $s_1 = s$. She then places the pair $(s_0, s_1)$ into the 1-out-of-2 oblivious transfer box. The receiver places his bit $c$ into the same box, and then the box reveals $s_c$ to him and $\perp$ to the sender. Notice that if $b = c$, then $s_c = s$, and otherwise $s_c = r$. Once $\perp$ is revealed to the sender, she sends $b$ to the receiver. The recieiver checks whether or not $b = c$. If $b = c$, then he knows that the bit revealed to him was $s$. Otherwise, he knows that the bit revealed to him was the nonsense bit $r$ and he regards it as $\perp$. \\

It is easy to see that this procedure satisfies the security requirements of Rabin's oblivious transfer protocol. Indeed, as we saw above, $s_c = s$ if and only if $b = c$, and since the sender knows $b$, we see that knowledge of whether or not the bit $s_c$ received by the receiver is equal to $s$ is equivalent to knowledge of $c$, and the security requirements of 1-out-of-2 oblivious transfer prevent the sender from knowing $c$. Also, if the receiver receives $r$ (or, equivalently, $\perp$), then knowledge of $s$ is knowledge of the bit that was not revealed to him by the box, which is again prevented by the security requirements of 1-out-of-2 oblivious transfer.  $\qed$

\begin{lemma}
A system implementing Rabin's oblivious transfer can be used to implement 1-out-of-2 oblivious transfer.
\end{lemma}

\proofsketch
The sender has two bits $s_0, s_1 \in \{0,1\}$ and the receiver has a single bit $c$. The sender randomly samples $3n$ random bits $x_1, \dotsc, x_{3n} \in \{0,1\}$. Each bit is placed into its own a Rabin oblivious transfer box. The $i$th box then reveals either $x_i$ or else $\perp$ to the receiver. Let 
\[ S := \{i \in \{1, \dotsc, 3n\} : \text{the receiver knows } x_i\}. \]
The receiver picks two sets $I_0, I_1 \subseteq \{1, \dotsc, 3n\}$ such that $\# I_0 = \# I_1 = n$, $I_c \subseteq S$ and $I_{1-c} \subseteq \{1, \dotsc, 3n\} \setminus S$. This is possible except with probability negligible in $n$. He then sends the pair $(I_0, I_1)$ to the sender. The sender then computes $t_j= \left(\bigoplus_{i \in I_j}x_i \right) \oplus s_j$ for both $j \in \{0,1\}$ and sends $(t_0, t_1)$ to the receiver. \\

Notice that the receiver can uncover $s_c$ from $t_c$ since he knows $x_i$ for all $i \in I_c$, but cannot uncover $s_{1-c}$. One can show that the security requirement of Rabin's oblivious transfer implies that this system satisfies the security requirement necessary for 1-out-of-2 oblivious transfer. $\qed$ \\

We will see below that length-preserving one-way trapdoor permutations can be used to realize 1-out-of-2 oblivious transfer. 

\begin{theorem}
The following protocol realizes 1-out-of-2 oblivious transfer in the presence of computationally bounded and semi-honest adversaries. 
\begin{enumerate}
\item The sender, who has two bits $s_0$ and $s_1$, samples a random length-preserving one-way trapdoor permutation $(f, f^{-1})$ and sends $f$ to the receiver.  Let $b(\cdot)$ be a hard-core bit for $f$.
\item The receiver, who has a bit $c$, randomly samples an $n$-bit string $x_c \in \{0,1\}^n$ and computes $y_c = f(x_c)$. He then samples another random $n$-bit string $y_{1-c} \in \{0,1\}^n$, and then sends $(y_0, y_1)$ to the sender.
\item The sender computes $x_0 := f^{-1}(y_0)$ and $x_1 := f^{-1}(y_1)$. She computes $b_0 := b(x_0) \oplus s_0$ and $b_1 := b(x_1) \oplus s_1$, and then sends the pair $(b_0, b_1)$ to the receiver.
\item The receiver knows $c$ and $x_c$, and can therefore compute $s_c = b_c \oplus b(x_c)$. 
\end{enumerate}
\end{theorem}
\proof
Correctness is clear from the protocol.	
For security, from the sender side, since $f$ is a length-preserving permutation, $(y_0, y_1)$ is statistically indistinguishable from two random strings, hence she can't learn anything about $c$.
From the receiver side, guessing $s_{1-c}$ correctly is equivalent to guessing the hard-core bit for $y_{1-c}$.
\qed


\subsection{1-out-of-4 oblivious transfer}
  We describe how to implement a 1-out-of-4 OT using 1-out-of-2 OT:\@
  \begin{enumerate}
    \item
      The sender, $\PartyA$ samples 5 random values $S_i \gets \bits, i \in \set{1,\dotsc, 5}$.
    \item
      $\PartyA$ computes
      \begin{align*}
        \alpha_0 &= S_0 \xor S_2 \xor m_0\\
        \alpha_1 &= S_0 \xor S_3 \xor m_1\\
        \alpha_2 &= S_1 \xor S_4 \xor m_2\\
        \alpha_3 &= S_1 \xor S_5 \xor m_3
      \end{align*}
      It sends these values to $\PartyB$.
    \item
      The parties engage in 3 1-out-of-2 Oblivious Transfer protocols for the following
      messages: $(S_0, S_1)$, $(S_2, S_3)$, $(S_4, S_5)$. THe receiver's input for
      the first OT is the first choice bit, and for the second and third ones is
      the second choice bit.
    \item
      The receiver can only decrypt one ciphertext.
  \end{enumerate}


\section{Yao's Garbled Circuit}


%\input{HWsolution.tex}
%\part{Yao's Garbled Circuit}

% ===========
\section{Setup}


Yao's Garbled Circuits is presented as a solution to Yao's Millionaires' problem, 
which asks whether 
two millionaires can compete for bragging rights of which is richer
without revealing their wealth to each other. 
It started the area of secure computation. 
We will present a solution for the two party problem;
it can be extended to a polynomial number of parties,
using the techniques from last lecture.

The solution we saw previously needed an interaction for each AND gate.
Yao's solution requires only one message,
so it provides a constant size of interaction.
We present a solution only for semi honest security. 
This can be amplified to malicious security, 
but there are more efficient ways of amplifying this than what we saw last lecture.

\subsection{Secure Computation}

Recall our definition of secure computation. 
We define ideal and real worlds. 
Security is defined to hold if 
anything an attacker can achieve in the real world 
 can also be achieved by an ideal attacker in the ideal world. 
We define the ideal world to have the properties that we desire. 
For security to hold these properties must also hold in the real world.

\subsection{$(\Garble, \Eval)$}
We will provide a definition, similar to how we define encryption, that allows us avoid dealing with simulators all the time. 


Yao's Garbled Circuit is defined as two efficient algorithms $(\Garble, \Eval)$. Let the circuit $C$ have $n$ input wires.
$\Garble$ produces the garbled circuit and two labels for each input wire. The labels are for each of 0 and 1 on that wire and are like encryption keys. 

\[
(\tilde{C}, \{\ell_{i,b}\}_{i \in [n], b \in \{0,1\}}) \leftarrow \Garble(1^k, C) 
\]

To evaluate the circuit on a single input we must choose a value for each of the n input wires.
Given n of 2n input keys, $\Eval$ can evaluate the circuit on those keys and get the circuit result.
\[
C(x) \leftarrow \Eval(\tilde{C}, \{\ell_{i, x_i}\}_{i \in [n]}) 
\]

\paragraph{Correctness}
Correctness is as usual, if you garble honestly, evaluation should produce the correct result. 
\[
\forall C, x 
Pr[ C(x) = \Eval(\tilde{C}, \{l_{i, x_i}\}),  (\tilde{C}, \{\ell_{i,b}\}) = \Garble(1^k, C)] = 1 - neg(k)
\]


\paragraph{Security}
For security we require that a party receiving 
a garbled circuit and n inputs labels 
can not computationally distinguish the joint distribution of the circuits and labels
from the distribution produced by 
a simulator with access to the circuit and its evaluation on the input that the labels represent. 
The simulator does not have access to the actual inputs.
If this holds, the party receiving the garbled circuit and n labels can not determine the inputs.

\begin{align*}
&\exists \Sim : \forall C, x\\
&(\tilde{C}, \{\ell_{i,x_i}\}_{i \in [n]}) \simeq \Sim(1^k, C, C(x)) \text{ where} \\
&(\tilde{C}, \{\ell_{i,b}\}_{i \in [n], b \in \{0,1\}}) \leftarrow \Garble(1^k, C) 
\end{align*}

For simplicity we pass the circuit to the simulator.
You could also use universal circuits and pass 
in with the inputs the specific circuit that the universal circuit should realize. 



\section{Use for Semi-honest two party secure communication}
Alice, with input $x^1$, and Bob, with input $x^2$, have a circuit, C, that they want to evaluate securely. 
The size of their combined inputs is n, so $|x^1| = n_1, |x^2| = n - n_1, |x^1| + |x^2| = n$.
They can do this by Alice garbling a circuit and sending input wire labels to Bob, as in Figure \ref{fig:message}.

Alice garbles the circuit and passes it to Bob, $\tilde{C}$.
Alice passes the labels for her input directly to Bob, $\{\ell_{i, x^1_i}\}_{i \in [n] / [n_2]}$.
Alice passes all the labels for Bob's input wires into oblivious transfer, $\{\ell_{i, b_i}\}_{i \in [n] / [n_1], b \in \{0,1\}}$, 
from which Bob can retrieve the labels for his actual inputs, $\{\ell_{i, x^2_i}\}_{i \in [n] / [n_1]}$.
Bob now has the garbled circuit and one label for each input wire. 
He evaluates the garbled circuit on those garbled inputs and learns $C(x^1||x^2)$.
Bob does not learn anything besides the result as he has only the garbled circuit and n garbled inputs.
Alice does not learn anything as she uses oblivious transfer to give Bob his input labels and receives nothing in reply.

\begin{figure}[htbp]
\begin{center}
\setlength{\unitlength}{1cm}
\begin{picture}(10, 7)(-5, -4)
% \put(-.5,2){\makebox(1,1){C}}
 \put(-6,2){\makebox{Alice: $C, x^1$}}
 \put(-6,1.3){\makebox{$(\tilde{C}, \{\ell_{i,b}\}) \leftarrow \Garble$}}
 \put(4,2){\makebox{Bob: $C, x^2$}}

 \put(-1,0){\makebox(2,2){$\underrightarrow{\tilde{C}}$}}
  \put(-1,-0.8){\makebox(2,2){$\underrightarrow{ S_{out}^0 \text{ is 0 }, S_{out}^1 \text{ is 1 } }$}}


 \put(-1,-2){\makebox(2,2){$\underrightarrow{\{\ell_{i, x^1_i}\}_{i \in [n] / [n_2]}}$}}
% \put(-1,-1){\makebox(2,2){$\underrightarrow{\ell_{i,0}, \, \ell_{i,1} \forall i \in  [n]/[n_1] }$}}

 \put(-.5,-3){\framebox(1,1){OT}}
  \put(-1,-2.8){\line(1,0){.5}}
   \put(-1.6,-2.8){\makebox{$\ell_{i,1}$}}

  \put(-1,-2.2){\line(1,0){.5}}
     \put(-1.6,-2.2){\makebox{$\ell_{i,0}$}}

  \put(.5,-2.5){\line(1,0){.5}}
     \put(1.2,-2.5){\makebox{$\{\ell_{i, x^2_i}\}_{i \in [n] / [n_1]}$}}
     
  \put(-1,-4.5){\makebox(2,2){$\underrightarrow{ \forall i \in  [n]/[n_1] }$}}


\end{picture}
\caption{Messages in Yao's Garbeled Circuit}
\label{fig:message}
\end{center}
\end{figure}






%\paragraph{Malicious Bob}
%Alice semi-honest, and oblivious transfer is maliciously secure.
%Holds against malicious $Bob^*$
% What of deliberate circuit that shows first input  

\subsection{Construction of Garbled Circuits}

We would like to garble a circuit such that there are two keys for each input wire.
Correctness should be that 
given one of the two keys for each wire we can compute the output for the inputs those keys correspond to.
Security should be that 
given one key for each wire you can only learn the output, not the actual inputs.

%---

We build the circuit as a bunch of NAND gates that outputs one bit. 
If more bits are required, this can be done multiple times.
NAND gates can create any logic needed. 
We define the following sets:
\begin{align*}
W &= \text{the set of wires in the circuit}\\
G &= \text{the set of gates in the circuit.}
\end{align*}

For  each wire in the circuit, sample two keys
to label the possible inputs $0$ and $1$  to the wire
\[
\forall w \in W  \quad S_w^0, S_w^1 \,  \leftarrow{} \{0,1\}^k.
\]
We can think of these as the secret keys to an encryption scheme
(Gen, Enc, Dec).
For such a scheme we can always replace the secret key with the random bits fed into Gen.


\paragraph{Wires}
For each wire in the circuit we will have an invariant that the evaluator can only get one of the wires two encrypted values.
Consider an internal wire fed by the evaluation of a gate. The gate receives two encrypted values as inputs
and produces one encrypted output. The output will be one of the two labels for that wire and the evaluator will have no 
way of obtaining the other label for that wire. 
For example on wire $w_i$, the evaluator will only learn the value for $1$,  $S_{w_i}^1$.
We ensure this for the input wires by giving the evaluator only one of the two encrypted values for the wire.

\paragraph{Gates}
For every gate in the circuit we create four cipher texts. 
For each choice of inputs we encrypt the output key under each of the input keys. 
Let gate $g$ have inputs $w_1, w_2$ and output $w_3$,
\begin{align*}
e_g^{00} &= \Enc_{S_{w_1}^0} ( \Enc_{S_{w_2}^0}  ( S_{w_3}^1, 0^k) )\\
e_g^{01} &= \Enc_{S_{w_1}^0} ( \Enc_{S_{w_2}^1}  ( S_{w_3}^1, 0^k) )\\
e_g^{10} &= \Enc_{S_{w_1}^1} ( \Enc_{S_{w_2}^0}  ( S_{w_3}^1, 0^k) )\\
e_g^{11} &= \Enc_{S_{w_1}^1} ( \Enc_{S_{w_2}^1}  ( S_{w_3}^0, 0^k) ).
\end{align*}
We add $k$ zeros at the end.

\paragraph{Final Output}
For the final output wire, $S_{out}$, we can just give out their values,
\begin{align*}
S_{out}^0 &\text{ corresponds to 0}\\
S_{out}^1 &\text{ corresponds to 1.}
\end{align*}

\paragraph{$\bold{\tilde{C}}$}
For each gate, Alice sends Bob a random permutation of the set of four encrypted output values.
\[
\{e_g^{C_1, C_2} \} \quad \forall g \in G \quad C_1, C_2 \in \{0,1\}.
\]
For each gate, Alice sends Bob a random permutation of the set of four encrypted output values

\paragraph{Evaluation}
With an encrypted gate $g$,
input keys $S_{w_1} \, S_{w_2}$ for the input wires,
and four randomly permuted encryptions of the output keys, $e_g^{a}, e_g^{b}, e_g^{c}, e_g^{d}$,
Bob can evaluate the gate to find the corresponding key $S_{w3}$ for the output wire.
Bob can decrypt each of the encrypted output keys until he finds one that decrypts 
to a string ending in the proper number of $0$'s, which is very likely to contain the proper output key.
We can increase the probability of the correct key by increasing the number of $0$'s. 
\[
\exists  i \in \{a, b, c, d\} : \Dec_{S_{w_2}} ( \Dec_{S_{w_1}} ( e_g^{i} ))  = S_{w_3}, 0^k
\]

Given input wire labels 
$\{ \ell_{i, x_i} \}_{i \in [n]}$
the complete encrypted circuit $\tilde{C}$ is evaluated by working up from the input gates. 

%$l_{i,b} = \{S_{i,b}\}$

%as with PRF encryption scheme
%$Enc(_s(m) = (r,  m \oplus F_s(r)$


The evaluator should not be able to infer anything except what they could infer in the ideal world.
As a simple example, if the evaluator supplies one input to a circuit of just one NAND gate,
 they would be able to infer the input of the other party. However, this is true is the ideal world as well.

\section{Proof Intuition}

What intuition can we offer that the 
distribution of $\tilde{C}$ with one label per input wire 
is indistinguishable from what which a simulator could produce with access to the output?
%
For each input wire we are only given one key.
As we are doing double encryption,
for each input gate we only have the keys needed to decrypt one of the four possible outputs.
The other three are protected by semantic security.
%
So from each input gate we learn only one key compounding to its output wire.
As the output labels were randomized, we also do not know if that key corresponds to a 0 or a 1. 
%
For the next level of gates we again have only one key per input wire, and our argument continues. 
%
 So for each wire of the circuit we can only know one key corresponding to an output value for the wire. 
 Everything else is random garbage.
% 
As we control the mapping from output keys to output values, we can set this to whatever is needed to
match the expected output. 


Security only holds for evaluation of the circuit with one set of input values and 
we assume that the circuit is combinatorial and thus acyclic. 

% with two input all 0 or all 1 all broken
%  even with just 2 keys for one  input wire - broken. 




% !TEX root = collection.tex

\section{Malicious attacker intead of semi-honest attacker}

The assumption we had before consisted of a semi-honest attacker instead of a malicious attacker. A malicious attacker does not have to follow the protocol, and may instead alter the original protocol. The main idea here is that we can convert a protocol aimed at semi-honest attackers into one that will work with malicious attackers.

At the beginning of the protocol, we have each party commit to its inputs:
Given a commitment protocol $com$, Party 1 produces
\begin{center}
$c_1 = com(x_1; w_1)$ \\
$d_1 = com(r_1; \phi_1)$ \\
\end{center}
Party 2 produces
\begin{center}
$c_2 = com(x_2; w_2)$\\
$d_2 = com(r_2; \phi_2)$
\end{center}

We have the following guarantee: $\exists x_i, r_i, w_i, \phi_i$ such that $c_i = com(x_i; w_i) \wedge d_i = com(r_i; \phi_i) \wedge t = \pi(i,\text{transcript}, x_i, r_i)$, where transcript is the set of messages sent in the protocol so far.

Here we have a potential problem. Since both parties are choosing their own random coins, we have to be able to enforce that the coins are \emph{indeed} random. We can solve this by using the following protocol:

\begin{center}
  \begin{picture}(200,100)(10,20)
    \put(20, 90){$d_1 = com(s_1; \phi_1)$}
    \put(20,80){\vector(1,0){50}}
    \put(150, 90){$d_2 = com(s_2; \phi_2)$}
    \put(200, 80){\vector(-1,0){50}}

    \put(20, 60){$s_2^{'}$}
    \put(20,50){\vector(1,0){50}}
    \put(200, 60){$s_1^{'}$}
    \put(200, 50){\vector(-1,0){50}}
  \end{picture}
\end{center}

We calculate $r_1 = s_1 \oplus s_1^{'}$, and $r_2 = s_2 \oplus s_2^{'}$. As long as one party is picking the random coins honestly, both parties would have truly random coins.

Furthermore, during the first commitment phase, we want to make sure that the committing party actually knows the value that is being committed to. Thus, we also attach along with the commitment a zero-knowledge proof of knowledge (ZK-PoK) to prove that the committing party knows the value that is being committed to.

\subsection{Zero-knowledge proof of knowledge (ZK-PoK)}

\begin{definition}[ZK-PoK] Zero-knowlwedge proof of knowledge (ZK-PoK) is a zero-knowledge proof system $(P,V)$ with the property proof of knowledge with knowledge error $\kappa$:

$\exists$ a PPT $E$ (knowledge extractor) such that $\forall x \in L$ and $\forall P^{*}$ (possibly unbounded), it holds that if $\Pr[Out_V(P^{*}(x,w) \leftrightarrow V(x))]> \kappa(x)$, then 
\[ \Pr[E^{P^*}(x) \in R(x)] \geq \Pr[Out_V(P^{*} \leftrightarrow V(x))] = 1]- \kappa(x).\]
Here we have $L$ be the language, $R$ be the relation, and $R(x)$ is the set such that $\forall w \in R(x)$, $(x, w) \in R$.
\end{definition}

Given a zero-knowledge proof system, we can construct a ZK-PoK system for statement $x\in L$ with witness $w$ as follows:
\begin{center}
  \begin{picture}(300,300)(10,20)

    \put(10, 290){$P$}
    \put(290, 290){$V$}

    \put(10, 270){$r \leftarrow \{0, 1\}^{|w|}$}

    \put(100, 260){$c_1 = com(r; \omega)$}
    \put(100, 250){$c_2 = com(r \oplus w; \phi)$}
    \put(100, 240){\vector(1,0){100}}

    \put(150, 210){$b$}
    \put(200, 200){\vector(-1,0){100}}

    \put(120, 160){if $b = 0$, open $c_1$ to reveal $r$}
    \put(120, 150){else open $c_2$ to reveal $r \oplus w$}
    \put(100, 140){\vector(1,0){100}}

    \put(120, 60){\framebox(50,50)[c]{ZK Proof}}
  \end{picture}
\end{center}

The last ZK proof proves that $\exists r, w, \omega, \phi$ such that $(x, w) \in R$ and $c_1 = com(r; \omega)$, $c_2 = com(r \oplus w; \phi)$.


\section*{Exercises}
\begin{exercise}
Given a (secure against malicious adversaries) two-party secure computation protocol (and nothing else) construct a (secure against malicious adversaries) three-party secure computation protocol.
\end{exercise}

%% !TEX root = collection.tex

\chapter{Obfustopia}


\section{Witness Encryption: A Story}\label{story}

Imagine that a billionaire who loves mathematics, would like to award with 1 million dollars the mathematician(s) who will prove the Riemann Hypothesis. Of course, neither does the billionaire know if the Riemann Hypothesis is true, nor if he will be still alive (if and) when a mathematician will come up with a proof. To overcome these couple of problems, the billionaire decides to:

\begin{enumerate}

\item Put 1 million dollars in gold in a big treasure chest.

\item Choose an arbitrary place of the world, dig up a hole, and hide the treasure chest.

\item Encrypt the coordinates of the treasure chest in a message so that only the mathematician(s) who can actually prove the Riemann Hypothesis can decrypt it.

\item Publish the ciphertext in every newspaper in the world.

\end{enumerate}

The goal of this lecture is to help the billionaire with step 3. To do so, we will assume for simplicity  that the proof is at most 10000 pages long. The latter assumption implies that the language
\begin{align*}
L = \{ x \text{ such that } x \text{ is an acceptable Riemann Hypothesis proof} \}
\end{align*}
 is in NP and therefore, using a reduction, we can come up with a circuit $C$ that takes as input $x$ and outputs $1$ if $x$ is a proof for the Riemann Hypothesis and $0$ otherwise.

\smallskip
Our goal now is to  design a pair of PPT machines $(\mathrm{Enc},\mathrm{Dec})$ such that:

\begin{enumerate}
\item $\mathrm{Enc}(C,m)$ takes as input the circuit $C$ and $m \in \{0,1\}$ and outputs a ciphertext $e \in \{0,1\}^{*}$.

\item $\mathrm{Dec}(C,e,w)$ takes as input the circuit $C$, the cipertext $e$ and a witness $w \in \{0,1\}^{*}$ and outputs $m$ if if $C(w) = 1$ or $\perp$ otherwise.
\end{enumerate}

and so that they satisfy the following correctness and security requirements:

\begin{itemize}

\item \textbf{Correctness:} If $\exists w$ such that $C(w) = 1$ then $\mathrm{Dec}(C,e,w)$ outputs $m$.

\item \textbf{Security:} If $\nexists w$ such that $C(w) = 1$ then $\mathrm{Enc}(C,0)   \approx^{c} \mathrm{Enc}(C,1) \!\ $ (where $ \approx^{c}$ means  ``computationally indistinguishable'').

\end{itemize}


\section{A Simple Language }

As a first example, we show how we can design such an encryption scheme for a simple language. Let $G$ be a group of prime order and  $g$ be a generator of the group. For elements $A, B, T \in G$ consider the language $L = \{(a,b): A = g^a, B = g^b, T = g^{ab} \}$. An encryption scheme for that language with the correctness and security requirements of Section~\ref{story} is the following:

\smallskip

\begin{itemize}

\item \textbf{Encryption$(g,A,B,T,G)$:}

\begin{itemize}
\item Choose elements $r_1, r_2 \in \mathbb{Z}_p^*$ uniformly and independently.

\item Let $c_1 = A^{r_1} g^{r_2} $, $c_2 =  g^m T^{r_1} B^{r_2}$, where $m \in \{0,1\}$ is the message we want to encrypt.

\item Output $c = (c_1, c_2)$

\end{itemize}

\item \textbf{Decryption($b$):}


\begin{itemize}

\item Output $\frac{c_2}{c_1^b}$

\end{itemize}

\end{itemize}


\textbf{Correctness:}
The correcntess of the above encryption scheme follows from the fact that if there exist $(a,b) \in L$ then:

\begin{eqnarray*}
\frac{c_2}{c_1^b} & = &  \frac{g^m T^{r_1} B^{r_2}  }{ \left( A^{r_1}g^{r_2}\right)^b } \\
& = & \frac{g^m \left(g^{ab}\right)^{r_1} \left( g^{b} \right)^{r_2}  }{ \left( g^{a} \right)^{r_1 b} g^{r_2 b} } \\
& = & g^{m}
\end{eqnarray*}

Since $m \in \{0,1\}$ and we know $g$, the value of $g^m$ implies the value of $m$.

\smallskip
\textbf{Security:}
As far as the security of the scheme is concerned, since $L$ is quite simple, we can actually prove that $m$ is information-theoretically hidden. To see this, assume there does not exist $(a,b) \in L$, but an adversary has the power to compute discrete logarithms. In that case, given $c_1$ and $c_2$ the adversary could get a system of the form:
\begin{eqnarray*}
ar_1 + r_2 & = & s_1 \\
m + r r_1 + b r_2 &=& s_2
\end{eqnarray*}
where $s_1$ and $s_2$ are   the discrete logarithms of $c_1$ and $c_2$ respectively (with base $g$), and $r \ne ab$ is an element of $  \mathbb{Z}_{p}^*$ such that $T = g^r$. Observe now that for each value of $m$ there exist numbers $r_1$ and $r_2$ so that the above system has a solution, and thus $m$ is indeed information-theoretically hidden (on the other hand, if we had that $ab = r$ then the equations are linearly dependent).

\newpage
\section{An  NP Complete Language }

In this section we focus on our original goal of designing an encryption for an NP complete language $L$. Specifically, we will consider the NP-complete problem \emph{exact cover}. Besides that, we introduce the $n$-Multilinear  Decisional Diffie-Hellman ($n$-MDDH) assumption  and the Decisional Multilinear No-Exact-Cover Assumption.  %(see also~\cite{Sanjam}). 
The latter will guarantee the security of our construction.

\subsection{ Exact Cover}

We are given as input $x = (n, S_1, S_2, \ldots, S_l)$, where $n$ is an integer and each $S_i, i \in [l]$ is a subset of $[n]$, and our goal is to find a subset of indices $T \subseteq [l]$ such that:

\begin{enumerate}
\item $\cup_{i \in T} S_i = [n] $ and

\item $\forall i, j \in T$ such that $i \ne j$ we have that $S_i \cap S_j = \emptyset$.
\end{enumerate}

If such a $T$ exists, we say that $T$ is an exact cover of $x$.

\subsection{Multilinear Maps}

Mutlinear maps is a generalization of bilinear maps (which we have already seen) that will be useful in our construction. Specifically, we assume the existence of a group generator $\mathcal{G}$, which takes as input a security parameter $\lambda$ and a positive integer $n$ to indicate the number of allowed operations. $\mathcal{G}(1^{\lambda},n)$ outputs a sequence of groups $\vec{\mathbb{G}}= (\mathbb{G}_1, \mathbb{G}_2, \ldots, \mathbb{G}_n)$  each of large prime order $P > 2^{\lambda}$. In addition, we let $g_i$ be a canonical generator of $\mathbb{G}_i$  (and is known from the group's description).

We also assume the existence of a set of bilinear maps $\{e_{i,j}: \mathbb{G}_i \times \mathbb{G}_j \rightarrow \mathbb{G}_{i+j} \mid i, j \ge 1; i+j \le n \}.$ The map $e_{i,j}$ satisfies the following relation:
\begin{align}
e_{i,j}\left(g_i^{a},g_j^{b}\right) = g^{ab}_{i+j}: \forall a,b \in \mathbb{Z}_p \label{vasikoni}
\end{align}
and we observe that one consequence of this is that $e_{i,j} (g_i, g_j) = g_{i+j}$ for each valid $i,j$.

\subsection{The $n$-MDDH Assumption  }

The $n$-Multilinear Decisional Diffie-Hellman ($n$-MDDH) problem states the following: A challenger runs $\mathcal{G}(1^{\lambda},n ) $ to generate groups and generators of order $p$. Then it picks random $s, c_1, \ldots, c_n  \in \mathbb{Z}_p$.  The assumption then states that given $g= g_1, g^{s}, g^{c_1}, \ldots,g^{c_n}$ it is hard to distinguish $T = g_n^{s \prod_{j \in [1,n ] } c_j}$ from a random group element in $G_n$, with better than negligible advantage (in security parameter $\lambda$).

\newpage


\subsection{Decisional Multilinear  No-Exact-Cover  Assumption}
Let $x = (n, S_1, \ldots, S_l)$ be an instance of the exact cover problem that has no solution. Let $\mathrm{param} \leftarrow \mathcal{G}(1^{1+n},n)$ be a description of a multilinear group family with order $p = p(\lambda)$. Let $a_1, a_2, \ldots, a_n,r$ be uniformly random in $\mathbb{Z}_p$. For $i \in [l]$, let $c_i  = g_{|S_i|}^{ \prod_{j \in S_i} a_j}$. Distiguish between the two distributions:
\begin{align*}
(\mathrm{params}, c_1, \ldots,c_l,g_n^{a_1a_2\ldots a_n}) \text{ and } (\mathrm{params},c_1, \ldots,c_l,g_n^r)
\end{align*}

The Decisional Multilinear No-Exact-Cover Assumption is that for all adversaries $\mathcal{A}$, there exists a fixed negligible function $\nu(\cdot)$ such that for all instances $x$ with no solution, $\mathcal{A}$'s distinguishing advantage against the Decisional Multilinear No-Exact-Cover Problem  for $x$ is at most $\nu(\lambda)$.

\subsection{The Encryption Scheme  }

We are now ready to give the description of our encryption scheme.

\begin{itemize}

\item $\mathrm{Enc}(x,m)$ takes as input $x=(n, S_1, \ldots,S_l)$ and the message $m \in \{0,1\}$ and:

\begin{itemize}

\item Samples $a_{0}, a_1, \ldots, a_{n}$ uniformly and independently from $\mathbb{Z}_p^*$.

\item $\forall i \in [l]$ let $c_i = g^{\prod_{j\in S_j} a_j}_{|S_i|}$

\item Sample uniformly an element $r \in \mathbb{Z}_p^*$

\item Let $d = d(m) $ be $  g_n^{\prod_{j \in [n]}a_j}$ if $m = 1 $ or $g_n^r$ if $m = 0$.

\item Output $c = (d, c_1, \ldots,c_l)$
\end{itemize}

\item $\mathrm{Dec}(x,T)$, where $T \subseteq[l]$ is a set of indices, computes $\prod_{i \in T}c_i$ and outputs $1$ if the latter value equals to $d$ or $0$ otherwise.

\end{itemize}


\begin{itemize}

\item \textbf{Correctness:} Assume that $T$ is an exact cover of $x$. Then, it is not hard to see that:
\begin{eqnarray*}
\prod_{i \in T} c_i & = & \prod_{i \in T} g^{\prod_{j\in S_j} a_j}_{|S_i|} \\
& = & g_n^{\prod_{j \in [n]}a_j}
\end{eqnarray*}
where we have used~\eqref{vasikoni} repeatedly and the fact that $T$ is an exact cover (to show that $\sum_{i \in T} |S_i| = n$ and that $\prod_{i \in T} \prod_{j \in S_i} a_j = \prod_{i \in [n]} a_i$).

\item \textbf{Security:} Intuitively, the construction is secure, since the only way to make $g_n^{\prod_{ j \in [n] }a_i}$ is to find an exact cover of $[n]$.  As a matter of fact, observe that if an exact cover does not exist, then for each subset of indices $T'$( such that $\cup_{i \in T'}S_j  = [n]$) we have that
\begin{align*}
\sum_{i =1 }^{n} |S_i| > n,
\end{align*}
which means that   $\prod_{i \in T} \prod_{j \in S_i} a_j$ is different than $\prod_{j \in [n]}a_j$. Formally, the security is based on the Decisional Multilinear No-Exact-Cover Assumption.

\end{itemize}


%\bibliographystyle{plain}
%\bibliography{smoser}


% !TEX root = collection.tex

%\newcommand{\norm}[1]{\left\Vert#1\right\Vert}
\newcommand{\ABS}[1]{\left\vert#1\right\vert}
\newcommand{\SET}[1]{\left\{#1\right\}}  
\newcommand{\INP}[1]{\left(#1\right)}
\newcommand{\POLY}[1]{\ensuremath{\mathop{\mathrm{poly}}\INP{#1}}}
%\newcommand{\iO}[1]{\ensuremath{\mathop{i\mathcal{O}}\INP{#1}}}
\newcommand{\ENC}[1]{\ensuremath{\mathop{\mathrm{Enc}}\INP{#1}}}
\newcommand{\DEC}[1]{\ensuremath{\mathop{\mathrm{Dec}}\INP{#1}}}
%\bibliographystyle{plain}

\section{Obfuscation}
The problem of program obfuscation asks whether one can transform a program (e.g., circuits, Turing machines) to another semantically equivalent program (i.e., having the same input/output behavior), but is otherwise intelligible.
It was originally formalized by Barak et al. who constructed a family of circuits that are non-obfuscatable under the most natural virtual black box (VBB) security.
\section{VBB Obfuscation}
As a motivation, recall that in a private-key encryption setting, we have a secret key $k$, encryption $E_k$ and decryption $D_k$.
A natural candidate for public-key encryption would be to simply release an encryption $E'_k \equiv E_k$ (i.e. $E'_k$ semantically equivalent to $E_k$, but computationally bounded adversaries would have a hard time figuring out $k$ from $E'_k$.

\begin{definition}[Obfuscator of circuits under VBB]
	$O$ is an \emph{obfuscator} of circuits if %for every circuit $c$ we have,
	\begin{enumerate}
		\item
			Correctness:
	$\forall c, O(c) \equiv c$.
	\item
		Efficiency:
		$\forall c, \ABS{O(c)} \le \POLY{\ABS{c}}$.
	\item
		VBB:
		$\forall A, A$ is PPT bounded, $\exists$ S (also PPT) s.t. $\forall c$,
		\[
			\ABS{\Pr\left[ A\left( O(c) \right) = 1\right] - \Pr\left[ S^c(1^{\ABS{c}}) = 1 \right]} \le \mathrm{negl}(\ABS{c}).
		\]
	\end{enumerate}
\end{definition}

Similarly we can define it for Turing machines.
\begin{definition}[Obfuscator of TMs under VBB]
	$O$ is an \emph{obfuscator} of Turing machines if %for every circuit $c$ we have,
	\begin{enumerate}
		\item
			Correctness:
	$\forall M, O(M) \equiv M$.
	\item
		Efficiency:
		$\exists q(\cdot) = \POLY{\cdot}, \forall M \left( M(x) \hbox{ halts in }t \hbox{ steps} \implies O(M)(x) \hbox{ halts in }q(t) \hbox{ steps}\right)$.
	\item
		VBB:
		Let $M'(t,x)$ be a TM that runs $M(x)$ for $t$ steps.
		$\forall A, A$ is PPT bounded, $\exists$ Sim (also PPT) s.t. $\forall c$,
		\[
			\ABS{\Pr\left[ A\left( O(M) \right) = 1\right] - \Pr\left[ S^{M'}(1^{\ABS{M'}}) = 1 \right]} \le \mathrm{negl}(\ABS{M'}).
		\]
	\end{enumerate}
\end{definition}

Let's show that our candidate PKE from VBB obfuscator $O$ is semantic secure, using a simple hybrid argument.

\proof
Recall the public key $PK=O(E_k)$.
Let's assume $E_k$ is a circuit.
\begin{align*}
	H_0 :& A(\SET{(PK, E_k(m_0))}) & \\
	H_1 :& S^c(\SET{E_k(m_0)}) & \hbox{ by VBB} \\
	H_2 :& S^c(\SET{E_k(m_1)}) & \hbox{ by semantic security of private key encryption} \\
	H_3 :& A(\SET{(PK, E_k(m_1))}) & \hbox{ by VBB}
\end{align*}
\qed

Unfortunately VBB obfuscator for all circuits does not exist. Now we show the impossiblity result of VBB obfuscator.
\begin{theorem}
	Let $O$ be an obfuscator.
	There exists PPT bounded $A$, and a family (ensemble) of functions $\SET{H_n}$, $\SET{Z_n}$ s.t.
	for every PPT bounded simulator $S$,
\begin{gather*}
	A\left( O(H_n) \right) = 1 \ \ \& \ \ A\left( O(Z_n) \right) = 0\\
	\ABS{\Pr\left[ S^{H_n} \left( 1^{\ABS{H_n}} \right) = 1 \right] - \Pr \left[ S^{Z_n} \left(1^{\ABS{Z_n}}\right) =1 \right]} \le\mathrm{negl}(n).
\end{gather*}
\end{theorem}

\proof
Let $\alpha, \beta \overset{\$}{\leftarrow} \SET{0,1}^n$.

We start by constructing $A',C_{\alpha,\beta}, D_{\alpha,\beta}$ s.t.
\begin{gather*}
	A'\left( O(C_{\alpha,\beta}), O(D_{\alpha,\beta}) \right) = 1 \ \ \& \ \ A'\left( O(Z_n), O(D_{\alpha,\beta}) \right) = 0\\
	\ABS{\Pr\left[ S^{C_{\alpha,\beta},D_{\alpha,\beta}} \left( \mathbf{1} \right) = 1 \right] - \Pr \left[ S^{Z_n,D_{\alpha,\beta}} \left(\mathbf{1}\right) =1 \right]} \le\mathrm{negl}(n).
\end{gather*}

\begin{gather*}
C_{\alpha,\beta}(x) =
\begin{cases}
	\beta, & \hbox{if } x = \alpha,\\
	0^n, & \hbox{o/w}
\end{cases} \\
D_{\alpha,\beta}(c)=
\begin{cases}
	1,& \hbox{if } c(\alpha) = \beta,\\
	0, & \hbox{o/w}.
\end{cases}
\end{gather*}

Clearly $A'(X,Y) = Y(X)$ works.
Now notice that input length to $D$ grows as the size of $O(C)$.

However for Turing machines which can have the same description length, one could combine the two in the following way:

$F_{\alpha,\beta}(b, x) =
\begin{cases}
	C_{\alpha,\beta}(x), & b=0\\
	D_{\alpha,\beta}(x), & b=1\\
\end{cases}.$

Let $OF= O(F_{\alpha,\beta})$, $OF_0(x) = OF(0,x)$, similarly for $OF_1$, then $A$ would be just $A(OF) = OF_1(OF_0)$.

Now assuming OWF exists, specifically we already have priavte-key encryption, we modify $D$ as follows.
\begin{gather*}
	D_k^{\alpha,\beta}(1,i) = \mathrm{Enc}_k(\alpha_i) \\
	D_k^{\alpha,\beta}(2,c,d,\odot) = \mathrm{Enc}_k(\mathrm{Dec}_k(c) \odot \mathrm{Dec}_k(d)), \hbox{where $\odot$ is a gate of AND, OR, NOT} \\
	D_k^{\alpha,\beta}(3, \gamma_1,\cdots,\gamma_n) =
	\begin{cases}
		1,& \forall i, \mathrm{Dec}_k(\gamma_i) = \beta_i,\\
		0, & \hbox{o/w}.
	\end{cases}
\end{gather*}

Now the adversary $A$ just simulate $O(C)$ gate by gate with a much smaller $O(D)$, thus we can use the combining tricks as for the Turing machines.
\qed

\section{Indistinguishability Obfuscation}

%\begin{definition}[Indistinguishability Obfuscation]
%	$\iO{\cdot}$ is an \emph{indistinguishability obfuscation} if $\forall c_1, c_2$ such that $\ABS{c_1}= \ABS{c_2}$ and $c_1\equiv c_2$, we have
%	\[
%		\iO{c_1} \overset{c}{\approx} \iO{c_2}.
%	\]
%\end{definition}


%\newcommand{\iO}{\ensuremath{i\mathcal{O}}}
\newcommand{\Ck}{\ensuremath{\mathcal{C}_\kappa}}

\begin{definition}[Indistinguishability Obfuscator]
A uniform PPT machine $\iO$ is an \emph{indistinguishability obfuscator}
for a collection of circuits $\Ck$ if the following conditions hold:
\begin{itemize}

\item \emph{Correctness.}
For every circuit $C \in \Ck$ and for all inputs $x$,
$C(x) = \iO(C(x))$.

\item \emph{Polynomial slowdown.}
For every circuit $C \in \Ck$, $|\iO(C)| \leq p(|C|)$ for some
polynomial $p$.

\item \emph{Indistinguishability.}
For all pairs of circuits $C_1, C_2 \in \Ck$, if $|C_1| = |C_2|$ and
$C_1(x) = C_2(x)$ for all inputs $x$, then
$\iO(C_1) \overset{c}{\simeq} \iO(C_2)$.
More precisely, there is a negligible function $\nu(k)$ such that for
any (possibly non-uniform) PPT $A$,
\begin{equation*}
\big| \Pr[A(\iO(C_1)) = 1] - \Pr[A(\iO(C_2)) = 1] \big| \leq \nu(k).
\end{equation*}

\end{itemize}
\end{definition}


\begin{lemma}
	Indistinguishability obfuscation implies witness encryption.
\end{lemma}
\proof
Recall the witness encryption scheme, with which one could encrypt a message $m$ to an instance $x$ of an NP language $L$, such that $\DEC{x,w,\ENC{x,m}}=
\begin{cases}
	m, \hbox{if} (x,w)\in L, \\
	\bot, \hbox{o/w}
\end{cases}$

Let $C_{x,m}(w)$ be a circuit that on input $w$, outputs $m$ if and only if $(x,w) \in L$.
Now we construct witness encryption as follows:
$\ENC{x,m}=\iO{C_{x,m}}, \DEC{x,w,c}=c(w)$.

Semantic security follows from the fact that, for $x\not\in L$, $C_{x,m}$ is just a circuit that always output $\bot$, and by indistinguishability obfuscation, we could replace it with a constant circuit (padding if necessary), and then change the message, and change the circuit back.
\qed


\begin{lemma}
	Indistinguishability obfuscation and OWFs imply public key encryption.
\end{lemma}
\proof
We'll use a length doubling PRG $F: \SET{0,1}^n \to \SET{0,1}^{2n}$, together with a witness encryption scheme $(E,D)$.
The NP language for the encryption scheme would be the image of $F$.
\begin{align*}
	&\mathrm{Gen}(1^n) = (PK = F(s), SK=s), s\overset{\$}{\leftarrow} \SET{0,1}^n\\
	&\ENC{PK,m} = E(x=PK,m)\\
	&\DEC{e,SK=s} = D(x=PK,w=s,c=e).
\end{align*}
\qed

\begin{lemma}
	Every best possible obfuscator could be equivalently achieved with an indistinguishability obfuscation (up to padding and computationally bounded).
\end{lemma}

\proof
Consider circuit $c$, the \emph{best possible obfuscated} $BPO(c)$, and $c'$ which is just padding $c$ to the same size of $BPO(c)$.
Computationally bounded adversaries cannot distinguish between $\iO{c'}$ and $\iO{BPO(c)}$.

Note that doing iO never decreases the ``entropy'' of a circuit, so $\iO{BPO(c)}$ is at least as secure as $BPO(c)$.
\qed



% !TEX root = collection.tex

%\section{Using Indistinguishability Obfuscation}



%\newcommand{\cO}{\ensuremath{\mathcal{O}}}
%\newcommand{\NC}[1]{\ensuremath{\mathbf{NC}^{#1}}}
\newcommand{\Ck}{\mathcal{C}_{\kappa}}


\chapter{Obfuscation}
\section{$\iO$ for Polynomial-sized Circuits}


\begin{definition}[Indistinguishability Obfuscator for $NC^1$]
Let $\Ck$ be the collection of circuits of size $O(\kappa)$ and depth
$O(\log{\kappa})$ with respect to gates of bounded fan-in.
Then a uniform PPT machine $\iO_{\NC{1}}$ is an
\emph{indistinguishability obfuscator} for circuit class $\NC{1}$ if it
is an indistinguishability obfuscator for $\Ck$.
\end{definition}

Given an indistinguishability obfuscator $\iO_{\NC{1}}$ for circuit
class $\NC{1}$, we shall demonstrate how to achieve an
indistinguishability obfuscator $\iO$ for all polynomial-sized circuits.
The amplification relies on fully homomorphic encryption (FHE).

\newcommand{\GEN}{\ensuremath{\mathsf{Gen}}}
%\newcommand{\Enc}{\ensuremath{\mathsf{Enc}}}
%\newcommand{\Dec}{\ensuremath{\mathsf{Dec}}}
%\newcommand{\Eval}{\ensuremath{\mathsf{Eval}}}
%\newcommand{\pk}{\ensuremath{\mathsf{pk}}}
%\newcommand{\sk}{\ensuremath{\mathsf{sk}}}

\begin{definition}[Homomorphic Encryption]
A \emph{homomorphic encryption scheme} is a tuple of PPT algorithms
$(\GEN, \Enc, \Dec, \Eval)$ as follows:
\begin{itemize}
\item
	$(\GEN, \Enc, \Dec)$ is a semantically-secure public-key
	encryption scheme.
\item
	$\Eval(\pk, C, e)$ takes public key $\pk$, an arithmetic circuit
	$C$, and ciphertext $e = \Enc(\pk, x)$ of some circuit input
	$x$, and outputs $\Enc(\pk, C(x))$.
\end{itemize}
\end{definition}

As an example, the ElGamal encryption scheme is homomorphic over the multiplication function.
Consider a cyclic group $G$ of order $q$ and generator $g$, and let
$\sk = a$ and $\pk = g^a$.
For ciphertexts $\Enc(\pk, m_1) = (g^{r_1}, g^{a r_1} \cdot m_1)$
and $\Enc(\pk, m_2) = (g^{r_2}, g^{a r_2} \cdot m_2)$, observe that
\begin{equation*}
\Enc(\pk, m_1) \cdot \Enc(\pk, m_2) = (g^{r_1 + r_2}, g^{a (r_1 + r_2)}
\cdot m_1 \cdot m_2) = \Enc(\pk, m_1 \cdot m_2)
\end{equation*}
Note that this scheme becomes additively homomorphic by encrypting $g^m$
instead of $m$.

\begin{definition}[Fully Homomorphic Encryption]
An encryption scheme is \emph{fully homomorphic} if it is both compact
and homomorphic for the class of all arithmetic circuits.
Compactness requires that the size of the output of $\Eval(\cdot, \cdot,
\cdot)$ is at most polynomial in the security parameter $\kappa$.
\end{definition}

\subsection{Construction}

%We first present a simpler construction under the virtual black box
%model, assuming the existence of a circuit obfuscator $\cO_{\NC{1}}$ for
%$\NC{1}$.

\newcommand{\prog}[1]{\ensuremath{P_{\pk_1,\pk_2,\sk_{#1},e_1,e_2}}}

Let $(\GEN, \Enc, \Dec, \Eval)$ be a fully homomorphic encryption
scheme.
We require that $\Dec$ be realizable by a circuit in $\NC{1}$.
The obfuscation procedure accepts a security parameter $\kappa$ and
a circuit $C$ whose size is at most polynomial in $\kappa$.
\begin{enumerate}
\item
	Generate $(\pk_1, \sk_1) \gets \GEN(1^\kappa)$ and
	$(\pk_2, \sk_2) \gets \GEN(1^\kappa)$.
\item
	Encrypt $C$, encoded in canonical form, as
	$e_1 \gets \Enc(\pk_1, C)$ and $e_2 \gets \Enc(\pk_2, C)$.
\item
	Output an obfuscation
	$\sigma = (\iO_{\NC{1}}(P), \pk_1, \pk_2, e_1, e_2)$
	of program $\prog{1}$ as described below.
\end{enumerate}

The evaluation procedure accepts the obfuscation $\sigma$ and program
input $x$.
\begin{enumerate}
\item
	Let $U$ be a universal circuit that computes $C(x)$ given a
	circuit description $C$ and input $x$, and denote by $U_x$ the
	circuit $U(\cdot, x)$ where $x$ is hard-wired.
	Let $R_1$ and $R_2$ be the circuits which compute
	$f_1 \gets \Eval(U_x, e_1)$ and $f_2 \gets \Eval(U_x, e_2)$,
	respectively.

\item
	Denote by $\omega_1$ and $\omega_2$ the set of all wires in $R_1$
	and $R_2$, respectively.
	Compute $\pi_1 : \omega_1 \to \{ 0, 1 \}$ and
	$\pi_2 : \omega_2 \to \{ 0, 1 \}$, which yield the value of internal
	wire $w \in \omega_1, \omega_2$ when applying $x$ as the input
	to $R_1$ and $R_2$.

\item
	Output the result of running $\prog{1}(x, f_1, \pi_1, f_2, \pi_2)$.
\end{enumerate}

Program $\prog{1}$ has $\pk_1$, $\pk_2$, $\sk_1$, $e_1$, and $e_2$
embedded.
\begin{enumerate}
\item
	Check whether $R_1(x) = f_1 \land R_2(x) = f_2$.
	$\pi_1$ and $\pi_2$ enable this check in logarithmic depth.
\item
	If the check succeeds, output $\Dec(\sk_1, f_1)$;
	otherwise output $\bot$.
\end{enumerate}

The use of two key pairs and two encryptions of $C$, similar to
CCA1-secure schemes seen previously, eliminates the virtual black-box
requirement for concealing $\sk_1$ within $\iO_{\NC{1}}(\prog{1})$.

\subsection{Proof of Security}

We prove the indistinguishability property for this construction
through a hybrid argument.

\newcommand{\Hyb}[1]{\ensuremath{\mathsf{H_{#1}}}}
\begin{proof}
Through the sequence of hybrids, we gradually transform an obfuscation
of circuit $C_1$ into an obfuscation of circuit $C_2$, with each
successor being indistinguishable from its antecedent.
\begin{description}
\item[$\Hyb{0}$]:
	This corresponds to an honest execution of $\iO(C_1)$.
	Recall that $e_1 = \Enc(\pk_1, C_1)$, $e_2 = \Enc(\pk_2, C_1)$,
	and $\sigma = (\iO_{\NC{1}}(\prog{1}), \ldots)$.

\item[$\Hyb{1}$]:
	We instead generate $e_2 = \Enc(\pk_2, C_2)$, relying on the
	semantic security of the underlying fully homomorphic encryption
	scheme.

\item[$\Hyb{2}$]:
	We alter program $\prog{2}$ such that it instead embeds $\sk_2$
	and outputs $\Dec(\sk_2, f_2)$.
	The output of the obfuscation procedure becomes
	$\sigma = (\iO_{\NC{1}}(\prog{2}, \ldots)$;
	we rely on the properties of functional equivalence and
	indistinguishability of $\iO_{\NC{1}}$.

\item[$\Hyb{3}$]:
	We generate $e_1 = \Enc(\pk_1, C_1)$ since $\sk_1$ is now
	unused, relying again on the semantic security of the fully
	homomorphic encryption scheme.

\item[$\Hyb{4}$]:
	We revert to the original program $\prog{1}$ and arrive
	at an honest execution of $\iO(C_1)$.

\end{description}
\end{proof}


\section{Identity-Based Encryption}

Another use of indistinguishability obfuscation is to realize
identity-based encryption (IBE).

\newcommand{\SETUP}{\ensuremath{\mathsf{Setup}}}
\newcommand{\KEYGEN}{\ensuremath{\mathsf{KeyGen}}}
\newcommand{\mpk}{\ensuremath{\mathsf{mpk}}}
\newcommand{\msk}{\ensuremath{\mathsf{msk}}}
\newcommand{\id}{\ensuremath{\mathsf{id}}}

\begin{definition}[Identity-Based Encryption]
An \emph{identity-based encryption scheme} is a tuple of PPT algorithms
$(\SETUP, \KEYGEN, \Enc, \Dec)$ as follows:
\begin{itemize}
\item
	$\SETUP(1^\kappa)$ generates and outputs a master public/private
	key pair $(\mpk, \msk)$.
\item
	$\KEYGEN(\msk, \id)$ derives and outputs a secret key
	$\sk_{\id}$ for identity $\id$.
\item
	$\Enc(\mpk, \id, m)$ encrypts message $m$ under identity $\id$
	and outputs the ciphertext.
\item
	$\Dec(\sk_{\id}, c)$ decrypts ciphertext $c$ and outputs the
	corresponding message if $c$ is a valid encryption under
	identity $\id$, or $\bot$ otherwise.
\end{itemize}
\end{definition}

\newcommand{\SIGN}{\ensuremath{\mathsf{Sign}}}
\newcommand{\VERIFY}{\ensuremath{\mathsf{Verify}}}

We combine an indistinguishability obfuscator $\iO$ with a digital
signature scheme $(\GEN, \SIGN, \VERIFY)$.
\begin{itemize}
\item
	Let $\SETUP \equiv \GEN$ and $\KEYGEN \equiv \SIGN$.
\item
	$\Enc$ outputs $\iO(P_m)$, where $P_m$ is a program that
	outputs (embedded) message $m$ if input $\sk$ is a secret key for
	the given $\id$, or $\bot$ otherwise.
\item
	$\Dec$ outputs the result of $c(\sk_{\id})$.
\end{itemize}
However, this requires that we have encryption scheme where the
``signatures'' do not exist.
We therefore investigate an alternative scheme.
%
%\newcommand{\Com}{\ensuremath{\mathsf{Com}}}
%
Let $(K, P, V)$ be a non-interactive zero-knowledge (NIZK) proof system.
Denote by $\Com(\cdot ; r)$ the commitment algorithm of a non-interactive
commitment scheme with explicit random coin $r$.

\begin{itemize}
\item
	Let $\sigma$ be a common random string.
	$\SETUP(1^\kappa)$ outputs $(\mpk = (\sigma, c_1, c_2), \msk =
	r_1)$, where $c_1 = \Com(0 ; r_1)$ and
	$c_2 = \Com(0^{|\id|} ; r_2)$.

\item
	$\KEYGEN(\msk, \id)$ produces a proof
	$\pi = P(\sigma, x_{\id}, s)$ for the following language $L$:
	$x \in L$ if there exists $s$ such that
\begin{equation*}
\underbrace{c_1 = \Com(0 ; s)}_{\text{Type I witness}} \lor
\underbrace{\left( c_2 = \Com(\id^* ; s) \land \id^* \ne \id
	\right)}_{\text{Type II witness}}
\end{equation*}

\item
	Let $P_{\id,m}$ be a program which outputs $m$ if
	$V(\sigma, x_{\id}, \pi_{\id}) = 1$ or outputs $\bot$ otherwise.

	$\Enc(\mpk, \id, m)$ outputs $\iO(P_{\id,m})$.
\end{itemize}

We briefly sketch the hybrid argument:
\begin{description}
\item[$\Hyb{0}$]:
	This corresponds to an honest execution as described above.
\item[$\Hyb{1}$]:
	We let $c_2 = \Com(\id^* ; r_2)$, relying on the hiding property
	of the commitment scheme.
\item[$\Hyb{2}$]:
	We switch to the Type II witness using
	$\pi_{\id_i} \forall i \in [q]$, corresponding to the queries
	issued by the adversary during the first phase of the
	selective-identity security game.
\item[$\Hyb{3}$]:
	We let $c_1 = \Com(1 ; r_1)$.
\end{description}

%\nocite{*}
%\printbibliography



% !TEX root = collection.tex

\newcommand{\extline}{$\scriptsize$-$\normalsize$\!}
\newcommand{\lextlineend}{$\scriptsize$\lhd\!$\normalsize$}
\newcommand{\rextlineend}{$\scriptsize\rule{.1ex}{0ex}$\rhd$\normalsize$}

\newcounter{index}

\newcommand\extlines[1]{%
  \setcounter{index}{0}%
  \whiledo {\value{index}< #1}
  {\addtocounter{index}{1}\extline}
}

\newcommand\rextlinearrow[2]{$
  \setbox0\hbox{$\extlines{#2}\rextlineend$}%
  \tiny$%
  \!\!\!\!\begin{array}{c}%
  \mathrm{#1}\\%
  \usebox0%
  \end{array}%
  $\normalsize$\!\!%
}

\newcommand\lextlinearrow[2]{$
  \setbox0\hbox{$\lextlineend\extlines{#2}$}%
  \tiny%
  $%
  \!\!\!\!\begin{array}{c}%
  \mathrm{#1}\\%
  \usebox0%
  \end{array}%
  $\normalsize$\!\!%
}

\renewcommand\lextlinearrow[2]{%
}

\renewcommand\rextlinearrow[2]{%
}
\renewcommand\lextlinearrow[2]{%
%  \setbox0\hbox{$\lextlineend\extlines{#2}$}%
   $\stackrel{\mathrm{#1}}{\leftarrow}$%
}

\renewcommand\rextlinearrow[2]{%
  %\setbox0\hbox{$\extlines{#2}\rextlineend$}%
  $\stackrel{\mathrm{#1}}{\rightarrow}$%
}



%\section{Indistinguishable Obfuscation Constructions using Puncturing}
\section{Digital Signature Scheme via Indistinguishable Obfuscation}
A digital signature scheme can be constructed via indistinguishable obfuscation (iO).  A digital signature scheme is made up of $(\SETUP, \SIGN, \VERIFY)$.\\

%\newcommand{\vk}{\mathsf{vk}}

\noindent $(\vk, \sk) \leftarrow \SETUP(1^k)$:\\
\indent $\sk$ = key of puncturable function and the seed of the PRF $F_k$\\
\indent $\vk = iO(P_k)$ where $P_k$ is the program:\\
\indent \indent $P_k(m, \sigma)$:\\
\indent \indent \indent for some OWF function $f$\\
\indent \indent \indent \indent return 1 if $f(\sigma) = f(F_k(m))$\\
\indent \indent \indent \indent return 0 otherwise\\

\noindent $\sigma \leftarrow \SIGN(\sk, m)$: Output $F_k(m)$.\\

\noindent $\VERIFY(\vk, m, \sigma)$: Output $P_k(m, \sigma)$.\\

\noindent Our security requirements will be that the adversary does wins the following game negligibly:\\

\begin{tabular}{llc}
{\large Challenger} & & {\large Adversary}\\
$(\vk, \sk) = \SETUP(1^k)$ and&&\\
picks random $m$&&\\
& \rextlinearrow{P_{k},m}{46} &\\
& \lextlinearrow{\sigma, m^*}{46} &\\
& Adversary wins game if $\VERIFY(\vk, m^*, \sigma) = 1$&
\end{tabular}\\

\noindent To prove the security of this system, we use a hybrid argument.  $H_0$ is as above.

\noindent $H_1$: Adjust $\vk$ so that $\vk = iO(P_{k, m, \alpha})$ where $\alpha = F_k(m)$ and $P_{k, m, \alpha}$ is the program such that:\\
\indent $P_{k,m, \alpha}(m^*, \sigma)$:\\
\indent \indent for some OWF $f$\\
\indent \indent \indent if $m = m^*$:\\
\indent \indent \indent \indent if $f(\sigma) = f(\alpha)$ return 1\\
\indent \indent \indent \indent otherwise return 0\\
\indent \indent \indent else proceed as $P_{k}$ from before\\
\indent \indent \indent \indent if $f(\sigma) = f(F_k(m^*))$ return 1\\
\indent \indent \indent \indent otherwise return 0\\
\noindent Note that this program does not change its output for any value. This is indistinguishable from $H_0$  by indistinguishability obfuscation.\\

\noindent $H_2$: Adjust $\alpha$ so that it is a randomly sampled value. The indistinguishability of $H_2$ and $H_1$ follows from the security of PRG.  \\
\noindent $H_3$: Adjust the program such that instead of $\alpha$ it relies on some $\beta$ that is compared instead $f(\alpha)$ in the third line.\\

Any adversary that can break $H_3$ non-negligibly can break the OWF $f$ with at the value $\beta$.

\section{Public Key Encryption via Indistinguishable Obfuscation}
A public key encryption scheme can be constructed via indistinguishable obfuscation.  A public key encryption scheme is made up of $(Gen, Enc, Dec)$.  The PRG used below is a length doubling PRG.\\

\noindent $(\pk, \sk) \leftarrow Gen(1^k)$\\
\indent $\sk$ = key of puncturable function and the seed of the PRF $F_k$\\
\indent $\pk = iO(P_k)$ where $P_k$ is the program:\\
\indent \indent $P_k(m, r)$:\\
\indent \indent \indent $t = PRG(r)$\\
\indent \indent \indent Output $c = (t, F_k(t) \oplus m)$\\

\noindent $Enc(\pk, m)$: Sample $r$ and output $(\pk(m,r))$.\\

\noindent $Dec(\sk = k, c = (c_1, c_2))$: Output $F_k(\sk,c_1) \oplus c_2$.\\

\noindent Our security requirements will be that the adversary does wins the following game negligibly:\\

\begin{tabular}{llc}
{\large Challenger} & & {\large Adversary}\\
$(\pk, \sk) = Gen(1^k)$ and&&\\
Randomly sample $b$ from $\{0,1\}$ and&&\\
$c^* = Enc(\pk, b)$ and&&\\
& \rextlinearrow{P_{k},c^*}{26} &\\
& \lextlinearrow{b^*}{26} &\\
& Adversary wins game if $b=b^*$&
\end{tabular}\\

\noindent To prove the security of this system, we use a hybrid argument.  $H_0$ is as above.

\noindent $H_1$: Adjust $\pk$ so that $\pk = iO(P_{k, \alpha, t})$ where $\alpha = F_k(t)$ and $P_{k, \alpha, t}$ is the program such that:\\
\indent $P_{k, \alpha, t}(m, r)$:\\
\indent \indent $t^* = PRG(r)$\\
\indent \indent if $t^* = t$, output $(t^*, \alpha \oplus m)$\\
\indent \indent else output $(t^*, F_k(t^*) \oplus m)$\\

\noindent Note that this program does not change its output for any value. This is indistinguishable from $H_0$  by indistinguishability obfuscation.\\

\noindent $H_2$: Adjust $\alpha$ so that it is a randomly sampled value.\\
\noindent $H_3$: Adjust the program such that $t^*$ is randomly sampled and is not in the range of the PRG.\\

Any adversary that can win $H_3$ can guess a random value non-negligibly.\\

\section{Indistinguishable Obfuscation Construction from $NC^1$ $iO$}
A construction of indistinguishable obfuscation from $iO$ for circuits in $NC^1$ is as follows:\\
Let $P_{k,C}(x)$ be the circuit that outputs the garbled circuit $\widetilde{UC(C,x)}$ with randomness $F_k(x)$ which is a punctured (at $k$) PRF in $NC^1$\\
\indent Note that $UC(C,x)$ outputs $C(x)$ ($UC$ is the ``universal'' circuit)\\
$iO(C) \rightarrow $ sample $k$ randomly from $\{0,1\}^{|x|}$ and output $iO_{NC^1}(P_{k,C})$ padded to a length $l$\\

As before, we use a hybrid argument to show the security for $iO$.\\
\noindent $H_0$: $iO(C) = iO_{NC^1}(P_{k,C})$ as above.\\
\noindent $H_{final} = H_{2^n}$: $iO(\pk, c_2)$\\
\noindent $H_1 \cdots H_i$: Create a program $Q_{k, c_1, c_2, i}(x)$ and obfuscate it.\\
$Q_{k,c_1,c_2,i}(x)$:\\
\indent Sample $k$ randomly\\
\indent if $x \ge i$, return $P_{k,c_1}(x)$\\
\indent else , return $P_{k,c_2}(x)$\\
\noindent Note that $H_i$ and $H_{i+1}$ are indistinguishable for any value other than $x=i$.\\
\noindent $H_{i,1}$ (between $H_i$ and $H_{i+1}$): Create a program $Q_{k, c_1, c_2, i, \alpha}(x)$, where $\alpha = Q_{k, c_1, c_2, i}(x)$ and obfuscate it.\\
$Q_{k, c_1, c_2, i, \alpha}(x)$:\\
\indent Sample $k$ randomly\\
\indent if $x = i$, return $\alpha$\\
\indent else , return $Q_{k,c_1,c_2, i}(x)$\\

\noindent $H_{i,2}$: Replace $\alpha$ with a random $\beta$ using fresh coins\\
\noindent $H_{i,3}$: Create the $c_2(x)$ value using fresh coins\\
\noindent $H_{i,4}$: Create the $c_2(x)$ value using $F_k(x)$\\
\noindent $H_{i,5}$: Finish the migration to $Q_{k,c_1,c_2,i+1}$\\

Note that at $H_{final}$, the circuit being obfuscated is completely changed from $c_1$ to $c_2$.





%%
% The back matter contains appendices, bibliographies, indices, glossaries, etc.



\backmatter

\DIFdelbegin %DIFDELCMD < \begin{thebibliography}{0}
%DIFDELCMD < \providecommand{\natexlab}[1]{#1}
%DIFDELCMD < \providecommand{\url}[1]{\texttt{#1}}
%DIFDELCMD < \expandafter\ifx\csname %%%
\DIFdel{urlstyle}%DIFDELCMD < \endcsname\relax
%DIFDELCMD <   \providecommand{\doi}[1]{doi: #1}\else
%DIFDELCMD <   \providecommand{\doi}{doi: \begingroup \urlstyle{rm}\Url}\fi
%DIFDELCMD < %%%
\DIFdelend \DIFaddbegin \begin{thebibliography}{10}

\bibitem{EPRINT:AlbPlaSco15}
\DIFadd{Martin~R. Albrecht, Rachel Player, and Sam Scott.
}\newblock \DIFadd{On the concrete hardness of learning with errors.
}\newblock \DIFadd{Cryptology ePrint Archive, Report 2015/046, 2015.
}

\bibitem{JC:Bellare02}
\DIFadd{Mihir Bellare.
}\newblock \DIFadd{A note on negligible functions.
}\newblock {\em \DIFadd{Journal of Cryptology}}\DIFadd{, 15(4):271--284, September 2002.
}

\bibitem{AC:BonLynSha01}
\DIFadd{Dan Boneh, Ben Lynn, and Hovav Shacham.
}\newblock \DIFadd{Short signatures from the }{\DIFadd{Weil}} \DIFadd{pairing.
}\newblock \DIFadd{In Colin Boyd, editor, }{\em \DIFadd{Advances in Cryptology --
  }{\DIFadd{ASIACRYPT}}\DIFadd{~2001}}\DIFadd{, volume 2248 of }{\em \DIFadd{Lecture Notes in Computer Science}}\DIFadd{,
  pages 514--532, Gold Coast, Australia, December~9--13, 2001. Springer,
  Berlin, Heidelberg, Germany.
}

\bibitem{ITCS:BraGenVai12}
\DIFadd{Zvika Brakerski, Craig Gentry, and Vinod Vaikuntanathan.
}\newblock \DIFadd{(}{\DIFadd{L}}\DIFadd{eveled) fully homomorphic encryption without bootstrapping.
}\newblock \DIFadd{In Shafi Goldwasser, editor, }{\em \DIFadd{ITCS 2012: 3rd Innovations in
  Theoretical Computer Science}}\DIFadd{, pages 309--325, Cambridge, MA, USA,
  January~8--10, 2012. Association for Computing Machinery.
}

\bibitem{C:GarHaj18}
\DIFadd{Sanjam Garg and Mohammad Hajiabadi.
}\newblock \DIFadd{Trapdoor functions from the computational }{\DIFadd{Diffie}}\DIFadd{-}{\DIFadd{Hellman}}
  \DIFadd{assumption.
}\newblock \DIFadd{In Hovav Shacham and Alexandra Boldyreva, editors, }{\em \DIFadd{Advances in
  Cryptology -- }{\DIFadd{CRYPTO}}\DIFadd{~2018, Part~II}}\DIFadd{, volume 10992 of }{\em \DIFadd{Lecture Notes in
  Computer Science}}\DIFadd{, pages 362--391, Santa Barbara, CA, USA, August~19--23,
  2018. Springer, Cham, Switzerland.
}

\bibitem{STOC:Gentry09}
\DIFadd{Craig Gentry.
}\newblock \DIFadd{Fully homomorphic encryption using ideal lattices.
}\newblock \DIFadd{In Michael Mitzenmacher, editor, }{\em \DIFadd{41st Annual }{\DIFadd{ACM}} \DIFadd{Symposium on
  Theory of Computing}}\DIFadd{, pages 169--178, Bethesda, MD, USA, May~31~--~June~2,
  2009. }{\DIFadd{ACM}} \DIFadd{Press.
}

\bibitem{C:GenSahWat13}
\DIFadd{Craig Gentry, Amit Sahai, and Brent Waters.
}\newblock \DIFadd{Homomorphic encryption from learning with errors:
  Conceptually-simpler, asymptotically-faster, attribute-based.
}\newblock \DIFadd{In Ran Canetti and Juan~A. Garay, editors, }{\em \DIFadd{Advances in
  Cryptology -- }{\DIFadd{CRYPTO}}\DIFadd{~2013, Part~I}}\DIFadd{, volume 8042 of }{\em \DIFadd{Lecture Notes in
  Computer Science}}\DIFadd{, pages 75--92, Santa Barbara, CA, USA, August~18--22, 2013.
  Springer, Berlin, Heidelberg, Germany.
}

\bibitem{STOC:ImpRud89}
\DIFadd{Russell Impagliazzo and Steven Rudich.
}\newblock \DIFadd{Limits on the provable consequences of one-way permutations.
}\newblock \DIFadd{In }{\em \DIFadd{21st Annual }{\DIFadd{ACM}} \DIFadd{Symposium on Theory of Computing}}\DIFadd{, pages
  44--61, Seattle, WA, USA, May~15--17, 1989. }{\DIFadd{ACM}} \DIFadd{Press.
}

\bibitem{DCC:LanSte15}
\DIFadd{Adeline Langlois and Damien Stehl}{\DIFadd{\'e}}\DIFadd{.
}\newblock \DIFadd{Worst-case to average-case reductions for module lattices.
}\newblock {\em \DIFadd{Designs, Codes and Cryptography}}\DIFadd{, 75(3):565--599, 2015.
}

\bibitem{RSA:LinPei11}
\DIFadd{Richard Lindner and Chris Peikert.
}\newblock \DIFadd{Better key sizes (and attacks) for }{\DIFadd{LWE}}\DIFadd{-based encryption.
}\newblock \DIFadd{In Aggelos Kiayias, editor, }{\em \DIFadd{Topics in Cryptology --
  CT-RSA~2011}}\DIFadd{, volume 6558 of }{\em \DIFadd{Lecture Notes in Computer Science}}\DIFadd{, pages
  319--339, San Francisco, CA, USA, February~14--18, 2011. Springer, Berlin,
  Heidelberg, Germany.
}

\bibitem{EC:LyuPeiReg10}
\DIFadd{Vadim Lyubashevsky, Chris Peikert, and Oded Regev.
}\newblock \DIFadd{On ideal lattices and learning with errors over rings.
}\newblock \DIFadd{In Henri Gilbert, editor, }{\em \DIFadd{Advances in Cryptology --
  }{\DIFadd{EUROCRYPT}}\DIFadd{~2010}}\DIFadd{, volume 6110 of }{\em \DIFadd{Lecture Notes in Computer Science}}\DIFadd{,
  pages 1--23, French Riviera, May~30~--~June~3, 2010. Springer, Berlin,
  Heidelberg, Germany.
}

\bibitem{EC:LyuPeiReg13}
\DIFadd{Vadim Lyubashevsky, Chris Peikert, and Oded Regev.
}\newblock \DIFadd{A toolkit for ring-}{\DIFadd{LWE}} \DIFadd{cryptography.
}\newblock \DIFadd{In Thomas Johansson and Phong~Q. Nguyen, editors, }{\em \DIFadd{Advances in
  Cryptology -- }{\DIFadd{EUROCRYPT}}\DIFadd{~2013}}\DIFadd{, volume 7881 of }{\em \DIFadd{Lecture Notes in
  Computer Science}}\DIFadd{, pages 35--54, Athens, Greece, May~26--30, 2013. Springer,
  Berlin, Heidelberg, Germany.
}

\bibitem{EC:MicPei12}
\DIFadd{Daniele Micciancio and Chris Peikert.
}\newblock \DIFadd{Trapdoors for lattices: Simpler, tighter, faster, smaller.
}\newblock \DIFadd{In David Pointcheval and Thomas Johansson, editors, }{\em \DIFadd{Advances in
  Cryptology -- }{\DIFadd{EUROCRYPT}}\DIFadd{~2012}}\DIFadd{, volume 7237 of }{\em \DIFadd{Lecture Notes in
  Computer Science}}\DIFadd{, pages 700--718, Cambridge, UK, April~15--19, 2012.
  Springer, Berlin, Heidelberg, Germany.
}

\bibitem{STOC:Regev05}
\DIFadd{Oded Regev.
}\newblock \DIFadd{On lattices, learning with errors, random linear codes, and
  cryptography.
}\newblock \DIFadd{In Harold~N. Gabow and Ronald Fagin, editors, }{\em \DIFadd{37th Annual }{\DIFadd{ACM}}
  \DIFadd{Symposium on Theory of Computing}}\DIFadd{, pages 84--93, Baltimore, MA, USA,
  May~22--24, 2005. }{\DIFadd{ACM}} \DIFadd{Press.
}

\bibitem{FOCS:Shor94}
\DIFadd{Peter~W. Shor.
}\newblock \DIFadd{Algorithms for quantum computation: Discrete logarithms and
  factoring.
}\newblock \DIFadd{In }{\em \DIFadd{35th Annual Symposium on Foundations of Computer Science}}\DIFadd{,
  pages 124--134, Santa Fe, NM, USA, November~20--22, 1994. }{\DIFadd{IEEE}} \DIFadd{Computer
  Society Press.
}\DIFaddend 

\end{thebibliography}

%\bibliographystyle{plainnat}
\bibliographystyle{plain}


%\printindex

\end{document}


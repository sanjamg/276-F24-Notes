% !TEX root = collection.tex

\section{BLS Signatures}

\paragraph{Bilinear Pairings.} Recall that we used prime order groups $\mathbb{G}$ to build the Schnorr signature scheme. Today we will introduce another mathematical object, pairing friendly groups, which support a new ``bilinear pairing'' operation defined as follows. Let $\mathbb{G}_1$, $\mathbb{G}_2$ and $\mathbb{G}_T$ be groups of prime order $p$. We define a bilinear pairing $e: G_1 \times G_2 \rightarrow \mathbb{G}_T$ to be an \emph{efficiently} computable map that satisfies the following properties:
\begin{itemize}
    \item Bilinearity: For all $a, b \in \mathbb{Z}_p$ and $g_1 \in G_1, g_2 \in G_2$, we have $e(g_1^a, g_2^b) = e(g_1, g_2)^{ab}$.
    \item Non-degenerate: $e(g_1, g_2) \neq 1_{\mathbb{G}_T}$.
\end{itemize}
Observe that in any group, by using the group operation, one can compute ``additions'' of the scalar exponents, but in pairing friendly groups one can additionally compute \emph{one} ``multiplication'' of the scalar exponents.

Note that while a pairing provides us with additional \emph{functionality}, it also means that we need to reevaluate any hardness assumptions that we have made. Consider for instance the decisional Diffie-Hellman (DDH) assumption which states that it is hard to distinguish between $(g, g^a, g^b, g^{ab})$ and $(g, g^a, g^b, g^c)$ for random $a, b, c \in \mathbb{Z}_p$. In the presence of a bilinear pairing, natural variants of the DDH assumption are no longer hard such as distinguishing between $(g_1, g_2, g_1^a, g_2^b, g_1^{ab})$ and $(g_1, g_2, g_1^a, g_2^b, g_1^{c})$. Instead, we will define new assumptions that are conjectured to be hard even given a bilinear pairing.

\paragraph{co-CDH assumption.} Let $\mathbb{G}_1$, $\mathbb{G}_2$ be groups of prime order with generators $g_1$ and $g_2$ and let $e: \mathbb{G}_1 \times \mathbb{G}_2 \rightarrow \mathbb{G}_T$ be an efficiently computable bilinear pairing. The co-CDH assumption states that for all non-uniform PPT adversaries $\adv$:
\begin{align*}
    \Pr\left[ \adv(g_1, g_2, g_1^a, g_1^b, g_2^b) \to g_1^{ab} \mid a,b \sample \mathbb{Z}_p \right] \leq \negl
\end{align*}

\paragraph{BLS Signature scheme.} We are now ready to describe the BLS signature scheme~\cite{AC:BonLynSha01}.
\begin{itemize}
    \item $\Gen(1^\lambda)$: Sample $\sk \sample \mathbb{Z}_p$ and set $\vk \gets g_2^\sk$. Output $(\vk, \sk)$.
    \item $\Sign(\sk, m)$: Given a message $m \in \{0,1\}^*$, output $\sigma \gets H(m)^{\sk}$, where $H: \{0,1\}^* \to \mathbb{G}_1$ is a hash function that maps arbitrary strings to elements in $\mathbb{G}_1$.
    \item $\Verify(\vk, m, \sigma)$: Output $e(\sigma, g_2) \stackrel{?}{=} e(H(m), \vk)$.
\end{itemize}

\paragraph{Correctness.} Correctness is easy to see by plugging in explicit expressions for the signature and verification key -- $e(\sigma=H(m)^\sk, g_2) \stackrel{?}{=} e(H(m), \vk=g_2^\sk)$.

\paragraph{Security.} If $H$ is modelled as a random oracle, we will show that if there exists an adversary $\adv$ that can forge a signature with non-negligible probability, then we can use $\adv$ to build $\advb$ that can solve the co-CDH problem with non-negligible probability.

\begin{theorem}
    The BLS signature scheme is is existentially unforgeable under chosen message attacks assuming the co-CDH problem is hard in the random oracle model.
\end{theorem}

\begin{proof}
    We will use a similar strategy as in the proof of RSA full domain hash. Let $\adv$ be a non-uniform PPT adversary that can forge a signature with non-negligible probability. We will use $\adv$ to build a non-uniform PPT adversary $\advb$ that can solve the co-CDH problem with non-negligible probability.

    $\advb$ works as follows:
    \begin{itemize}
        \item $\advb$ receives $(g_1, g_2, g_1^a, g_1^b, g_2^b)$ as input and is tasked with computing $g_1^{ab}$.
        \item $\advb$ now runs the EUF-CMA game with $\adv$ where $\adv$ makes signing queries and in the end outputs a forgery $(m^*, \sigma^*)$, where $m^*$ has not been previously queried.
        \item At the start of the protocol, $\advb$ sets $\vk = g_1^b$ and sends it to $\adv$. For all random oracle queries $H(m)$, $\adv$ samples $r \sample \mathbb{Z}_p$ and sets $H(m) = g_1^r$. When $\adv$ asks for a signature on $m_i$, $\advb$ samples $r_i \sample \mathbb{Z}_p$ and sets $H(m_i) = g_1^{r_i}$ (if the query has not been previously made). It then responds with $\sigma_i = H(m_i)^b = (g_1^b)^{r_i}$. However, for a randomly chosen index $i^*$ (out of the maximum number of queries $\adv$ can make), $\advb$ sets $H(m_{i^*}) = g_1^a$. Now, if $\adv$ asks for a signature on the chosen $m_{i^*}$, $\advb$ aborts and restarts the EUF-CMA game. In the ends, $\adv$ outputs a forgery with non-negligible probability. And since there are only a polynomial number of queries, $\adv$ outputs a forgery on $m_{i^*}$ with non-negligible probability. In which case $\advb$ outputs $\sigma_{i^*}$ as its output in the co-CDH game.
    \end{itemize}

    We now analyze the probability that $\advb$ solves the co-CDH problem.
    \begin{align*}
        \Pr[\advb \to g_1^{ab}] & = \Pr[\adv \text{ outputs forgery} \wedge m_{i^*} \text{ was not queried} \wedge \text{forgery on } m_{i^*}] \\
                                & \geq \epsilon \times \left( 1 -\frac{1}{q_h}\right)^{q_s} \times \frac{1}{q_h}
    \end{align*}
    which is non-negligible.
    Thus, by contradiction, the BLS signature scheme is existentially unforgeable under chosen message attacks assuming the co-CDH problem is hard in the random oracle model.
\end{proof}
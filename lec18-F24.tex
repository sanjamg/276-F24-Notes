\chapter{Proving Computation Integrity}
\section{Zero-Knowledge Proofs}
Traditional Euclidean style proofs allow us to prove veracity of statements to others. However, such proof systems have two shortcomings: (1) the running time of the verifier needs to grow with the length of the proof, and (2) the proof itself needs to be disclosed to the verifier. In this chapter, we will provide methods enabling provers to prove veracity of statements of their choice to verifiers while avoiding the aforementioned limitations. In realizing such methods we will allow the prover and verifier to be probabilistic and also allow them to interact with each other.\footnote{Formally, they can be modeled as interactive PPT Turing Machines.}

\section{Interactive Proofs}
\begin{definition} {\normalfont\textbf{(Interactive Proof System)}} For a language L we have an \textit{interactive proof system} if $\exists$ a pair of algorithms (or better, interacting machines) $(\mathcal{P},\mathcal{V})$, where $\mathcal{V}$ runs in polynomial time in its input length, and both can flip coins, such that:
		\begin{itemize}
			\item Completeness: $\forall x\in L$
		$$\Pr_{\mathcal{P},\mathcal{V}} \left[Output_{\mathcal{V}}(\mathcal{P}(x) \leftrightarrow \mathcal{V}(x))=1\right]=1,$$
			\item Soundness: $\forall x\notin L$, $\forall \mathcal{P}^*$ (unbounded)
		$$\Pr_{\mathcal{V}} \left[Output_{\mathcal{V}}(\mathcal{P}^*(x) \leftrightarrow \mathcal{V}(x))=1\right]<\mathsf{negl}(|x|),$$
		\end{itemize} where $Output_{\mathcal{V}}(\mathcal{P}(x) \leftrightarrow \mathcal{V}(x))$ denotes the output of $\mathcal{V}$ in the interaction between $\mathcal{P}$ and $\mathcal{V}$ where both parties get $x$ as input.
		We stress that $\mathcal{P}$ and $\mathcal{P}^*$ can be computationally unbounded. 
  \end{definition}
We can also consider other variants of this definition, e.g. imperfect completeness.

To understand the above definition, let's consider two languages over a pair of graphs $G_0$ and $G_1$: 
\begin{enumerate}
	\item Graph Isomorphism (GI): We say that two graphs $G_0$ and $G_1$ are isomorphic, denoted $G_0 \cong G_1$, if $\exists$ an isomorphism $f: V(G_0) \rightarrow V(G_1)$ s.t. $(u,v)\in E(G_0)$ iff $(f(u),f(v))\in E(G_1)$, where $V(G)$ and $E(G)$ are the vertex and edge sets of some graph $G$. Let $GI=\lbrace(G_0,G_1)|\  G_0\cong G_1\rbrace$ be the language that consists of pairs of graphs that are isomorphic.
	\item Graph Non-Isomorphism (GNI): On the other hand, $G_0$ and $G_1$ are said to be non-isomorphic, $G_0 \ncong G_1$, if $\nexists$ any such $f$, and let $GNI=\lbrace(G_0,G_1)|\  G_0\ncong G_1\rbrace$ be the language that consists of pairs of graphs that are not isomorphic.
\end{enumerate}
 
\paragraph{Trivial Case of Graph Isomorphism (GI).} A prover can easily prove to a verifier that two graphs are isomorphic by directly providing the isomorphism $f$ between them. The verifier can confirm the isomorphism in time polynomial in the size of the graphs (i.e., its input); hence we have perfect completeness. If the graphs are not isomorphic, no isomorphism exists, and the verifier always rejects; we have perfect soundness too. This proof was trivial, and we didn't even require (back-and-forth) interaction. We now look at a more interesting case of GNI. Moreover, looking ahead, we will see more interesting properties that we can ask of proof systems, like zero-knowledge, where this trivial proof system terribly fails, and we will revisit the GI problem to see how we can prove it with zero-knowledge.

\paragraph{Interactive Proof for Graph Non-Isomorphism (GNI).}  Unlike the case of GI, for GNI, there is no succinct (e.g., linear in the size of graphs) information that the prover can provide, and consequently, no ``efficient'' (polynomial time in the graphs) verification that the verifier can do. This is where the {\em power of interaction} comes in. In other words, since GNI is not believed to have short proofs, an {\em interactive} proof could offer the prover a mechanism to prove to a polynomially bounded verifier that two graphs are non-isomorphic. We will now describe an interactive proof system for GNI.

The intuition is simple. Consider a verifier that randomly renames the vertices of one of the graphs and give it to the prover. Can the prover, given the relabeled graph, figure out which graph did the verifier start with?  If $G_0$ and $G_1$ were not isomorphic, then an unbounded-time prover can figure this out. However, in case $G_0$ and $G_1$ {are} isomorphic, then the distributions resulting form random relabelings of $G_0$ and $G_1$ are actually identical. Therefore, even an unbounded prover has no way of distinguishing which graph the verifier started with. So the prover has only a $\frac12$ probability of guessing which graph the verifier started with. Note that by repeating this process we can reduce the success probability of a cheating prover to negligible\footnote{This strategy is called soundness amplification by ``sequential'' repetition. Later, we might cover proof systems where we additionally consider ``parallel'' repetition to achieve different security properties.}. More formally, given a claim $(G_0,G_1)\in GNI$, we define the following interactive proof system:

%		If they consistently answer correctly, however, it would be hard to remain skeptical against $G_0 \ncong G_1$ as they beat the odds to almost impossible limits.  And so this interaction can ``prove" very strongly to the verifier that $(G_0,G_1)\in$ GNI.  Consider the protocol we can define from this:

		\begin{center}
			\includegraphics[scale=.51094]{Old Scribe Notes/GNI_IP_Protocol.png}
		\end{center}

		\begin{itemize}
			\item Completeness: If $(G_0,G_1)\in$ GNI, then the unbounded $\mathcal{P}$ can distinguish isomorphism of $G_0$ against those of $G_1$ and can always return the correct $b'$.  Thus, $\mathcal{V}$ will always output 1 for this case.
			\item Soundness: If $(G_0,G_1)\notin$ GNI, then it is equiprobable that $H$ is a random isomorphism of $G_0$ as it is of $G_1$, and so $\mathcal{P}$'s guess for $b'$ can be correct only with a probability $\frac{1}{2}$\footnote{A curious reader might notice that the challenge bit $b$ sampled by $\mathcal{V}$ is information-theoretically hidden from $\mathcal{P}$ (hidden in $H$) when $\mathcal{P}$'s claim is false. This is similar to what we saw in Hash Proof Systems before.}.  Repeating this protocol $k$ times, with fresh verifier randomness each time, means the probability of guessing the correct $b'$ for all $k$ interactions is $\frac{1}{2^k}$.  And so the probability of $\mathcal{V}$ outputting 0 (e.g. rejecting $\mathcal{P}$'s proof at the first sign of falter) is $1-\frac{1}{2^k}$.  

		\end{itemize}

		%The interaction between prover and verifier captures the notion of a proof system for GNI, a problem previously not known to have an efficient method of proof.  By interacting, we can prove what seemed impossible to efficiently prove before!
		To conclude, the interactive proof system we described above enabled something that wasn't possible without interaction. 

\section{Zero Knowledge Proofs}
We saw a crucial difference between GI and GNI: in GI, the prover already holds a succinct proof to back its claim, we call this a ``witness'', while in GNI, no such succinct proof exists (i.e., there is nothing that the prover can directly send to the verifier to back its claim). From this point onwards, we exclusively focus on the languages of the first kind, i.e., where a witness for the claim exists; these languages a vast majority of the use-cases of verifiable computation. These languages are formalized as follows:

\begin{definition} {\normalfont\textbf{(NP-Verifier)}} A language L has an NP-verifier if $\exists$ a verifier $\mathcal{V}$ that is polynomial time in $|x|$ such that:
		\begin{itemize}
			\item Completeness: $\forall x\in L,\ \exists\ a\ proof\ \pi\ s.t.\ \mathcal{V}(x,\pi)=1$
			\item Soundness: $\forall x \notin L$, and $\forall$ purported proof $\pi$, we have $\mathcal{V}(x,\pi)=0$
		\end{itemize}
  \end{definition}

		That is, the conventional idea of a proof is formalized in terms of what a computer can efficiently verify.\smallskip %A language $L$ has an NP-verifier if for all claims of statements $\in L$, a proof can be written down that can be ``easily'' and ``rigorously'' verified if and only if a statement is in the language.

	\noindent \textit{Keeping the witness private}. The goal of a proof system is for the verifier to learn if the prover's claim is valid or not. Let's focus on what a verifier actually learns at the end of its interaction with the prover. In the trivial GI proof system we saw above, the verifier learns the entire automorphism --- in other words, the verifier learns {\em everything} that the prover knew. 
	This is too much leakage. Imagine the prover holding some secret or valuable information (e.g., its secret key) which is leaked to the verifier. This is not desirable. We want the verifier to learn only the validity of the claim, and nothing more. This is where the notion of zero-knowledge comes in.
	For a proof system for a language with an NP-verifier, this translates to the verifier not learning the witness from the prover.
	
	We now revisit the GI problem\footnote{GI is not NP-complete.} for which an NP-verifier exists, as we saw earlier. Later, we will consider NP-complete languages like graph 3-coloring --- giving us proof systems for all of NP. \smallskip

	\noindent \textit{Hiding witness for Graph Isomorphism}. We will build the ideas for our proof system with zero-knowledge gradually by iterating through a series of straw-man approaches. On the way, we will formally define zero knowledge. 

	When $G_0$ and $G_1$ are isomorphic, the isomorphism between them would be a \textit{witness}, $w$, to that fact, that can be used in the proof.  The prover doesn't want to reveal the isomorphism, $w:V(G_0)\rightarrow V(G_1)$, that they claim to have.  The prover is comfortable however giving us a ``scrambled'' version, $\phi$, of $w$ as long as it doesn't leak any information about their precious $w$.  For example, the prover is willing to divulge $\phi = \pi \circ w$ where $\pi$ is a privately chosen random permutation of $|V|=|V(G_0)|=|V(G_1)|$ vertices.  Since $\pi$ renames vertices completely randomly, it scrambles what $w$ is doing entirely and $\phi$ is just a random permutation of $|V|$ elements.  At this point, we might be a little annoyed at the prover since we could have just created a random permutation on our own.  Let's look at why this is a good starting point.
		
	If we want to be convinced that $\phi$ really is of the form $\pi \circ w$, thus containing $w$ in its definition, and isn't just a completely random permuation, we can note that if it is of that form then $\phi(G_0)=\pi(w(G_0))=\pi(G_1)$ (since $w$ being an isomorphism implies that $w(G_0)=G_1$).  Note that we started with a mapping on input $G_0$ and ended with a mapping on input $G_1$.  With an isormphism, one could get from one graph to the other seamlessly; if the prover \textit{really} has the isomorphism it claims to have, then it should have no problem displaying this ability.  So, what if we force the prover to give us $H=\pi (G_1)$ just after randomly choosing its $\pi$ and then let it show us its ability to go from $G_1$ to $G_0$ with ease: give us a $\phi$ so that $\phi(G_0)=\pi(G_1)=H$.  The only way the prover can give a mapping that jumps from $G_0$ to $G_1$ is if they know an isomorphism; in fact, if the prover could find a $\phi$ efficiently but did \textit{not} know an isomorphism then they would have been able to see that $\pi^{-1}(\phi(G_0))=G_1$ and thus have $\pi^{-1}\circ\phi$ as an isomorphism from $G_0$ to $G_1$, which would contradict the assumed hardness of finding isomorphisms in the GI problem. So by forcing the prover to give us $H$, as we've defined, and to produce a $\phi$ so that $\phi(G_0)=H$, we've found a way to expose provers that don't really have an isomorphism and we can then be convinced that they really do know $w$ when they pass our test.  Importantly, the prover didn't directly tell us $w$, so we are headed in the right direction.
		
	But not everything is airtight about this interaction.  Why, for instance, would the prover be willing to provide $H=\pi(G_1)$ when they're trying to divulge as little information as possible?  The prover was comfortable giving us $\phi$ since we could have just simulated the process of getting a completely random permutation of vertices ourselves, but couldn't the additional information of $H$ reveal information about $w$?  At this point, if we look closely, we realize that $H=\pi(G_1)=\pi'(G_0)$, for some $\pi'$, is just a random isomorphic copy of $G_0$ \textit{and} $G_1$ as long as $G_0 \cong G_1$; we could have just chosen a random $\pi'$, set $H=\pi'(G_0)$, and let $\phi=\pi'$ and would have created our very own random isomorphic copy, $H$, of $G_1$ that satisfies our test condition $H=\phi(G_0)$, just like what we got from our interaction with the prover. To our annoyance, the prover can easily fool this test. Indeed, the test has a hole in it: how can we force the prover to give us $H=\pi(G_1)$ like we asked?  If the prover is lying and it knows our test condition is to verify that $H=\phi(G_0)$, the prover might just cheat and give us $H=\pi(G_0)$ so it doesn't have to use knowledge of $w$ to switch from $G_1$ to $G_0$.  And, in fact, by doing this and sending $\phi=\pi$, the prover would fool us!
		
	To keep the prover on their toes, though, we can randomly switch whether we want $H$ to equal $\phi(G_0)$ or $\phi(G_1)$.  In our interaction, the prover must first provide $H=\pi(G_1)$ before we let them know which we want. By sending $H$, the prover locks itself into a commitment to either $G_0$ or $G_1$ if it is cheating, but if not, then it can easily move between the two graphs. A prover only has a $50\%$ chance of committing to the same case we want on a given round and so, if they don't have $w$ to deftly switch between $G_0$ and $G_1$ to always answer correctly, they again have to be an extremely lucky guesser if they're trying to lie.
		
	Therefore, we've created an interactive scheme that can catch dishonest provers with probability 1-$\frac{1}{2^k}$ and where we always believe honest provers!
		
		\begin{center}
			\includegraphics[scale=.51094]{Old Scribe Notes/GI_ZK_Protocol.png}
		\end{center}
		
		\begin{itemize}
			\item Completeness: If $(G_0,G_1)\in GI$ and $\mathcal{P}$ knows $w$, then whether $\mathcal{V}$ chooses $b=0$ or 1, $\mathcal{P}$ can always give the correct $\phi$ which, by definition, will always result in $H=\phi(G_b)$ and so $\mathcal{V}$ will always output 1.
			\item Soundness: If $(G_0,G_1)\notin GI$, then $\mathcal{P}$ can only cheat, as discussed earlier, if the original $H$ it commits to ends up being $\pi(G_b)$ for the $b$ that is randomly chosen at the next step.  Since $b$ isn't even chosen yet, this can only happen by chance with probability $\frac{1}{2}$.  And so the probability $\mathcal{V}$ outputs 0 is $1-\frac{1}{2^k}$ for $k$ rounds.
		\end{itemize}
		
		We have just shown that what have so far is an interactive proof system. We now think of how the notion of zero-knowledge can be formalized here.
		
		As a verifier, we've seen some things in interacting with the prover.  Surely, clever folks like ourselves must be able to glean \textit{some} information about $w$ after seeing enough to thoroughly convince us that the prover knows $w$.  We've first seen $H$, and we've also seen the random $b$ that we chose, along with $\phi$ at the end;  this is our whole view of information during the interaction.  But we're more bewildered than annoyed this time when we realize we could have always just chosen $b$ and $\phi$ randomly and set $H=\phi(G_b)$ on our own.  Again, everything checks out when $G_0 \cong G_1$ and we could have produced everything that we saw during the interaction before it even began.  That is, the distribution of the random variable triple ($H$, $b$, $\phi$) is identical whether it is what we saw from the prover during the interaction or it is yielded from the solitary process we just described.  We've just constructed a complete interactive proof system that entirely convinces us of the prover's knowledge of $w$, yet we could have simulated the whole experience on our own!  We couldn't have gained any knowledge about $w$ since we didn't see anything we couldn't have manufactured on our own, yet we are entirely convinced that $(G_0,G_1)\in$ GI and that $\mathcal{P}$ knows $w$!  And so the prover has proven something to us yet has given us absolutely zero additional knowledge!
		
		This may feel very surprising or as if you've been swindled by a fast talker, and it very much should feel this way; it was certainly an amazing research discovery!  But this is true, and it can be made rigorous, as we do next.
		
		We should first be sure what we want out of this new proof system.  We of course want it to be complete and sound so that we accept proofs iff they're true.  But we also want the verifier to gain zero knowledge from the interaction; that is, the verifier should have been able to simulate the whole experience on its own without the verifier.
		Finally, we would also like all witnesses to a true statement to each be sufficient to prove the veracity of that statement and so we let $R$ be the relation s.t. $x \in L$ iff $\exists$ a witness $w$ s.t. $(x,w)\in R$.  We can then gather all witness by defining $R(x)$ to be the set of all such witnesses. We will first look at a weaker notion of zero-knowledge, called \textit{Honest Verifier Zero Knowledge} (HVZK), where we only require that an {\em honest} verifier does not learn anything from the prover. We will then move on to the stronger notion of \textit{Zero Knowledge} (ZK), where we extend this to all verifiers, including malicious verifiers.
		
		\begin{definition} {\normalfont\textbf{(Honest Verifier Zero Knowledge Proof [HVZK])}} 
			$(\mathcal{P},\mathcal{V})$ is a (perfect) HVZK proof system for a language $L$ w.r.t. witness relation $R$ if 
			$\exists$ a PPT machine $\mathcal{S}$ (called the simulator) s.t. $\forall x \in L$, $\forall w\in R(x)$, the following distributions are (identical) indistinguishable:
		$$\{View_{\mathcal{V}}(\mathcal{P}(x,w) \leftrightarrow \mathcal{V}(x))\} \approx \{\mathcal{S}(x)\}$$
		where $View_{\mathcal{V}}(\mathcal{P}(x,w) \leftrightarrow \mathcal{V}(x))$ is the random coins of $\mathcal{V}$ and all the messages $\mathcal{V}$ saw.
  \end{definition}

\begin{remark}
In the above definition, $View_{\mathcal{V}}(\mathcal{P}(x,w) \leftrightarrow \mathcal{V}(x))$ contains both the random coins of $\mathcal{V}$ and all the messages that $\mathcal{V}$ saw, because they together constitute the view of $\mathcal{V}$, and they are correlated. If the random coins of $\mathcal{V}$ are not included in the definition of $View_{\mathcal{V}}(\mathcal{P}(x,w) \leftrightarrow \mathcal{V}(x))$, then even if $\mathcal{S}$ can generate all messages that $\mathcal{V}$ saw with the same distribution as in the real execution, the verifier may still be able to distinguish the two views using its random coins.
\end{remark}

\begin{remark}
In the above definition, the order of quantifiers is quite important. We cannot change it to: $\forall x \in L$, $\forall w\in R(x)$, $\exists$ a PPT machine $\mathcal{S}$. This is because the definition would be trivially satisfied by hardcoding the witness $w$ in the simulator $\mathcal{S}$.
\end{remark}

To prove HVZK property of the GI proof system we described earlier, we now construct a simulator $\mathcal{S}$, with input $x$, as follows:
\begin{enumerate}
	\item Sample $b\in\{0,1\}$ uniformly at random.
	\item Sample a random permutation $\sigma$ of the vertices.
	\item Set $H \gets \sigma(G_b)$.
	\item Output $(H, b, \sigma)$.
\end{enumerate}
It is straightforward to see that this simulator produces the same distribution as the real interaction between the prover and the verifier. This is because $H = \sigma(G_b) = \sigma'(G_{1-b})$, i.e., $H$ is a random permutation of both $G_0$ amd $G_1$. 
	
To recap: There is an interesting progression of the requirements of a proof system: Completeness, Soundness, and the Zero Knowledge property.  Completeness first cares that a prover-verifier pair exist and can capture all true things as a team that works together; they both honestly obey the protocol trying prove true statements.  Soundness, however, assumes that the prover is a liar and cares about having a strong enough verifier that can stand up to any type of prover and not be misled.  Finally, Zero Knowledge assumes that the verifier is hoping to glean information from the proof to learn the prover's secrets and this requirement makes sure the prover is clever enough that it gives no information away in its proof. Unlike the soundness' requirment for a verifier to combat \textit{all} malicious provers, HVZK is only concerned with the verifier in the original prover-verifier pair that follows the set protocol. Verifiers that stray from the protocol or cheat, however, are captured in the natural generalization to Zero Knowledge proofs.

\section{Graph Isomorphism}

In this section, we construct our final zero-knowledge interactive proof system for GI where we don't have to assume an honest verifier for zero knowledge to hold. The proof system construction is exactly the same as the one we saw earlier. What changes is the definition of zero knowledge, and therefore, the simulator. 

\begin{definition} {\normalfont\textbf{(Zero Knowledge Proof [ZK])}} 
	$(\mathcal{P},\mathcal{V})$ is a (perfect) ZK proof system for a language $L$ w.r.t. witness relation $R$ if $\forall$ PPT machines $\mathcal{V}^*$,
	$\exists$ a PPT machine $\mathcal{S}$ (called the simulator) s.t. $\forall x \in L$, $\forall w\in R(x)$, the following distributions are (identical) indistinguishable:
$$\{View_{\mathcal{V^*}}(\mathcal{P}(x,w) \leftrightarrow \mathcal{V^*}(x))\} \approx \{\mathcal{S}(x)\}$$
where $View_{\mathcal{V^*}}(\mathcal{P}(x,w) \leftrightarrow \mathcal{V^*}(x))$ is the random coins of $\mathcal{V^*}$ and all the messages $\mathcal{V^*}$ saw.
\end{definition}
\begin{remark}
	Note that the order of quantifiers matters again. The definition would be stronger if we switch the order to: $\exists$ a PPT machine $\mathcal{S}$ (called the simulator) s.t. $\forall$ PPT machines $\mathcal{V}^*$. This is because the same simulator would need to work for all possible efficient verifiers.
\end{remark}

TODO: HERE 
FIRST: WHY HVZK simulator doesn't work.
Recall our protocol for graph isomorphism: the interaction is $P(x,w) \leftrightarrow V(x)$ where $x$ represents graphs $G_0 = (V, E_0)$ and $G_1 = (V, E_1)$ and $w$ represents a permutation $\pi$ on $V$ such that $\pi (G_0) = G_1$.

\begin{enumerate}
\item $P$ samples a random permutation $\sigma: V \to V$ and sends the graph $H = \sigma(G_1)$ to $V$.

\item $V$ samples a random bit $b$ and sends it to $P$.

\item If $b = 1$, then $P$ defines a permutation $\tau$ to be $\sigma$. If $b = 0$, then instead $\tau = \sigma \circ \pi$. $P$ then sends $\tau$ to $V$.

\item $V$ verifies that $\tau(G_b) = H$ and accepts if so.

\end{enumerate}

We will show that this is an efficient prover zero-knowledge proof system. It is clear that if $G_0$ and $G_1$ are isomorphic, then this protocol will succeed with probability 1.

For soundness, observe that if $G_0$ is not isomorphic to $G_1$, then the graph $H$ that $P$ sends to $V$ in step 1 of the protocol can be isomorphic to at most one of $G_0$ or $G_1$. Since $V$ samples a bit $b$ uniformly at random in step 2, then there is a probability of at most 1/2 that $P$ can produce a valid isomorphism in step 3.

For zero knowledge, consider the following simulator $S$ with input $G_0$ and $G_1$ (with vertex set $V$) and verifier $V^*$:

\begin{enumerate}
\item Guess a bit $b$ uniformly at random.

\item Sample a permutation $\pi: V \to V$ uniformly at random and send $\pi (G_b)$ to $V^*$.

\item Receive $b'$ form $V^*$.

\item If $b=b'$, then output $(\pi (G_b), b, \pi)$ and terminate. Otherwise, restart at step 1.

\end{enumerate}

Note that if $G_0 \simeq G_1$, then $\pi(G_b)$ is statistically independent of $b$ because $b$ and $\pi$ are sampled uniformly. Thus, with probability 1/2, $V^*$ will output $b$ so on average, two attempts will be needed before $S$ terminates. It follows that $S$ will terminate in \emph{expected} polynomial time.

Since $b$ is sampled uniformly at random, $\pi (G_b)$ is uniformly distributed with all graphs of the form $\sigma (G_1)$ where $\sigma$ is sampled uniformly at random from permutations on $V$. Thus, the output $\pi(G_b)$ in our simulator will be identically distributed with the output $H$ in our graph isomorphism protocol.

In step 3 of our graph isomorphism protocol, note that $\tau$ is distributed uniformly at random. This is because composing a uniformly random permutation with a fixed permutation will not change its distribution. Thus $\tau$ will be identically distributed with $\pi$ in our simulator. It follows that the transcripts outputted by our simulator will be identically distributed with the transcripts produced by the graph isomorphism protocol.
		
		  
%		These are mostly discussed (including auxiliary inputs) in the next class, although the first definition is given below:
%		
%		{\definition {\normalfont\textbf{(Zero Knowledge Proof [ZK])}} For a language L we have a (perfect) \textit{ZK proof system} w.r.t. witness relation $R$ if $\exists$ an interactive proof system, $(\mathcal{P},\mathcal{V})$ s.t. $\exists$ a PPT machine $\mathcal{S}$ (called the simulator) s.t. $\forall x \in L$, $\forall w\in R(x)$, $\forall \mathcal{V}^*$, the following distributions are identical:
%		$$View_{\mathcal{V}^*}(\mathcal{P}(x,w) \leftarrow \mathcal{V}^*(x))$$
%		$$\mathcal{S}^{\mathcal{V}^*}(x)$$
%		where $\mathcal{S}^{\mathcal{V}^*}(x)$ is the simulator with oracle access to $\mathcal{V}^*$.}


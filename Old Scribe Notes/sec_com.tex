% !TEX root = collection.tex

%\RestyleAlgo{boxruled}
%\floatstyle{boxed}
%\restylefloat{figure}
%
%\DeclarePairedDelimiter{\floor}{\lfloor}{\rfloor}
%\DeclarePairedDelimiter{\parens}{\(}{\)}
%\DeclarePairedDelimiter{\ceil}{\lceil}{\rceil}

\section{Yao's Two Party Computation Protocol}

Yao's Two Party Protocol is a protocol conducted between two parties for
computing any function. It is obtained by combining two primitives: a scheme
for garbling circuits and oblivious transfer. Informally, the idea is that one
party (say $\PartyA$) \emph{garbles} a circuit. This involves assigning random
labels to each wire of the circuit, including the input and output wires.
$\PartyA$ then sends the garbled circuit and the labels corresponding to its
input wires to $\PartyB$. The two parties then engage in an oblivious
transfer protocol to transfer the labels corresponding to $\PartyB$'s inputs to
$\PartyB$. $\PartyB$ then evaluates the garbled circuit using the two sets of
input labels to get the result of the computation.

\paragraph{Problem Setup:}
Let  $G$ and $W$ be the gates and wires respectively in $\mathcal{U}$, the universal circuit.
Let $w_i$ be the $i^{th}$ input wire, $i \in [n]$, and $w_\out$ be the output wire.

\begin{definition}[Garbling Scheme]
  A \emph{garbling scheme} is a pair of ppt.\ algorithms $(\Garble, \Eval)$:
  \begin{itemize}
    \item
      $\Garble(1^\Param, \Cir) \rightarrow \left(\GCir, \set{\Lab_{i, b_i}}_{i\in
      [n], b_i \in \bits}\right)$. The circuit $\Cir$ has $n$ input wires and one
      output wire.

    \item $\Eval(1^\Param, \GCir, \set{\Lab_{i, x_i}}_{i \in [n]}) \rightarrow y$.
  \end{itemize}
  It satisfies the following two properties:
  \begin{description}
    \item[Correctness:] $\forall\ C, x$, we have that
      \begin{equation*}\prob{(\GCir, \Lab_{i, b_i}) \gets \Garble(1^\Param, \Cir) \logAnd
      y \gets \Eval(1^\Param, \GCir,\Lab_{i,x_i}) \logAnd
      y = \Cir(x)} = 1\end{equation*}
    \item[Security:]
      $\exists$ simulator $\IdealAdv$ s.t.\ $\forall\ \Cir, x$, we have that
      \begin{equation*}
        (\GCir, \set{\Lab_{i, x_i}}) \indis \IdealAdv(1^\Param, C(x))
      \end{equation*}
  \end{description}
\end{definition}

\subsection{Construction:}
  We construct a garbling scheme:
  \begin{figure}[H]
    \setlength{\parskip}{\baselineskip}%
    \setlist{noitemsep,topsep=0pt,parsep=0pt,partopsep=0pt,leftmargin=*}
    \begin{minipage}[t]{\textwidth}
    $\Garble(1^\Param, \Cir)$:
    \begin{enumerate}
      \item
        For each $w \in W$, sample $(k_w^0, k_w^1)$ as encryption keys.
      \item
        For each $g \in G$ with input wires $w_0, w_1$, and output wire $w_2$, set
      \begin{equation*}e_g \Assign {(\Enc_{k_{w_0}^a}(\Enc_{k_{w_1}^b}(0^\Param || k_{w_2}^{g(a,b)})))}_{a,b \in \bits}\end{equation*}\vspace{-0.3cm}
      \item
        Set $\GCir \Assign ({(e_g)}_{g \in G}, k_{w_\out}^0 \rightarrow 0, k_{w_\out}^1 \rightarrow 1)$.
      \item
        Set $\Lab_{i, b_i} \Assign k_{w_i}^{b_i}$.
      \item
        Output $(\GCir, \Lab_{i, b_i})$.
    \end{enumerate}
    \end{minipage}

    \begin{minipage}[t]{\textwidth}
    $\Eval(1^\Param, \GCir, \set{\Lab_{i, x_i}}_{i \in [n]})$:
    \begin{enumerate}
      \item
        For each gate $g \in G$, obtain $\Lab_{w_2} \gets \Dec_{\Lab_{w_0}}(\Dec_{\Lab_{w_1}}(e_g))$.
    \end{enumerate}
    \end{minipage}
    \caption{Definition of a garbling scheme}
    \label{alg:garbling-scheme}
  \end{figure}

\subsection{Proof of Security}
\proof
  We construct a ppt.\ simulator $\Sim$ such that $\forall x \in \bits^n$:
  \begin{equation*}
    \set{\GCir, \Lab_{i, x_i}} \indis \set{\Sim(1^\Param, \Cir(x))}
  \end{equation*}

  \begin{figure}[H]
    \setlist{noitemsep,topsep=0pt,parsep=0pt,partopsep=0pt,leftmargin=*}
    \begin{minipage}[t]{\textwidth}
      $\Sim(1^\Param, \Cir(x))$:
      \begin{enumerate}
        \item
          Sample random $k_w, k_w'$ $\forall w \in W$.
        \item
          For each $g \in G$ with input wires $w_0, w_1$ and output wire $w_2$, set $e_g$ as
          \begin{itemize}
            \item
              $\Enc_{k_{w_0}}(\Enc_{k_{w_1}}(0^\Param||k_{w_2}))$
            \item
              $\Enc_{k_{w_0}}(\Enc_{k_{w_1}'}(0^\Param||k_{w_2}))$
            \item
              $\Enc_{k_{w_0}'}(\Enc_{k_{w_1}}(0^\Param||k_{w_2}))$
            \item
              $\Enc_{k_{w_0}'}(\Enc_{k_{w_1}'}(0^\Param||k_{w_2}))$
          \end{itemize}
        \item
          Output $\GCir \Assign ({(e_g)}_{g \in G}, k_{w_\out} \rightarrow \Cir(x), k_{w_\out}' \rightarrow 1-\Cir(x))$.
      \end{enumerate}
    \end{minipage}
    \caption{Simulator for security of garbled circuits}
    \label{alg:sim}
  \end{figure}

  We prove that the output of this simulator is indistinguishable from the actual
  view of the circuit evaluator via a series of hybrids:
  \begin{equation*}
    H_0 = \set{\GCir, \Lab_{i, x_i} } \indis H_1 \indis \cdots \indis H_{T-1}
    \indis \set{\Sim(1^\Param, \Cir(x))} = H_T
  \end{equation*}
  We proceed by replacing $e_g$'s gate by gate. The idea is to replace the three
  non-opened entries with encryptions of the only wire which is correct. Say
  the input labels are $k_{w_0}^0, k_{w_1}^1$. Originally, $e_g$ consists of
  \begin{align*}
    \Enc_{k_{w_0}^0}(\Enc_{k_{w_1}^0}&{}(0^\Param||k_{w_2}^{g(0,0)}))\\
    \Enc_{k_{w_0}^0}(\Enc_{k_{w_1}^1}&{}(0^\Param||k_{w_2}^{g(0,1)}))\\
                                     &{}\vdots
  \end{align*}
  For each $a,b \in \bits$, we replace the encrypted values with
  $\Enc_{k_{w_0}^a}(\Enc_{k_{w_1}^b}(0^\Param||k_{w_2}^{g(0,1)}))$. By the semantic security
  of the encryption scheme, this new $e_g'$ is indistinguishable from the original
  $e_g$. Hence each hybrid is indistinguishable from the previous one.
  \qed

\section{GMW Protocol}
  Yao's protocol is limited to two party computation. Goldreich, Micali and Wigderson
  (GMW) created the first secure multiparty computation protocol. Here we give an
  informal sketch of the protocol, limiting ourselves to two parties for simplicity
  of exposition.

  Let $\PartyA$'s input be $\InputA$, and $\PartyB$'s input be $\InputB$. Each
  party creates a secret share of their input and sends it to the other party.
  For example, $\PartyA$ samples a random $a$, and computes $b_1 = a_1 \xor \InputA$.
  $\PartyB$ does the same for $\InputB$. $\PartyA$ gets the $a$ shares, and
  $\PartyB$ gets the $b$ shares. Computing $\NOT,\allowbreak \XOR$, and $\AND$
  gates is done as follows:
  \begin{itemize}%[leftmargin=*,nolistsep]
    \item $\NOT$ gates: Each party simply flips their share of the input bit.

    \item $\XOR$ gates: Since $\InputA \xor \InputB = (a_1 \xor b_1) \xor (a_2
      \xor b_2) = (a_1 \xor a_2) \xor (b_1 \xor b_2)$, each party can simply
      XOR their shares separately.

    \item $\AND$ gates: $(a_1 \xor b_1)\cdot (a_2 \xor b_2) = ((a_1 \cdot a_2) \xor (b_1 \cdot b_2))
      \xor ((a_1\cdot b_2) \xor (a_2 \cdot b_1))$. We see that each party can
      compute the first parts with the shares they possess. However, to get the remaining,
      they need to know the other party's shares, which compromises security.
      So instead, the parties utilize 1-out-of-4 oblivious transfer to compute the
      $\AND$ gate.

      The setup is as follows: $\PartyA$ samples a random bit $a$. This its share
      of the result of the gate. It computes the following table:
      \begin{center}
        \begin{tabular}{lcc}
          \hline
          $b_1$ & $b_2$ & $b_3$\\
          \hline
          0   &   0   & $(a_1\cdot a_2)\xor a$\\
          0   &   1   & $(a_1\cdot \neg a_2)\xor a$\\
          1   &   0   & $(\neg a_1\cdot a_2)\xor a$\\
          1   &   1   & $(\neg a_1\cdot \neg a_2)\xor a$\\
          \hline
        \end{tabular}
      \end{center}
      The intuition for the entries in the table is that when $b_1 = b_2 = 1$,
      the $\AND$ gate becomes $(a_1 \xor b_1)\cdot (a_2 \xor b_2)  = (a_1 \xor 1) \cdot (a_2 \xor 1)
      = (\neg a_1)\cdot (\neg a_2)$. The other entries are computed similarly.

      $\PartyA$ computes each of the possible values of $b_3$, and feeds them as input
      to the OT.\ $\PartyB$ feeds in $(b_1, b_2)$ to the OT, and gets some secret
      share. Thus both parties possess a share of the correct output.
  \end{itemize}

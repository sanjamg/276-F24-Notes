% !TEX root = collection.tex

\newcommand{\extline}{$\scriptsize$-$\normalsize$\!}
\newcommand{\lextlineend}{$\scriptsize$\lhd\!$\normalsize$}
\newcommand{\rextlineend}{$\scriptsize\rule{.1ex}{0ex}$\rhd$\normalsize$}

\newcounter{index}

\newcommand\extlines[1]{%
  \setcounter{index}{0}%
  \whiledo {\value{index}< #1}
  {\addtocounter{index}{1}\extline}
}

\newcommand\rextlinearrow[2]{$
  \setbox0\hbox{$\extlines{#2}\rextlineend$}%
  \tiny$%
  \!\!\!\!\begin{array}{c}%
  \mathrm{#1}\\%
  \usebox0%
  \end{array}%
  $\normalsize$\!\!%
}

\newcommand\lextlinearrow[2]{$
  \setbox0\hbox{$\lextlineend\extlines{#2}$}%
  \tiny%
  $%
  \!\!\!\!\begin{array}{c}%
  \mathrm{#1}\\%
  \usebox0%
  \end{array}%
  $\normalsize$\!\!%
}

\renewcommand\lextlinearrow[2]{%
}

\renewcommand\rextlinearrow[2]{%
}
\renewcommand\lextlinearrow[2]{%
%  \setbox0\hbox{$\lextlineend\extlines{#2}$}%
   $\stackrel{\mathrm{#1}}{\leftarrow}$%
}

\renewcommand\rextlinearrow[2]{%
  %\setbox0\hbox{$\extlines{#2}\rextlineend$}%
  $\stackrel{\mathrm{#1}}{\rightarrow}$%
}



%\section{Indistinguishable Obfuscation Constructions using Puncturing}
\section{Digital Signature Scheme via Indistinguishable Obfuscation}
A digital signature scheme can be constructed via indistinguishable obfuscation (iO).  A digital signature scheme is made up of $(\SETUP, \SIGN, \VERIFY)$.\\

\newcommand{\vk}{\mathsf{vk}}

\noindent $(\vk, \sk) \leftarrow \SETUP(1^k)$:\\
\indent $\sk$ = key of puncturable function and the seed of the PRF $F_k$\\
\indent $\vk = iO(P_k)$ where $P_k$ is the program:\\
\indent \indent $P_k(m, \sigma)$:\\
\indent \indent \indent for some OWF function $f$\\
\indent \indent \indent \indent return 1 if $f(\sigma) = f(F_k(m))$\\
\indent \indent \indent \indent return 0 otherwise\\

\noindent $\sigma \leftarrow \SIGN(\sk, m)$: Output $F_k(m)$.\\

\noindent $\VERIFY(\vk, m, \sigma)$: Output $P_k(m, \sigma)$.\\

\noindent Our security requirements will be that the adversary does wins the following game negligibly:\\

\begin{tabular}{llc}
{\large Challenger} & & {\large Adversary}\\
$(\vk, \sk) = \SETUP(1^k)$ and&&\\
picks random $m$&&\\
& \rextlinearrow{P_{k},m}{46} &\\
& \lextlinearrow{\sigma, m^*}{46} &\\
& Adversary wins game if $\VERIFY(\vk, m^*, \sigma) = 1$&
\end{tabular}\\\\

\noindent To prove the security of this system, we use a hybrid argument.  $H_0$ is as above.

\noindent $H_1$: Adjust $\vk$ so that $\vk = iO(P_{k, m, \alpha})$ where $\alpha = F_k(m)$ and $P_{k, m, \alpha}$ is the program such that:\\
\indent $P_{k,m, \alpha}(m^*, \sigma)$:\\
\indent \indent for some OWF $f$\\
\indent \indent \indent if $m = m^*$:\\
\indent \indent \indent \indent if $f(\sigma) = f(\alpha)$ return 1\\
\indent \indent \indent \indent otherwise return 0\\
\indent \indent \indent else proceed as $P_{k}$ from before\\
\indent \indent \indent \indent if $f(\sigma) = f(F_k(m^*))$ return 1\\
\indent \indent \indent \indent otherwise return 0\\
\noindent Note that this program does not change its output for any value. This is indistinguishable from $H_0$  by indistinguishability obfuscation.\\

\noindent $H_2$: Adjust $\alpha$ so that it is a randomly sampled value. The indistinguishability of $H_2$ and $H_1$ follows from the security of PRG.  \\
\noindent $H_3$: Adjust the program such that instead of $\alpha$ it relies on some $\beta$ that is compared instead $f(\alpha)$ in the third line.\\

Any adversary that can break $H_3$ non-negligibly can break the OWF $f$ with at the value $\beta$.

\section{Public Key Encryption via Indistinguishable Obfuscation}
A public key encryption scheme can be constructed via indistinguishable obfuscation.  A public key encryption scheme is made up of $(Gen, Enc, Dec)$.  The PRG used below is a length doubling PRG.\\

\noindent $(\pk, \sk) \leftarrow Gen(1^k)$\\
\indent $\sk$ = key of puncturable function and the seed of the PRF $F_k$\\
\indent $\pk = iO(P_k)$ where $P_k$ is the program:\\
\indent \indent $P_k(m, r)$:\\
\indent \indent \indent $t = PRG(r)$\\
\indent \indent \indent Output $c = (t, F_k(t) \oplus m)$\\

\noindent $Enc(\pk, m)$: Sample $r$ and output $(\pk(m,r))$.\\

\noindent $Dec(\sk = k, c = (c_1, c_2))$: Output $F_k(\sk,c_1) \oplus c_2$.\\

\noindent Our security requirements will be that the adversary does wins the following game negligibly:\\

\begin{tabular}{llc}
{\large Challenger} & & {\large Adversary}\\
$(\pk, \sk) = Gen(1^k)$ and&&\\
Randomly sample $b$ from $\{0,1\}$ and&&\\
$c^* = Enc(\pk, b)$ and&&\\
& \rextlinearrow{P_{k},c^*}{26} &\\
& \lextlinearrow{b^*}{26} &\\
& Adversary wins game if $b=b^*$&
\end{tabular}\\\\

\noindent To prove the security of this system, we use a hybrid argument.  $H_0$ is as above.

\noindent $H_1$: Adjust $\pk$ so that $\pk = iO(P_{k, \alpha, t})$ where $\alpha = F_k(t)$ and $P_{k, \alpha, t}$ is the program such that:\\
\indent $P_{k, \alpha, t}(m, r)$:\\
\indent \indent $t^* = PRG(r)$\\
\indent \indent if $t^* = t$, output $(t^*, \alpha \oplus m)$\\
\indent \indent else output $(t^*, F_k(t^*) \oplus m)$\\

\noindent Note that this program does not change its output for any value. This is indistinguishable from $H_0$  by indistinguishability obfuscation.\\

\noindent $H_2$: Adjust $\alpha$ so that it is a randomly sampled value.\\
\noindent $H_3$: Adjust the program such that $t^*$ is randomly sampled and is not in the range of the PRG.\\

Any adversary that can win $H_3$ can guess a random value non-negligibly.\\

\section{Indistinguishable Obfuscation Construction from $NC^1$ $iO$}
A construction of indistinguishable obfuscation from $iO$ for circuits in $NC^1$ is as follows:\\
Let $P_{k,C}(x)$ be the circuit that outputs the garbled circuit $\widetilde{UC(C,x)}$ with randomness $F_k(x)$ which is a punctured (at $k$) PRF in $NC^1$\\
\indent Note that $UC(C,x)$ outputs $C(x)$ ($UC$ is the ``universal'' circuit)\\
$iO(C) \rightarrow $ sample $k$ randomly from $\{0,1\}^{|x|}$ and output $iO_{NC^1}(P_{k,C})$ padded to a length $l$\\

As before, we use a hybrid argument to show the security for $iO$.\\
\noindent $H_0$: $iO(C) = iO_{NC^1}(P_{k,C})$ as above.\\
\noindent $H_{final} = H_{2^n}$: $iO(\pk, c_2)$\\
\noindent $H_1 \cdots H_i$: Create a program $Q_{k, c_1, c_2, i}(x)$ and obfuscate it.\\
$Q_{k,c_1,c_2,i}(x)$:\\
\indent Sample $k$ randomly\\
\indent if $x \ge i$, return $P_{k,c_1}(x)$\\
\indent else , return $P_{k,c_2}(x)$\\\\
\noindent Note that $H_i$ and $H_{i+1}$ are indistinguishable for any value other than $x=i$.\\
\noindent $H_{i,1}$ (between $H_i$ and $H_{i+1}$): Create a program $Q_{k, c_1, c_2, i, \alpha}(x)$, where $\alpha = Q_{k, c_1, c_2, i}(x)$ and obfuscate it.\\
$Q_{k, c_1, c_2, i, \alpha}(x)$:\\
\indent Sample $k$ randomly\\
\indent if $x = i$, return $\alpha$\\
\indent else , return $Q_{k,c_1,c_2, i}(x)$\\

\noindent $H_{i,2}$: Replace $\alpha$ with a random $\beta$ using fresh coins\\
\noindent $H_{i,3}$: Create the $c_2(x)$ value using fresh coins\\
\noindent $H_{i,4}$: Create the $c_2(x)$ value using $F_k(x)$\\
\noindent $H_{i,5}$: Finish the migration to $Q_{k,c_1,c_2,i+1}$\\

Note that at $H_{final}$, the circuit being obfuscated is completely changed from $c_1$ to $c_2$.




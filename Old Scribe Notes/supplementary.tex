
\newcommand{\bin}{\{0,1\}}
\newcommand{\adv}{\mathcal{A}}
\newcommand{\advb}{\mathcal{B}}
\newcommand{\advc}{\mathcal{C}}
%\usepackage[utf8]{inputenc}
%\usepackage{amsmath,amssymb,fullpage}


\section{Examples}

The goal of this section is to illustrate the general strategy for the problems of the form,
\begin{center}
\textit{``If $f$ is one-way function, then show that $f'$ (derived from $f$) is not a one-way function"}
\end{center}
Some of the examples include:
\begin{itemize}
\item If $f$ is a one-way function, prove that $f'$ defined as $f(f(\cdot))$ is not one-way.
\item If $f$ is a one-way function, prove that $f'$ defined by dropping the first bit the output of $f$ is not one-way.
\end{itemize}

In order to give such a proof, we need to give an example of an one-way function $f$ and show that $f'$ (derived from $f$) is not one-way. The general strategy for these types of problems is the following:
\begin{enumerate}
\item Come up with a contrived function $g$ and show that $g$ is one-way. 
\item Construct the new function $g'$ that is derived from $g$.
\item Show that $g'$ can be inverted with non-negligible probability and thus show that $g'$ is not one-way.
\end{enumerate}
The reason why we need to come-up with a contrived function is that for specific one-way function $f$, $f'$ (derived from $f$) could be one-way. To see why this is the case, consider a one-way function $f: \bin^n \rightarrow \bin^n$ that is additionally injective. Then, one can show that $f^2(\cdot)$ is in fact a one-way function.\footnote{Try to prove this!} On the other hand, in the previous section, we showed that there exists a (contrived) function $g$ such that $g$ is one-way but $g^2$ is not one-way.
Hence, we might not always be able to start from any one-way function $f$ and show that $f'$ (derived from $f$) is not one-way. The first step where we come up a suitable $g$ requires some ingenuity. Once that is done, the second and the third steps would generally be not so hard.

To illustrate these three steps, let us consider a concrete example. We want to show that if $f$ is one-way then $f'$ that is defined by dropping the first bit of the output of $f$ is not one-way.


\paragraph{Step-1: Designing the function $g$.} We want to come up with a (contrived) function $g$ and prove that it is one-way.
Let us assume that there exists a one-way function $h : \bin^n \rightarrow \bin^n$. We define the function $g : \bin^{2n} \rightarrow \bin^{2n}$ as follows:
$$
g(x\|y) = \begin{cases}
 0^{n}\|y &\text{    if } x = 0^n\\
1\|0^{n-1}\|g(y) &\text{    otherwise }
\end{cases}
$$
\begin{claim}
If $h$ is a one-way function, then so is $g$.
\end{claim}
\begin{proof}
Assume for the sake of contradiction that $g$ is not one-way. Then there exists a polynomial time adversary $\adv$ and a non-negligible function $\mu(\cdot)$ such that:
$$
\Pr_{x,y}[\adv(1^n,g(x\|y)) \in g^{-1}(g(x\|y))] = \mu(n)
$$
We will use such an adversary $\adv$ to invert $h$ with some non-negligible probability. This contradicts the one-wayness of $h$ and thus our assumption that $g$ is not one-way function is false.

Let us now construct an $\advb$ that uses $\adv$ and inverts $h$. $\advb$ is given $1^n,h(y)$ for a randomly chosen $y$ and its goal is to output $y' \in h^{-1}(h(y))$ with some non-negligible probability. $\advb$ works as follows:
\begin{enumerate}
\item It samples $x \gets \bin^n$ randomly.
\item If $x = 0^n$, it samples a random $y' \gets \bin^n$ and outputs it.
\item Otherwise, it runs $\adv(10^{n-1}\|h(y))$ and obtains $x' \| y'$. It outputs $y'$.
\end{enumerate}

Let us first analyze the running time of $\advb$. The first two steps are clearly polynomial (in $n$) time. In the third step, $\advb$ runs $\adv$ and uses its output. Note that the running time of since $\adv$ runs in polynomial (in $n$) time, this step also takes polynomial (in $n$) time. Thus, the overall running time of $\advb$ is polynomial (in $n$).

Let us now calculate the probability that $\advb$ outputs the correct inverse. If $x = 0^n$, the probability that $y'$ is the correct inverse is at least $\frac{1}{2^n}$ (because it guesses $y'$ randomly and probability that a random $y'$ is the correct inverse is $\geq 1/2^n$). On the other hand, if $x \neq 0^n$, then the probability that $\advb$ outputs the correct inverse is $\mu(n)$. Thus,
\begin{eqnarray*}
\Pr[\advb(1^n,h(y)) \in h^{-1}(h(y))] & \geq & \Pr[x = 0^n](\frac{1}{2^n}) + \Pr[x \neq 0^n]\mu(n)\\
& = & \frac{1}{2^{2n}} + (1 - \frac{1}{2^n}) \mu(n) \\
& \geq & \mu(n) - (\frac{1}{2^{n}} - \frac{1}{2^{2n}})
\end{eqnarray*}

Since $\mu(n)$ is a non-negligible function and $(\frac{1}{2^{n}} - \frac{1}{2^{2n}})$ is a negligible function, their difference is non-negligible.\footnote{Exercise: Prove that if $\alpha(\cdot)$ is a non-negligible function and $\beta(\cdot)$ is a negligible function, then $(\alpha - \beta)(\cdot)$ is a non-negligible function.} This contradicts the one-wayness of $h$.

\end{proof}

\paragraph{Step-2: Constructing $g'$.} We construct the new function $g': \bin^{2n} \rightarrow \bin^{2n-1}$ by dropping the first bit of $g$. That is,
$$
g'(x\|y) = \begin{cases}
 0^{n-1}\|y &\text{    if } x = 0^n\\
0^{n-1}\|g(y) &\text{    otherwise }
\end{cases}
$$

\paragraph{Step-3: Inverting $g'$.} We now want to prove that $g'$ is not one-way. That is, we want to design an adversary $\advc$ such that given $1^{2n}$ and $g'(x \| y)$ for a randomly chosen $x,y$, it outputs an element in the set $g^{-1}(g(x \| y)$. The description of $\advc$ is as follows:

\begin{itemize}
\item On input $1^{2n}$ and $g'(x \| y)$, the adversary $\advc$ parses $g'(x \| y)$ as $0^{n-1} \| \overline{y}$.
\item It outputs $0^{n} \| \overline{y}$ as the inverse.
\end{itemize}
Notice that $g'(0^{n} \| \overline{y}) = 0^{n-1} \| \overline{y}$. Thus, $\advc$ succeeds with probability $1$ and this breaks the one-wayness of $g'$.


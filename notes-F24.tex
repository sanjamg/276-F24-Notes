\documentclass[12pt]{tufte-book}
\usepackage{amsthm,amssymb,amsmath,thmtools,datetime,tikz}
\setcounter{secnumdepth}{3}

\declaretheorem[numberwithin=chapter,shaded={bgcolor=Lavender}]{definition}

\declaretheorem[numberwithin=chapter,shaded={bgcolor=Thistle}]{lemma}
\declaretheorem[numberwithin=chapter,shaded={bgcolor=Thistle}]{claim}

\declaretheorem[numberwithin=chapter,shaded={bgcolor=Apricot}]{theorem}

\declaretheorem[numberwithin=chapter,shaded={bgcolor=yellow}]{remark}
\declaretheorem[numberwithin=chapter]{exercise}
\declaretheorem[numberwithin=chapter,shaded={bgcolor=pink}]{construction}
\usepackage[
    type={CC},
    modifier={by-nc-nd},
    version={4.0},
]{doclicense}

\usepackage{graphicx,xcolor,mdframed}
%\usepackage[version=0.96]{pgf}
\usepackage{enumitem}


\def\chpcolor{blue!45}
\def\chpcolortxt{blue!60}

\iffalse
\titleformat{\chapter}%
  {\huge\rmfamily\itshape\color{red}}% format applied to label+text
  {\llap{\colorbox{red}{\parbox{1.5cm}{\hfill\itshape\huge\color{white}\thechapter}}}}% label
  {2pt}% horizontal separation between label and title body
  {}% before the title body
  []% after the title body
\fi

\hypersetup{colorlinks}% uncomment this line if you prefer colored hyperlinks (e.g., for onscreen viewing)

%%
% Book metadata
\title{A Course in Theory of Cryptography}
\author[Sanjam Garg]{Sanjam Garg}
%\publisher{Publisher of This Book}

%%
% If they're installed, use Bergamo and Chantilly from www.fontsite.com.
% They're clones of Bembo and Gill Sans, respectively.
%\IfFileExists{bergamo.sty}{\usepackage[osf]{bergamo}}{}% Bembo
%\IfFileExists{chantill.sty}{\usepackage{chantill}}{}% Gill Sans

%\usepackage{microtype}

%%
% Just some sample text
\usepackage{lipsum}
\input{macros}
%%
% For nicely typeset tabular material
\usepackage{booktabs}

\usepackage[n,advantage,operators,sets,adversary,landau,probability,notions,logic,ff,mm,primitives,events,complexity,oracles,asymptotics,keys]{cryptocode} 
%%
% For graphics / images
\usepackage{graphicx,algpseudocode}
\setkeys{Gin}{width=\linewidth,totalheight=\textheight,keepaspectratio}
\graphicspath{{graphics/}}

% The fancyvrb package lets us customize the formatting of verbatim
% environments.  We use a slightly smaller font.
\usepackage{fancyvrb}
\fvset{fontsize=\normalsize}

%%
% Prints argument within hanging parentheses (i.e., parentheses that take
% up no horizontal space).  Useful in tabular environments.
\newcommand{\hangp}[1]{\makebox[0pt][r]{(}#1\makebox[0pt][l]{)}}

%%
% Prints an asterisk that takes up no horizontal space.
% Useful in tabular environments.
\newcommand{\hangstar}{\makebox[0pt][l]{*}}

%%
% Prints a trailing space in a smart way.
\usepackage{xspace}

%%
% Some shortcuts for Tufte's book titles.  The lowercase commands will
% produce the initials of the book title in italics.  The all-caps commands
% will print out the full title of the book in italics.
\newcommand{\vdqi}{\textit{VDQI}\xspace}
\newcommand{\ei}{\textit{EI}\xspace}
\newcommand{\ve}{\textit{VE}\xspace}
\newcommand{\be}{\textit{BE}\xspace}
\newcommand{\VDQI}{\textit{The Visual Display of Quantitative Information}\xspace}
\newcommand{\EI}{\textit{Envisioning Information}\xspace}
\newcommand{\VE}{\textit{Visual Explanations}\xspace}
\newcommand{\BE}{\textit{Beautiful Evidence}\xspace}

\newcommand{\TL}{Tufte-\LaTeX\xspace}

% Prints the month name (e.g., January) and the year (e.g., 2008)
\newcommand{\monthyear}{%
  \ifcase\month\or January\or February\or March\or April\or May\or June\or
  July\or August\or September\or October\or November\or
  December\fi\space\number\year
}


% Prints an epigraph and speaker in sans serif, all-caps type.
\newcommand{\openepigraph}[2]{%
  %\sffamily\fontsize{14}{16}\selectfont
  \begin{fullwidth}
  \sffamily\large
  \begin{doublespace}
  \noindent\allcaps{#1}\\% epigraph
  \noindent\allcaps{#2}% author
  \end{doublespace}
  \end{fullwidth}
}

% Inserts a blank page
\newcommand{\blankpage}{\newpage\hbox{}\thispagestyle{empty}\newpage}

\usepackage{units}

% Typesets the font size, leading, and measure in the form of 10/12x26 pc.
\newcommand{\measure}[3]{#1/#2$\times$\unit[#3]{pc}}

% Macros for typesetting the documentation
\newcommand{\hlred}[1]{\textcolor{Maroon}{#1}}% prints in red
\newcommand{\hangleft}[1]{\makebox[0pt][r]{#1}}
\newcommand{\hairsp}{\hspace{1pt}}% hair space
\newcommand{\hquad}{\hskip0.5em\relax}% half quad space
\newcommand{\TODO}{\textcolor{red}{\bf TODO!}\xspace}
\newcommand{\ie}{\textit{i.\hairsp{}e.}\xspace}
\newcommand{\eg}{\textit{e.\hairsp{}g.}\xspace}
\newcommand{\na}{\quad--}% used in tables for N/A cells
\providecommand{\XeLaTeX}{X\lower.5ex\hbox{\kern-0.15em\reflectbox{E}}\kern-0.1em\LaTeX}
\newcommand{\tXeLaTeX}{\XeLaTeX\index{XeLaTeX@\protect\XeLaTeX}}
% \index{\texttt{\textbackslash xyz}@\hangleft{\texttt{\textbackslash}}\texttt{xyz}}
\newcommand{\tuftebs}{\symbol{'134}}% a backslash in tt type in OT1/T1
\newcommand{\doccmdnoindex}[2][]{\texttt{\tuftebs#2}}% command name -- adds backslash automatically (and doesn't add cmd to the index)
\newcommand{\doccmddef}[2][]{%
  \hlred{\texttt{\tuftebs#2}}\label{cmd:#2}%
  \ifthenelse{\isempty{#1}}%
    {% add the command to the index
      \index{#2 command@\protect\hangleft{\texttt{\tuftebs}}\texttt{#2}}% command name
    }%
    {% add the command and package to the index
      \index{#2 command@\protect\hangleft{\texttt{\tuftebs}}\texttt{#2} (\texttt{#1} package)}% command name
      \index{#1 package@\texttt{#1} package}\index{packages!#1@\texttt{#1}}% package name
    }%
}% command name -- adds backslash automatically
\newcommand{\doccmd}[2][]{%
  \texttt{\tuftebs#2}%
  \ifthenelse{\isempty{#1}}%
    {% add the command to the index
      \index{#2 command@\protect\hangleft{\texttt{\tuftebs}}\texttt{#2}}% command name
    }%
    {% add the command and package to the index
      \index{#2 command@\protect\hangleft{\texttt{\tuftebs}}\texttt{#2} (\texttt{#1} package)}% command name
      \index{#1 package@\texttt{#1} package}\index{packages!#1@\texttt{#1}}% package name
    }%
}% command name -- adds backslash automatically
\newcommand{\docopt}[1]{\ensuremath{\langle}\textrm{\textit{#1}}\ensuremath{\rangle}}% optional command argument
\newcommand{\docarg}[1]{\textrm{\textit{#1}}}% (required) command argument
\newenvironment{docspec}{\begin{quotation}\ttfamily\parskip0pt\parindent0pt\ignorespaces}{\end{quotation}}% command specification environment
\newcommand{\docenv}[1]{\texttt{#1}\index{#1 environment@\texttt{#1} environment}\index{environments!#1@\texttt{#1}}}% environment name
\newcommand{\docenvdef}[1]{\hlred{\texttt{#1}}\label{env:#1}\index{#1 environment@\texttt{#1} environment}\index{environments!#1@\texttt{#1}}}% environment name
\newcommand{\docpkg}[1]{\texttt{#1}\index{#1 package@\texttt{#1} package}\index{packages!#1@\texttt{#1}}}% package name
\newcommand{\doccls}[1]{\texttt{#1}}% document class name
\newcommand{\docclsopt}[1]{\texttt{#1}\index{#1 class option@\texttt{#1} class option}\index{class options!#1@\texttt{#1}}}% document class option name
\newcommand{\docclsoptdef}[1]{\hlred{\texttt{#1}}\label{clsopt:#1}\index{#1 class option@\texttt{#1} class option}\index{class options!#1@\texttt{#1}}}% document class option name defined
\newcommand{\docmsg}[2]{\bigskip\begin{fullwidth}\noindent\ttfamily#1\end{fullwidth}\medskip\par\noindent#2}
\newcommand{\docfilehook}[2]{\texttt{#1}\index{file hooks!#2}\index{#1@\texttt{#1}}}
\newcommand{\doccounter}[1]{\texttt{#1}\index{#1 counter@\texttt{#1} counter}}

% Generates the index
\usepackage{makeidx}
\makeindex

\begin{document}
\iffalse
% Front matter
\frontmatter

% r.1 blank page
\blankpage


% v.2 epigraphs
\newpage\thispagestyle{empty}
\openepigraph{%
The public is more familiar with bad design than good design.
It is, in effect, conditioned to prefer bad design, 
because that is what it lives with. 
The new becomes threatening, the old reassuring.
}{Paul Rand%, {\itshape Design, Form, and Chaos}
}
\vfill
\openepigraph{%
A designer knows that he has achieved perfection 
not when there is nothing left to add, 
but when there is nothing left to take away.
}{Antoine de Saint-Exup\'{e}ry}
\vfill
\openepigraph{%
\ldots the designer of a new system must not only be the implementor and the first 
large-scale user; the designer should also write the first user manual\ldots 
If I had not participated fully in all these activities, 
literally hundreds of improvements would never have been made, 
because I would never have thought of them or perceived 
why they were important.
}{Donald E. Knuth}
\fi

% r.3 full title page
\maketitle


% v.4 copyright page
%\newpage
\begin{fullwidth}
~\vfill
\thispagestyle{empty}
\setlength{\parindent}{0pt}
\setlength{\parskip}{\baselineskip}
Copyright \copyright\ \the\year\ \thanklessauthor

%\par\smallcaps{Published by \thanklesspublisher}

\par\smallcaps{This document is continually being updated. Please send us your feedback.}


\par \doclicenseThis
 \index{license}

\par\textit{This draft was compiled on \today.}
\end{fullwidth}

% r.5 contents
\tableofcontents

%\listoffigures

%\listoftables

% r.7 dedication
\iffalse
\cleardoublepage
~\vfill

\begin{doublespace}
\noindent\fontsize{18}{22}\selectfont\itshape
\nohyphenation
Dedicated to those who appreciate \LaTeX{} 
and the work of \mbox{Edward R.~Tufte} 
and \mbox{Donald E.~Knuth}.
\end{doublespace}
\vfill
\vfill

% r.9 introduction
\cleardoublepage
\fi
\chapter*{Preface}
Cryptography enables many paradoxical objects, such as public key encryption, verifiable electronic signatures, zero-knowledge protocols, and fully homomorphic encryption.  The two main steps in developing such seemingly impossible primitives are (i) defining the desired security properties formally and (ii) obtaining a construction satisfying the security property provably. In modern cryptography, the second step typically assumes (unproven) computational assumptions, which are conjectured to be computationally intractable. In this course, we will define several cryptographic primitives and argue their security based on well-studied computational hardness assumptions. However, we will largely ignore the mathematics underlying the assumed computational intractability assumptions.

\section*{Acknowledgements}
These lecture notes are based on scribe notes taken by students in CS 276 over the years. Also, thanks to Peihan Miao, Akshayaram Srinivasan, and Bhaskar Roberts for helping to improve these notes.
%%
% Start the main matter (normal chapters)
\newcommand{\sanjam}[1]{{\color{red} Sanjam: #1}}

\newcommand{\bhaskar}[1]{{\color{ForestGreen} Bhaskar: #1}}

\mainmatter
\input{lec00-F24}
\input{lec01-F24}
\input{lec02-F24}
\input{lec03-F24}
\input{lec04-F24}
\input{lec05-F24}
\input{lec06-F24}
\input{lec07-F24}
\input{lec08-F24}
\input{lec09-F24}
\input{lec10-F24}
\input{lec11-F24}
\input{lec12-F24}
\input{lec13-F24}
\input{lec14-F24}
\chapter{Advanced Encryption Schemes}
\section{Identity-Based Encryption}

We introduce Bilinear Maps and two of its applications: NIKE, Non-Interactive Key Exchange; and IBE, Identity Based Encryption.


\section{Diffie-Hellman Key Exchange}

\begin{figure}
\label{fig:dh}
\centering
  \includegraphics[width=0.7\textwidth]{Old Scribe Notes/fig1.pdf}
\caption{Diffie-Hellman Key Exchange}
\end{figure}


Fig \ref{fig:dh} illustrates Diffie-Hellman key exchange. Alice and Bob each has a private key ($a$ and $b$ respectively), and they want to build a shared key for symmetric encryption communication. They can only communicate over a insecure link, which is eavesdropped by Eve.
So Alice generates a public key $A$ and Bob generates a public key $B$, and they send their public key to each other at the same time. Then Alice generates the shared key $K$ from $a$ and $B$, and likewise, Bob generates the shared key $K$ from $b$ and $A$.
And we have $\forall$ PPT Eve, $Pr[k=Eve(A,B)]=neg(k)$, where $k$ is the length of $a$.


\subsection{Discussion 1}

Assume that $\forall (g, p)$, and $a_1,b_1 \stackrel{\$}{\gets} Z^*_p$, and $a_2,b_2,r \stackrel{\$}{\gets}Z^*_p$, we have $(g^{a_1}, g^{b_1}, g^{a_1b_1}) \stackrel{c}{\simeq} (g^{a_2}, g^{b_2}, g^r)$. How to apply this to Diffie-Hellman Key Exchange?


Make $A=g^a$, $B=g^b$, $K=A^b=g^{ab}$, and $K=B^a=g^{ab}$.


\subsection{Discussion 2}


How does Diffie-Hellman Key Exchange imply Public Key Encryption?


Alice
$pk = A$, $sk = a$, $Enc(pk, m \in \{0, 1\})$.

Bob
$b,r \gets Z^*_p$
$(g^b, mA^b+(1-m)g^r)$

Alice $Dec(sk, (c_1, c_2))$

$c_1^a \stackrel{?}{=} c_2$




\section{Bilinear Maps}

\begin{definition}{Bilinear Maps}

Bilinear Maps is $(G,P,G_T,g,e)$, where $e$ is an efficient function $G \times G \to G_T$ such that

\begin{itemize}
\item if $g$ is generator of $G$, then $e(g, g)$ is the generator of $G_T$.
\item $\forall a,b \in Z_p$, we have $e(g^a, g^b) = e(g, g)^{ab} = e(g^b, g^a)$.
\end{itemize}

\end{definition}

\subsection{Discussion 1}


How does Bilinear Maps apply to Diffie-Hellman?

Make $A=g^a$, $B=g^b$, and $T=g^{ab}$, then Diffie-Hellman has $e(A, B)=e(g, T)$.


\section{Tripartite Diffie-Hellman}

\begin{figure}
\label{fig:3dh}
\centering
  \includegraphics[width=0.7\textwidth]{Old Scribe Notes/fig2.pdf}
\caption{Tripartite Diffie-Hellman Key Exchange}
\end{figure}

Fig \ref{fig:3dh} illustrates Tripartite Diffie-Hellman key exchange. $a$, $b$, and $c$ are private key of Alice, Bob, and Carol, respectively.
They use $g^a$, $g^b$, $g^c$ as public key, and the shared key $K=e(g,g)^{abc}$.
Formally, we have
$$a,b,c \stackrel{\$}{\gets} Z^*_p, r \stackrel{\$}{\gets} Z^*_p$$
$$A=g^a, B=g^b, C=g^c$$
$$K=e(g,g)^{abc}$$



\section{IBE: Identity-Based Encryption}
When two parties communicate secure messages through a public key infrastructure, they need to go through a time-consuming and error-prone process to get each other's key and verify each other's identity through a Certificate Authority. 
Identity-based cryptography (IBC) seeks to reduce these barriers by requiring no preparation on the part of the message recipient, therefore saving the initial round trip. 
Identity based encryption can also be used to construct CCA-secure public key encryption and digital signatures. 

IBE contains four steps: \emph{Setup}, \emph{KeyGen}, \emph{Enc}, and \emph{Dec}. We illustrate it in Figure \ref{fig:ibe}.
IBE relies on a trusted third party called Private Key Generator(PKG). 
In first step, PKG gets a Master Public Key ($mpk$) and Master Signing Key ($msk$) from $Gen(1^n)$. 
Then a user with an ID (in this example, ``Mike''), sends his ID to the PKG. 
The PKG generates the Signing Key of Mike with $KeyGen(msk, id)$ ans sends it back. 
Another user, Alice, wants to send an encrypted message to Mike. 
She only has $mpk$ and Mike's ID. 
So she encrypts the message with $c=Enc(mpk, id=Mike, m)$, and sends the encrypted message $c$ to Mike. 
Mike decodes $c$ with $m=Dec(c, sk_{Mike})$. 
Notice that Alice never need to know Mike's public key. 
She only needs to remember MPK and other people's IDs.

\begin{figure}
\label{fig:ibe}
\centering
  \includegraphics[width=0.7\textwidth]{Old Scribe Notes/fig3.pdf}
\caption{Identity-Based Encryption}
\end{figure}

Then we define IBE formally, 

\begin{definition}[Identity-Based Encryption]
    An \textbf{identity based encryption scheme} $\mathcal{E}_{id} = (G,K,E,D)$ is a tuple of four efficient algorithms: a \textbf{setup algorithm} $G$, a \textbf{key generation algorithm} $K$, an \textbf{encryption algorithm} $E$, and a \textbf{decryption algorithm} $D$.
    \begin{itemize}
        \item $G$ is a probabilistic algorithm invoked as $(mpk, msk)\stackrel{\$}{\gets} G(1^n)$, where $mpk$ is called the \textbf{master public key} and $msk$ is called the \textbf{master secret key} for the IBE scheme.
        \item $K$ is a probabilistic algorithm invoked as $sk_{id}\stackrel{\$}{\gets}K(msk,id )$, where $msk$ is the master secret key (as output by $S$), $id \in \mathcal{ID}$ is an identity, and $sk_{id}$ is a secret key for id.
        \item $E$ is a probabilistic algorithm invoked as $c\stackrel{\$}{\gets}E(mpk,id, m)$.
        \item $D$ is a deterministic algorithm invoked as $m\gets D(sk_{id}, c)$. Here $m$ is either a message or a special reject value $\bot$ (distinct from all messages).
    \end{itemize}
\end{definition}

As usual, we define the correctness of IBE to be the decryption undoes encryption, formally we have 
\begin{definition}[Correctness of IBE]
$\forall n, id, m$, we have
\[
Pr\begin{bmatrix}
   (mpk,msk) \gets G(1^n), \\[0.3em]
   sk_{id} \gets K(msk, ID), \\[0.3em]
   c \gets E(mpk, id, m), \\[0.3em]
   m \gets D(sk_{id}, c)
\end{bmatrix}
  =1 - \text{negl}(n)
\]
\end{definition}

Next we define the security of IBE scheme. 
The basic security definition considers an adversary who obtains the secret keys for a number of identities of its choice. 
The adversary should not be able to break semantic security for some other identity of its choice for which it does not have the secret key.

\begin{definition}[Security of IBE]
    For an IBE scheme $\mathcal{E}_{id}=(G,K,E,D)$ is secure if $\forall$ nuPPT $\mathcal{A}$,
    \[
    \Pr[\text{Exp}_{\pi,\mathcal{A}}^{IBE,CPA}(n)=1]=\text{negl}(n)
    \]
\end{definition}

We then define the experiment $\text{Exp}_{\pi,\mathcal{A}}^{IBE,CPA}$.
\begin{definition}[IBE-CPA Experiment]
We denote the experiment in the following order: 
\begin{enumerate}
    \item The challenger invokes $(mpk, msk)\stackrel{\$}{\gets} G(1^n)$ and send $mpk$ to adversary $\mathcal{A}$.
    \item $\mathcal{A}$ can make multiple key queries and generate desired ID $id^{*}$ and two message $(m_0, m_1)$. 
    \item Challenger random selects $b\stackrel{\$}{\gets}\{0,1\}$ and encrypt $c^{*}=E(mpk, id^{*},m_b)$ and send $c^{*}$ to $\mathcal{A}$.
    \item $\mathcal{A}$ can make more encryption queries based on $c^{*}$ and other id and message $(m_0,m_1)$. Note that throughout the process $\mathcal{A}$ can \textbf{never} make any query on $id^{*}$. 
    \item At the end, $\mathcal{A}$ generate $b'$ and send to challenger. 
    \item Challenger output $1$ if $b=b'$, $0$ otherwise. 
\end{enumerate}
\end{definition}

\subsection{IBE Construction from Pairing}
We then present a concrete IBE construction from pairing.
First, we will give the hardness assumption in this scheme called \textbf{bilinear Diffe-Hellman} assumption, or BDH. 
This assumption says that given random element $g_0^{\alpha},g_0^{\beta},g_0^{\gamma}\in\mathbb{G_0}$ and a few additional terms, the quantity $e(g0,g1)^{\alpha\beta\gamma}\in\mathbb{G}_T$ is computationally indistinguishable from a random element in GT.

\begin{definition}[Decisional bilinear Diffe-Hellman]
    Let $e: \mathbb{G}_0\times\mathbb{G}_1\rightarrow\mathbb{G}_T$ be a pairing where $\mathbb{G}_0,\mathbb{G}_1,\mathbb{G}_T$ are cyclic groups of prime order $q$ with generators $g_0 \in \mathbb{G}_0$ and $g_1 \in \mathbb{G}_1$. For a given adversary $\mathcal{A}$, the following distribution is distinguishable: 
    \[
    \{g_0^{\alpha},g_1^{\alpha},g_0^{\beta},g_1^{\gamma},e(g_0,g_1)^{\alpha\beta\gamma},  \alpha,\beta,\gamma\stackrel{\$}{\gets}\mathbb{Z}_q\} \approx^{c}
    \{g_0^{\alpha},g_1^{\alpha},g_0^{\beta},g_1^{\gamma},e(g_0,g_1)^{\delta},  \alpha,\beta,\gamma,\delta\stackrel{\$}{\gets}\mathbb{Z}_q\}
    \]
\end{definition}

Note that this assumption work even with $g_0=g_1$.

We then present our IBE construction. 
\begin{itemize}
    \item $G(1^n)$: 
    \[
    \alpha\stackrel{\$}{\gets}\mathbb{Z}_q, mpk\gets g^\alpha, msk\gets\alpha
    \]
    and output ($mpk,msk$)
    \item $K(msk=\alpha,id)$:
    \[
    sk_{id}\gets H(id)^\alpha
    \]
    where $H$ is a hash function $H:\{0,1\}^{*}\rightarrow\mathbb{G}$
    \item $E(mpk, id, m)$: 
    \[
    \beta\stackrel{\$}{\gets}\mathbb{Z}_q, c_1\gets g^{\beta}, c_2\gets e(mpk,H(id)^{\beta})\cdot m
    \]
    and output $(c_1,c_2)$.
    \item $D(sk_{id}, c=(c_1, c_2))$:
    \[
    m=\frac{c_2}{e(c_1,sk_{id})}
    \]
\end{itemize}
By the property of bilinear map, we can verify the correctness of this scheme, 
\begin{align*}
    m &= \frac{c_2}{e(c_1,sk_{id})} \\
      &= \frac{e(mpk,H(id)^{\beta})\cdot m}{e(c_1,sk_{id})} \\
      &= \frac{e(g^\alpha,H(id)^{\beta})\cdot m}{e(g^{\beta}, H(id)^{\alpha})} \\
      &= \frac{e(g,H(id))^{\alpha\beta}}{e(g, H(id))^{\alpha\beta}}\cdot m \\
      &= m
\end{align*}

We then prove the security property under random oracle model.
\begin{theorem}
    If decision BDH holds for e, H is modelled as a random oracle, then the above construction is a secure IBE scheme. 
\end{theorem}
\begin{proof}
    Let $\mathcal{A}$ be an adversary that breaks the IBE scheme, we can construct another adversary $\mathcal{B}$ such that it breaks the DBDH assumption.

    The adversary $\mathcal{B}$ works as follows:
    \begin{enumerate}
        \item $\mathcal{B}$ receives 5 elements as input $\{g^{\alpha},g^{\beta},g^{\gamma},z,  \alpha,\beta,\gamma\stackrel{\$}{\gets}\mathbb{Z}_q,z\in\mathbb{G}\}$, and $\mathcal{B}$ need to determine whether $z=e(g,g)^{\alpha\beta\gamma}$ or not. 
        \item $\mathcal{B}$ send IBE public parameter $mpk=g^{\alpha}$ to IBE adversary $\mathcal{A}$.
        \item Then $\mathcal{A}$ will make multiple $sk_{id}$ queries to $\mathcal{B}$. $\mathcal{B}$ responds them by (1) Choose $\rho\stackrel{\$}{\gets}\mathbb{Z}_q$ (2) Setting $H(id)=g^{\rho}$ (3) set the secret key be $sk_{id}=H(id)^{\alpha}=g^{\rho\cdot\alpha}$. One \textbf{exception} is that $\mathcal{B}$ will set a random $id'$ whose $H(id')=g^\beta$.
        \item After receiving the key query, $\mathcal{A}$ outputs $(id^{*}, m_0,m_1)$.  
        \item When $\mathcal{B}$ receives encryption query $(id^{*}, m_0, m_1)$, it first check if $\mathcal{A}$ have previously query $sk_{id^*}$ before, if yes, then abort. Otherwise, $\mathcal{B}$ choose $b\stackrel{\$}{\gets}\{0,1\}$ and encrypt $m_b$ using $(c_1=g^\gamma, c_2=e(msk,H(id')^\gamma)\cdot m_b)$ and send back to $\mathcal{A}$.
        \item $\mathcal{A}$ eventually output $b'$ to $\mathcal{B}$. And $\mathcal{B}$ output $1$ if $b'=b$ and $0$ otherwise. 
    \end{enumerate}
    Since $e(msk,H(id')^{\gamma})=e(g^\alpha,g^{\beta\gamma})=e(g,g)^{\alpha\beta\gamma}$, $\mathcal{B}$ can embedded the challenge to $\mathcal{A}$ and break DBDH. \qed
\end{proof}

\subsection{Digital Signature from IBE}
We can directly derive a secure signature scheme from IBE. 
Given a secure IBE $\mathcal{E}_{id}=(G,K,E,D)$ with id space $\mathcal{ID}$ and message space $\mathcal{M}_{IBE}$, we construct a secure digital signature scheme $\mathcal{S} = (G',S',V')$ as follows: 
\begin{itemize}
    \item $G'(1^n)$: run $(mpk, msk)\stackrel{\$}{\gets} G(1^n)$ and output $(mpk, msk)$ as the sign key pair, with $mpk$ as verification key and $msk$ as sign key.
    \item $S'(msk,m)$: Given message $m\in\mathcal{ID}$, compute $\sigma\stackrel{\$}{\gets}K(msk,m)$, output $\sigma$ as signature. 
    \item $V'(mpk,m,\sigma)$: Choose $r\stackrel{\$}{\gets}\mathcal{M}_{IBE}$, compute $c\stackrel{\$}{\gets}E(mpk,m,r)$, and accept if $D(\sigma,c) = r$. 
\end{itemize}

We have the following theorem,
\begin{theorem}
    Let $\mathcal{E}_{id}$ be a secure IBE with message space is super-poly. Then the derive signature scheme $\mathcal{S}$ is a secure digital signature scheme. 
\end{theorem}
\begin{proof}
    Let $\mathcal{A}$ be an adversary that breaks the digital signature scheme, we can construct another adversary $\mathcal{B}$ such that breaks the IBE security.

    The BIE adversary $\mathcal{B}$ is modelled as follows:
    \begin{enumerate}
        \item $\mathcal{B}$ receives $mpk$ from the challenger, and forward $mpk$ as a signature public key to $\mathcal{A}$. 
        \item $\mathcal{A}$ makes a series of signing queries $m_0,\dots,m_n$ to $\mathcal{B}$. $\mathcal{B}$ responds by issuing the corresponding key query to the challenger and forwarding the answer back the response to $\mathcal{A}$. 
        \item $\mathcal{A}$ will output a signature forgery $(m,\sigma)$ which it didn't issue the sign query $m$. 
        \item $\mathcal{B}$ then choose two random message $t_0,t_1\stackrel{\$}{\gets}\mathcal{M}_{IBE}$ and issue encryption query with the identity $m$.
        \item $\mathcal{B} $gets back the ciphertext $c\stackrel{\$}{\gets}E(mpk,m,t_b)$ for $b\in{0,1}$. Then it runs $t'\gets D(\sigma,c)$ and output $b'=t'$. 
    \end{enumerate}
    We observe that 
    \begin{itemize}
        \item when $b = 1$, then $c\stackrel{\$}{\gets}E(mpk, m,t_1)$, and $\mathcal{B}$ outputs 1 with probability the same as the probability of $\mathcal{A}$ breaks digital signature, we note as $SIGadv[\mathcal{A,S}]$
        \item when $b = 0$, then $c\stackrel{\$}{\gets}E(mpk, m,t_0)$, and $\mathcal{B}$ outputs 1 with probability $1/|\mathcal{M}_{IBE}|$ since $\mathcal{B}$ can only make random guess in the message space. 
    \end{itemize}
    We then have the probability of $\mathcal{B}$ break IBE scheme
    \[
        \frac{1}{2}(SIGadv[\mathcal{A,S}] + 1/|\mathcal{M}_{IBE}|)
    \]
    which is not negligible if $SIGadv[\mathcal{A,S}]$ is not negligible. \qed

\end{proof}

\input{lec16-F24}
\input{lec17-F24}
\input{lec18-F24}
\input{lec19-F24}
\input{lec20-F24}
\input{lec21-F24}
\input{lec22-F24}
\input{lec23-F24}
\input{lec24-F24}
\input{lec25-F24}
\input{lec26-F24}
% \algrenewcommand\algorithmicfunction{\textbf{Machine}}

\renewcommand{\O}{\ensuremath{\mathcal{O}}}

\renewcommand{\P}{\cclass{P}}

\renewcommand{\Enc}{\mathsf{Enc}}
\renewcommand{\Dec}{\mathsf{Dec}}
\renewcommand{\sk}{\mathsf{sk}}

\chapter{Secure Computation}

\section{Introduction}
Secure multiparty computation considers the problem of different parties
computing a joint function of their separate, private inputs without revealing
any extra information about these inputs than that is leaked by just the result
of the computation. This setting is well motivated, and captures many different
applications. Considering some of these applications will provide intuition
about how security should be defined for secure computation:
\begin{description}
  \item[Voting:] Electronic voting can be thought of as a multi party computation
	  between $n$ players: the voters. Their input is their choice $b \in \{0,1\}$
    (we restrict ourselves to the binary choice setting without loss of generality), and the function
    they wish to compute is the majority function.

    Now consider what happens when only one user votes: their input is trivially
    revealed as the output of the computation. What does privacy of inputs mean
    in this scenario?

  \item[Searchable Encryption:] Searchable encryption schemes allow clients
    to store their data with a server, and subsequently grant servers tokens
    to conduct specific searches. However, most schemes do not consider access
    pattern leakage. This leakage tells the server potentially valuable information
    about the underlying plaintext. How do we model all the different kinds
    information that is leaked?
\end{description}

From these examples we see that defining security is tricky, with lots of
potential edge cases to consider. We want to ensure that no party can learn
anything more from the secure computation protocol than it can from just its
input and the result of the computation. To formalize this, we adopt the
\textbf{real/ideal paradigm}.



\section{Real/Ideal Paradigm}
\paragraph{Notation.} 
Suppose there are $n$ parties, and party $P_i$ has access to some data $x_i$. They are trying to compute some function of their inputs $f(x_1, \dotsc, x_n)$. The goal is to do this securely: even if some parties are corrupted, no one should learn more than is strictly necessitated by the computation.

\paragraph{Real World.} In the real world, the $n$ parties execute a protocol $\Pi$
to compute the function $f$. This protocol can involve multiple rounds of
interaction. %Each party can additionally have some randomness.
The real world adversary $\RealAdv$ can corrupt arbitrarily many (but not all) parties.


\paragraph{Ideal World.} In the ideal world, an angel helps in the
computation of $f$:
each party sends their input to the angel and receives the output of the computation $f(x_1, \dotsc, x_n)$.
Here the ideal world adversary $\IdealAdv$ can again corrupt arbitrarily many (but not all) parties.

To model malicious adversaries, we need to modify the ideal world model as follows. 
Some parties are honest, and each honest party $P_i$ simply sends $x_i$ to the angel. The other parties are corrupted and are under control of the adversary $\IdealAdv$. The adversary chooses an input $x_i'$ for each corrupted party $P_i$ (where possibly $x_i' \neq x_i$) and that party then sends $x_i'$ to the angel. The angel computes a function $f$ of the values she receives (for example, if only party 1 is honest, then the angel computes $f(x_1, x_2', x_3', \dotsc, x_n')$) in order to obtain a tuple $(y_1, \dotsc, y_n)$. 
She then sends $y_i$ of corrupted parties to the adversary, who gets to decide whether or not honest parties will receive their response from the angel. The angel obliges. Each honest party $P_i$ then outputs $y_i$ if they receive $y_i$ from the angel and $\perp$ otherwise, and corrupted parties output whatever the adversary tells them to. 

 



\paragraph{Definition of Security.} 
A protocol $\Pi$ is secure against computationally bounded adversaries if for every PPT adversary $\RealAdv$ in the real world, there exists an PPT adversary $\IdealAdv$ in the ideal world such that for all tuples of bit strings $(x_1, \dotsc, x_n)$, we have
\[ \mathrm{Real}_{\Pi, \RealAdv}(x_1, \dotsc x_n) \stackrel{c}{\simeq} \mathrm{Ideal}_{F,\IdealAdv}(x_1, \dotsc, x_n) \]
where the left-hand side denotes the output distribution induced by $\Pi$ running with $\RealAdv$, and the right-hand side denotes the output distribution induced by running the ideal protocol $F$ with $\IdealAdv$. 
The ideal protocol is either the original one described for semi-honest adversaries, or the modified one described for malicious adversaries. 





%We require that the views of the parties
%in each of the scenarios be identical, i.e.\ that a real-world execution of the
%protocol $\Pi$ should not leak any information not leaked by the ideal-world
%execution. Hence, the parties can only learn what they can infer from their
%inputs and the output $f(\InputA, \InputB)$. More formally, assuming $\RealAdv$
%corrupts one party (say $\PartyA$, wlog), we define random variables
%$\RealVar_{\Pi, \RealAdv}(\InputA, \InputB) = \RealAdv(\InputA, r_1, \text{messages
%sent in } \Pi)$ and $\IdealVar_{F, \IdealAdv}(\InputA, \InputB) = \IdealAdv(\InputA,
%f(\InputA,\InputB))$.  These random variables represent the views of the
%adversary in each of the two settings. Our definition of security thus requires
%that
%
%\begin{equation*}
%\RealVar_{\Pi, \RealAdv}(\InputA, \InputB) \indis \IdealVar_{F, \IdealAdv}(x_1, x_2).
%\end{equation*}

\paragraph{Assumptions.} We have brushed over some details of the above setting.
Below we state these assumptions explicitly:
\begin{enumerate}
  \item \textbf{Communication channel:} We assume that the communication channel
    between the involved parties is completely insecure, i.e., it does not preserve
    the privacy of the messages. However, we assume that it is reliable, which means
    that the adversary can drop messages, but if a message is delivered, then
    the receiver knows the origin.

  \item \textbf{Corruption model:} We have different models of how and when the
    adversary can corrupt parties involved in the protocol:
    \begin{itemize}
      \item
        \emph{Static:} The adversary chooses which parties to corrupt before the
        protocol execution starts, and during the protocol, the malicious parties
        remain fixed.
      \item
        \emph{Adaptive:} The adversary can corrupt parties dynamically during
        the protocol execution, but the simulator can do the same.
      \item
        \emph{Mobile:} Parties corrupted by the adversary can be ``uncorrupted''
        at any time during the protocol execution at the adversary's discretion.
    \end{itemize}

  \item \textbf{Fairness:} The protocols we consider are not ``fair'', i.e.,
    the adversary can cause corrupted parties to abort arbitrarily. This can
    mean that one party does not get its share of the output of the computation.

  \item \textbf{Bounds on corruption:} In some scenarios, we place upper bounds
    on the number of parties that the adversary can corrupt.

  \item \textbf{Power of the adversary:} We consider primarily two types of
    adversaries:
    \begin{itemize}
      \item \emph{Semi-honest adversaries:} Corrupted parties follow the protocol
        execution $\Pi$ honestly, but attempt to learn as much information as they
        can from the protocol transcript.

      \item \emph{Malicious adversaries:} Corrupted parties can deviate arbitrarily
        from the protocol $\Pi$.
    \end{itemize}

  \item \textbf{Standalone vs.\ Multiple execution:} In some settings, protocols
    can be executed in isolation; only one instance of a particular protocol
    is ever executed at any given time. In other settings, many different protocols
    can be executed concurrently. This can compromise security.
\end{enumerate}







\section{Oblivious transfer}

\emph{Rabin's oblivious transfer} sets out to accomplish the following special task of two-party secure computation. The sender has a bit $s \in \{0,1\}$. She places the bit in a box. Then the box reveals the bit to the receiver with probability 1/2, and reveals $\perp$ to the receiver with probability 1/2. The sender cannot know whether the receiver received $s$ or $\perp$, and the receiver cannot have any information about $s$ if they receive $\perp$.

\subsection{1-out-of-2 oblivious transfer}
\emph{1-out-of-2 oblivious transfer} sets out to accomplish the following related task. The sender has two bits $s_0, s_1 \in \{0,1\}$ and the receiver has a bit $c \in \{0,1\}$. The sender places the pair $(s_0, s_1)$ into a box, and the receiver places $c$ into the same box. The box then reveals $s_c$ to the receiver, and reveals $\perp$ to the sender (in order to inform the sender that the receiver has placed his bit $c$ into the box and has been shown $s_c$). The sender cannot learn any information about $c$, so she cannot know anything about which of her bits the receiver received. Also, the receiver cannot know any information about $s_{1-c}$. This exchange can be modeled by a function $f$, such that $f((s_0, s_1), c) = (\perp, s_c)$. The sender sends the input $(s_0, s_1)$ and receives the output $\perp$. The receiver sends the input $c$ and receives the output $s_c$. We will assume that the two parties follow this protocol, and we will later change to a malicious setting.

\begin{lemma}
A system implementing 1-out-of-2 oblivious transfer can be used to implement Rabin's oblivious transfer.
\end{lemma}

\proof
The sender has a bit $s$. She randomly samples a bit $b \in \{0,1\}$ and $r \in \{0,1\}$, and the receiver randomly samples a bit $c \in \{0,1\}$. If $b = 0$, the sender defines $s_0 = s$ and $s_1 = r$, and otherwise, if $b = 1$, she defines $s_0 = r$ and $s_1 = s$. She then places the pair $(s_0, s_1)$ into the 1-out-of-2 oblivious transfer box. The receiver places his bit $c$ into the same box, and then the box reveals $s_c$ to him and $\perp$ to the sender. Notice that if $b = c$, then $s_c = s$, and otherwise $s_c = r$. Once $\perp$ is revealed to the sender, she sends $b$ to the receiver. The recieiver checks whether or not $b = c$. If $b = c$, then he knows that the bit revealed to him was $s$. Otherwise, he knows that the bit revealed to him was the nonsense bit $r$ and he regards it as $\perp$. \\

It is easy to see that this procedure satisfies the security requirements of Rabin's oblivious transfer protocol. Indeed, as we saw above, $s_c = s$ if and only if $b = c$, and since the sender knows $b$, we see that knowledge of whether or not the bit $s_c$ received by the receiver is equal to $s$ is equivalent to knowledge of $c$, and the security requirements of 1-out-of-2 oblivious transfer prevent the sender from knowing $c$. Also, if the receiver receives $r$ (or, equivalently, $\perp$), then knowledge of $s$ is knowledge of the bit that was not revealed to him by the box, which is again prevented by the security requirements of 1-out-of-2 oblivious transfer.  $\qed$

\begin{lemma}
A system implementing Rabin's oblivious transfer can be used to implement 1-out-of-2 oblivious transfer.
\end{lemma}

\proofsketch
The sender has two bits $s_0, s_1 \in \{0,1\}$ and the receiver has a single bit $c$. The sender randomly samples $3n$ random bits $x_1, \dotsc, x_{3n} \in \{0,1\}$. Each bit is placed into its own a Rabin oblivious transfer box. The $i$th box then reveals either $x_i$ or else $\perp$ to the receiver. Let 
\[ S := \{i \in \{1, \dotsc, 3n\} : \text{the receiver knows } x_i\}. \]
The receiver picks two sets $I_0, I_1 \subseteq \{1, \dotsc, 3n\}$ such that $\# I_0 = \# I_1 = n$, $I_c \subseteq S$ and $I_{1-c} \subseteq \{1, \dotsc, 3n\} \setminus S$. This is possible except with probability negligible in $n$. He then sends the pair $(I_0, I_1)$ to the sender. The sender then computes $t_j= \left(\bigoplus_{i \in I_j}x_i \right) \oplus s_j$ for both $j \in \{0,1\}$ and sends $(t_0, t_1)$ to the receiver. \\

Notice that the receiver can uncover $s_c$ from $t_c$ since he knows $x_i$ for all $i \in I_c$, but cannot uncover $s_{1-c}$. One can show that the security requirement of Rabin's oblivious transfer implies that this system satisfies the security requirement necessary for 1-out-of-2 oblivious transfer. $\qed$ \\

We will see below that length-preserving one-way trapdoor permutations can be used to realize 1-out-of-2 oblivious transfer. 

\begin{theorem}
The following protocol realizes 1-out-of-2 oblivious transfer in the presence of computationally bounded and semi-honest adversaries. 
\begin{enumerate}
\item The sender, who has two bits $s_0$ and $s_1$, samples a random length-preserving one-way trapdoor permutation $(f, f^{-1})$ and sends $f$ to the receiver.  Let $b(\cdot)$ be a hard-core bit for $f$.
\item The receiver, who has a bit $c$, randomly samples an $n$-bit string $x_c \in \{0,1\}^n$ and computes $y_c = f(x_c)$. He then samples another random $n$-bit string $y_{1-c} \in \{0,1\}^n$, and then sends $(y_0, y_1)$ to the sender.
\item The sender computes $x_0 := f^{-1}(y_0)$ and $x_1 := f^{-1}(y_1)$. She computes $b_0 := b(x_0) \oplus s_0$ and $b_1 := b(x_1) \oplus s_1$, and then sends the pair $(b_0, b_1)$ to the receiver.
\item The receiver knows $c$ and $x_c$, and can therefore compute $s_c = b_c \oplus b(x_c)$. 
\end{enumerate}
\end{theorem}
\proof
Correctness is clear from the protocol.	
For security, from the sender side, since $f$ is a length-preserving permutation, $(y_0, y_1)$ is statistically indistinguishable from two random strings, hence she can't learn anything about $c$.
From the receiver side, guessing $s_{1-c}$ correctly is equivalent to guessing the hard-core bit for $y_{1-c}$.
\qed


\subsection{1-out-of-4 oblivious transfer}
  With messages $m_{00},\ m_{01},\ m_{10},$ and $m_{11}$, we describe how to implement a 1-out-of-4 oblivious transfer (OT) using 1-out-of-2 OT:\@
  \begin{enumerate}
    \item
      The sender $\PartyA$ samples 6 random values $(S_0, S_1, S_{00}, S_{01}, S_{10}, S_{11}) \gets \bits^6$. Note that each of these values are sampled uniformly at random so as to not leak any information about the messages.
    \item
      $\PartyA$ computes
      \begin{align*}
        \alpha_{00} &= S_0 \xor S_{00} \xor m_{00}\\
        \alpha_{01} &= S_0 \xor S_{01} \xor m_{01}\\
        \alpha_{10} &= S_1 \xor S_{10} \xor m_{10}\\
        \alpha_{11} &= S_1 \xor S_{11} \xor m_{11}
      \end{align*}
      It sends these values to $\PartyB$.
    \item
      The parties engage in 3 1-out-of-2 oblivious transfer protocols for the following
      messages: $(S_0, S_1)$, $(S_{00}, S_{01})$, $(S_{10}, S_{11})$. The receiver's input for the first OT is the first choice bit, and for the second and third ones is
      the second choice bit.
    \item
      The receiver can only decrypt one ciphertext.
  \end{enumerate}

\subsection{Computation for Input Sharing}
Consider $n$ provers $P_i$ for all $i \{1, \dots, n\}$ that are trying to compute shares of $\gamma$. Consider inputs $\alpha = \alpha_1 \xor \dots \xor \alpha_n$ and $\beta = \beta_1 \xor \dots \xor \beta_n$ such that for all $i \in \{1, \dots, n\}$, each prover $P_i$ only has access to $\alpha_i$ and $\beta_i$.

The XOR operation can be accomplished with secret sharing, since $\gamma = \alpha \xor \beta = (\alpha_1 \xor \dots \alpha_n) \xor (\beta_1 \xor \dots \beta_n) = (\alpha_1 \xor \beta_1) \xor (\alpha_2 \xor \beta_2) \xor \dots \xor (\alpha_n \xor \beta_n)$.

The NOT operation can be accomplished with secret sharing, where $\alpha$ is the input and $1 - \alpha$ is the output, since the first party can simply flip a bit by calculating $1 - \alpha_1$.

The AND operation can be accomplished with secret sharing, since $\gamma = \alpha \cdot \beta = (\alpha_1 \xor \dots \xor \alpha_n) \cdot (\beta_1 \xor \dots \xor \beta_n) = \Sigma_i \alpha_i \beta_i + \Sigma_{i,j\ |\ i \ne j} \alpha_i \beta_j$. The first summation because for each $i \in \{1, \dots, n\}$, $P_i$ can calculate $\alpha_i \beta_i$. The second summation with the ``cross terms" is equal to $\Sigma_{1 \leq i < j \leq n}(\alpha_i \beta_j + \alpha_j + \beta_i) = \Sigma_{1 \leq i < j \leq n}$. Note that $z_{ji} = \alpha_i\beta_j + \alpha_j\beta_i + z_{ij}$. To compute $\alpha_i \beta_j + \alpha_j \beta_i$, $P_i$ and $P_j$ need to work together. If $\alpha_j = \beta_i = 0$, then $z_{ji} = z_{ij}$. If $\alpha_j = 0$ an d $\beta_i = 1$, then $z_{ji} = \alpha_i + z_{ij}$. If $\alpha_j = 1$ an d $\beta_i = 0$, then $z_{ji} = \beta_i + z_{ij}$. If $\alpha_j = \beta_i = 1$, then $z_{ji} = \alpha_i + \beta_i + z_{ij}$.

Note that the goal here is not the solve the problem of whether someone does not reveal their results, such as if their computer dies. In fact, it may be unclear what ``solving" would even mean in such a context. For example, in the context of voting, it is unclear whether a ``solve" would mean a tie between candidates or something else.

\section{Yao's Garbled Circuit}


%\input{HWsolution.tex}
%\part{Yao's Garbled Circuit}

% ===========
\section{Setup}


Yao's Garbled Circuits is presented as a solution to Yao's Millionaires' problem, 
which asks whether 
two millionaires can compete for bragging rights of which is richer
without revealing their wealth to each other. 
It started the area of secure computation. 
We will present a solution for the two party problem;
it can be extended to a polynomial number of parties,
using the techniques from last lecture.

The solution we saw previously needed an interaction for each AND gate.
Yao's solution requires only one message,
so it provides a constant size of interaction.
We present a solution only for semi honest security. 
This can be amplified to malicious security, 
but there are more efficient ways of amplifying this than what we saw last lecture.

\subsection{Secure Computation}

Recall our definition of secure computation. 
We define ideal and real worlds. 
Security is defined to hold if 
anything an attacker can achieve in the real world 
 can also be achieved by an ideal attacker in the ideal world. 
We define the ideal world to have the properties that we desire. 
For security to hold these properties must also hold in the real world.

\subsection{$(\Garble, \Eval)$}
We will provide a definition, similar to how we define encryption, that allows us avoid dealing with simulators all the time. 


Yao's Garbled Circuit is defined as two efficient algorithms $(\Garble, \Eval)$. Let the circuit $C$ have $n$ input wires.
$\Garble$ produces the garbled circuit and two labels for each input wire. The labels are for each of 0 and 1 on that wire and are like encryption keys. 

\[
(\tilde{C}, \{\ell_{i,b}\}_{i \in [n], b \in \{0,1\}}) \leftarrow \Garble(1^k, C) 
\]

To evaluate the circuit on a single input we must choose a value for each of the n input wires.
Given n of 2n input keys, $\Eval$ can evaluate the circuit on those keys and get the circuit result.
\[
C(x) \leftarrow \Eval(\tilde{C}, \{\ell_{i, x_i}\}_{i \in [n]}) 
\]

\paragraph{Correctness}
Correctness is as usual, if you garble honestly, evaluation should produce the correct result. 
\[
\forall C, x 
Pr[ C(x) \ne \Eval(\tilde{C}, \{l_{i, x_i}\})\ |\ (\tilde{C}, \{\ell_{i,b}\}) = \Garble(1^k, C)] = 0
\]


\paragraph{Security}
For security we require that a party receiving 
a garbled circuit and n inputs labels 
can not computationally distinguish the joint distribution of the circuits and labels
from the distribution produced by 
a simulator with access to the circuit and its evaluation on the input that the labels represent. 
The simulator does not have access to the actual inputs.
If this holds, the party receiving the garbled circuit and n labels can not determine the inputs.

\begin{align*}
&\exists\ \text{PPT}\ \Sim : \forall C, x\\
&(\tilde{C}, \{\ell_{i,x_i}\}_{i \in [n]}) \simeq \Sim(1^k, C, C(x)) \text{ where} \\
&(\tilde{C}, \{\ell_{i,b}\}_{i \in [n], b \in \{0,1\}}) \leftarrow \Garble(1^k, C) 
\end{align*}

For simplicity we pass the circuit to the simulator.
You could also use universal circuits and pass 
in with the inputs the specific circuit that the universal circuit should realize. Letting $U$ be the universal circuit such that $U(C, x) = C(x)$, the structure of $U$ does not need to be hidden, just its input $(C, x)$.



\section{Use for Semi-honest two party secure communication}
Alice, with input $x^1$, and Bob, with input $x^2$, have a circuit, C, that they want to evaluate securely. 
The size of their combined inputs is n, so $|x^1| = n_1, |x^2| = n - n_1, |x^1| + |x^2| = n$.
They can do this by Alice garbling a circuit and sending input wire labels to Bob, as in Figure \ref{fig:message}.

Alice garbles the circuit and passes it to Bob, $\tilde{C}$.
Alice passes the labels for her input directly to Bob, $\{\ell_{i, x^1_i}\}_{i \in [n] / [n_2]}$.
Alice passes all the labels for Bob's input wires into oblivious transfer, $\{\ell_{i, b_i}\}_{i \in [n] / [n_1], b \in \{0,1\}}$, 
from which Bob can retrieve the labels for his actual inputs, $\{\ell_{i, x^2_i}\}_{i \in [n] / [n_1]}$.
Bob now has the garbled circuit and one label for each input wire. 
He evaluates the garbled circuit on those garbled inputs and learns $C(x^1||x^2)$.
Bob does not learn anything besides the result as he has only the garbled circuit and n garbled inputs.
Alice does not learn anything as she uses oblivious transfer to give Bob his input labels and receives nothing in reply.

\begin{figure}[htbp]
\begin{center}
\setlength{\unitlength}{1cm}
\begin{picture}(10, 7)(-5, -4)
% \put(-.5,2){\makebox(1,1){C}}
 \put(-6,2){\makebox{Alice: $C, x^1$}}
 \put(-6,1.3){\makebox{$(\tilde{C}, \{\ell_{i,b}\}) \leftarrow \Garble$}}
 \put(4,2){\makebox{Bob: $C, x^2$}}

 \put(-1,0){\makebox(2,2){$\underrightarrow{\tilde{C}}$}}
  \put(-1,-0.8){\makebox(2,2){$\underrightarrow{ S_{out}^0 \text{ is 0 }, S_{out}^1 \text{ is 1 } }$}}


 \put(-1,-2){\makebox(2,2){$\underrightarrow{\{\ell_{i, x^1_i}\}_{i \in [n] / [n_2]}}$}}
% \put(-1,-1){\makebox(2,2){$\underrightarrow{\ell_{i,0}, \, \ell_{i,1} \forall i \in  [n]/[n_1] }$}}

 \put(-.5,-3){\framebox(1,1){OT}}
  \put(-1,-2.8){\line(1,0){.5}}
   \put(-1.6,-2.8){\makebox{$\ell_{i,1}$}}

  \put(-1,-2.2){\line(1,0){.5}}
     \put(-1.6,-2.2){\makebox{$\ell_{i,0}$}}

  \put(.5,-2.5){\line(1,0){.5}}
     \put(1.2,-2.5){\makebox{$\{\ell_{i, x^2_i}\}_{i \in [n] / [n_1]}$}}
     
  \put(-1,-4.5){\makebox(2,2){$\underrightarrow{ \forall i \in  [n]/[n_1] }$}}


\end{picture}
\caption{Messages in Yao's Garbeled Circuit}
\label{fig:message}
\end{center}
\end{figure}






%\paragraph{Malicious Bob}
%Alice semi-honest, and oblivious transfer is maliciously secure.
%Holds against malicious $Bob^*$
% What of deliberate circuit that shows first input  

\subsection{Construction of Garbled Circuits}

We would like to garble a circuit such that there are two keys for each input wire.
Correctness should be that 
given one of the two keys for each wire we can compute the output for the inputs those keys correspond to.
Security should be that 
given one key for each wire you can only learn the output, not the actual inputs.

%---

We build the circuit as a bunch of NAND gates that outputs one bit. 
If more bits are required, this can be done multiple times.
NAND gates can create any logic needed. 
We define the following sets:
\begin{align*}
W &= \text{the set of wires in the circuit}\\
G &= \text{the set of gates in the circuit.}
\end{align*}

For  each wire in the circuit, sample two keys
to label the possible inputs $0$ and $1$  to the wire
\[
\forall w \in W  \quad S_w^0, S_w^1 \,  \leftarrow{} \{0,1\}^k.
\]
We can think of these as the secret keys to an encryption scheme
(Gen, Enc, Dec).
For such a scheme we can always replace the secret key with the random bits fed into Gen.


\paragraph{Wires}
For each wire in the circuit we will have an invariant that the evaluator can only get one of the wires two encrypted values.
Consider an internal wire fed by the evaluation of a gate. The gate receives two encrypted values as inputs
and produces one encrypted output. The output will be one of the two labels for that wire and the evaluator will have no 
way of obtaining the other label for that wire. 
For example on wire $w_i$, the evaluator will only learn the value for $1$,  $S_{w_i}^1$.
We ensure this for the input wires by giving the evaluator only one of the two encrypted values for the wire.

\paragraph{Gates}
For every gate in the circuit we create four cipher texts. 
For each choice of inputs we encrypt the output key under each of the input keys. 
Let gate $g$ have inputs $w_1, w_2$ and output $w_3$,
\begin{align*}
e_g^{00} &= \Enc_{S_{w_1}^0} ( \Enc_{S_{w_2}^0}  ( S_{w_3}^1, 0^k) )\\
e_g^{01} &= \Enc_{S_{w_1}^0} ( \Enc_{S_{w_2}^1}  ( S_{w_3}^1, 0^k) )\\
e_g^{10} &= \Enc_{S_{w_1}^1} ( \Enc_{S_{w_2}^0}  ( S_{w_3}^1, 0^k) )\\
e_g^{11} &= \Enc_{S_{w_1}^1} ( \Enc_{S_{w_2}^1}  ( S_{w_3}^0, 0^k) ).
\end{align*}
We add $k$ zeros at the end.

\paragraph{Final Output}
For the final output wire, $S_{out}$, we can just give out their values,
\begin{align*}
S_{out}^0 &\text{ corresponds to 0}\\
S_{out}^1 &\text{ corresponds to 1.}
\end{align*}

\paragraph{$\bold{\tilde{C}}$}
For each gate, Alice sends Bob a random permutation of the set of four encrypted output values.
\[
\{e_g^{C_1, C_2} \} \quad \forall g \in G \quad C_1, C_2 \in \{0,1\}.
\]
For each gate, Alice sends Bob a random permutation of the set of four encrypted output values

\paragraph{Evaluation}
With an encrypted gate $g$,
input keys $S_{w_1} \, S_{w_2}$ for the input wires,
and four randomly permuted encryptions of the output keys, $e_g^{a}, e_g^{b}, e_g^{c}, e_g^{d}$,
Bob can evaluate the gate to find the corresponding key $S_{w3}$ for the output wire.
Bob can decrypt each of the encrypted output keys until he finds one that decrypts 
to a string ending in the proper number of $0$'s, which is very likely to contain the proper output key.
We can increase the probability of the correct key by increasing the number of $0$'s. 
\[
\exists  i \in \{a, b, c, d\} : \Dec_{S_{w_2}} ( \Dec_{S_{w_1}} ( e_g^{i} ))  = S_{w_3}, 0^k
\]

Given input wire labels 
$\{ \ell_{i, x_i} \}_{i \in [n]}$
the complete encrypted circuit $\tilde{C}$ is evaluated by working up from the input gates. 

%$l_{i,b} = \{S_{i,b}\}$

%as with PRF encryption scheme
%$Enc(_s(m) = (r,  m \oplus F_s(r)$


The evaluator should not be able to infer anything except what they could infer in the ideal world.
As a simple example, if the evaluator supplies one input to a circuit of just one NAND gate,
 they would be able to infer the input of the other party. However, this is true is the ideal world as well.

\section{Proof Intuition}

What intuition can we offer that the 
distribution of $\tilde{C}$ with one label per input wire 
is indistinguishable from what which a simulator could produce with access to the output?
%
For each input wire we are only given one key.
As we are doing double encryption,
for each input gate we only have the keys needed to decrypt one of the four possible outputs.
The other three are protected by semantic security.
%
So from each input gate we learn only one key compounding to its output wire.
As the output labels were randomized, we also do not know if that key corresponds to a 0 or a 1. 
%
For the next level of gates we again have only one key per input wire, and our argument continues. 
%
 So for each wire of the circuit we can only know one key corresponding to an output value for the wire. 
 Everything else is random garbage.
% 
As we control the mapping from output keys to output values, we can set this to whatever is needed to
match the expected output. 


Security only holds for evaluation of the circuit with one set of input values and 
we assume that the circuit is combinatorial and thus acyclic. 

% with two input all 0 or all 1 all broken
%  even with just 2 keys for one  input wire - broken. 




% !TEX root = collection.tex

\section{Malicious attacker intead of semi-honest attacker}

The assumption we had before consisted of a semi-honest attacker instead of a malicious attacker. A malicious attacker does not have to follow the protocol, and may instead alter the original protocol. The main idea here is that we can convert a protocol aimed at semi-honest attackers into one that will work with malicious attackers.

At the beginning of the protocol, we have each party commit to its inputs:
Given a commitment protocol $com$, Party 1 produces
\begin{center}
$c_1 = com(x_1; w_1)$ \\
$d_1 = com(r_1; \phi_1)$ \\
\end{center}
Party 2 produces
\begin{center}
$c_2 = com(x_2; w_2)$\\
$d_2 = com(r_2; \phi_2)$
\end{center}

We have the following guarantee: $\exists x_i, r_i, w_i, \phi_i$ such that $c_i = com(x_i; w_i) \wedge d_i = com(r_i; \phi_i) \wedge t = \pi(i,\text{transcript}, x_i, r_i)$, where transcript is the set of messages sent in the protocol so far.

Here we have a potential problem. Since both parties are choosing their own random coins, we have to be able to enforce that the coins are \emph{indeed} random. We can solve this by using the following protocol:

\begin{center}
  \begin{picture}(200,100)(10,20)
    \put(20, 90){$d_1 = com(s_1; \phi_1)$}
    \put(20,80){\vector(1,0){50}}
    \put(150, 90){$d_2 = com(s_2; \phi_2)$}
    \put(200, 80){\vector(-1,0){50}}

    \put(20, 60){$s_2^{'}$}
    \put(20,50){\vector(1,0){50}}
    \put(200, 60){$s_1^{'}$}
    \put(200, 50){\vector(-1,0){50}}
  \end{picture}
\end{center}

We calculate $r_1 = s_1 \oplus s_1^{'}$, and $r_2 = s_2 \oplus s_2^{'}$. As long as one party is picking the random coins honestly, both parties would have truly random coins.

Furthermore, during the first commitment phase, we want to make sure that the committing party actually knows the value that is being committed to. Thus, we also attach along with the commitment a zero-knowledge proof of knowledge (ZK-PoK) to prove that the committing party knows the value that is being committed to.

\subsection{Zero-knowledge proof of knowledge (ZK-PoK)}

\begin{definition}[ZK-PoK] Zero-knowlwedge proof of knowledge (ZK-PoK) is a zero-knowledge proof system $(P,V)$ with the property proof of knowledge with knowledge error $\kappa$:

$\exists$ a PPT $E$ (knowledge extractor) such that $\forall x \in L$ and $\forall P^{*}$ (possibly unbounded), it holds that if $\Pr[Out_V(P^{*}(x,w) \leftrightarrow V(x))]> \kappa(x)$, then 
\[ \Pr[E^{P^*}(x) \in R(x)] \geq \Pr[Out_V(P^{*} \leftrightarrow V(x))] = 1]- \kappa(x).\]
Here we have $L$ be the language, $R$ be the relation, and $R(x)$ is the set such that $\forall w \in R(x)$, $(x, w) \in R$.
\end{definition}

Given a zero-knowledge proof system, we can construct a ZK-PoK system for statement $x\in L$ with witness $w$ as follows:
\begin{center}
  \begin{picture}(300,300)(10,20)

    \put(10, 290){$P$}
    \put(290, 290){$V$}

    \put(10, 270){$r \leftarrow \{0, 1\}^{|w|}$}

    \put(100, 260){$c_1 = com(r; \omega)$}
    \put(100, 250){$c_2 = com(r \oplus w; \phi)$}
    \put(100, 240){\vector(1,0){100}}

    \put(150, 210){$b$}
    \put(200, 200){\vector(-1,0){100}}

    \put(120, 160){if $b = 0$, open $c_1$ to reveal $r$}
    \put(120, 150){else open $c_2$ to reveal $r \oplus w$}
    \put(100, 140){\vector(1,0){100}}

    \put(120, 60){\framebox(50,50)[c]{ZK Proof}}
  \end{picture}
\end{center}

The last ZK proof proves that $\exists r, w, \omega, \phi$ such that $(x, w) \in R$ and $c_1 = com(r; \omega)$, $c_2 = com(r \oplus w; \phi)$.


\section*{Exercises}
\begin{exercise}
Given a (secure against malicious adversaries) two-party secure computation protocol (and nothing else) construct a (secure against malicious adversaries) three-party secure computation protocol.
\end{exercise}


% % !TEX root = collection.tex

% \chapter{Obfustopia}


% \section{Witness Encryption: A Story}\label{story}

% Imagine that a billionaire who loves mathematics, would like to award with 1 million dollars the mathematician(s) who will prove the Riemann Hypothesis. Of course, neither does the billionaire know if the Riemann Hypothesis is true, nor if he will be still alive (if and) when a mathematician will come up with a proof. To overcome these couple of problems, the billionaire decides to:

% \begin{enumerate}

% \item Put 1 million dollars in gold in a big treasure chest.

% \item Choose an arbitrary place of the world, dig up a hole, and hide the treasure chest.

% \item Encrypt the coordinates of the treasure chest in a message so that only the mathematician(s) who can actually prove the Riemann Hypothesis can decrypt it.

% \item Publish the ciphertext in every newspaper in the world.

% \end{enumerate}

% The goal of this lecture is to help the billionaire with step 3. To do so, we will assume for simplicity  that the proof is at most 10000 pages long. The latter assumption implies that the language
% \begin{align*}
% L = \{ x \text{ such that } x \text{ is an acceptable Riemann Hypothesis proof} \}
% \end{align*}
%  is in NP and therefore, using a reduction, we can come up with a circuit $C$ that takes as input $x$ and outputs $1$ if $x$ is a proof for the Riemann Hypothesis and $0$ otherwise.

% \smallskip
% Our goal now is to  design a pair of PPT machines $(\mathrm{Enc},\mathrm{Dec})$ such that:

% \begin{enumerate}
% \item $\mathrm{Enc}(C,m)$ takes as input the circuit $C$ and $m \in \{0,1\}$ and outputs a ciphertext $e \in \{0,1\}^{*}$.

% \item $\mathrm{Dec}(C,e,w)$ takes as input the circuit $C$, the cipertext $e$ and a witness $w \in \{0,1\}^{*}$ and outputs $m$ if if $C(w) = 1$ or $\perp$ otherwise.
% \end{enumerate}

% and so that they satisfy the following correctness and security requirements:

% \begin{itemize}

% \item \textbf{Correctness:} If $\exists w$ such that $C(w) = 1$ then $\mathrm{Dec}(C,e,w)$ outputs $m$.

% \item \textbf{Security:} If $\nexists w$ such that $C(w) = 1$ then $\mathrm{Enc}(C,0)   \approx^{c} \mathrm{Enc}(C,1) \!\ $ (where $ \approx^{c}$ means  ``computationally indistinguishable'').

% \end{itemize}


% \section{A Simple Language }

% As a first example, we show how we can design such an encryption scheme for a simple language. Let $G$ be a group of prime order and  $g$ be a generator of the group. For elements $A, B, T \in G$ consider the language $L = \{(a,b): A = g^a, B = g^b, T = g^{ab} \}$. An encryption scheme for that language with the correctness and security requirements of Section~\ref{story} is the following:

% \smallskip

% \begin{itemize}

% \item \textbf{Encryption$(g,A,B,T,G)$:}

% \begin{itemize}
% \item Choose elements $r_1, r_2 \in \mathbb{Z}_p^*$ uniformly and independently.

% \item Let $c_1 = A^{r_1} g^{r_2} $, $c_2 =  g^m T^{r_1} B^{r_2}$, where $m \in \{0,1\}$ is the message we want to encrypt.

% \item Output $c = (c_1, c_2)$

% \end{itemize}

% \item \textbf{Decryption($b$):}


% \begin{itemize}

% \item Output $\frac{c_2}{c_1^b}$

% \end{itemize}

% \end{itemize}


% \textbf{Correctness:}
% The correcntess of the above encryption scheme follows from the fact that if there exist $(a,b) \in L$ then:

% \begin{eqnarray*}
% \frac{c_2}{c_1^b} & = &  \frac{g^m T^{r_1} B^{r_2}  }{ \left( A^{r_1}g^{r_2}\right)^b } \\
% & = & \frac{g^m \left(g^{ab}\right)^{r_1} \left( g^{b} \right)^{r_2}  }{ \left( g^{a} \right)^{r_1 b} g^{r_2 b} } \\
% & = & g^{m}
% \end{eqnarray*}

% Since $m \in \{0,1\}$ and we know $g$, the value of $g^m$ implies the value of $m$.

% \smallskip
% \textbf{Security:}
% As far as the security of the scheme is concerned, since $L$ is quite simple, we can actually prove that $m$ is information-theoretically hidden. To see this, assume there does not exist $(a,b) \in L$, but an adversary has the power to compute discrete logarithms. In that case, given $c_1$ and $c_2$ the adversary could get a system of the form:
% \begin{eqnarray*}
% ar_1 + r_2 & = & s_1 \\
% m + r r_1 + b r_2 &=& s_2
% \end{eqnarray*}
% where $s_1$ and $s_2$ are   the discrete logarithms of $c_1$ and $c_2$ respectively (with base $g$), and $r \ne ab$ is an element of $  \mathbb{Z}_{p}^*$ such that $T = g^r$. Observe now that for each value of $m$ there exist numbers $r_1$ and $r_2$ so that the above system has a solution, and thus $m$ is indeed information-theoretically hidden (on the other hand, if we had that $ab = r$ then the equations are linearly dependent).

% \newpage
% \section{An  NP Complete Language }

% In this section we focus on our original goal of designing an encryption for an NP complete language $L$. Specifically, we will consider the NP-complete problem \emph{exact cover}. Besides that, we introduce the $n$-Multilinear  Decisional Diffie-Hellman ($n$-MDDH) assumption  and the Decisional Multilinear No-Exact-Cover Assumption.  %(see also~\cite{Sanjam}). 
% The latter will guarantee the security of our construction.

% \subsection{ Exact Cover}

% We are given as input $x = (n, S_1, S_2, \ldots, S_l)$, where $n$ is an integer and each $S_i, i \in [l]$ is a subset of $[n]$, and our goal is to find a subset of indices $T \subseteq [l]$ such that:

% \begin{enumerate}
% \item $\cup_{i \in T} S_i = [n] $ and

% \item $\forall i, j \in T$ such that $i \ne j$ we have that $S_i \cap S_j = \emptyset$.
% \end{enumerate}

% If such a $T$ exists, we say that $T$ is an exact cover of $x$.

% \subsection{Multilinear Maps}

% Mutlinear maps is a generalization of bilinear maps (which we have already seen) that will be useful in our construction. Specifically, we assume the existence of a group generator $\mathcal{G}$, which takes as input a security parameter $\lambda$ and a positive integer $n$ to indicate the number of allowed operations. $\mathcal{G}(1^{\lambda},n)$ outputs a sequence of groups $\vec{\mathbb{G}}= (\mathbb{G}_1, \mathbb{G}_2, \ldots, \mathbb{G}_n)$  each of large prime order $P > 2^{\lambda}$. In addition, we let $g_i$ be a canonical generator of $\mathbb{G}_i$  (and is known from the group's description).

% We also assume the existence of a set of bilinear maps $\{e_{i,j}: \mathbb{G}_i \times \mathbb{G}_j \rightarrow \mathbb{G}_{i+j} \mid i, j \ge 1; i+j \le n \}.$ The map $e_{i,j}$ satisfies the following relation:
% \begin{align}
% e_{i,j}\left(g_i^{a},g_j^{b}\right) = g^{ab}_{i+j}: \forall a,b \in \mathbb{Z}_p \label{vasikoni}
% \end{align}
% and we observe that one consequence of this is that $e_{i,j} (g_i, g_j) = g_{i+j}$ for each valid $i,j$.

% \subsection{The $n$-MDDH Assumption  }

% The $n$-Multilinear Decisional Diffie-Hellman ($n$-MDDH) problem states the following: A challenger runs $\mathcal{G}(1^{\lambda},n ) $ to generate groups and generators of order $p$. Then it picks random $s, c_1, \ldots, c_n  \in \mathbb{Z}_p$.  The assumption then states that given $g= g_1, g^{s}, g^{c_1}, \ldots,g^{c_n}$ it is hard to distinguish $T = g_n^{s \prod_{j \in [1,n ] } c_j}$ from a random group element in $G_n$, with better than negligible advantage (in security parameter $\lambda$).

% \newpage


% \subsection{Decisional Multilinear  No-Exact-Cover  Assumption}
% Let $x = (n, S_1, \ldots, S_l)$ be an instance of the exact cover problem that has no solution. Let $\mathrm{param} \leftarrow \mathcal{G}(1^{1+n},n)$ be a description of a multilinear group family with order $p = p(\lambda)$. Let $a_1, a_2, \ldots, a_n,r$ be uniformly random in $\mathbb{Z}_p$. For $i \in [l]$, let $c_i  = g_{|S_i|}^{ \prod_{j \in S_i} a_j}$. Distiguish between the two distributions:
% \begin{align*}
% (\mathrm{params}, c_1, \ldots,c_l,g_n^{a_1a_2\ldots a_n}) \text{ and } (\mathrm{params},c_1, \ldots,c_l,g_n^r)
% \end{align*}

% The Decisional Multilinear No-Exact-Cover Assumption is that for all adversaries $\mathcal{A}$, there exists a fixed negligible function $\nu(\cdot)$ such that for all instances $x$ with no solution, $\mathcal{A}$'s distinguishing advantage against the Decisional Multilinear No-Exact-Cover Problem  for $x$ is at most $\nu(\lambda)$.

% \subsection{The Encryption Scheme  }

% We are now ready to give the description of our encryption scheme.

% \begin{itemize}

% \item $\mathrm{Enc}(x,m)$ takes as input $x=(n, S_1, \ldots,S_l)$ and the message $m \in \{0,1\}$ and:

% \begin{itemize}

% \item Samples $a_{0}, a_1, \ldots, a_{n}$ uniformly and independently from $\mathbb{Z}_p^*$.

% \item $\forall i \in [l]$ let $c_i = g^{\prod_{j\in S_j} a_j}_{|S_i|}$

% \item Sample uniformly an element $r \in \mathbb{Z}_p^*$

% \item Let $d = d(m) $ be $  g_n^{\prod_{j \in [n]}a_j}$ if $m = 1 $ or $g_n^r$ if $m = 0$.

% \item Output $c = (d, c_1, \ldots,c_l)$
% \end{itemize}

% \item $\mathrm{Dec}(x,T)$, where $T \subseteq[l]$ is a set of indices, computes $\prod_{i \in T}c_i$ and outputs $1$ if the latter value equals to $d$ or $0$ otherwise.

% \end{itemize}


% \begin{itemize}

% \item \textbf{Correctness:} Assume that $T$ is an exact cover of $x$. Then, it is not hard to see that:
% \begin{eqnarray*}
% \prod_{i \in T} c_i & = & \prod_{i \in T} g^{\prod_{j\in S_j} a_j}_{|S_i|} \\
% & = & g_n^{\prod_{j \in [n]}a_j}
% \end{eqnarray*}
% where we have used~\eqref{vasikoni} repeatedly and the fact that $T$ is an exact cover (to show that $\sum_{i \in T} |S_i| = n$ and that $\prod_{i \in T} \prod_{j \in S_i} a_j = \prod_{i \in [n]} a_i$).

% \item \textbf{Security:} Intuitively, the construction is secure, since the only way to make $g_n^{\prod_{ j \in [n] }a_i}$ is to find an exact cover of $[n]$.  As a matter of fact, observe that if an exact cover does not exist, then for each subset of indices $T'$( such that $\cup_{i \in T'}S_j  = [n]$) we have that
% \begin{align*}
% \sum_{i =1 }^{n} |S_i| > n,
% \end{align*}
% which means that   $\prod_{i \in T} \prod_{j \in S_i} a_j$ is different than $\prod_{j \in [n]}a_j$. Formally, the security is based on the Decisional Multilinear No-Exact-Cover Assumption.

% \end{itemize}


% %\bibliographystyle{plain}
% %\bibliography{smoser}


% % !TEX root = collection.tex

% %\newcommand{\norm}[1]{\left\Vert#1\right\Vert}
% \newcommand{\ABS}[1]{\left\vert#1\right\vert}
% \newcommand{\SET}[1]{\left\{#1\right\}}  
% \newcommand{\INP}[1]{\left(#1\right)}
% \newcommand{\POLY}[1]{\ensuremath{\mathop{\mathrm{poly}}\INP{#1}}}
% %\newcommand{\iO}[1]{\ensuremath{\mathop{i\mathcal{O}}\INP{#1}}}
% \newcommand{\ENC}[1]{\ensuremath{\mathop{\mathrm{Enc}}\INP{#1}}}
% \newcommand{\DEC}[1]{\ensuremath{\mathop{\mathrm{Dec}}\INP{#1}}}
% %\bibliographystyle{plain}

% \section{Obfuscation}
% The problem of program obfuscation asks whether one can transform a program (e.g., circuits, Turing machines) to another semantically equivalent program (i.e., having the same input/output behavior), but is otherwise intelligible.
% It was originally formalized by Barak et al. who constructed a family of circuits that are non-obfuscatable under the most natural virtual black box (VBB) security.
% \section{VBB Obfuscation}
% As a motivation, recall that in a private-key encryption setting, we have a secret key $k$, encryption $E_k$ and decryption $D_k$.
% A natural candidate for public-key encryption would be to simply release an encryption $E'_k \equiv E_k$ (i.e. $E'_k$ semantically equivalent to $E_k$, but computationally bounded adversaries would have a hard time figuring out $k$ from $E'_k$.

% \begin{definition}[Obfuscator of circuits under VBB]
% 	$O$ is an \emph{obfuscator} of circuits if %for every circuit $c$ we have,
% 	\begin{enumerate}
% 		\item
% 			Correctness:
% 	$\forall c, O(c) \equiv c$.
% 	\item
% 		Efficiency:
% 		$\forall c, \ABS{O(c)} \le \POLY{\ABS{c}}$.
% 	\item
% 		VBB:
% 		$\forall A, A$ is PPT bounded, $\exists$ S (also PPT) s.t. $\forall c$,
% 		\[
% 			\ABS{\Pr\left[ A\left( O(c) \right) = 1\right] - \Pr\left[ S^c(1^{\ABS{c}}) = 1 \right]} \le \mathrm{negl}(\ABS{c}).
% 		\]
% 	\end{enumerate}
% \end{definition}

% Similarly we can define it for Turing machines.
% \begin{definition}[Obfuscator of TMs under VBB]
% 	$O$ is an \emph{obfuscator} of Turing machines if %for every circuit $c$ we have,
% 	\begin{enumerate}
% 		\item
% 			Correctness:
% 	$\forall M, O(M) \equiv M$.
% 	\item
% 		Efficiency:
% 		$\exists q(\cdot) = \POLY{\cdot}, \forall M \left( M(x) \hbox{ halts in }t \hbox{ steps} \implies O(M)(x) \hbox{ halts in }q(t) \hbox{ steps}\right)$.
% 	\item
% 		VBB:
% 		Let $M'(t,x)$ be a TM that runs $M(x)$ for $t$ steps.
% 		$\forall A, A$ is PPT bounded, $\exists$ Sim (also PPT) s.t. $\forall c$,
% 		\[
% 			\ABS{\Pr\left[ A\left( O(M) \right) = 1\right] - \Pr\left[ S^{M'}(1^{\ABS{M'}}) = 1 \right]} \le \mathrm{negl}(\ABS{M'}).
% 		\]
% 	\end{enumerate}
% \end{definition}

% Let's show that our candidate PKE from VBB obfuscator $O$ is semantic secure, using a simple hybrid argument.

% \proof
% Recall the public key $PK=O(E_k)$.
% Let's assume $E_k$ is a circuit.
% \begin{align*}
% 	H_0 :& A(\SET{(PK, E_k(m_0))}) & \\
% 	H_1 :& S^c(\SET{E_k(m_0)}) & \hbox{ by VBB} \\
% 	H_2 :& S^c(\SET{E_k(m_1)}) & \hbox{ by semantic security of private key encryption} \\
% 	H_3 :& A(\SET{(PK, E_k(m_1))}) & \hbox{ by VBB}
% \end{align*}
% \qed

% Unfortunately VBB obfuscator for all circuits does not exist. Now we show the impossiblity result of VBB obfuscator.
% \begin{theorem}
% 	Let $O$ be an obfuscator.
% 	There exists PPT bounded $A$, and a family (ensemble) of functions $\SET{H_n}$, $\SET{Z_n}$ s.t.
% 	for every PPT bounded simulator $S$,
% \begin{gather*}
% 	A\left( O(H_n) \right) = 1 \ \ \& \ \ A\left( O(Z_n) \right) = 0\\
% 	\ABS{\Pr\left[ S^{H_n} \left( 1^{\ABS{H_n}} \right) = 1 \right] - \Pr \left[ S^{Z_n} \left(1^{\ABS{Z_n}}\right) =1 \right]} \le\mathrm{negl}(n).
% \end{gather*}
% \end{theorem}

% \proof
% Let $\alpha, \beta \overset{\$}{\leftarrow} \SET{0,1}^n$.

% We start by constructing $A',C_{\alpha,\beta}, D_{\alpha,\beta}$ s.t.
% \begin{gather*}
% 	A'\left( O(C_{\alpha,\beta}), O(D_{\alpha,\beta}) \right) = 1 \ \ \& \ \ A'\left( O(Z_n), O(D_{\alpha,\beta}) \right) = 0\\
% 	\ABS{\Pr\left[ S^{C_{\alpha,\beta},D_{\alpha,\beta}} \left( \mathbf{1} \right) = 1 \right] - \Pr \left[ S^{Z_n,D_{\alpha,\beta}} \left(\mathbf{1}\right) =1 \right]} \le\mathrm{negl}(n).
% \end{gather*}

% \begin{gather*}
% C_{\alpha,\beta}(x) =
% \begin{cases}
% 	\beta, & \hbox{if } x = \alpha,\\
% 	0^n, & \hbox{o/w}
% \end{cases} \\
% D_{\alpha,\beta}(c)=
% \begin{cases}
% 	1,& \hbox{if } c(\alpha) = \beta,\\
% 	0, & \hbox{o/w}.
% \end{cases}
% \end{gather*}

% Clearly $A'(X,Y) = Y(X)$ works.
% Now notice that input length to $D$ grows as the size of $O(C)$.

% However for Turing machines which can have the same description length, one could combine the two in the following way:

% $F_{\alpha,\beta}(b, x) =
% \begin{cases}
% 	C_{\alpha,\beta}(x), & b=0\\
% 	D_{\alpha,\beta}(x), & b=1\\
% \end{cases}.$

% Let $OF= O(F_{\alpha,\beta})$, $OF_0(x) = OF(0,x)$, similarly for $OF_1$, then $A$ would be just $A(OF) = OF_1(OF_0)$.

% Now assuming OWF exists, specifically we already have priavte-key encryption, we modify $D$ as follows.
% \begin{gather*}
% 	D_k^{\alpha,\beta}(1,i) = \mathrm{Enc}_k(\alpha_i) \\
% 	D_k^{\alpha,\beta}(2,c,d,\odot) = \mathrm{Enc}_k(\mathrm{Dec}_k(c) \odot \mathrm{Dec}_k(d)), \hbox{where $\odot$ is a gate of AND, OR, NOT} \\
% 	D_k^{\alpha,\beta}(3, \gamma_1,\cdots,\gamma_n) =
% 	\begin{cases}
% 		1,& \forall i, \mathrm{Dec}_k(\gamma_i) = \beta_i,\\
% 		0, & \hbox{o/w}.
% 	\end{cases}
% \end{gather*}

% Now the adversary $A$ just simulate $O(C)$ gate by gate with a much smaller $O(D)$, thus we can use the combining tricks as for the Turing machines.
% \qed

% \section{Indistinguishability Obfuscation}

% %\begin{definition}[Indistinguishability Obfuscation]
% %	$\iO{\cdot}$ is an \emph{indistinguishability obfuscation} if $\forall c_1, c_2$ such that $\ABS{c_1}= \ABS{c_2}$ and $c_1\equiv c_2$, we have
% %	\[
% %		\iO{c_1} \overset{c}{\approx} \iO{c_2}.
% %	\]
% %\end{definition}


% %\newcommand{\iO}{\ensuremath{i\mathcal{O}}}
% \newcommand{\Ck}{\ensuremath{\mathcal{C}_\kappa}}

% \begin{definition}[Indistinguishability Obfuscator]
% A uniform PPT machine $\iO$ is an \emph{indistinguishability obfuscator}
% for a collection of circuits $\Ck$ if the following conditions hold:
% \begin{itemize}

% \item \emph{Correctness.}
% For every circuit $C \in \Ck$ and for all inputs $x$,
% $C(x) = \iO(C(x))$.

% \item \emph{Polynomial slowdown.}
% For every circuit $C \in \Ck$, $|\iO(C)| \leq p(|C|)$ for some
% polynomial $p$.

% \item \emph{Indistinguishability.}
% For all pairs of circuits $C_1, C_2 \in \Ck$, if $|C_1| = |C_2|$ and
% $C_1(x) = C_2(x)$ for all inputs $x$, then
% $\iO(C_1) \overset{c}{\simeq} \iO(C_2)$.
% More precisely, there is a negligible function $\nu(k)$ such that for
% any (possibly non-uniform) PPT $A$,
% \begin{equation*}
% \big| \Pr[A(\iO(C_1)) = 1] - \Pr[A(\iO(C_2)) = 1] \big| \leq \nu(k).
% \end{equation*}

% \end{itemize}
% \end{definition}


% \begin{lemma}
% 	Indistinguishability obfuscation implies witness encryption.
% \end{lemma}
% \proof
% Recall the witness encryption scheme, with which one could encrypt a message $m$ to an instance $x$ of an NP language $L$, such that $\DEC{x,w,\ENC{x,m}}=
% \begin{cases}
% 	m, \hbox{if} (x,w)\in L, \\
% 	\bot, \hbox{o/w}
% \end{cases}$

% Let $C_{x,m}(w)$ be a circuit that on input $w$, outputs $m$ if and only if $(x,w) \in L$.
% Now we construct witness encryption as follows:
% $\ENC{x,m}=\iO{C_{x,m}}, \DEC{x,w,c}=c(w)$.

% Semantic security follows from the fact that, for $x\not\in L$, $C_{x,m}$ is just a circuit that always output $\bot$, and by indistinguishability obfuscation, we could replace it with a constant circuit (padding if necessary), and then change the message, and change the circuit back.
% \qed


% \begin{lemma}
% 	Indistinguishability obfuscation and OWFs imply public key encryption.
% \end{lemma}
% \proof
% We'll use a length doubling PRG $F: \SET{0,1}^n \to \SET{0,1}^{2n}$, together with a witness encryption scheme $(E,D)$.
% The NP language for the encryption scheme would be the image of $F$.
% \begin{align*}
% 	&\mathrm{Gen}(1^n) = (PK = F(s), SK=s), s\overset{\$}{\leftarrow} \SET{0,1}^n\\
% 	&\ENC{PK,m} = E(x=PK,m)\\
% 	&\DEC{e,SK=s} = D(x=PK,w=s,c=e).
% \end{align*}
% \qed

% \begin{lemma}
% 	Every best possible obfuscator could be equivalently achieved with an indistinguishability obfuscation (up to padding and computationally bounded).
% \end{lemma}

% \proof
% Consider circuit $c$, the \emph{best possible obfuscated} $BPO(c)$, and $c'$ which is just padding $c$ to the same size of $BPO(c)$.
% Computationally bounded adversaries cannot distinguish between $\iO{c'}$ and $\iO{BPO(c)}$.

% Note that doing iO never decreases the ``entropy'' of a circuit, so $\iO{BPO(c)}$ is at least as secure as $BPO(c)$.
% \qed



% % !TEX root = collection.tex

% %\section{Using Indistinguishability Obfuscation}


 % Near-duplicate of lec26-F24
\newcommand{\cO}{\ensuremath{\mathcal{O}}}
%\newcommand{\NC}[1]{\ensuremath{\mathbf{NC}^{#1}}}
\newcommand{\Ck}{\mathcal{C}_{\kappa}}


\chapter{Obfuscation}
\section{$\iO$ for Polynomial-sized Circuits}


\begin{definition}[Indistinguishability Obfuscator for $NC^1$]
Let $\Ck$ be the collection of circuits of size $O(\kappa)$ and depth
$O(\log{\kappa})$ with respect to gates of bounded fan-in.
Then a uniform PPT machine $\iO_{\NC{1}}$ is an
\emph{indistinguishability obfuscator} for circuit class $\NC{1}$ if it
is an indistinguishability obfuscator for $\Ck$.
\end{definition}

Given an indistinguishability obfuscator $\iO_{\NC{1}}$ for circuit
class $\NC{1}$, we shall demonstrate how to achieve an
indistinguishability obfuscator $\iO$ for all polynomial-sized circuits.
The amplification relies on fully homomorphic encryption (FHE).

\newcommand{\GEN}{\ensuremath{\mathsf{Gen}}}
%\newcommand{\Enc}{\ensuremath{\mathsf{Enc}}}
%\newcommand{\Dec}{\ensuremath{\mathsf{Dec}}}
%\newcommand{\Eval}{\ensuremath{\mathsf{Eval}}}
%\newcommand{\pk}{\ensuremath{\mathsf{pk}}}
%\newcommand{\sk}{\ensuremath{\mathsf{sk}}}

\begin{definition}[Homomorphic Encryption]
A \emph{homomorphic encryption scheme} is a tuple of PPT algorithms
$(\GEN, \Enc, \Dec, \Eval)$ as follows:
\begin{itemize}
\item
	$(\GEN, \Enc, \Dec)$ is a semantically-secure public-key
	encryption scheme.
\item
	$\Eval(\pk, C, e)$ takes public key $\pk$, an arithmetic circuit
	$C$, and ciphertext $e = \Enc(\pk, x)$ of some circuit input
	$x$, and outputs $\Enc(\pk, C(x))$.
\end{itemize}
\end{definition}

As an example, the ElGamal encryption scheme is homomorphic over the multiplication function.
Consider a cyclic group $G$ of order $q$ and generator $g$, and let
$\sk = a$ and $\pk = g^a$.
For ciphertexts $\Enc(\pk, m_1) = (g^{r_1}, g^{a r_1} \cdot m_1)$
and $\Enc(\pk, m_2) = (g^{r_2}, g^{a r_2} \cdot m_2)$, observe that
\begin{equation*}
\Enc(\pk, m_1) \cdot \Enc(\pk, m_2) = (g^{r_1 + r_2}, g^{a (r_1 + r_2)}
\cdot m_1 \cdot m_2) = \Enc(\pk, m_1 \cdot m_2)
\end{equation*}
Note that this scheme becomes additively homomorphic by encrypting $g^m$
instead of $m$.

\begin{definition}[Fully Homomorphic Encryption]
An encryption scheme is \emph{fully homomorphic} if it is both compact
and homomorphic for the class of all arithmetic circuits.
Compactness requires that the size of the output of $\Eval(\cdot, \cdot,
\cdot)$ is at most polynomial in the security parameter $\kappa$.
\end{definition}

\subsection{Construction}

%We first present a simpler construction under the virtual black box
%model, assuming the existence of a circuit obfuscator $\cO_{\NC{1}}$ for
%$\NC{1}$.

\newcommand{\prog}[1]{\ensuremath{P_{\pk_1,\pk_2,\sk_{#1},e_1,e_2}}}

Let $(\GEN, \Enc, \Dec, \Eval)$ be a fully homomorphic encryption
scheme.
We require that $\Dec$ be realizable by a circuit in $\NC{1}$.
The obfuscation procedure accepts a security parameter $\kappa$ and
a circuit $C$ whose size is at most polynomial in $\kappa$.
\begin{enumerate}
\item
	Generate $(\pk_1, \sk_1) \gets \GEN(1^\kappa)$ and
	$(\pk_2, \sk_2) \gets \GEN(1^\kappa)$.
\item
	Encrypt $C$, encoded in canonical form, as
	$e_1 \gets \Enc(\pk_1, C)$ and $e_2 \gets \Enc(\pk_2, C)$.
\item
	Output an obfuscation
	$\sigma = (\iO_{\NC{1}}(P), \pk_1, \pk_2, e_1, e_2)$
	of program $\prog{1}$ as described below.
\end{enumerate}

The evaluation procedure accepts the obfuscation $\sigma$ and program
input $x$.
\begin{enumerate}
\item
	Let $U$ be a universal circuit that computes $C(x)$ given a
	circuit description $C$ and input $x$, and denote by $U_x$ the
	circuit $U(\cdot, x)$ where $x$ is hard-wired.
	Let $R_1$ and $R_2$ be the circuits which compute
	$f_1 \gets \Eval(U_x, e_1)$ and $f_2 \gets \Eval(U_x, e_2)$,
	respectively.

\item
	Denote by $\omega_1$ and $\omega_2$ the set of all wires in $R_1$
	and $R_2$, respectively.
	Compute $\pi_1 : \omega_1 \to \{ 0, 1 \}$ and
	$\pi_2 : \omega_2 \to \{ 0, 1 \}$, which yield the value of internal
	wire $w \in \omega_1, \omega_2$ when applying $x$ as the input
	to $R_1$ and $R_2$.

\item
	Output the result of running $\prog{1}(x, f_1, \pi_1, f_2, \pi_2)$.
\end{enumerate}

Program $\prog{1}$ has $\pk_1$, $\pk_2$, $\sk_1$, $e_1$, and $e_2$
embedded.
\begin{enumerate}
\item
	Check whether $R_1(x) = f_1 \land R_2(x) = f_2$.
	$\pi_1$ and $\pi_2$ enable this check in logarithmic depth.
\item
	If the check succeeds, output $\Dec(\sk_1, f_1)$;
	otherwise output $\bot$.
\end{enumerate}

The use of two key pairs and two encryptions of $C$, similar to
CCA1-secure schemes seen previously, eliminates the virtual black-box
requirement for concealing $\sk_1$ within $\iO_{\NC{1}}(\prog{1})$.

\subsection{Proof of Security}

We prove the indistinguishability property for this construction
through a hybrid argument.

\newcommand{\Hyb}[1]{\ensuremath{\mathsf{H_{#1}}}}
\begin{proof}
Through the sequence of hybrids, we gradually transform an obfuscation
of circuit $C_1$ into an obfuscation of circuit $C_2$, with each
successor being indistinguishable from its antecedent.
\begin{description}
\item[$\Hyb{0}$]:
	This corresponds to an honest execution of $\iO(C_1)$.
	Recall that $e_1 = \Enc(\pk_1, C_1)$, $e_2 = \Enc(\pk_2, C_1)$,
	and $\sigma = (\iO_{\NC{1}}(\prog{1}), \ldots)$.

\item[$\Hyb{1}$]:
	We instead generate $e_2 = \Enc(\pk_2, C_2)$, relying on the
	semantic security of the underlying fully homomorphic encryption
	scheme.

\item[$\Hyb{2}$]:
	We alter program $\prog{2}$ such that it instead embeds $\sk_2$
	and outputs $\Dec(\sk_2, f_2)$.
	The output of the obfuscation procedure becomes
	$\sigma = (\iO_{\NC{1}}(\prog{2}, \ldots)$;
	we rely on the properties of functional equivalence and
	indistinguishability of $\iO_{\NC{1}}$.

\item[$\Hyb{3}$]:
	We generate $e_1 = \Enc(\pk_1, C_1)$ since $\sk_1$ is now
	unused, relying again on the semantic security of the fully
	homomorphic encryption scheme.

\item[$\Hyb{4}$]:
	We revert to the original program $\prog{1}$ and arrive
	at an honest execution of $\iO(C_1)$.

\end{description}
\end{proof}


\section{Identity-Based Encryption}

Another use of indistinguishability obfuscation is to realize
identity-based encryption (IBE).

\newcommand{\SETUP}{\ensuremath{\mathsf{Setup}}}
\newcommand{\KEYGEN}{\ensuremath{\mathsf{KeyGen}}}
\newcommand{\mpk}{\ensuremath{\mathsf{mpk}}}
\newcommand{\msk}{\ensuremath{\mathsf{msk}}}
\newcommand{\id}{\ensuremath{\mathsf{id}}}

\begin{definition}[Identity-Based Encryption]
An \emph{identity-based encryption scheme} is a tuple of PPT algorithms
$(\SETUP, \KEYGEN, \Enc, \Dec)$ as follows:
\begin{itemize}
\item
	$\SETUP(1^\kappa)$ generates and outputs a master public/private
	key pair $(\mpk, \msk)$.
\item
	$\KEYGEN(\msk, \id)$ derives and outputs a secret key
	$\sk_{\id}$ for identity $\id$.
\item
	$\Enc(\mpk, \id, m)$ encrypts message $m$ under identity $\id$
	and outputs the ciphertext.
\item
	$\Dec(\sk_{\id}, c)$ decrypts ciphertext $c$ and outputs the
	corresponding message if $c$ is a valid encryption under
	identity $\id$, or $\bot$ otherwise.
\end{itemize}
\end{definition}

\newcommand{\SIGN}{\ensuremath{\mathsf{Sign}}}
\newcommand{\VERIFY}{\ensuremath{\mathsf{Verify}}}

We combine an indistinguishability obfuscator $\iO$ with a digital
signature scheme $(\GEN, \SIGN, \VERIFY)$.
\begin{itemize}
\item
	Let $\SETUP \equiv \GEN$ and $\KEYGEN \equiv \SIGN$.
\item
	$\Enc$ outputs $\iO(P_m)$, where $P_m$ is a program that
	outputs (embedded) message $m$ if input $\sk$ is a secret key for
	the given $\id$, or $\bot$ otherwise.
\item
	$\Dec$ outputs the result of $c(\sk_{\id})$.
\end{itemize}
However, this requires that we have encryption scheme where the
``signatures'' do not exist.
We therefore investigate an alternative scheme.
%
%\newcommand{\Com}{\ensuremath{\mathsf{Com}}}
%
Let $(K, P, V)$ be a non-interactive zero-knowledge (NIZK) proof system.
Denote by $\Com(\cdot ; r)$ the commitment algorithm of a non-interactive
commitment scheme with explicit random coin $r$.

\begin{itemize}
\item
	Let $\sigma$ be a common random string.
	$\SETUP(1^\kappa)$ outputs $(\mpk = (\sigma, c_1, c_2), \msk =
	r_1)$, where $c_1 = \Com(0 ; r_1)$ and
	$c_2 = \Com(0^{|\id|} ; r_2)$.

\item
	$\KEYGEN(\msk, \id)$ produces a proof
	$\pi = P(\sigma, x_{\id}, s)$ for the following language $L$:
	$x \in L$ if there exists $s$ such that
\begin{equation*}
\underbrace{c_1 = \Com(0 ; s)}_{\text{Type I witness}} \lor
\underbrace{\left( c_2 = \Com(\id^* ; s) \land \id^* \ne \id
	\right)}_{\text{Type II witness}}
\end{equation*}

\item
	Let $P_{\id,m}$ be a program which outputs $m$ if
	$V(\sigma, x_{\id}, \pi_{\id}) = 1$ or outputs $\bot$ otherwise.

	$\Enc(\mpk, \id, m)$ outputs $\iO(P_{\id,m})$.
\end{itemize}

We briefly sketch the hybrid argument:
\begin{description}
\item[$\Hyb{0}$]:
	This corresponds to an honest execution as described above.
\item[$\Hyb{1}$]:
	We let $c_2 = \Com(\id^* ; r_2)$, relying on the hiding property
	of the commitment scheme.
\item[$\Hyb{2}$]:
	We switch to the Type II witness using
	$\pi_{\id_i} \forall i \in [q]$, corresponding to the queries
	issued by the adversary during the first phase of the
	selective-identity security game.
\item[$\Hyb{3}$]:
	We let $c_1 = \Com(1 ; r_1)$.
\end{description}

%\nocite{*}
%\printbibliography



% !TEX root = collection.tex

\newcommand{\extline}{$\scriptsize$-$\normalsize$\!}
\newcommand{\lextlineend}{$\scriptsize$\lhd\!$\normalsize$}
\newcommand{\rextlineend}{$\scriptsize\rule{.1ex}{0ex}$\rhd$\normalsize$}

\newcounter{index}

\newcommand\extlines[1]{%
  \setcounter{index}{0}%
  \whiledo {\value{index}< #1}
  {\addtocounter{index}{1}\extline}
}

\newcommand\rextlinearrow[2]{$
  \setbox0\hbox{$\extlines{#2}\rextlineend$}%
  \tiny$%
  \!\!\!\!\begin{array}{c}%
  \mathrm{#1}\\%
  \usebox0%
  \end{array}%
  $\normalsize$\!\!%
}

\newcommand\lextlinearrow[2]{$
  \setbox0\hbox{$\lextlineend\extlines{#2}$}%
  \tiny%
  $%
  \!\!\!\!\begin{array}{c}%
  \mathrm{#1}\\%
  \usebox0%
  \end{array}%
  $\normalsize$\!\!%
}

\renewcommand\lextlinearrow[2]{%
}

\renewcommand\rextlinearrow[2]{%
}
\renewcommand\lextlinearrow[2]{%
%  \setbox0\hbox{$\lextlineend\extlines{#2}$}%
   $\stackrel{\mathrm{#1}}{\leftarrow}$%
}

\renewcommand\rextlinearrow[2]{%
  %\setbox0\hbox{$\extlines{#2}\rextlineend$}%
  $\stackrel{\mathrm{#1}}{\rightarrow}$%
}



%\section{Indistinguishable Obfuscation Constructions using Puncturing}
\section{Digital Signature Scheme via Indistinguishable Obfuscation}
A digital signature scheme can be constructed via indistinguishable obfuscation (iO).  A digital signature scheme is made up of $(\SETUP, \SIGN, \VERIFY)$.\\

%\newcommand{\vk}{\mathsf{vk}}

\noindent $(\vk, \sk) \leftarrow \SETUP(1^k)$:\\
\indent $\sk$ = key of puncturable function and the seed of the PRF $F_k$\\
\indent $\vk = iO(P_k)$ where $P_k$ is the program:\\
\indent \indent $P_k(m, \sigma)$:\\
\indent \indent \indent for some OWF function $f$\\
\indent \indent \indent \indent return 1 if $f(\sigma) = f(F_k(m))$\\
\indent \indent \indent \indent return 0 otherwise\\

\noindent $\sigma \leftarrow \SIGN(\sk, m)$: Output $F_k(m)$.\\

\noindent $\VERIFY(\vk, m, \sigma)$: Output $P_k(m, \sigma)$.\\

\noindent Our security requirements will be that the adversary does wins the following game negligibly:\\

\begin{tabular}{llc}
{\large Challenger} & & {\large Adversary}\\
$(\vk, \sk) = \SETUP(1^k)$ and&&\\
picks random $m$&&\\
& \rextlinearrow{P_{k},m}{46} &\\
& \lextlinearrow{\sigma, m^*}{46} &\\
& Adversary wins game if $\VERIFY(\vk, m^*, \sigma) = 1$&
\end{tabular}\\

\noindent To prove the security of this system, we use a hybrid argument.  $H_0$ is as above.

\noindent $H_1$: Adjust $\vk$ so that $\vk = iO(P_{k, m, \alpha})$ where $\alpha = F_k(m)$ and $P_{k, m, \alpha}$ is the program such that:\\
\indent $P_{k,m, \alpha}(m^*, \sigma)$:\\
\indent \indent for some OWF $f$\\
\indent \indent \indent if $m = m^*$:\\
\indent \indent \indent \indent if $f(\sigma) = f(\alpha)$ return 1\\
\indent \indent \indent \indent otherwise return 0\\
\indent \indent \indent else proceed as $P_{k}$ from before\\
\indent \indent \indent \indent if $f(\sigma) = f(F_k(m^*))$ return 1\\
\indent \indent \indent \indent otherwise return 0\\
\noindent Note that this program does not change its output for any value. This is indistinguishable from $H_0$  by indistinguishability obfuscation.\\

\noindent $H_2$: Adjust $\alpha$ so that it is a randomly sampled value. The indistinguishability of $H_2$ and $H_1$ follows from the security of PRG.  \\
\noindent $H_3$: Adjust the program such that instead of $\alpha$ it relies on some $\beta$ that is compared instead $f(\alpha)$ in the third line.\\

Any adversary that can break $H_3$ non-negligibly can break the OWF $f$ with at the value $\beta$.

\section{Public Key Encryption via Indistinguishable Obfuscation}
A public key encryption scheme can be constructed via indistinguishable obfuscation.  A public key encryption scheme is made up of $(Gen, Enc, Dec)$.  The PRG used below is a length doubling PRG.\\

\noindent $(\pk, \sk) \leftarrow Gen(1^k)$\\
\indent $\sk$ = key of puncturable function and the seed of the PRF $F_k$\\
\indent $\pk = iO(P_k)$ where $P_k$ is the program:\\
\indent \indent $P_k(m, r)$:\\
\indent \indent \indent $t = PRG(r)$\\
\indent \indent \indent Output $c = (t, F_k(t) \oplus m)$\\

\noindent $Enc(\pk, m)$: Sample $r$ and output $(\pk(m,r))$.\\

\noindent $Dec(\sk = k, c = (c_1, c_2))$: Output $F_k(\sk,c_1) \oplus c_2$.\\

\noindent Our security requirements will be that the adversary does wins the following game negligibly:\\

\begin{tabular}{llc}
{\large Challenger} & & {\large Adversary}\\
$(\pk, \sk) = Gen(1^k)$ and&&\\
Randomly sample $b$ from $\{0,1\}$ and&&\\
$c^* = Enc(\pk, b)$ and&&\\
& \rextlinearrow{P_{k},c^*}{26} &\\
& \lextlinearrow{b^*}{26} &\\
& Adversary wins game if $b=b^*$&
\end{tabular}\\

\noindent To prove the security of this system, we use a hybrid argument.  $H_0$ is as above.

\noindent $H_1$: Adjust $\pk$ so that $\pk = iO(P_{k, \alpha, t})$ where $\alpha = F_k(t)$ and $P_{k, \alpha, t}$ is the program such that:\\
\indent $P_{k, \alpha, t}(m, r)$:\\
\indent \indent $t^* = PRG(r)$\\
\indent \indent if $t^* = t$, output $(t^*, \alpha \oplus m)$\\
\indent \indent else output $(t^*, F_k(t^*) \oplus m)$\\

\noindent Note that this program does not change its output for any value. This is indistinguishable from $H_0$  by indistinguishability obfuscation.\\

\noindent $H_2$: Adjust $\alpha$ so that it is a randomly sampled value.\\
\noindent $H_3$: Adjust the program such that $t^*$ is randomly sampled and is not in the range of the PRG.\\

Any adversary that can win $H_3$ can guess a random value non-negligibly.\\

\section{Indistinguishable Obfuscation Construction from $NC^1$ $iO$}
A construction of indistinguishable obfuscation from $iO$ for circuits in $NC^1$ is as follows:\\
Let $P_{k,C}(x)$ be the circuit that outputs the garbled circuit $\widetilde{UC(C,x)}$ with randomness $F_k(x)$ which is a punctured (at $k$) PRF in $NC^1$\\
\indent Note that $UC(C,x)$ outputs $C(x)$ ($UC$ is the ``universal'' circuit)\\
$iO(C) \rightarrow $ sample $k$ randomly from $\{0,1\}^{|x|}$ and output $iO_{NC^1}(P_{k,C})$ padded to a length $l$\\

As before, we use a hybrid argument to show the security for $iO$.\\
\noindent $H_0$: $iO(C) = iO_{NC^1}(P_{k,C})$ as above.\\
\noindent $H_{final} = H_{2^n}$: $iO(\pk, c_2)$\\
\noindent $H_1 \cdots H_i$: Create a program $Q_{k, c_1, c_2, i}(x)$ and obfuscate it.\\
$Q_{k,c_1,c_2,i}(x)$:\\
\indent Sample $k$ randomly\\
\indent if $x \ge i$, return $P_{k,c_1}(x)$\\
\indent else , return $P_{k,c_2}(x)$\\
\noindent Note that $H_i$ and $H_{i+1}$ are indistinguishable for any value other than $x=i$.\\
\noindent $H_{i,1}$ (between $H_i$ and $H_{i+1}$): Create a program $Q_{k, c_1, c_2, i, \alpha}(x)$, where $\alpha = Q_{k, c_1, c_2, i}(x)$ and obfuscate it.\\
$Q_{k, c_1, c_2, i, \alpha}(x)$:\\
\indent Sample $k$ randomly\\
\indent if $x = i$, return $\alpha$\\
\indent else , return $Q_{k,c_1,c_2, i}(x)$\\

\noindent $H_{i,2}$: Replace $\alpha$ with a random $\beta$ using fresh coins\\
\noindent $H_{i,3}$: Create the $c_2(x)$ value using fresh coins\\
\noindent $H_{i,4}$: Create the $c_2(x)$ value using $F_k(x)$\\
\noindent $H_{i,5}$: Finish the migration to $Q_{k,c_1,c_2,i+1}$\\

Note that at $H_{final}$, the circuit being obfuscated is completely changed from $c_1$ to $c_2$.






% The back matter contains appendices, bibliographies, indices, glossaries, etc.



\backmatter

\bibliography{cryptobib/abbrev0,cryptobib/crypto}
%\bibliographystyle{plainnat}
\bibliographystyle{plain}


%\printindex

\end{document}


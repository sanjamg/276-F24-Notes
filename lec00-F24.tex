\chapter{Mathematical Background}
\label{sec:mb}

In modern cryptography, (1) we typically assume that our attackers cannot run in unreasonably large amounts of time, and (2) we allow security to be broken with a \emph{very small}, but non-zero, probability.

Without these assumptions, we must work in the realm of information-theoretic cryptography, which is often unachievable or impractical for many applications. For example, the one-time pad
\footnote{For a message $m \in \{0,1\}^n$ and a random key $k \in \{0,1\}^n$, the encryption of $m$ is $c = m \oplus k$. The decryption is $m = c \oplus k$.}
-- an information-theoretically secure cipher -- is not very useful because it requires very large keys.

In this chapter, we define items (1) and (2) more formally. We require our adversaries to run in polynomial time, which captures the idea that their runtime is not unreasonably large (sections~\ref{ssec:ppt}). We also allow security to be broken with negligible -- very small -- probability (section ~\ref{ssec:nnf}). 


\section{Probabilistic Polynomial Time}
\label{ssec:ppt}
A probabilistic Turing Machine is a generic computer that is allowed to make random choices during its execution. A probabilistic \textit{polynomial time} Turing Machine is one which halts in time polynomial in its input length. More formally:

\begin{definition}[Probabilistic Polynomial Time]
A probabilistic Turing Machine $M$ is said to be PPT (a Probabilistic Polynomial Time Turing Machine) if $\exists c \in \mathbb{Z}^+$ such that $\forall x \in\{0,1\}^*$, $M(x)$ halts in $|x|^c$ steps.
\end{definition}

A {\em non-uniform} PPT Turing Machine is a collection of machines one for each input length, as opposed to a single machine that must work for all input lengths.

\begin{definition}[Non-uniform PPT]
A non-uniform PPT machine is a sequence of Turing Machines $\{ M_1, M_2, \cdots \}$ such that $\exists c \in \mathbb{Z}^+$ such that $\forall x \in\{0,1\}^*$, $M_{|x|}(x)$ halts in $|x|^c$ steps.
\end{definition}



\section{Noticeable and Negligible Functions}
\label{ssec:nnf}
Noticeable and negligible functions are used to characterize the ``largeness'' or ``smallness'' of a function describing the probability of some event.  Intuitively, a noticeable function is required to be larger than some inverse-polynomially function in the input parameter. On the other hand, a negligible function must be smaller than any inverse-polynomial function of the input parameter. More formally:


\begin{definition}[Noticeable Function]
A function $\mu(\cdot): \mathbb{Z}^+ \rightarrow [0,1]$ is noticeable iff $\exists c \in \mathbb{Z}^+, n_0 \in \mathbb{Z}^+$ such that $\forall n \geq n_0 , \; \mu(n) > n^{-c}$.
\end{definition}

\paragraph{Example.} Observe that $\mu(n) = n^{-3}$ is a noticeable function.  (Notice that the above definition is satisfied for $c = 4$ and $n_0 = 1$.)

\begin{definition}[Negligible Function]
A function $\mu(\cdot): \mathbb{Z}^+ \rightarrow [0,1]$ is negligible iff $\forall c \in \mathbb{Z}^+ \; \exists n_0 \in \mathbb{Z}^+$ such that $\forall n \geq n_0 , \; \mu(n) < n^{-c}$.
\end{definition}

\paragraph{Example.} $\mu(n) = 2^{-n}$ is an example of a negligible function. This can be observed as follows.
Consider an arbitrary $c \in \mathbb{Z}^+$ and set $n_0 = c^2$. Now, observe that for all $n \geq n_0$, we have that $\frac{n}{\log_2 n} \geq \frac{n_0}{\log_2 n_0} > \frac{n_0}{\sqrt{n_0}} = \sqrt{n_0} = c$. This allows us to conclude that $$\mu(n) = 2^{-n} = n^{-\frac{n}{\log_2 n}} < n^{-c}.$$

Thus, we have proved that for any $c \in \mathbb{Z}^+$, there exists $n_0 \in \mathbb{Z}^+$ such that for any $n \geq n_0$, $\mu(n) < n^{-c}$.

\paragraph{Gap between Noticeable and Negligible Functions.}
At first thought it might seem that a function that is {not} negligible (or, a non-negligible function) must be a noticeable. This is not true!\cite{JC:Bellare02} Negating the definition of a negligible function, we obtain that a non-negligible function $\mu(\cdot)$ is such that $\exists c \in \mathbb{Z}^+$ such that $\forall n_0 \in \mathbb{Z}^+$, $\exists n \geq n_0$ such that $\mu(n) > n^{-c}$.
Note that this requirement is satisfied as long as $\mu(n) > n^{-c}$ for infinitely many choices of $n \in \mathbb{Z}^+$. However, a noticeable function requires this condition to be true for every $n \geq n_0$.

Below we give example of a function $\mu(\cdot)$ that is neither negligible nor noticeable.
$$\mu(n) = \Big\{
\begin{array}{ll}
  2^{-n} & : x \mod 2 = 0\\
  n^{-3} & : x \mod 2 \neq 0
\end{array}
$$
This function is obtained by interleaving negligible and  noticeable functions. It cannot be negligible (resp., noticeable) because it is greater (resp., less) than an inverse-polynomially function for infinitely many input choices.

\paragraph{Properties of Negligible Functions.} Sum and product of two negligible functions is still a negligible function. We argue this for the sum function below and defer the problem for products to Exercise~\ref{ex:product}. These properties together imply that any polynomial function of a negligible function is still negligible.

\begin{exercise}
If $\mu(n)$ and $\nu(n)$ are negligible functions from domain $\mathbb{Z}^+$ to range $[0,1]$ then prove that the following functions are also negligible:
\begin{enumerate}
    \item $\psi_1(n) = \frac{1}{2} \cdot \left(\mu(n) + \nu(n)\right)$
    \item $\psi_2(n) = \min\{\mu(n) + \nu(n), 1\}$
    \item $\psi_3(n) = \mu(n)\cdot \nu(n)$
    \item $\psi_4(n) = \mathsf{poly}(\mu(n))$, where $\mathsf{poly}(\cdot)$ is an unspecified polynomial function. (Assume that the output is also clamped to $[0,1]$ to satisfy the definition)
\end{enumerate}function.
\end{exercise}
\proof 
$ $
\begin{enumerate}
    \item We need to show that for any $c \in \mathbb{Z}^+$, we can find $n_0$ such that $\forall n \geq n_0$, $\psi_1(n) \leq n^{-c}$. Our argument proceeds as follows. Given the fact that $\mu$ and $\nu$ are negligible we can conclude that there exist $n_1$ and $n_2$ such that $\forall n \geq n_1$, $\mu(n) < n^{-c}$ and $\forall n \geq n_2$, $g(n) < n^{-c}$. Combining the above two facts and setting $n_0 = \max(n_1, n_2)$ we have that for every $n \geq n_0$,
    \begin{align*}
        \psi_1(n) &= \frac{1}{2} \cdot (\mu(n) + \nu(n)) < \frac{1}{2} \cdot (n^{-c} + n^{-c}) = n^{-c}
    \end{align*}
    Thus, $\psi_1(n) \leq n^{-c}$ and hence is negligible.

    \item We need to show that for any $c \in \mathbb{Z}^+$, we can find $n_0$ such that $\forall n \geq n_0$, $\psi_2(n) \leq n^{-c}$. Given the fact that $\mu$ and $\nu$ are negligible, there exist $n_1$ and $n_2$ such that $\forall n \geq n_1$, $\mu(n) \leq n^{-c-1}$ and $\forall n \geq n_2$, $g(n) \leq n^{-c-1}$. Setting $n_0 = \max(n_1, n_2, 3)$ we have that for every $n \geq n_0$,
    \begin{align*}
        \psi_2(n) &= \min\{\mu(n) + \nu(n), 1\} < n^{-c-1} + n^{-c-1} < n^{-c}
    \end{align*}
\end{enumerate}
\qed

%\begin{corollary}
%If $f(n)$ is non-negligible and $g(n)$ is negligible, then $h(n) = f(n) - g(n)$ is non-negligible.
%\end{corollary}
%
%\proof If $h(n)$ was negligible, then $f(n) = g(n) + h(n)$ would be the sum of two negligible functions, but would be non-negligible, which is a contradiction.  \qed

\section{Computationally  Hard Problems}\label{sec:assumptions}
%So far, much of our investigation relied on the existence of one-way-functions or in certain cases on the existence of one-one one-way functions. However, just the mere existence of an object is not enough for real-world implementations. 
We will next provide certain number theoretical problems that are conjectured to be computationally intractable. We will use the conjectured hardness of these problems in subsequent chapters to o provide concrete instantiations.

\subsection{The Discrete-Log Family of Problem}
Consider a group $\mathbb{G}$ of prime order. For example, consider the group $\mathbb{Z}_p^*$ where $p$ is a large prime. Let $g$ be a generator of this group $\mathbb{G}$. In this group, given $g^x$ for a random $x \in \{1,\ldots p-1\}$ consider the problem of finding $x$. This problem, referred to as the discrete-log problem, is believed to be computationally hard.

The asymptotic definition of the discrete-log problem needs to consider an infinite family of groups or what we will a group ensemble. 

\paragraph{Group Ensemble.} A group ensemble is a set of finite cyclic groups $\mathcal{G} =\{ \mathbb{G}_n\}_{n \in \mathbb{N}}$. For the group $G_n$, we assume that given two group elements in $G_n$, their sum can be computed in polynomial in $n$ time. Additionally, we assume that given $n$ the generator $g$ of $\mathbb{G}_n$ can be computed in polynomial time. 

\begin{definition}[Discrete-Log Assumption]\label{def:dl}
We say that the discrete-log assumption holds for the group ensemble $\mathcal{G} =\{ \mathbb{G}_n\}_{n \in \mathbb{N}}$, if for every non-uniform PPT algorithm $\mathcal{A}$ we have that
\[\mu_\mathcal{A}(n) := \Pr_{x \leftarrow |G_n|}[\mathcal{A}(g,g^x) = x]\]
is a negligible function.
\end{definition}

\paragraph{The Diffie-Hellman Problems.} In addition to the discrete-log assumption, we also define the Computational Diffie-Hellman Assumption and the Decisional Diffie-Hellman Assumption. 

\begin{definition}[Computational Diffie-Hellman (CDH) Assumption]\label{def:cdh}
We say that the Computational Diffie-Hellman Assumption holds for the group ensemble $\mathcal{G} =\{ \mathbb{G}_n\}_{n \in \mathbb{N}}$, if for every non-uniform PPT algorithm $\mathcal{A}$ we have that
\[\mu_\mathcal{A}(n) := \Pr_{x,y \leftarrow |G_n|}[\mathcal{A}(g,g^x,g^y) = g^{xy}]\]
is a negligible function.
\end{definition}

\begin{definition}[Decisional Diffie-Hellman (DDH) Assumption]\label{def:ddh}
We say that the Computational Diffie-Hellman Assumption holds for the group ensemble $\mathcal{G} =\{ \mathbb{G}_n\}_{n \in \mathbb{N}}$, if for every non-uniform PPT algorithm $\mathcal{A}$ we have that
\[\mu_\mathcal{A}(n) = \mid\Pr_{x,y \leftarrow |G_n|}[\mathcal{A}(g,g^x,g^y,g^{xy}) = 1] - \Pr_{x,y,z \leftarrow |G_n|}[\mathcal{A}(g,g^x,g^y,g^{z}) = 1]\mid\]
is a negligible function.
\end{definition}

It is not hard to observe that the discrete-log assumption is the weakest of the three assumptions above. In fact, it is not difficult to show that the Discrete-Log Assumption for $\mathcal{G}$ implies the CDH and the DDH Assumptions for $\mathcal{G}$.  Additionally, we leave it as an exercise to show that the CDH Assumption for $\mathcal{G}$ implies the  DDH Assumptions for $\mathcal{G}$.

\paragraph{Examples of Groups where these assumptions hold.} Now we provide some examples of group where these assumptions hold. 
\begin{enumerate}
    \item Consider the group $\mathbb{Z}_p^*$ for a prime $p$.\footnote{Since the number of primes is infinite we can define an infinite family of such groups. For the sake of simplicity, here we only consider a single group.} For this group the CDH Assumption is conjectured to be true. However, using the Legendre symbol,\footnote{Let $p$ be an odd prime number. An integer $a$ is said to be a \emph{quadratic residue} modulo $p$ if it is congruent to a perfect square modulo $p$ and  is said to be a \emph{quadratic non-residue} modulo $p$ otherwise. The \emph{Legendre symbol} is a function of $a$ and $p$ defined as
    \begin{equation*}
        \left(\frac{a}{p}\right) = \begin{cases}
    1 &\text{if $a$ is quadration residue mod $p $ and $a \not\equiv 0 \mod p$}\\
    -1 &\text{if $a$ is quadration non-residue mod $p $}\\
    0 &\text{if $a \equiv 0 \mod p $}
    \end{cases}
    \end{equation*}
    
Legendre symbol can be efficiently computed as $\left(\frac{a}{p}\right) = a^{\frac{p-1}{2}}\mod p$.} 
the DDH Assumption in this group can be shown to be false. Can you show how?\footnote{This is because given $g^x, g^y$ one can easily compute deduce the Legendre symbol of $g^{xy}$.  Observe that if $\left(\frac{g}{p}\right) = -1$ then we have that $\left(\frac{g^{xy}}{p}\right) = 1$ if and only if $ \left(\frac{g^x}{p}\right) =1 $ or $\left(\frac{g^y}{p}\right) = 1$. Using this fact, we can construct an adversary that breaks the DDH problem with a non-negligible (in fact, noticeable) probability.}
\item Let $p = 2q+1$ where both $p$ and $q$ are prime.\footnote{By Dirichet's Theorem on primes in arithmetic progression, we have that there are infinite choices of primes $(p,q)$ for which $p = 2q+1$. This allows us to generalize this group to a group ensemble.} Next, let $\mathbb{Q}$ be the order-$q$ subgroup of quadratic residues in $\mathbb{Z}^*_p$. For this group, the DDH assumption is believed to hold. 
\item Let $N = pq$ where $p,q, \frac{p-1}{2}$ and $\frac{q-1}{2}$ are primes. Let $\mathbb{QR}_N$ be the cyclic subgroup of qudratic resides of order $\phi(N) = (p-1)(q-1)$. For this group $\mathbb{QR}_N$, the DDH assumption is also believed to hold.
%\item Pairings:
\end{enumerate}

\paragraph{Is DDH strictly stronger than Discrete-Log?} In the example cases above, where DDH is known believed to be hard, the
best known algorithms for DDH are no better than the best known algorithms for the discrete-log problem. Whether the DDH assumption is strictly stronger than the discrete-log assumption is an open problem. 



\subsection{CDH in $\mathbb{QR}_N$ implies Factoring}

In this section, we will show that the CDH assumption in $\mathbb{QR}_N$ implies the factoring assumption. 
\begin{lemma}
Given an algorithm $\mathcal{A}$ that breaks the CDH assumption in $\mathbb{QR}_N$, we construct an non-uniform PPT adversary $\mathcal{B}$ that on input $N$ outputs its prime factors $p$ and $q$.
\end{lemma}
\begin{proof}
Given that $\mathcal{A}$ is an algorithm that solves the CDH problem in $\mathbb{QR}_N$ with a non-negligible probability, we construct an algorithm $\mathcal{B}$ that can factor $N$. Specifically, $\mathcal{B}$ on input $N$ proceeds as follows:
\begin{enumerate}
\item Sample $v \leftarrow \mathbb{QR}_N$ (such a $v$ can be obtained by sampling a random value in $\mathbb{Z}_N^*$ and squaring it) and compute $g := v^2 \mod N$.
\item Sample $x, y \leftarrow [N]$.\footnote{Note that sampling $x,y$ uniformly from $[N]$ is statistically close to sampling $x,y$ uniformly from $[\phi(N)]$.}
\item Let $ u := \mathcal{A}(g, g^{x}\cdot v, g^y\cdot v)$\footnote{Note that $g^x\cdot v$ where $x \leftarrow [N]$ is statistically close to $g^x$ where $x \leftarrow [N]$.} and compute $w := \frac{u}{g^{xy}\cdot v^{x+y}}$.
\item If $w^2 = v^2 \mod N$ and $u \neq \pm v$, then compute the factors of $N$ as $\mathsf{gcd}(N,u+v)$ and $N/\mathsf{gcd}(N,u+v)$. Otherwise, output $\bot$.
\end{enumerate}
Observe that if $\mathcal{A}$ solves the CDH problem then the returned values $u = g^{(x+ 2^{-1})(y + 2^{-1})} = v^{2xy + x+ y + 2^{-1}}$. Consequently, the computed value $w = v^{2^{-1}}$. Furthermore, with probability $\frac12$ we have that $w \neq v$. In this case, $\mathcal{B}$ can factor $N$.
\end{proof}
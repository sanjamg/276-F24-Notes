\chapter{Mathematical Background}
\label{sec:mb}

In modern cryptography, we typically assume that our attackers cannot run in unreasonably large amounts of time, and we allow security to be broken with a \emph{very small}, but non-zero, probability.

Without these assumptions, we must work in the realm of information-theoretic cryptography, which is often unachievable or impractical for many applications. For example, the one-time pad --- an information-theoretically secure cipher --- is not very useful because it requires very large keys.

In this chapter, we define items (1) and (2) more formally. We require our adversaries to run in polynomial time, which captures the idea that their runtime is not unreasonably large (sections~\ref{ssec:ppt}). We also allow security to be broken with negligible -- very small -- probability (section ~\ref{ssec:nnf}). 


\section{Probabilistic Polynomial Time}
\label{ssec:ppt}
A probabilistic Turing Machine is a generic computer that is allowed to make random choices during its execution. A probabilistic \textit{polynomial time} Turing Machine is one which halts in time polynomial in its input length. More formally:

\begin{definition}[Probabilistic Polynomial Time]
A probabilistic Turing Machine $M$ is said to be PPT (a Probabilistic Polynomial Time Turing Machine) if $\exists c \in \mathbb{Z}^+$ such that $\forall x \in\{0,1\}^*$, $M(x)$ halts in $|x|^c$ steps.
\end{definition}

A {\em non-uniform} PPT Turing Machine is a collection of machines one for each input length, as opposed to a single machine that must work for all input lengths.

\begin{definition}[Non-uniform PPT]
A non-uniform PPT machine is a sequence of Turing Machines $\{ M_1, M_2, \cdots \}$ such that $\exists c \in \mathbb{Z}^+$ such that $\forall x \in\{0,1\}^*$, $M_{|x|}(x)$ halts in $|x|^c$ steps.
\end{definition}



\section{Noticeable and Negligible Functions}
\label{ssec:nnf}
Noticeable and negligible functions are used to characterize the ``largeness'' or ``smallness'' of a function describing the probability of some event.  Intuitively, a noticeable function is required to be larger than some inverse-polynomially function in the input parameter. On the other hand, a negligible function must be smaller than any inverse-polynomial function of the input parameter. More formally:


\begin{definition}[Noticeable Function]
A function $\mu(\cdot): \mathbb{Z}^+ \rightarrow [0,1]$ is noticeable iff $\exists c \in \mathbb{Z}^+, n_0 \in \mathbb{Z}^+$ such that $\forall n \geq n_0 , \; \mu(n) \geq n^{-c}$.
\end{definition}

\paragraph{Example.} Observe that $\mu(n) = n^{-3}$ is a noticeable function.  (Notice that the above definition is satisfied for $c = 3$ and $n_0 = 1$.)

\begin{definition}[Negligible Function]
A function $\mu(\cdot): \mathbb{Z}^+ \rightarrow [0,1]$ is negligible iff $\forall c \in \mathbb{Z}^+ \; \exists n_0 \in \mathbb{Z}^+$ such that $\forall n \geq n_0 , \; \mu(n) < n^{-c}$.
\end{definition}

\paragraph{Example.} $\mu(n) = 2^{-n}$ is an example of a negligible function. This can be observed as follows.
Consider an arbitrary $c \in \mathbb{Z}^+$ and set $n_0 = c^2$. Now, observe that for all $n \geq n_0$, we have that $\frac{n}{\log_2 n} \geq \frac{n_0}{\log_2 n_0} \geq \frac{n_0}{\sqrt{n_0}} = \sqrt{n_0} = c$. This allows us to conclude that $$\mu(n) = 2^{-n} = n^{-\frac{n}{\log_2 n}} \leq n^{-c}.$$

Thus, we have proved that for any $c \in \mathbb{Z}^+$, there exists $n_0 \in \mathbb{Z}^+$ such that for any $n \geq n_0$, $\mu(n) \leq n^{-c}$.

\paragraph{Gap between Noticeable and Negligible Functions.}
At first thought it might seem that a function that is {not} negligible (or, a non-negligible function) must be a noticeable. This is not true!\cite{JC:Bellare02} Negating the definition of a negligible function, we obtain that a non-negligible function $\mu(\cdot)$ is such that $\exists c \in \mathbb{Z}^+$ such that $\forall n_0 \in \mathbb{Z}^+$, $\exists n \geq n_0$ such that $\mu(n) \geq n^{-c}$.
Note that this requirement is satisfied as long as $\mu(n) \geq n^{-c}$ for infinitely many choices of $n \in \mathbb{Z}^+$. However, a noticeable function requires this condition to be true for every $n \geq n_0$.

Below we give example of a function $\mu(\cdot)$ that is neither negligible nor noticeable.
$$\mu(n) = \Big\{
\begin{array}{ll}
  2^{-n} & : x \mod 2 = 0\\
  n^{-3} & : x \mod 2 \neq 0
\end{array}
$$
This function is obtained by interleaving negligible and  noticeable functions. It cannot be negligible (resp., noticeable) because it is greater (resp., less) than an inverse-polynomially function for infinitely many input choices.

\paragraph{Properties of Negligible Functions.} Sum and product of two negligible functions is still a negligible function. We argue this for the sum function below and defer the problem for products to Exercise~\ref{ex:product}.

\begin{exercise}
If $\mu(n)$ and $\nu(n)$ are negligible functions from domain $\mathbb{Z}^+$ to range $[0,1]$ then prove that the following functions are also negligible:
\begin{enumerate}
    \item $\psi_1(n) = \frac{1}{2} \cdot \left(\mu(n) + \nu(n)\right)$
    \item $\psi_2(n) = \mu(n)\cdot \nu(n)$
    \item $\psi_3(n) = \mathsf{poly}(\mu(n))$, where $\mathsf{poly}(\cdot)$ is an unspecified polynomial function.
\end{enumerate}function.
\end{exercise}
\proof 
$ $
\begin{enumerate}
    \item We need to show that for any $c \in \mathbb{Z}^+$, we can find $n_0$ such that $\forall n \geq n_0$, $\psi_1(n) \leq n^{-c}$. Our argument proceeds as follows. Given the fact that $\mu$ and $\nu$ are negligible we can conclude that there exist $n_1$ and $n_2$ such that $\forall n \geq n_1$, $\mu(n) \leq n^{-c}$ and $\forall n \geq n_2$, $g(n) \leq n^{-c}$. Combining the above two facts and setting $n_0 = \max(n_1, n_2)$ we have that for every $n \geq n_0$,
    \begin{align*}
        \psi_1(n) &= \frac{1}{2} \cdot (\mu(n) + \nu(n)) \leq \frac{1}{2} \cdot (n^{-c} + n^{-c}) = n^{-c}
    \end{align*}
    Thus, $\psi_1(n) \leq n^{-c}$ and hence is negligible.
\end{enumerate}
\qed

%\begin{corollary}
%If $f(n)$ is non-negligible and $g(n)$ is negligible, then $h(n) = f(n) - g(n)$ is non-negligible.
%\end{corollary}
%
%\proof If $h(n)$ was negligible, then $f(n) = g(n) + h(n)$ would be the sum of two negligible functions, but would be non-negligible, which is a contradiction.  \qed


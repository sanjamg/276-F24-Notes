\chapter{Advanced Encryption Schemes}
\section{Identity-Based Encryption}

We introduce Bilinear Maps and two of its applications: NIKE, Non-Interactive Key Exchange; and IBE, Identity Based Encryption.


\section{Diffie-Hellman Key Exchange}

\begin{figure}
\label{fig:dh}
\centering
  \includegraphics[width=0.7\textwidth]{Old Scribe Notes/fig1.pdf}
\caption{Diffie-Hellman Key Exchange}
\end{figure}


Fig \ref{fig:dh} illustrates Diffie-Hellman key exchange. Alice and Bob each has a private key ($a$ and $b$ respectively), and they want to build a shared key for symmetric encryption communication. They can only communicate over a insecure link, which is eavesdropped by Eve.
So Alice generates a public key $A$ and Bob generates a public key $B$, and they send their public key to each other at the same time. Then Alice generates the shared key $K$ from $a$ and $B$, and likewise, Bob generates the shared key $K$ from $b$ and $A$.
And we have $\forall$ PPT Eve, $Pr[k=Eve(A,B)]=neg(k)$, where $k$ is the length of $a$.


\subsection{Discussion 1}

Assume that $\forall (g, p)$, and $a_1,b_1 \stackrel{\$}{\gets} Z^*_p$, and $a_2,b_2,r \stackrel{\$}{\gets}Z^*_p$, we have $(g^{a_1}, g^{b_1}, g^{a_1b_1}) \stackrel{c}{\simeq} (g^{a_2}, g^{b_2}, g^r)$. How to apply this to Diffie-Hellman Key Exchange?


Make $A=g^a$, $B=g^b$, $K=A^b=g^{ab}$, and $K=B^a=g^{ab}$.


\subsection{Discussion 2}


How does Diffie-Hellman Key Exchange imply Public Key Encryption?


Alice
$pk = A$, $sk = a$, $Enc(pk, m \in \{0, 1\})$.

Bob
$b,r \gets Z^*_p$
$(g^b, mA^b+(1-m)g^r)$

Alice $Dec(sk, (c_1, c_2))$

$c_1^a \stackrel{?}{=} c_2$




\section{Bilinear Maps}

\begin{definition}{Bilinear Maps}

Bilinear Maps is $(G,P,G_T,g,e)$, where $e$ is an efficient function $G \times G \to G_T$ such that

\begin{itemize}
\item if $g$ is generator of $G$, then $e(g, g)$ is the generator of $G_T$.
\item $\forall a,b \in Z_p$, we have $e(g^a, g^b) = e(g, g)^{ab} = e(g^b, g^a)$.
\end{itemize}

\end{definition}

\subsection{Discussion 1}


How does Bilinear Maps apply to Diffie-Hellman?

Make $A=g^a$, $B=g^b$, and $T=g^{ab}$, then Diffie-Hellman has $e(A, B)=e(g, T)$.


\section{Tripartite Diffie-Hellman}

\begin{figure}
\label{fig:3dh}
\centering
  \includegraphics[width=0.7\textwidth]{Old Scribe Notes/fig2.pdf}
\caption{Tripartite Diffie-Hellman Key Exchange}
\end{figure}

Fig \ref{fig:3dh} illustrates Tripartite Diffie-Hellman key exchange. $a$, $b$, and $c$ are private key of Alice, Bob, and Carol, respectively.
They use $g^a$, $g^b$, $g^c$ as public key, and the shared key $K=e(g,g)^{abc}$.
Formally, we have
$$a,b,c \stackrel{\$}{\gets} Z^*_p, r \stackrel{\$}{\gets} Z^*_p$$
$$A=g^a, B=g^b, C=g^c$$
$$K=e(g,g)^{abc}$$



\section{IBE: Identity-Based Encryption}
When two parties communicate secure messages through a public key infrastructure, they need to go through a time-consuming and error-prone process to get each other's key and verify each other's identity through a Certificate Authority. 
Identity-based cryptography (IBC) seeks to reduce these barriers by requiring no preparation on the part of the message recipient, therefore saving the initial round trip. 
Identity based encryption can also be used to construct CCA-secure public key encryption and digital signatures. 

IBE contains four steps: \emph{Setup}, \emph{KeyGen}, \emph{Enc}, and \emph{Dec}. We illustrate it in Figure \ref{fig:ibe}.
IBE relies on a trusted third party called Private Key Generator(PKG). 
In first step, PKG gets a Master Public Key ($mpk$) and Master Signing Key ($msk$) from $Gen(1^n)$. 
Then a user with an ID (in this example, ``Mike''), sends his ID to the PKG. 
The PKG generates the Signing Key of Mike with $KeyGen(msk, id)$ ans sends it back. 
Another user, Alice, wants to send an encrypted message to Mike. 
She only has $mpk$ and Mike's ID. 
So she encrypts the message with $c=Enc(mpk, id=Mike, m)$, and sends the encrypted message $c$ to Mike. 
Mike decodes $c$ with $m=Dec(c, sk_{Mike})$. 
Notice that Alice never need to know Mike's public key. 
She only needs to remember MPK and other people's IDs.

\begin{figure}
\label{fig:ibe}
\centering
  \includegraphics[width=0.7\textwidth]{Old Scribe Notes/fig3.pdf}
\caption{Identity-Based Encryption}
\end{figure}

Then we define IBE formally, 

\begin{definition}[Identity-Based Encryption]
    An \textbf{identity based encryption scheme} $\mathcal{E}_{id} = (G,K,E,D)$ is a tuple of four efficient algorithms: a \textbf{setup algorithm} $G$, a \textbf{key generation algorithm} $K$, an \textbf{encryption algorithm} $E$, and a \textbf{decryption algorithm} $D$.
    \begin{itemize}
        \item $G$ is a probabilistic algorithm invoked as $(mpk, msk)\stackrel{\$}{\gets} G(1^n)$, where $mpk$ is called the \textbf{master public key} and $msk$ is called the \textbf{master secret key} for the IBE scheme.
        \item $K$ is a probabilistic algorithm invoked as $sk_{id}\stackrel{\$}{\gets}K(msk,id )$, where $msk$ is the master secret key (as output by $S$), $id \in \mathcal{ID}$ is an identity, and $sk_{id}$ is a secret key for id.
        \item $E$ is a probabilistic algorithm invoked as $c\stackrel{\$}{\gets}E(mpk,id, m)$.
        \item $D$ is a deterministic algorithm invoked as $m\gets D(sk_{id}, c)$. Here $m$ is either a message or a special reject value $\bot$ (distinct from all messages).
    \end{itemize}
\end{definition}

As usual, we define the correctness of IBE to be the decryption undoes encryption, formally we have 
\begin{definition}[Correctness of IBE]
$\forall n, id, m$, we have
\[
Pr\begin{bmatrix}
   (mpk,msk) \gets G(1^n), \\[0.3em]
   sk_{id} \gets K(msk, ID), \\[0.3em]
   c \gets E(mpk, id, m), \\[0.3em]
   m \gets D(sk_{id}, c)
\end{bmatrix}
  =1 - \text{negl}(n)
\]
\end{definition}

Next we define the security of IBE scheme. 
The basic security definition considers an adversary who obtains the secret keys for a number of identities of its choice. 
The adversary should not be able to break semantic security for some other identity of its choice for which it does not have the secret key.

\begin{definition}[Security of IBE]
    For an IBE scheme $\mathcal{E}_{id}=(G,K,E,D)$ is secure if $\forall$ nuPPT $\mathcal{A}$,
    \[
    \Pr[\text{Exp}_{\pi,\mathcal{A}}^{IBE,CPA}(n)=1]=\text{negl}(n)
    \]
\end{definition}

We then define the experiment $\text{Exp}_{\pi,\mathcal{A}}^{IBE,CPA}$.
\begin{definition}[IBE-CPA Experiment]
We denote the experiment in the following order: 
\begin{enumerate}
    \item The challenger invokes $(mpk, msk)\stackrel{\$}{\gets} G(1^n)$ and send $mpk$ to adversary $\mathcal{A}$.
    \item $\mathcal{A}$ can make multiple key queries and generate desired ID $id^{*}$ and two message $(m_0, m_1)$. 
    \item Challenger random selects $b\stackrel{\$}{\gets}\{0,1\}$ and encrypt $c^{*}=E(mpk, id^{*},m_b)$ and send $c^{*}$ to $\mathcal{A}$.
    \item $\mathcal{A}$ can make more encryption queries based on $c^{*}$ and other id and message $(m_0,m_1)$. Note that throughout the process $\mathcal{A}$ can \textbf{never} make any query on $id^{*}$. 
    \item At the end, $\mathcal{A}$ generate $b'$ and send to challenger. 
    \item Challenger output $1$ if $b=b'$, $0$ otherwise. 
\end{enumerate}
\end{definition}

\subsection{IBE Construction from Pairing}
We then present a concrete IBE construction from pairing.
First, we will give the hardness assumption in this scheme called \textbf{bilinear Diffe-Hellman} assumption, or BDH. 
This assumption says that given random element $g_0^{\alpha},g_0^{\beta},g_0^{\gamma}\in\mathbb{G_0}$ and a few additional terms, the quantity $e(g0,g1)^{\alpha\beta\gamma}\in\mathbb{G}_T$ is computationally indistinguishable from a random element in GT.

\begin{definition}[Decisional bilinear Diffe-Hellman]
    Let $e: \mathbb{G}_0\times\mathbb{G}_1\rightarrow\mathbb{G}_T$ be a pairing where $\mathbb{G}_0,\mathbb{G}_1,\mathbb{G}_T$ are cyclic groups of prime order $q$ with generators $g_0 \in \mathbb{G}_0$ and $g_1 \in \mathbb{G}_1$. For a given adversary $\mathcal{A}$, the following distribution is distinguishable: 
    \[
    \{g_0^{\alpha},g_1^{\alpha},g_0^{\beta},g_1^{\gamma},e(g_0,g_1)^{\alpha\beta\gamma},  \alpha,\beta,\gamma\stackrel{\$}{\gets}\mathbb{Z}_q\} \approx^{c}
    \{g_0^{\alpha},g_1^{\alpha},g_0^{\beta},g_1^{\gamma},e(g_0,g_1)^{\delta},  \alpha,\beta,\gamma,\delta\stackrel{\$}{\gets}\mathbb{Z}_q\}
    \]
\end{definition}

Note that this assumption work even with $g_0=g_1$.

We then present our IBE construction. 
\begin{itemize}
    \item $G(1^n)$: 
    \[
    \alpha\stackrel{\$}{\gets}\mathbb{Z}_q, mpk\gets g^\alpha, msk\gets\alpha
    \]
    and output ($mpk,msk$)
    \item $K(msk=\alpha,id)$:
    \[
    sk_{id}\gets H(id)^\alpha
    \]
    where $H$ is a hash function $H:\{0,1\}^{*}\rightarrow\mathbb{G}$
    \item $E(mpk, id, m)$: 
    \[
    \beta\stackrel{\$}{\gets}\mathbb{Z}_q, c_1\gets g^{\beta}, c_2\gets e(mpk,H(id)^{\beta})\cdot m
    \]
    and output $(c_1,c_2)$.
    \item $D(sk_{id}, c=(c_1, c_2))$:
    \[
    m=\frac{c_2}{e(c_1,sk_{id})}
    \]
\end{itemize}
By the property of bilinear map, we can verify the correctness of this scheme, 
\begin{align*}
    m &= \frac{c_2}{e(c_1,sk_{id})} \\
      &= \frac{e(mpk,H(id)^{\beta})\cdot m}{e(c_1,sk_{id})} \\
      &= \frac{e(g^\alpha,H(id)^{\beta})\cdot m}{e(g^{\beta}, H(id)^{\alpha})} \\
      &= \frac{e(g,H(id))^{\alpha\beta}}{e(g, H(id))^{\alpha\beta}}\cdot m \\
      &= m
\end{align*}

We then prove the security property under random oracle model.
\begin{theorem}
    If decision BDH holds for e, H is modelled as a random oracle, then the above construction is a secure IBE scheme. 
\end{theorem}
\begin{proof}
    Let $\mathcal{A}$ be an adversary that breaks the IBE scheme, we can construct another adversary $\mathcal{B}$ such that it breaks the DBDH assumption.

    The adversary $\mathcal{B}$ works as follows:
    \begin{enumerate}
        \item $\mathcal{B}$ receives 5 elements as input $\{g^{\alpha},g^{\beta},g^{\gamma},z,  \alpha,\beta,\gamma\stackrel{\$}{\gets}\mathbb{Z}_q,z\in\mathbb{G}\}$, and $\mathcal{B}$ need to determine whether $z=e(g,g)^{\alpha\beta\gamma}$ or not. 
        \item $\mathcal{B}$ send IBE public parameter $mpk=g^{\alpha}$ to IBE adversary $\mathcal{A}$.
        \item Then $\mathcal{A}$ will make multiple $sk_{id}$ queries to $\mathcal{B}$. $\mathcal{B}$ responds them by (1) Choose $\rho\stackrel{\$}{\gets}\mathbb{Z}_q$ (2) Setting $H(id)=g^{\rho}$ (3) set the secret key be $sk_{id}=H(id)^{\alpha}=g^{\rho\cdot\alpha}$. One \textbf{exception} is that $\mathcal{B}$ will set a random $id'$ whose $H(id')=g^\beta$.
        \item After receiving the key query, $\mathcal{A}$ outputs $(id^{*}, m_0,m_1)$.  
        \item When $\mathcal{B}$ receives encryption query $(id^{*}, m_0, m_1)$, it first check if $\mathcal{A}$ have previously query $sk_{id^*}$ before, if yes, then abort. Otherwise, $\mathcal{B}$ choose $b\stackrel{\$}{\gets}\{0,1\}$ and encrypt $m_b$ using $(c_1=g^\gamma, c_2=e(msk,H(id')^\gamma)\cdot m_b)$ and send back to $\mathcal{A}$.
        \item $\mathcal{A}$ eventually output $b'$ to $\mathcal{B}$. And $\mathcal{B}$ output $1$ if $b'=b$ and $0$ otherwise. 
    \end{enumerate}
    Since $e(msk,H(id')^{\gamma})=e(g^\alpha,g^{\beta\gamma})=e(g,g)^{\alpha\beta\gamma}$, $\mathcal{B}$ can embedded the challenge to $\mathcal{A}$ and break DBDH. \qed
\end{proof}

\subsection{Digital Signature from IBE}
We can directly derive a secure signature scheme from IBE. 
Given a secure IBE $\mathcal{E}_{id}=(G,K,E,D)$ with id space $\mathcal{ID}$ and message space $\mathcal{M}_{IBE}$, we construct a secure digital signature scheme $\mathcal{S} = (G',S',V')$ as follows: 
\begin{itemize}
    \item $G'(1^n)$: run $(mpk, msk)\stackrel{\$}{\gets} G(1^n)$ and output $(mpk, msk)$ as the sign key pair, with $mpk$ as verification key and $msk$ as sign key.
    \item $S'(msk,m)$: Given message $m\in\mathcal{ID}$, compute $\sigma\stackrel{\$}{\gets}K(msk,m)$, output $\sigma$ as signature. 
    \item $V'(mpk,m,\sigma)$: Choose $r\stackrel{\$}{\gets}\mathcal{M}_{IBE}$, compute $c\stackrel{\$}{\gets}E(mpk,m,r)$, and accept if $D(\sigma,c) = r$. 
\end{itemize}

We have the following theorem,
\begin{theorem}
    Let $\mathcal{E}_{id}$ be a secure IBE with message space is super-poly. Then the derive signature scheme $\mathcal{S}$ is a secure digital signature scheme. 
\end{theorem}
\begin{proof}
    Let $\mathcal{A}$ be an adversary that breaks the digital signature scheme, we can construct another adversary $\mathcal{B}$ such that breaks the IBE security.

    The BIE adversary $\mathcal{B}$ is modelled as follows:
    \begin{enumerate}
        \item $\mathcal{B}$ receives $mpk$ from the challenger, and forward $mpk$ as a signature public key to $\mathcal{A}$. 
        \item $\mathcal{A}$ makes a series of signing queries $m_0,\dots,m_n$ to $\mathcal{B}$. $\mathcal{B}$ responds by issuing the corresponding key query to the challenger and forwarding the answer back the response to $\mathcal{A}$. 
        \item $\mathcal{A}$ will output a signature forgery $(m,\sigma)$ which it didn't issue the sign query $m$. 
        \item $\mathcal{B}$ then choose two random message $t_0,t_1\stackrel{\$}{\gets}\mathcal{M}_{IBE}$ and issue encryption query with the identity $m$.
        \item $\mathcal{B} $gets back the ciphertext $c\stackrel{\$}{\gets}E(mpk,m,t_b)$ for $b\in{0,1}$. Then it runs $t'\gets D(\sigma,c)$ and output $b'=t'$. 
    \end{enumerate}
    We observe that 
    \begin{itemize}
        \item when $b = 1$, then $c\stackrel{\$}{\gets}E(mpk, m,t_1)$, and $\mathcal{B}$ outputs 1 with probability the same as the probability of $\mathcal{A}$ breaks digital signature, we note as $SIGadv[\mathcal{A,S}]$
        \item when $b = 0$, then $c\stackrel{\$}{\gets}E(mpk, m,t_0)$, and $\mathcal{B}$ outputs 1 with probability $1/|\mathcal{M}_{IBE}|$ since $\mathcal{B}$ can only make random guess in the message space. 
    \end{itemize}
    We then have the probability of $\mathcal{B}$ break IBE scheme
    \[
        \frac{1}{2}(SIGadv[\mathcal{A,S}] + 1/|\mathcal{M}_{IBE}|)
    \]
    which is not negligible if $SIGadv[\mathcal{A,S}]$ is not negligible. \qed

\end{proof}
